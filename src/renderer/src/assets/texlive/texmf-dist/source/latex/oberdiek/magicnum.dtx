% \iffalse meta-comment
%
% File: magicnum.dtx
% Version: 2011/04/10 v1.4
% Info: Magic numbers
%
% Copyright (C) 2007, 2009-2011 by
%    Heiko Oberdiek <heiko.oberdiek at googlemail.com>
%
% This work may be distributed and/or modified under the
% conditions of the LaTeX Project Public License, either
% version 1.3c of this license or (at your option) any later
% version. This version of this license is in
%    http://www.latex-project.org/lppl/lppl-1-3c.txt
% and the latest version of this license is in
%    http://www.latex-project.org/lppl.txt
% and version 1.3 or later is part of all distributions of
% LaTeX version 2005/12/01 or later.
%
% This work has the LPPL maintenance status "maintained".
%
% This Current Maintainer of this work is Heiko Oberdiek.
%
% The Base Interpreter refers to any `TeX-Format',
% because some files are installed in TDS:tex/generic//.
%
% This work consists of the main source file magicnum.dtx
% and the derived files
%    magicnum.sty, magicnum.pdf, magicnum.ins, magicnum.drv, magicnum.txt,
%    magicnum-test1.tex, magicnum-test2.tex, magicnum-test3.tex,
%    magicnum-test4.tex, magicnum.lua, oberdiek.magicnum.lua.
%
% Distribution:
%    CTAN:macros/latex/contrib/oberdiek/magicnum.dtx
%    CTAN:macros/latex/contrib/oberdiek/magicnum.pdf
%
% Unpacking:
%    (a) If magicnum.ins is present:
%           tex magicnum.ins
%    (b) Without magicnum.ins:
%           tex magicnum.dtx
%    (c) If you insist on using LaTeX
%           latex \let\install=y% \iffalse meta-comment
%
% File: magicnum.dtx
% Version: 2011/04/10 v1.4
% Info: Magic numbers
%
% Copyright (C) 2007, 2009-2011 by
%    Heiko Oberdiek <heiko.oberdiek at googlemail.com>
%
% This work may be distributed and/or modified under the
% conditions of the LaTeX Project Public License, either
% version 1.3c of this license or (at your option) any later
% version. This version of this license is in
%    http://www.latex-project.org/lppl/lppl-1-3c.txt
% and the latest version of this license is in
%    http://www.latex-project.org/lppl.txt
% and version 1.3 or later is part of all distributions of
% LaTeX version 2005/12/01 or later.
%
% This work has the LPPL maintenance status "maintained".
%
% This Current Maintainer of this work is Heiko Oberdiek.
%
% The Base Interpreter refers to any `TeX-Format',
% because some files are installed in TDS:tex/generic//.
%
% This work consists of the main source file magicnum.dtx
% and the derived files
%    magicnum.sty, magicnum.pdf, magicnum.ins, magicnum.drv, magicnum.txt,
%    magicnum-test1.tex, magicnum-test2.tex, magicnum-test3.tex,
%    magicnum-test4.tex, magicnum.lua, oberdiek.magicnum.lua.
%
% Distribution:
%    CTAN:macros/latex/contrib/oberdiek/magicnum.dtx
%    CTAN:macros/latex/contrib/oberdiek/magicnum.pdf
%
% Unpacking:
%    (a) If magicnum.ins is present:
%           tex magicnum.ins
%    (b) Without magicnum.ins:
%           tex magicnum.dtx
%    (c) If you insist on using LaTeX
%           latex \let\install=y% \iffalse meta-comment
%
% File: magicnum.dtx
% Version: 2011/04/10 v1.4
% Info: Magic numbers
%
% Copyright (C) 2007, 2009-2011 by
%    Heiko Oberdiek <heiko.oberdiek at googlemail.com>
%
% This work may be distributed and/or modified under the
% conditions of the LaTeX Project Public License, either
% version 1.3c of this license or (at your option) any later
% version. This version of this license is in
%    http://www.latex-project.org/lppl/lppl-1-3c.txt
% and the latest version of this license is in
%    http://www.latex-project.org/lppl.txt
% and version 1.3 or later is part of all distributions of
% LaTeX version 2005/12/01 or later.
%
% This work has the LPPL maintenance status "maintained".
%
% This Current Maintainer of this work is Heiko Oberdiek.
%
% The Base Interpreter refers to any `TeX-Format',
% because some files are installed in TDS:tex/generic//.
%
% This work consists of the main source file magicnum.dtx
% and the derived files
%    magicnum.sty, magicnum.pdf, magicnum.ins, magicnum.drv, magicnum.txt,
%    magicnum-test1.tex, magicnum-test2.tex, magicnum-test3.tex,
%    magicnum-test4.tex, magicnum.lua, oberdiek.magicnum.lua.
%
% Distribution:
%    CTAN:macros/latex/contrib/oberdiek/magicnum.dtx
%    CTAN:macros/latex/contrib/oberdiek/magicnum.pdf
%
% Unpacking:
%    (a) If magicnum.ins is present:
%           tex magicnum.ins
%    (b) Without magicnum.ins:
%           tex magicnum.dtx
%    (c) If you insist on using LaTeX
%           latex \let\install=y% \iffalse meta-comment
%
% File: magicnum.dtx
% Version: 2011/04/10 v1.4
% Info: Magic numbers
%
% Copyright (C) 2007, 2009-2011 by
%    Heiko Oberdiek <heiko.oberdiek at googlemail.com>
%
% This work may be distributed and/or modified under the
% conditions of the LaTeX Project Public License, either
% version 1.3c of this license or (at your option) any later
% version. This version of this license is in
%    http://www.latex-project.org/lppl/lppl-1-3c.txt
% and the latest version of this license is in
%    http://www.latex-project.org/lppl.txt
% and version 1.3 or later is part of all distributions of
% LaTeX version 2005/12/01 or later.
%
% This work has the LPPL maintenance status "maintained".
%
% This Current Maintainer of this work is Heiko Oberdiek.
%
% The Base Interpreter refers to any `TeX-Format',
% because some files are installed in TDS:tex/generic//.
%
% This work consists of the main source file magicnum.dtx
% and the derived files
%    magicnum.sty, magicnum.pdf, magicnum.ins, magicnum.drv, magicnum.txt,
%    magicnum-test1.tex, magicnum-test2.tex, magicnum-test3.tex,
%    magicnum-test4.tex, magicnum.lua, oberdiek.magicnum.lua.
%
% Distribution:
%    CTAN:macros/latex/contrib/oberdiek/magicnum.dtx
%    CTAN:macros/latex/contrib/oberdiek/magicnum.pdf
%
% Unpacking:
%    (a) If magicnum.ins is present:
%           tex magicnum.ins
%    (b) Without magicnum.ins:
%           tex magicnum.dtx
%    (c) If you insist on using LaTeX
%           latex \let\install=y\input{magicnum.dtx}
%        (quote the arguments according to the demands of your shell)
%
% Documentation:
%    (a) If magicnum.drv is present:
%           latex magicnum.drv
%    (b) Without magicnum.drv:
%           latex magicnum.dtx; ...
%    The class ltxdoc loads the configuration file ltxdoc.cfg
%    if available. Here you can specify further options, e.g.
%    use A4 as paper format:
%       \PassOptionsToClass{a4paper}{article}
%
%    Programm calls to get the documentation (example):
%       pdflatex magicnum.dtx
%       makeindex -s gind.ist magicnum.idx
%       pdflatex magicnum.dtx
%       makeindex -s gind.ist magicnum.idx
%       pdflatex magicnum.dtx
%
% Installation:
%    TDS:tex/generic/oberdiek/magicnum.sty
%    TDS:scripts/oberdiek/magicnum.lua
%    TDS:scripts/oberdiek/oberdiek.magicnum.lua
%    TDS:doc/latex/oberdiek/magicnum.pdf
%    TDS:doc/latex/oberdiek/magicnum.txt
%    TDS:doc/latex/oberdiek/test/magicnum-test1.tex
%    TDS:doc/latex/oberdiek/test/magicnum-test2.tex
%    TDS:doc/latex/oberdiek/test/magicnum-test3.tex
%    TDS:doc/latex/oberdiek/test/magicnum-test4.tex
%    TDS:source/latex/oberdiek/magicnum.dtx
%
%<*ignore>
\begingroup
  \catcode123=1 %
  \catcode125=2 %
  \def\x{LaTeX2e}%
\expandafter\endgroup
\ifcase 0\ifx\install y1\fi\expandafter
         \ifx\csname processbatchFile\endcsname\relax\else1\fi
         \ifx\fmtname\x\else 1\fi\relax
\else\csname fi\endcsname
%</ignore>
%<*install>
\input docstrip.tex
\Msg{************************************************************************}
\Msg{* Installation}
\Msg{* Package: magicnum 2011/04/10 v1.4 Magic numbers (HO)}
\Msg{************************************************************************}

\keepsilent
\askforoverwritefalse

\let\MetaPrefix\relax
\preamble

This is a generated file.

Project: magicnum
Version: 2011/04/10 v1.4

Copyright (C) 2007, 2009-2011 by
   Heiko Oberdiek <heiko.oberdiek at googlemail.com>

This work may be distributed and/or modified under the
conditions of the LaTeX Project Public License, either
version 1.3c of this license or (at your option) any later
version. This version of this license is in
   http://www.latex-project.org/lppl/lppl-1-3c.txt
and the latest version of this license is in
   http://www.latex-project.org/lppl.txt
and version 1.3 or later is part of all distributions of
LaTeX version 2005/12/01 or later.

This work has the LPPL maintenance status "maintained".

This Current Maintainer of this work is Heiko Oberdiek.

The Base Interpreter refers to any `TeX-Format',
because some files are installed in TDS:tex/generic//.

This work consists of the main source file magicnum.dtx
and the derived files
   magicnum.sty, magicnum.pdf, magicnum.ins, magicnum.drv, magicnum.txt,
   magicnum-test1.tex, magicnum-test2.tex, magicnum-test3.tex,
   magicnum-test4.tex, magicnum.lua, oberdiek.magicnum.lua.

\endpreamble
\let\MetaPrefix\DoubleperCent

\generate{%
  \file{magicnum.ins}{\from{magicnum.dtx}{install}}%
  \file{magicnum.drv}{\from{magicnum.dtx}{driver}}%
  \usedir{tex/generic/oberdiek}%
  \file{magicnum.sty}{\from{magicnum.dtx}{package}}%
  \usedir{doc/latex/oberdiek/test}%
  \file{magicnum-test1.tex}{\from{magicnum.dtx}{test1}}%
  \file{magicnum-test2.tex}{\from{magicnum.dtx}{testplain,testdata}}%
  \file{magicnum-test3.tex}{\from{magicnum.dtx}{testlatex,testdata}}%
  \file{magicnum-test4.tex}{\from{magicnum.dtx}{test4}}%
  \nopreamble
  \nopostamble
  \usedir{doc/latex/oberdiek}%
  \file{magicnum.txt}{\from{magicnum.dtx}{data}}%
  \usedir{source/latex/oberdiek/catalogue}%
  \file{magicnum.xml}{\from{magicnum.dtx}{catalogue}}%
}
\def\MetaPrefix{-- }
\def\defaultpostamble{%
  \MetaPrefix^^J%
  \MetaPrefix\space End of File `\outFileName'.%
}
\def\currentpostamble{\defaultpostamble}%
\generate{%
  \usedir{scripts/oberdiek}%
  \file{magicnum.lua}{\from{magicnum.dtx}{lua}}%
  \file{oberdiek.magicnum.lua}{\from{magicnum.dtx}{lua}}%
}

\catcode32=13\relax% active space
\let =\space%
\Msg{************************************************************************}
\Msg{*}
\Msg{* To finish the installation you have to move the following}
\Msg{* file into a directory searched by TeX:}
\Msg{*}
\Msg{*     magicnum.sty}
\Msg{*}
\Msg{* And install the following script files:}
\Msg{*}
\Msg{*     magicnum.lua, oberdiek.magicnum.lua}
\Msg{*}
\Msg{* To produce the documentation run the file `magicnum.drv'}
\Msg{* through LaTeX.}
\Msg{*}
\Msg{* Happy TeXing!}
\Msg{*}
\Msg{************************************************************************}

\endbatchfile
%</install>
%<*ignore>
\fi
%</ignore>
%<*driver>
\NeedsTeXFormat{LaTeX2e}
\ProvidesFile{magicnum.drv}%
  [2011/04/10 v1.4 Magic numbers (HO)]%
\documentclass{ltxdoc}
\usepackage{holtxdoc}[2011/11/22]
\usepackage{array}
\begin{document}
  \DocInput{magicnum.dtx}%
\end{document}
%</driver>
% \fi
%
% \CheckSum{755}
%
% \CharacterTable
%  {Upper-case    \A\B\C\D\E\F\G\H\I\J\K\L\M\N\O\P\Q\R\S\T\U\V\W\X\Y\Z
%   Lower-case    \a\b\c\d\e\f\g\h\i\j\k\l\m\n\o\p\q\r\s\t\u\v\w\x\y\z
%   Digits        \0\1\2\3\4\5\6\7\8\9
%   Exclamation   \!     Double quote  \"     Hash (number) \#
%   Dollar        \$     Percent       \%     Ampersand     \&
%   Acute accent  \'     Left paren    \(     Right paren   \)
%   Asterisk      \*     Plus          \+     Comma         \,
%   Minus         \-     Point         \.     Solidus       \/
%   Colon         \:     Semicolon     \;     Less than     \<
%   Equals        \=     Greater than  \>     Question mark \?
%   Commercial at \@     Left bracket  \[     Backslash     \\
%   Right bracket \]     Circumflex    \^     Underscore    \_
%   Grave accent  \`     Left brace    \{     Vertical bar  \|
%   Right brace   \}     Tilde         \~}
%
% \GetFileInfo{magicnum.drv}
%
% \title{The \xpackage{magicnum} package}
% \date{2011/04/10 v1.4}
% \author{Heiko Oberdiek\\\xemail{heiko.oberdiek at googlemail.com}}
%
% \maketitle
%
% \begin{abstract}
% This packages allows to access magic numbers by a hierarchical
% name system.
% \end{abstract}
%
% \tableofcontents
%
% \hypersetup{bookmarksopenlevel=2}
% \section{Documentation}
%
% \subsection{Introduction}
%
% Especially since \eTeX\ there are many integer values
% with special meanings, such as catcodes, group types, \dots
% Package \xpackage{etex}, enabled by options, defines
% macros in the user namespace for these values.
%
% This package goes another approach for storing the names and values.
% \begin{itemize}
% \item If \LuaTeX\ is available, they
% are stored in Lua tables.
% \item Without \LuaTeX\ they are remembered using internal
% macros.
% \end{itemize}
%
% \subsection{User interface}
%
% The integer values and names are organized in a hierarchical
% scheme of categories with the property names as leaves.
% Example: \eTeX's \cs{currentgrouplevel} reports |2| for a
% group caused by \cs{hbox}. This package has choosen to organize
% the group types in a main category |etex| and its subcategory
% |grouptype|:
% \begin{quote}
%   |etex.grouptype.hbox| = |2|
% \end{quote}
% The property name |hbox| in category |etex.grouptype| has value |2|.
% Dots are used to separate components.
%
% If you want to have the value, the access key is constructed by
% the category with all its components and the property name.
% For the opposite the value is used instead of the property name.
%
% Values are always integers (including negative numbers).
%
% \subsubsection{\cs{magicnum}}
%
% \begin{declcs}{magicnum} \M{access key}
% \end{declcs}
% Macro \cs{magicnum} expects an access key as argument and
% expands to the requested data. The macro is always expandable.
% In case of errors the expansion result is empty.
%
% The same macro is also used for getting a property name.
% In this case the property name part in the access key is
% replaced by the value.
%
% The catcodes
% of the resulting numbers and strings follow \TeX's tradition of
% \cs{string}, \cs{meaning}, \dots: The space has catcode 10
% (|tex.catcode.space|) and the other characters have catcode
% 12 (|tex.catcode.other|).
%
% Examples:
% \begin{quote}
%   |\magicnum{etex.grouptype.hbox}| $\Rightarrow$ |2|\\
%   |\magicnum{tex.catcode.14}| $\Rightarrow$ |comment|\\
%   |\magicnum{tex.catcode.undefined}| $\Rightarrow$ $\emptyset$
% \end{quote}
%
% \subsubsection{Properties}
%
% \begin{itemize}
% \item The components of a category are either subcategories or
%       key value pairs, but not both.
% \item The full specified property names are unique and thus
%       has one integer value exactly.
% \item Also the values inside a category are unique.
%       This condition is a prerequisite for the reverse mapping
%       of \cs{magicnum}.
% \item All names start with a letter. Only letters or digits
%       may follow.
% \end{itemize}
%
% \subsection{Data}
%
%  \subsubsection{\texorpdfstring{Category }{}\texttt{tex.catcode}}
%
% \begin{quote}
% \begin{tabular}{@{}>{\ttfamily}l>{\ttfamily}l@{}}
%    tex.catcode.escape & 0\\
%    tex.catcode.begingroup & 1\\
%    tex.catcode.endgroup & 2\\
%    tex.catcode.math & 3\\
%    tex.catcode.align & 4\\
%    tex.catcode.eol & 5\\
%    tex.catcode.parameter & 6\\
%    tex.catcode.superscript & 7\\
%    tex.catcode.subscript & 8\\
%    tex.catcode.ignore & 9\\
%    tex.catcode.space & 10\\
%    tex.catcode.letter & 11\\
%    tex.catcode.other & 12\\
%    tex.catcode.active & 13\\
%    tex.catcode.comment & 14\\
%    tex.catcode.invalid & 15\\
%  \end{tabular}
%  \end{quote}
%
%  \subsubsection{\texorpdfstring{Category }{}\texttt{etex.grouptype}}
%
% \begin{quote}
% \begin{tabular}{@{}>{\ttfamily}l>{\ttfamily}l@{}}
%    etex.grouptype.bottomlevel & 0\\
%    etex.grouptype.simple & 1\\
%    etex.grouptype.hbox & 2\\
%    etex.grouptype.adjustedhbox & 3\\
%    etex.grouptype.vbox & 4\\
%    etex.grouptype.align & 5\\
%    etex.grouptype.noalign & 6\\
%    etex.grouptype.output & 8\\
%    etex.grouptype.math & 9\\
%    etex.grouptype.disc & 10\\
%    etex.grouptype.insert & 11\\
%    etex.grouptype.vcenter & 12\\
%    etex.grouptype.mathchoice & 13\\
%    etex.grouptype.semisimple & 14\\
%    etex.grouptype.mathshift & 15\\
%    etex.grouptype.mathleft & 16\\
%  \end{tabular}
%  \end{quote}
%
%  \subsubsection{\texorpdfstring{Category }{}\texttt{etex.iftype}}
%
% \begin{quote}
% \begin{tabular}{@{}>{\ttfamily}l>{\ttfamily}l@{}}
%    etex.iftype.none & 0\\
%    etex.iftype.char & 1\\
%    etex.iftype.cat & 2\\
%    etex.iftype.num & 3\\
%    etex.iftype.dim & 4\\
%    etex.iftype.odd & 5\\
%    etex.iftype.vmode & 6\\
%    etex.iftype.hmode & 7\\
%    etex.iftype.mmode & 8\\
%    etex.iftype.inner & 9\\
%    etex.iftype.void & 10\\
%    etex.iftype.hbox & 11\\
%    etex.iftype.vbox & 12\\
%    etex.iftype.x & 13\\
%    etex.iftype.eof & 14\\
%    etex.iftype.true & 15\\
%    etex.iftype.false & 16\\
%    etex.iftype.case & 17\\
%    etex.iftype.defined & 18\\
%    etex.iftype.csname & 19\\
%    etex.iftype.fontchar & 20\\
%  \end{tabular}
%  \end{quote}
%
%  \subsubsection{\texorpdfstring{Category }{}\texttt{etex.nodetype}}
%
% \begin{quote}
% \begin{tabular}{@{}>{\ttfamily}l>{\ttfamily}l@{}}
%    etex.nodetype.none & -1\\
%    etex.nodetype.char & 0\\
%    etex.nodetype.hlist & 1\\
%    etex.nodetype.vlist & 2\\
%    etex.nodetype.rule & 3\\
%    etex.nodetype.ins & 4\\
%    etex.nodetype.mark & 5\\
%    etex.nodetype.adjust & 6\\
%    etex.nodetype.ligature & 7\\
%    etex.nodetype.disc & 8\\
%    etex.nodetype.whatsit & 9\\
%    etex.nodetype.math & 10\\
%    etex.nodetype.glue & 11\\
%    etex.nodetype.kern & 12\\
%    etex.nodetype.penalty & 13\\
%    etex.nodetype.unset & 14\\
%    etex.nodetype.maths & 15\\
%  \end{tabular}
%  \end{quote}
%
%  \subsubsection{\texorpdfstring{Category }{}\texttt{etex.interactionmode}}
%
% \begin{quote}
% \begin{tabular}{@{}>{\ttfamily}l>{\ttfamily}l@{}}
%    etex.interactionmode.batch & 0\\
%    etex.interactionmode.nonstop & 1\\
%    etex.interactionmode.scroll & 2\\
%    etex.interactionmode.errorstop & 3\\
%  \end{tabular}
%  \end{quote}
%
%  \subsubsection{\texorpdfstring{Category }{}\texttt{luatex.pdfliteral.mode}}
%
% \begin{quote}
% \begin{tabular}{@{}>{\ttfamily}l>{\ttfamily}l@{}}
%    luatex.pdfliteral.mode.setorigin & 0\\
%    luatex.pdfliteral.mode.page & 1\\
%    luatex.pdfliteral.mode.direct & 2\\
%  \end{tabular}
%  \end{quote}
%
%
% \hypersetup{bookmarksopenlevel=1}
%
% \StopEventually{
% }
%
% \section{Implementation}
%
%    \begin{macrocode}
%<*package>
%    \end{macrocode}
%
% \subsection{Reload check and package identification}
%    Reload check, especially if the package is not used with \LaTeX.
%    \begin{macrocode}
\begingroup\catcode61\catcode48\catcode32=10\relax%
  \catcode13=5 % ^^M
  \endlinechar=13 %
  \catcode35=6 % #
  \catcode39=12 % '
  \catcode44=12 % ,
  \catcode45=12 % -
  \catcode46=12 % .
  \catcode58=12 % :
  \catcode64=11 % @
  \catcode123=1 % {
  \catcode125=2 % }
  \expandafter\let\expandafter\x\csname ver@magicnum.sty\endcsname
  \ifx\x\relax % plain-TeX, first loading
  \else
    \def\empty{}%
    \ifx\x\empty % LaTeX, first loading,
      % variable is initialized, but \ProvidesPackage not yet seen
    \else
      \expandafter\ifx\csname PackageInfo\endcsname\relax
        \def\x#1#2{%
          \immediate\write-1{Package #1 Info: #2.}%
        }%
      \else
        \def\x#1#2{\PackageInfo{#1}{#2, stopped}}%
      \fi
      \x{magicnum}{The package is already loaded}%
      \aftergroup\endinput
    \fi
  \fi
\endgroup%
%    \end{macrocode}
%    Package identification:
%    \begin{macrocode}
\begingroup\catcode61\catcode48\catcode32=10\relax%
  \catcode13=5 % ^^M
  \endlinechar=13 %
  \catcode35=6 % #
  \catcode39=12 % '
  \catcode40=12 % (
  \catcode41=12 % )
  \catcode44=12 % ,
  \catcode45=12 % -
  \catcode46=12 % .
  \catcode47=12 % /
  \catcode58=12 % :
  \catcode64=11 % @
  \catcode91=12 % [
  \catcode93=12 % ]
  \catcode123=1 % {
  \catcode125=2 % }
  \expandafter\ifx\csname ProvidesPackage\endcsname\relax
    \def\x#1#2#3[#4]{\endgroup
      \immediate\write-1{Package: #3 #4}%
      \xdef#1{#4}%
    }%
  \else
    \def\x#1#2[#3]{\endgroup
      #2[{#3}]%
      \ifx#1\@undefined
        \xdef#1{#3}%
      \fi
      \ifx#1\relax
        \xdef#1{#3}%
      \fi
    }%
  \fi
\expandafter\x\csname ver@magicnum.sty\endcsname
\ProvidesPackage{magicnum}%
  [2011/04/10 v1.4 Magic numbers (HO)]%
%    \end{macrocode}
%
% \subsection{Catcodes}
%
%    \begin{macrocode}
\begingroup\catcode61\catcode48\catcode32=10\relax%
  \catcode13=5 % ^^M
  \endlinechar=13 %
  \catcode123=1 % {
  \catcode125=2 % }
  \catcode64=11 % @
  \def\x{\endgroup
    \expandafter\edef\csname magicnum@AtEnd\endcsname{%
      \endlinechar=\the\endlinechar\relax
      \catcode13=\the\catcode13\relax
      \catcode32=\the\catcode32\relax
      \catcode35=\the\catcode35\relax
      \catcode61=\the\catcode61\relax
      \catcode64=\the\catcode64\relax
      \catcode123=\the\catcode123\relax
      \catcode125=\the\catcode125\relax
    }%
  }%
\x\catcode61\catcode48\catcode32=10\relax%
\catcode13=5 % ^^M
\endlinechar=13 %
\catcode35=6 % #
\catcode64=11 % @
\catcode123=1 % {
\catcode125=2 % }
\def\TMP@EnsureCode#1#2{%
  \edef\magicnum@AtEnd{%
    \magicnum@AtEnd
    \catcode#1=\the\catcode#1\relax
  }%
  \catcode#1=#2\relax
}
\TMP@EnsureCode{34}{12}% "
\TMP@EnsureCode{39}{12}% '
\TMP@EnsureCode{40}{12}% (
\TMP@EnsureCode{41}{12}% )
\TMP@EnsureCode{42}{12}% *
\TMP@EnsureCode{44}{12}% ,
\TMP@EnsureCode{45}{12}% -
\TMP@EnsureCode{46}{12}% .
\TMP@EnsureCode{47}{12}% /
\TMP@EnsureCode{58}{12}% :
\TMP@EnsureCode{60}{12}% <
\TMP@EnsureCode{62}{12}% >
\TMP@EnsureCode{91}{12}% [
\TMP@EnsureCode{93}{12}% ]
\edef\magicnum@AtEnd{\magicnum@AtEnd\noexpand\endinput}
%    \end{macrocode}
%
% \subsection{Check for previous definition}
%
%    \begin{macrocode}
\begingroup\expandafter\expandafter\expandafter\endgroup
\expandafter\ifx\csname newcommand\endcsname\relax
  \expandafter\ifx\csname magicnum\endcsname\relax
  \else
    \input infwarerr.sty\relax
    \@PackageError{magicnum}{%
      \string\magicnum\space is already defined%
    }\@ehc
  \fi
\else
  \newcommand*{\magicnum}{}%
\fi
%    \end{macrocode}
%
% \subsection{Without \LuaTeX}
%
%    \begin{macrocode}
\begingroup\expandafter\expandafter\expandafter\endgroup
\expandafter\ifx\csname directlua\endcsname\relax
%    \end{macrocode}
%
%    \begin{macro}{\magicnum}
%    \begin{macrocode}
  \begingroup\expandafter\expandafter\expandafter\endgroup
  \expandafter\ifx\csname ifcsname\endcsname\relax
    \def\magicnum#1{%
      \expandafter\ifx\csname MG@#1\endcsname\relax
      \else
        \csname MG@#1\endcsname
      \fi
    }%
  \else
    \begingroup
      \edef\x{\endgroup
        \def\noexpand\magicnum##1{%
          \expandafter\noexpand\csname
          ifcsname\endcsname MG@##1\noexpand\endcsname
            \noexpand\csname MG@##1%
                 \noexpand\expandafter\noexpand\endcsname
          \expandafter\noexpand\csname fi\endcsname
        }%
      }%
    \x
  \fi
%    \end{macrocode}
%    \end{macro}
%
%    \begin{macrocode}
\else
%    \end{macrocode}
%
% \subsection{With \LuaTeX}
%
%    \begin{macrocode}
  \begingroup\expandafter\expandafter\expandafter\endgroup
  \expandafter\ifx\csname RequirePackage\endcsname\relax
    \input ifluatex.sty\relax
    \input infwarerr.sty\relax
  \else
    \RequirePackage{ifluatex}[2010/03/01]%
    \RequirePackage{infwarerr}[2010/04/08]%
  \fi
%    \end{macrocode}
%
%    \begin{macro}{\magicnum@directlua}
%    \begin{macrocode}
  \ifnum\luatexversion<36 %
    \def\magicnum@directlua{\directlua0 }%
  \else
    \let\magicnum@directlua\directlua
  \fi
%    \end{macrocode}
%    \end{macro}
%    \begin{macrocode}
  \magicnum@directlua{%
    require("oberdiek.magicnum")%
  }%
  \begingroup
    \def\x{2011/04/10 v1.4}%
    \def\StripPrefix#1>{}%
    \edef\x{\expandafter\StripPrefix\meaning\x}%
    \edef\y{%
      \magicnum@directlua{%
        if oberdiek.magicnum.getversion then %
          oberdiek.magicnum.getversion()%
        end%
      }%
    }%
    \ifx\x\y
    \else
      \@PackageError{magicnum}{%
        Wrong version of lua module.\MessageBreak
        Package version: \x\MessageBreak
        Lua module: \y
      }\@ehc
    \fi
  \endgroup
%    \end{macrocode}
%    \begin{macro}{\luaescapestring}
%    \begin{macrocode}
  \begingroup
    \expandafter\ifx\csname luaescapestring\endcsname\relax
      \directlua{%
        if tex.enableprimitives then %
          tex.enableprimitives('magicnum@', {'luaescapestring'})%
        end%
      }%
      \global\let\luaescapestring\magicnum@luaescapestring
    \fi
    \expandafter\ifx\csname luaescapestring\endcsname\relax
      \escapechar=92 %
      \@PackageError{magicnum}{%
        Missing \string\luaescapestring
      }\@ehc
    \fi
  \endgroup
%    \end{macrocode}
%    \end{macro}
%    \begin{macro}{\magicnum}
%    \begin{macrocode}
  \def\magicnum#1{%
    \magicnum@directlua{%
      oberdiek.magicnum.get("\luaescapestring{#1}")%
    }%
  }%
%    \end{macrocode}
%    \end{macro}
%
%    \begin{macrocode}
  \expandafter\magicnum@AtEnd
\fi%
%</package>
%    \end{macrocode}
%
% \subsection{Data}
%
% \subsubsection{Plain data}
%
%    \begin{macrocode}
%<*data>
tex.catcode
  escape = 0
  begingroup = 1
  endgroup = 2
  math = 3
  align = 4
  eol = 5
  parameter = 6
  superscript = 7
  subscript = 8
  ignore = 9
  space = 10
  letter = 11
  other = 12
  active = 13
  comment = 14
  invalid = 15
etex.grouptype
  bottomlevel = 0
  simple = 1
  hbox = 2
  adjustedhbox = 3
  vbox = 4
  align = 5
  noalign = 6
  output = 8
  math = 9
  disc = 10
  insert = 11
  vcenter = 12
  mathchoice = 13
  semisimple = 14
  mathshift = 15
  mathleft = 16
etex.iftype
  none = 0
  char = 1
  cat = 2
  num = 3
  dim = 4
  odd = 5
  vmode = 6
  hmode = 7
  mmode = 8
  inner = 9
  void = 10
  hbox = 11
  vbox = 12
  x = 13
  eof = 14
  true = 15
  false = 16
  case = 17
  defined = 18
  csname = 19
  fontchar = 20
etex.nodetype
  none = -1
  char = 0
  hlist = 1
  vlist = 2
  rule = 3
  ins = 4
  mark = 5
  adjust = 6
  ligature = 7
  disc = 8
  whatsit = 9
  math = 10
  glue = 11
  kern = 12
  penalty = 13
  unset = 14
  maths = 15
etex.interactionmode
  batch = 0
  nonstop = 1
  scroll = 2
  errorstop = 3
luatex.pdfliteral.mode
  setorigin = 0
  page = 1
  direct = 2
%</data>
%    \end{macrocode}
%
% \subsubsection{Data for \TeX}
%
%    \begin{macrocode}
%<*package>
%    \end{macrocode}
%    \begin{macro}{\magicnum@add}
%    \begin{macrocode}
\begingroup\expandafter\expandafter\expandafter\endgroup
\expandafter\ifx\csname detokenize\endcsname\relax
  \def\magicnum@add#1#2#3{%
    \expandafter\magicnum@@add
        \csname MG@#1.#2\expandafter\endcsname
        \csname MG@#1.#3\endcsname
       {#3}{#2}%
  }%
  \def\magicnum@@add#1#2#3#4{%
    \def#1{#3}%
    \def#2{#4}%
    \edef#1{%
      \expandafter\strip@prefix\meaning#1%
    }%
    \edef#2{%
      \expandafter\strip@prefix\meaning#2%
    }%
  }%
  \expandafter\ifx\csname strip@prefix\endcsname\relax
    \def\strip@prefix#1->{}%
  \fi
\else
  \def\magicnum@add#1#2#3{%
    \expandafter\edef\csname MG@#1.#2\endcsname{%
      \detokenize{#3}%
    }%
    \expandafter\edef\csname MG@#1.#3\endcsname{%
      \detokenize{#2}%
    }%
  }%
\fi
%    \end{macrocode}
%    \end{macro}
%    \begin{macrocode}
\magicnum@add{tex.catcode}{escape}{0}
\magicnum@add{tex.catcode}{begingroup}{1}
\magicnum@add{tex.catcode}{endgroup}{2}
\magicnum@add{tex.catcode}{math}{3}
\magicnum@add{tex.catcode}{align}{4}
\magicnum@add{tex.catcode}{eol}{5}
\magicnum@add{tex.catcode}{parameter}{6}
\magicnum@add{tex.catcode}{superscript}{7}
\magicnum@add{tex.catcode}{subscript}{8}
\magicnum@add{tex.catcode}{ignore}{9}
\magicnum@add{tex.catcode}{space}{10}
\magicnum@add{tex.catcode}{letter}{11}
\magicnum@add{tex.catcode}{other}{12}
\magicnum@add{tex.catcode}{active}{13}
\magicnum@add{tex.catcode}{comment}{14}
\magicnum@add{tex.catcode}{invalid}{15}
\magicnum@add{etex.grouptype}{bottomlevel}{0}
\magicnum@add{etex.grouptype}{simple}{1}
\magicnum@add{etex.grouptype}{hbox}{2}
\magicnum@add{etex.grouptype}{adjustedhbox}{3}
\magicnum@add{etex.grouptype}{vbox}{4}
\magicnum@add{etex.grouptype}{align}{5}
\magicnum@add{etex.grouptype}{noalign}{6}
\magicnum@add{etex.grouptype}{output}{8}
\magicnum@add{etex.grouptype}{math}{9}
\magicnum@add{etex.grouptype}{disc}{10}
\magicnum@add{etex.grouptype}{insert}{11}
\magicnum@add{etex.grouptype}{vcenter}{12}
\magicnum@add{etex.grouptype}{mathchoice}{13}
\magicnum@add{etex.grouptype}{semisimple}{14}
\magicnum@add{etex.grouptype}{mathshift}{15}
\magicnum@add{etex.grouptype}{mathleft}{16}
\magicnum@add{etex.iftype}{none}{0}
\magicnum@add{etex.iftype}{char}{1}
\magicnum@add{etex.iftype}{cat}{2}
\magicnum@add{etex.iftype}{num}{3}
\magicnum@add{etex.iftype}{dim}{4}
\magicnum@add{etex.iftype}{odd}{5}
\magicnum@add{etex.iftype}{vmode}{6}
\magicnum@add{etex.iftype}{hmode}{7}
\magicnum@add{etex.iftype}{mmode}{8}
\magicnum@add{etex.iftype}{inner}{9}
\magicnum@add{etex.iftype}{void}{10}
\magicnum@add{etex.iftype}{hbox}{11}
\magicnum@add{etex.iftype}{vbox}{12}
\magicnum@add{etex.iftype}{x}{13}
\magicnum@add{etex.iftype}{eof}{14}
\magicnum@add{etex.iftype}{true}{15}
\magicnum@add{etex.iftype}{false}{16}
\magicnum@add{etex.iftype}{case}{17}
\magicnum@add{etex.iftype}{defined}{18}
\magicnum@add{etex.iftype}{csname}{19}
\magicnum@add{etex.iftype}{fontchar}{20}
\magicnum@add{etex.nodetype}{none}{-1}
\magicnum@add{etex.nodetype}{char}{0}
\magicnum@add{etex.nodetype}{hlist}{1}
\magicnum@add{etex.nodetype}{vlist}{2}
\magicnum@add{etex.nodetype}{rule}{3}
\magicnum@add{etex.nodetype}{ins}{4}
\magicnum@add{etex.nodetype}{mark}{5}
\magicnum@add{etex.nodetype}{adjust}{6}
\magicnum@add{etex.nodetype}{ligature}{7}
\magicnum@add{etex.nodetype}{disc}{8}
\magicnum@add{etex.nodetype}{whatsit}{9}
\magicnum@add{etex.nodetype}{math}{10}
\magicnum@add{etex.nodetype}{glue}{11}
\magicnum@add{etex.nodetype}{kern}{12}
\magicnum@add{etex.nodetype}{penalty}{13}
\magicnum@add{etex.nodetype}{unset}{14}
\magicnum@add{etex.nodetype}{maths}{15}
\magicnum@add{etex.interactionmode}{batch}{0}
\magicnum@add{etex.interactionmode}{nonstop}{1}
\magicnum@add{etex.interactionmode}{scroll}{2}
\magicnum@add{etex.interactionmode}{errorstop}{3}
\magicnum@add{luatex.pdfliteral.mode}{setorigin}{0}
\magicnum@add{luatex.pdfliteral.mode}{page}{1}
\magicnum@add{luatex.pdfliteral.mode}{direct}{2}
%    \end{macrocode}
%    \begin{macrocode}
\magicnum@AtEnd%
%</package>
%    \end{macrocode}
%
% \subsubsection{Lua module}
%
%    \begin{macrocode}
%<*lua>
%    \end{macrocode}
%    \begin{macrocode}
module("oberdiek.magicnum", package.seeall)
%    \end{macrocode}
%    \begin{macrocode}
function getversion()
  tex.write("2011/04/10 v1.4")
end
%    \end{macrocode}
%    \begin{macrocode}
local data = {
  ["tex.catcode"] = {
    [0] = "escape",
    [1] = "begingroup",
    [2] = "endgroup",
    [3] = "math",
    [4] = "align",
    [5] = "eol",
    [6] = "parameter",
    [7] = "superscript",
    [8] = "subscript",
    [9] = "ignore",
    [10] = "space",
    [11] = "letter",
    [12] = "other",
    [13] = "active",
    [14] = "comment",
    [15] = "invalid",
    ["active"] = 13,
    ["align"] = 4,
    ["begingroup"] = 1,
    ["comment"] = 14,
    ["endgroup"] = 2,
    ["eol"] = 5,
    ["escape"] = 0,
    ["ignore"] = 9,
    ["invalid"] = 15,
    ["letter"] = 11,
    ["math"] = 3,
    ["other"] = 12,
    ["parameter"] = 6,
    ["space"] = 10,
    ["subscript"] = 8,
    ["superscript"] = 7
  },
  ["etex.grouptype"] = {
    [0] = "bottomlevel",
    [1] = "simple",
    [2] = "hbox",
    [3] = "adjustedhbox",
    [4] = "vbox",
    [5] = "align",
    [6] = "noalign",
    [8] = "output",
    [9] = "math",
    [10] = "disc",
    [11] = "insert",
    [12] = "vcenter",
    [13] = "mathchoice",
    [14] = "semisimple",
    [15] = "mathshift",
    [16] = "mathleft",
    ["adjustedhbox"] = 3,
    ["align"] = 5,
    ["bottomlevel"] = 0,
    ["disc"] = 10,
    ["hbox"] = 2,
    ["insert"] = 11,
    ["math"] = 9,
    ["mathchoice"] = 13,
    ["mathleft"] = 16,
    ["mathshift"] = 15,
    ["noalign"] = 6,
    ["output"] = 8,
    ["semisimple"] = 14,
    ["simple"] = 1,
    ["vbox"] = 4,
    ["vcenter"] = 12
  },
  ["etex.iftype"] = {
    [0] = "none",
    [1] = "char",
    [2] = "cat",
    [3] = "num",
    [4] = "dim",
    [5] = "odd",
    [6] = "vmode",
    [7] = "hmode",
    [8] = "mmode",
    [9] = "inner",
    [10] = "void",
    [11] = "hbox",
    [12] = "vbox",
    [13] = "x",
    [14] = "eof",
    [15] = "true",
    [16] = "false",
    [17] = "case",
    [18] = "defined",
    [19] = "csname",
    [20] = "fontchar",
    ["case"] = 17,
    ["cat"] = 2,
    ["char"] = 1,
    ["csname"] = 19,
    ["defined"] = 18,
    ["dim"] = 4,
    ["eof"] = 14,
    ["false"] = 16,
    ["fontchar"] = 20,
    ["hbox"] = 11,
    ["hmode"] = 7,
    ["inner"] = 9,
    ["mmode"] = 8,
    ["none"] = 0,
    ["num"] = 3,
    ["odd"] = 5,
    ["true"] = 15,
    ["vbox"] = 12,
    ["vmode"] = 6,
    ["void"] = 10,
    ["x"] = 13
  },
  ["etex.nodetype"] = {
    [-1] = "none",
    [0] = "char",
    [1] = "hlist",
    [2] = "vlist",
    [3] = "rule",
    [4] = "ins",
    [5] = "mark",
    [6] = "adjust",
    [7] = "ligature",
    [8] = "disc",
    [9] = "whatsit",
    [10] = "math",
    [11] = "glue",
    [12] = "kern",
    [13] = "penalty",
    [14] = "unset",
    [15] = "maths",
    ["adjust"] = 6,
    ["char"] = 0,
    ["disc"] = 8,
    ["glue"] = 11,
    ["hlist"] = 1,
    ["ins"] = 4,
    ["kern"] = 12,
    ["ligature"] = 7,
    ["mark"] = 5,
    ["math"] = 10,
    ["maths"] = 15,
    ["none"] = -1,
    ["penalty"] = 13,
    ["rule"] = 3,
    ["unset"] = 14,
    ["vlist"] = 2,
    ["whatsit"] = 9
  },
  ["etex.interactionmode"] = {
    [0] = "batch",
    [1] = "nonstop",
    [2] = "scroll",
    [3] = "errorstop",
    ["batch"] = 0,
    ["errorstop"] = 3,
    ["nonstop"] = 1,
    ["scroll"] = 2
  },
  ["luatex.pdfliteral.mode"] = {
    [0] = "setorigin",
    [1] = "page",
    [2] = "direct",
    ["direct"] = 2,
    ["page"] = 1,
    ["setorigin"] = 0
  }
}
%    \end{macrocode}
%    \begin{macrocode}
function get(name)
  local startpos, endpos, category, entry =
      string.find(name, "^(%a[%a%d%.]*)%.(-?[%a%d]+)$")
  if not entry then
    return
  end
  local node = data[category]
  if not node then
    return
  end
  local num = tonumber(entry)
  local value
  if num then
    value = node[num]
    if not value then
      return
    end
  else
    value = node[entry]
    if not value then
      return
    end
    value = "" .. value
  end
  tex.write(value)
end
%    \end{macrocode}
%
%    \begin{macrocode}
%</lua>
%    \end{macrocode}
%
% \section{Test}
%
% \subsection{Catcode checks for loading}
%
%    \begin{macrocode}
%<*test1>
%    \end{macrocode}
%    \begin{macrocode}
\catcode`\{=1 %
\catcode`\}=2 %
\catcode`\#=6 %
\catcode`\@=11 %
\expandafter\ifx\csname count@\endcsname\relax
  \countdef\count@=255 %
\fi
\expandafter\ifx\csname @gobble\endcsname\relax
  \long\def\@gobble#1{}%
\fi
\expandafter\ifx\csname @firstofone\endcsname\relax
  \long\def\@firstofone#1{#1}%
\fi
\expandafter\ifx\csname loop\endcsname\relax
  \expandafter\@firstofone
\else
  \expandafter\@gobble
\fi
{%
  \def\loop#1\repeat{%
    \def\body{#1}%
    \iterate
  }%
  \def\iterate{%
    \body
      \let\next\iterate
    \else
      \let\next\relax
    \fi
    \next
  }%
  \let\repeat=\fi
}%
\def\RestoreCatcodes{}
\count@=0 %
\loop
  \edef\RestoreCatcodes{%
    \RestoreCatcodes
    \catcode\the\count@=\the\catcode\count@\relax
  }%
\ifnum\count@<255 %
  \advance\count@ 1 %
\repeat

\def\RangeCatcodeInvalid#1#2{%
  \count@=#1\relax
  \loop
    \catcode\count@=15 %
  \ifnum\count@<#2\relax
    \advance\count@ 1 %
  \repeat
}
\def\RangeCatcodeCheck#1#2#3{%
  \count@=#1\relax
  \loop
    \ifnum#3=\catcode\count@
    \else
      \errmessage{%
        Character \the\count@\space
        with wrong catcode \the\catcode\count@\space
        instead of \number#3%
      }%
    \fi
  \ifnum\count@<#2\relax
    \advance\count@ 1 %
  \repeat
}
\def\space{ }
\expandafter\ifx\csname LoadCommand\endcsname\relax
  \def\LoadCommand{\input magicnum.sty\relax}%
\fi
\def\Test{%
  \RangeCatcodeInvalid{0}{47}%
  \RangeCatcodeInvalid{58}{64}%
  \RangeCatcodeInvalid{91}{96}%
  \RangeCatcodeInvalid{123}{255}%
  \catcode`\@=12 %
  \catcode`\\=0 %
  \catcode`\%=14 %
  \LoadCommand
  \RangeCatcodeCheck{0}{36}{15}%
  \RangeCatcodeCheck{37}{37}{14}%
  \RangeCatcodeCheck{38}{47}{15}%
  \RangeCatcodeCheck{48}{57}{12}%
  \RangeCatcodeCheck{58}{63}{15}%
  \RangeCatcodeCheck{64}{64}{12}%
  \RangeCatcodeCheck{65}{90}{11}%
  \RangeCatcodeCheck{91}{91}{15}%
  \RangeCatcodeCheck{92}{92}{0}%
  \RangeCatcodeCheck{93}{96}{15}%
  \RangeCatcodeCheck{97}{122}{11}%
  \RangeCatcodeCheck{123}{255}{15}%
  \RestoreCatcodes
}
\Test
\csname @@end\endcsname
\end
%    \end{macrocode}
%    \begin{macrocode}
%</test1>
%    \end{macrocode}
%
% \subsection{Test data}
%
%    \begin{macrocode}
%<*testplain>
\input magicnum.sty\relax
\def\Test#1#2{%
  \edef\result{\magicnum{#1}}%
  \edef\expect{#2}%
  \edef\expect{\expandafter\stripprefix\meaning\expect}%
  \ifx\result\expect
  \else
    \errmessage{%
      Failed: [#1] % hash-ok
      returns [\result] instead of [\expect]%
    }%
  \fi
}
\def\stripprefix#1->{}
%</testplain>
%    \end{macrocode}
%    \begin{macrocode}
%<*testlatex>
\NeedsTeXFormat{LaTeX2e}
\documentclass{minimal}
\usepackage{magicnum}[2011/04/10]
\usepackage{qstest}
\IncludeTests{*}
\LogTests{log}{*}{*}
\newcommand*{\Test}[2]{%
  \Expect*{\magicnum{#1}}{#2}%
}
\begin{qstest}{magicnum}{magicnum}
%</testlatex>
%    \end{macrocode}
%    \begin{macrocode}
%<*testdata>
\Test{tex.catcode.escape}{0}
\Test{tex.catcode.invalid}{15}
\Test{tex.catcode.unknown}{}
\Test{tex.catcode.0}{escape}
\Test{tex.catcode.15}{invalid}
\Test{etex.iftype.true}{15}
\Test{etex.iftype.false}{16}
\Test{etex.iftype.15}{true}
\Test{etex.iftype.16}{false}
\Test{etex.nodetype.none}{-1}
\Test{etex.nodetype.-1}{none}
\Test{luatex.pdfliteral.mode.direct}{2}
\Test{luatex.pdfliteral.mode.1}{page}
\Test{}{}
\Test{unknown}{}
\Test{unknown.foo.bar}{}
\Test{unknown.foo.4}{}
%</testdata>
%    \end{macrocode}
%    \begin{macrocode}
%<*testplain>
\csname @@end\endcsname
\end
%</testplain>
%<*testlatex>
\end{qstest}
\csname @@end\endcsname
%</testlatex>
%    \end{macrocode}
%
% \subsection{Small test for \hologo{iniTeX}}
%
%    \begin{macrocode}
%<*test4>
\catcode`\{=1
\catcode`\}=2
\catcode`\#=6
\input magicnum.sty\relax
\edef\x{\magicnum{tex.catcode.15}}
\edef\y{invalid}
\def\Strip#1>{}
\edef\y{\expandafter\Strip\meaning\y}
\ifx\x\y
  \immediate\write16{Ok}%
\else
  \errmessage{\x<>\y}%
\fi
\csname @@end\endcsname\end
%</test4>
%    \end{macrocode}
%
% \section{Installation}
%
% \subsection{Download}
%
% \paragraph{Package.} This package is available on
% CTAN\footnote{\url{ftp://ftp.ctan.org/tex-archive/}}:
% \begin{description}
% \item[\CTAN{macros/latex/contrib/oberdiek/magicnum.dtx}] The source file.
% \item[\CTAN{macros/latex/contrib/oberdiek/magicnum.pdf}] Documentation.
% \end{description}
%
%
% \paragraph{Bundle.} All the packages of the bundle `oberdiek'
% are also available in a TDS compliant ZIP archive. There
% the packages are already unpacked and the documentation files
% are generated. The files and directories obey the TDS standard.
% \begin{description}
% \item[\CTAN{install/macros/latex/contrib/oberdiek.tds.zip}]
% \end{description}
% \emph{TDS} refers to the standard ``A Directory Structure
% for \TeX\ Files'' (\CTAN{tds/tds.pdf}). Directories
% with \xfile{texmf} in their name are usually organized this way.
%
% \subsection{Bundle installation}
%
% \paragraph{Unpacking.} Unpack the \xfile{oberdiek.tds.zip} in the
% TDS tree (also known as \xfile{texmf} tree) of your choice.
% Example (linux):
% \begin{quote}
%   |unzip oberdiek.tds.zip -d ~/texmf|
% \end{quote}
%
% \paragraph{Script installation.}
% Check the directory \xfile{TDS:scripts/oberdiek/} for
% scripts that need further installation steps.
% Package \xpackage{attachfile2} comes with the Perl script
% \xfile{pdfatfi.pl} that should be installed in such a way
% that it can be called as \texttt{pdfatfi}.
% Example (linux):
% \begin{quote}
%   |chmod +x scripts/oberdiek/pdfatfi.pl|\\
%   |cp scripts/oberdiek/pdfatfi.pl /usr/local/bin/|
% \end{quote}
%
% \subsection{Package installation}
%
% \paragraph{Unpacking.} The \xfile{.dtx} file is a self-extracting
% \docstrip\ archive. The files are extracted by running the
% \xfile{.dtx} through \plainTeX:
% \begin{quote}
%   \verb|tex magicnum.dtx|
% \end{quote}
%
% \paragraph{TDS.} Now the different files must be moved into
% the different directories in your installation TDS tree
% (also known as \xfile{texmf} tree):
% \begin{quote}
% \def\t{^^A
% \begin{tabular}{@{}>{\ttfamily}l@{ $\rightarrow$ }>{\ttfamily}l@{}}
%   magicnum.sty & tex/generic/oberdiek/magicnum.sty\\
%   magicnum.lua & scripts/oberdiek/magicnum.lua\\
%   oberdiek.magicnum.lua & scripts/oberdiek/oberdiek.magicnum.lua\\
%   magicnum.pdf & doc/latex/oberdiek/magicnum.pdf\\
%   magicnum.txt & doc/latex/oberdiek/magicnum.txt\\
%   test/magicnum-test1.tex & doc/latex/oberdiek/test/magicnum-test1.tex\\
%   test/magicnum-test2.tex & doc/latex/oberdiek/test/magicnum-test2.tex\\
%   test/magicnum-test3.tex & doc/latex/oberdiek/test/magicnum-test3.tex\\
%   test/magicnum-test4.tex & doc/latex/oberdiek/test/magicnum-test4.tex\\
%   magicnum.dtx & source/latex/oberdiek/magicnum.dtx\\
% \end{tabular}^^A
% }^^A
% \sbox0{\t}^^A
% \ifdim\wd0>\linewidth
%   \begingroup
%     \advance\linewidth by\leftmargin
%     \advance\linewidth by\rightmargin
%   \edef\x{\endgroup
%     \def\noexpand\lw{\the\linewidth}^^A
%   }\x
%   \def\lwbox{^^A
%     \leavevmode
%     \hbox to \linewidth{^^A
%       \kern-\leftmargin\relax
%       \hss
%       \usebox0
%       \hss
%       \kern-\rightmargin\relax
%     }^^A
%   }^^A
%   \ifdim\wd0>\lw
%     \sbox0{\small\t}^^A
%     \ifdim\wd0>\linewidth
%       \ifdim\wd0>\lw
%         \sbox0{\footnotesize\t}^^A
%         \ifdim\wd0>\linewidth
%           \ifdim\wd0>\lw
%             \sbox0{\scriptsize\t}^^A
%             \ifdim\wd0>\linewidth
%               \ifdim\wd0>\lw
%                 \sbox0{\tiny\t}^^A
%                 \ifdim\wd0>\linewidth
%                   \lwbox
%                 \else
%                   \usebox0
%                 \fi
%               \else
%                 \lwbox
%               \fi
%             \else
%               \usebox0
%             \fi
%           \else
%             \lwbox
%           \fi
%         \else
%           \usebox0
%         \fi
%       \else
%         \lwbox
%       \fi
%     \else
%       \usebox0
%     \fi
%   \else
%     \lwbox
%   \fi
% \else
%   \usebox0
% \fi
% \end{quote}
% If you have a \xfile{docstrip.cfg} that configures and enables \docstrip's
% TDS installing feature, then some files can already be in the right
% place, see the documentation of \docstrip.
%
% \subsection{Refresh file name databases}
%
% If your \TeX~distribution
% (\teTeX, \mikTeX, \dots) relies on file name databases, you must refresh
% these. For example, \teTeX\ users run \verb|texhash| or
% \verb|mktexlsr|.
%
% \subsection{Some details for the interested}
%
% \paragraph{Attached source.}
%
% The PDF documentation on CTAN also includes the
% \xfile{.dtx} source file. It can be extracted by
% AcrobatReader 6 or higher. Another option is \textsf{pdftk},
% e.g. unpack the file into the current directory:
% \begin{quote}
%   \verb|pdftk magicnum.pdf unpack_files output .|
% \end{quote}
%
% \paragraph{Unpacking with \LaTeX.}
% The \xfile{.dtx} chooses its action depending on the format:
% \begin{description}
% \item[\plainTeX:] Run \docstrip\ and extract the files.
% \item[\LaTeX:] Generate the documentation.
% \end{description}
% If you insist on using \LaTeX\ for \docstrip\ (really,
% \docstrip\ does not need \LaTeX), then inform the autodetect routine
% about your intention:
% \begin{quote}
%   \verb|latex \let\install=y\input{magicnum.dtx}|
% \end{quote}
% Do not forget to quote the argument according to the demands
% of your shell.
%
% \paragraph{Generating the documentation.}
% You can use both the \xfile{.dtx} or the \xfile{.drv} to generate
% the documentation. The process can be configured by the
% configuration file \xfile{ltxdoc.cfg}. For instance, put this
% line into this file, if you want to have A4 as paper format:
% \begin{quote}
%   \verb|\PassOptionsToClass{a4paper}{article}|
% \end{quote}
% An example follows how to generate the
% documentation with pdf\LaTeX:
% \begin{quote}
%\begin{verbatim}
%pdflatex magicnum.dtx
%makeindex -s gind.ist magicnum.idx
%pdflatex magicnum.dtx
%makeindex -s gind.ist magicnum.idx
%pdflatex magicnum.dtx
%\end{verbatim}
% \end{quote}
%
% \section{Catalogue}
%
% The following XML file can be used as source for the
% \href{http://mirror.ctan.org/help/Catalogue/catalogue.html}{\TeX\ Catalogue}.
% The elements \texttt{caption} and \texttt{description} are imported
% from the original XML file from the Catalogue.
% The name of the XML file in the Catalogue is \xfile{magicnum.xml}.
%    \begin{macrocode}
%<*catalogue>
<?xml version='1.0' encoding='us-ascii'?>
<!DOCTYPE entry SYSTEM 'catalogue.dtd'>
<entry datestamp='$Date$' modifier='$Author$' id='magicnum'>
  <name>magicnum</name>
  <caption>Access TeX systems' "magic numbers".</caption>
  <authorref id='auth:oberdiek'/>
  <copyright owner='Heiko Oberdiek' year='2007,2009-2011'/>
  <license type='lppl1.3'/>
  <version number='1.4'/>
  <description>
    This package allows access to the various parameter values in
    TeX (catcode values), e-TeX (group, if and node types, and
    interaction mode), and LuaTeX (pdfliteral mode) by a hierarchical
    name system.
    <p/>
    The package is part of the <xref refid='oberdiek'>oberdiek</xref> bundle.
  </description>
  <documentation details='Package documentation'
      href='ctan:/macros/latex/contrib/oberdiek/magicnum.pdf'/>
  <ctan file='true' path='/macros/latex/contrib/oberdiek/magicnum.dtx'/>
  <miktex location='oberdiek'/>
  <texlive location='oberdiek'/>
  <install path='/macros/latex/contrib/oberdiek/oberdiek.tds.zip'/>
</entry>
%</catalogue>
%    \end{macrocode}
%
% \begin{History}
%   \begin{Version}{2007/12/12 v1.0}
%   \item
%     First public version.
%   \end{Version}
%   \begin{Version}{2009/04/10 v1.1}
%   \item
%     Adaptation to \LuaTeX\ 0.40.
%   \end{Version}
%   \begin{Version}{2010/03/09 v1.2}
%   \item
%     Adaptation to package \xpackage{luatex} 0.4.
%   \end{Version}
%   \begin{Version}{2011/03/24 v1.3}
%   \item
%     Catcode fixes.
%   \end{Version}
%   \begin{Version}{2011/04/10 v1.4}
%   \item
%     Compatibility for \hologo{iniTeX}.
%   \item
%     Dependency from package \xpackage{luatex} removed.
%   \item
%     Version check for lua module.
%   \end{Version}
% \end{History}
%
% \PrintIndex
%
% \Finale
\endinput

%        (quote the arguments according to the demands of your shell)
%
% Documentation:
%    (a) If magicnum.drv is present:
%           latex magicnum.drv
%    (b) Without magicnum.drv:
%           latex magicnum.dtx; ...
%    The class ltxdoc loads the configuration file ltxdoc.cfg
%    if available. Here you can specify further options, e.g.
%    use A4 as paper format:
%       \PassOptionsToClass{a4paper}{article}
%
%    Programm calls to get the documentation (example):
%       pdflatex magicnum.dtx
%       makeindex -s gind.ist magicnum.idx
%       pdflatex magicnum.dtx
%       makeindex -s gind.ist magicnum.idx
%       pdflatex magicnum.dtx
%
% Installation:
%    TDS:tex/generic/oberdiek/magicnum.sty
%    TDS:scripts/oberdiek/magicnum.lua
%    TDS:scripts/oberdiek/oberdiek.magicnum.lua
%    TDS:doc/latex/oberdiek/magicnum.pdf
%    TDS:doc/latex/oberdiek/magicnum.txt
%    TDS:doc/latex/oberdiek/test/magicnum-test1.tex
%    TDS:doc/latex/oberdiek/test/magicnum-test2.tex
%    TDS:doc/latex/oberdiek/test/magicnum-test3.tex
%    TDS:doc/latex/oberdiek/test/magicnum-test4.tex
%    TDS:source/latex/oberdiek/magicnum.dtx
%
%<*ignore>
\begingroup
  \catcode123=1 %
  \catcode125=2 %
  \def\x{LaTeX2e}%
\expandafter\endgroup
\ifcase 0\ifx\install y1\fi\expandafter
         \ifx\csname processbatchFile\endcsname\relax\else1\fi
         \ifx\fmtname\x\else 1\fi\relax
\else\csname fi\endcsname
%</ignore>
%<*install>
\input docstrip.tex
\Msg{************************************************************************}
\Msg{* Installation}
\Msg{* Package: magicnum 2011/04/10 v1.4 Magic numbers (HO)}
\Msg{************************************************************************}

\keepsilent
\askforoverwritefalse

\let\MetaPrefix\relax
\preamble

This is a generated file.

Project: magicnum
Version: 2011/04/10 v1.4

Copyright (C) 2007, 2009-2011 by
   Heiko Oberdiek <heiko.oberdiek at googlemail.com>

This work may be distributed and/or modified under the
conditions of the LaTeX Project Public License, either
version 1.3c of this license or (at your option) any later
version. This version of this license is in
   http://www.latex-project.org/lppl/lppl-1-3c.txt
and the latest version of this license is in
   http://www.latex-project.org/lppl.txt
and version 1.3 or later is part of all distributions of
LaTeX version 2005/12/01 or later.

This work has the LPPL maintenance status "maintained".

This Current Maintainer of this work is Heiko Oberdiek.

The Base Interpreter refers to any `TeX-Format',
because some files are installed in TDS:tex/generic//.

This work consists of the main source file magicnum.dtx
and the derived files
   magicnum.sty, magicnum.pdf, magicnum.ins, magicnum.drv, magicnum.txt,
   magicnum-test1.tex, magicnum-test2.tex, magicnum-test3.tex,
   magicnum-test4.tex, magicnum.lua, oberdiek.magicnum.lua.

\endpreamble
\let\MetaPrefix\DoubleperCent

\generate{%
  \file{magicnum.ins}{\from{magicnum.dtx}{install}}%
  \file{magicnum.drv}{\from{magicnum.dtx}{driver}}%
  \usedir{tex/generic/oberdiek}%
  \file{magicnum.sty}{\from{magicnum.dtx}{package}}%
  \usedir{doc/latex/oberdiek/test}%
  \file{magicnum-test1.tex}{\from{magicnum.dtx}{test1}}%
  \file{magicnum-test2.tex}{\from{magicnum.dtx}{testplain,testdata}}%
  \file{magicnum-test3.tex}{\from{magicnum.dtx}{testlatex,testdata}}%
  \file{magicnum-test4.tex}{\from{magicnum.dtx}{test4}}%
  \nopreamble
  \nopostamble
  \usedir{doc/latex/oberdiek}%
  \file{magicnum.txt}{\from{magicnum.dtx}{data}}%
  \usedir{source/latex/oberdiek/catalogue}%
  \file{magicnum.xml}{\from{magicnum.dtx}{catalogue}}%
}
\def\MetaPrefix{-- }
\def\defaultpostamble{%
  \MetaPrefix^^J%
  \MetaPrefix\space End of File `\outFileName'.%
}
\def\currentpostamble{\defaultpostamble}%
\generate{%
  \usedir{scripts/oberdiek}%
  \file{magicnum.lua}{\from{magicnum.dtx}{lua}}%
  \file{oberdiek.magicnum.lua}{\from{magicnum.dtx}{lua}}%
}

\catcode32=13\relax% active space
\let =\space%
\Msg{************************************************************************}
\Msg{*}
\Msg{* To finish the installation you have to move the following}
\Msg{* file into a directory searched by TeX:}
\Msg{*}
\Msg{*     magicnum.sty}
\Msg{*}
\Msg{* And install the following script files:}
\Msg{*}
\Msg{*     magicnum.lua, oberdiek.magicnum.lua}
\Msg{*}
\Msg{* To produce the documentation run the file `magicnum.drv'}
\Msg{* through LaTeX.}
\Msg{*}
\Msg{* Happy TeXing!}
\Msg{*}
\Msg{************************************************************************}

\endbatchfile
%</install>
%<*ignore>
\fi
%</ignore>
%<*driver>
\NeedsTeXFormat{LaTeX2e}
\ProvidesFile{magicnum.drv}%
  [2011/04/10 v1.4 Magic numbers (HO)]%
\documentclass{ltxdoc}
\usepackage{holtxdoc}[2011/11/22]
\usepackage{array}
\begin{document}
  \DocInput{magicnum.dtx}%
\end{document}
%</driver>
% \fi
%
% \CheckSum{755}
%
% \CharacterTable
%  {Upper-case    \A\B\C\D\E\F\G\H\I\J\K\L\M\N\O\P\Q\R\S\T\U\V\W\X\Y\Z
%   Lower-case    \a\b\c\d\e\f\g\h\i\j\k\l\m\n\o\p\q\r\s\t\u\v\w\x\y\z
%   Digits        \0\1\2\3\4\5\6\7\8\9
%   Exclamation   \!     Double quote  \"     Hash (number) \#
%   Dollar        \$     Percent       \%     Ampersand     \&
%   Acute accent  \'     Left paren    \(     Right paren   \)
%   Asterisk      \*     Plus          \+     Comma         \,
%   Minus         \-     Point         \.     Solidus       \/
%   Colon         \:     Semicolon     \;     Less than     \<
%   Equals        \=     Greater than  \>     Question mark \?
%   Commercial at \@     Left bracket  \[     Backslash     \\
%   Right bracket \]     Circumflex    \^     Underscore    \_
%   Grave accent  \`     Left brace    \{     Vertical bar  \|
%   Right brace   \}     Tilde         \~}
%
% \GetFileInfo{magicnum.drv}
%
% \title{The \xpackage{magicnum} package}
% \date{2011/04/10 v1.4}
% \author{Heiko Oberdiek\\\xemail{heiko.oberdiek at googlemail.com}}
%
% \maketitle
%
% \begin{abstract}
% This packages allows to access magic numbers by a hierarchical
% name system.
% \end{abstract}
%
% \tableofcontents
%
% \hypersetup{bookmarksopenlevel=2}
% \section{Documentation}
%
% \subsection{Introduction}
%
% Especially since \eTeX\ there are many integer values
% with special meanings, such as catcodes, group types, \dots
% Package \xpackage{etex}, enabled by options, defines
% macros in the user namespace for these values.
%
% This package goes another approach for storing the names and values.
% \begin{itemize}
% \item If \LuaTeX\ is available, they
% are stored in Lua tables.
% \item Without \LuaTeX\ they are remembered using internal
% macros.
% \end{itemize}
%
% \subsection{User interface}
%
% The integer values and names are organized in a hierarchical
% scheme of categories with the property names as leaves.
% Example: \eTeX's \cs{currentgrouplevel} reports |2| for a
% group caused by \cs{hbox}. This package has choosen to organize
% the group types in a main category |etex| and its subcategory
% |grouptype|:
% \begin{quote}
%   |etex.grouptype.hbox| = |2|
% \end{quote}
% The property name |hbox| in category |etex.grouptype| has value |2|.
% Dots are used to separate components.
%
% If you want to have the value, the access key is constructed by
% the category with all its components and the property name.
% For the opposite the value is used instead of the property name.
%
% Values are always integers (including negative numbers).
%
% \subsubsection{\cs{magicnum}}
%
% \begin{declcs}{magicnum} \M{access key}
% \end{declcs}
% Macro \cs{magicnum} expects an access key as argument and
% expands to the requested data. The macro is always expandable.
% In case of errors the expansion result is empty.
%
% The same macro is also used for getting a property name.
% In this case the property name part in the access key is
% replaced by the value.
%
% The catcodes
% of the resulting numbers and strings follow \TeX's tradition of
% \cs{string}, \cs{meaning}, \dots: The space has catcode 10
% (|tex.catcode.space|) and the other characters have catcode
% 12 (|tex.catcode.other|).
%
% Examples:
% \begin{quote}
%   |\magicnum{etex.grouptype.hbox}| $\Rightarrow$ |2|\\
%   |\magicnum{tex.catcode.14}| $\Rightarrow$ |comment|\\
%   |\magicnum{tex.catcode.undefined}| $\Rightarrow$ $\emptyset$
% \end{quote}
%
% \subsubsection{Properties}
%
% \begin{itemize}
% \item The components of a category are either subcategories or
%       key value pairs, but not both.
% \item The full specified property names are unique and thus
%       has one integer value exactly.
% \item Also the values inside a category are unique.
%       This condition is a prerequisite for the reverse mapping
%       of \cs{magicnum}.
% \item All names start with a letter. Only letters or digits
%       may follow.
% \end{itemize}
%
% \subsection{Data}
%
%  \subsubsection{\texorpdfstring{Category }{}\texttt{tex.catcode}}
%
% \begin{quote}
% \begin{tabular}{@{}>{\ttfamily}l>{\ttfamily}l@{}}
%    tex.catcode.escape & 0\\
%    tex.catcode.begingroup & 1\\
%    tex.catcode.endgroup & 2\\
%    tex.catcode.math & 3\\
%    tex.catcode.align & 4\\
%    tex.catcode.eol & 5\\
%    tex.catcode.parameter & 6\\
%    tex.catcode.superscript & 7\\
%    tex.catcode.subscript & 8\\
%    tex.catcode.ignore & 9\\
%    tex.catcode.space & 10\\
%    tex.catcode.letter & 11\\
%    tex.catcode.other & 12\\
%    tex.catcode.active & 13\\
%    tex.catcode.comment & 14\\
%    tex.catcode.invalid & 15\\
%  \end{tabular}
%  \end{quote}
%
%  \subsubsection{\texorpdfstring{Category }{}\texttt{etex.grouptype}}
%
% \begin{quote}
% \begin{tabular}{@{}>{\ttfamily}l>{\ttfamily}l@{}}
%    etex.grouptype.bottomlevel & 0\\
%    etex.grouptype.simple & 1\\
%    etex.grouptype.hbox & 2\\
%    etex.grouptype.adjustedhbox & 3\\
%    etex.grouptype.vbox & 4\\
%    etex.grouptype.align & 5\\
%    etex.grouptype.noalign & 6\\
%    etex.grouptype.output & 8\\
%    etex.grouptype.math & 9\\
%    etex.grouptype.disc & 10\\
%    etex.grouptype.insert & 11\\
%    etex.grouptype.vcenter & 12\\
%    etex.grouptype.mathchoice & 13\\
%    etex.grouptype.semisimple & 14\\
%    etex.grouptype.mathshift & 15\\
%    etex.grouptype.mathleft & 16\\
%  \end{tabular}
%  \end{quote}
%
%  \subsubsection{\texorpdfstring{Category }{}\texttt{etex.iftype}}
%
% \begin{quote}
% \begin{tabular}{@{}>{\ttfamily}l>{\ttfamily}l@{}}
%    etex.iftype.none & 0\\
%    etex.iftype.char & 1\\
%    etex.iftype.cat & 2\\
%    etex.iftype.num & 3\\
%    etex.iftype.dim & 4\\
%    etex.iftype.odd & 5\\
%    etex.iftype.vmode & 6\\
%    etex.iftype.hmode & 7\\
%    etex.iftype.mmode & 8\\
%    etex.iftype.inner & 9\\
%    etex.iftype.void & 10\\
%    etex.iftype.hbox & 11\\
%    etex.iftype.vbox & 12\\
%    etex.iftype.x & 13\\
%    etex.iftype.eof & 14\\
%    etex.iftype.true & 15\\
%    etex.iftype.false & 16\\
%    etex.iftype.case & 17\\
%    etex.iftype.defined & 18\\
%    etex.iftype.csname & 19\\
%    etex.iftype.fontchar & 20\\
%  \end{tabular}
%  \end{quote}
%
%  \subsubsection{\texorpdfstring{Category }{}\texttt{etex.nodetype}}
%
% \begin{quote}
% \begin{tabular}{@{}>{\ttfamily}l>{\ttfamily}l@{}}
%    etex.nodetype.none & -1\\
%    etex.nodetype.char & 0\\
%    etex.nodetype.hlist & 1\\
%    etex.nodetype.vlist & 2\\
%    etex.nodetype.rule & 3\\
%    etex.nodetype.ins & 4\\
%    etex.nodetype.mark & 5\\
%    etex.nodetype.adjust & 6\\
%    etex.nodetype.ligature & 7\\
%    etex.nodetype.disc & 8\\
%    etex.nodetype.whatsit & 9\\
%    etex.nodetype.math & 10\\
%    etex.nodetype.glue & 11\\
%    etex.nodetype.kern & 12\\
%    etex.nodetype.penalty & 13\\
%    etex.nodetype.unset & 14\\
%    etex.nodetype.maths & 15\\
%  \end{tabular}
%  \end{quote}
%
%  \subsubsection{\texorpdfstring{Category }{}\texttt{etex.interactionmode}}
%
% \begin{quote}
% \begin{tabular}{@{}>{\ttfamily}l>{\ttfamily}l@{}}
%    etex.interactionmode.batch & 0\\
%    etex.interactionmode.nonstop & 1\\
%    etex.interactionmode.scroll & 2\\
%    etex.interactionmode.errorstop & 3\\
%  \end{tabular}
%  \end{quote}
%
%  \subsubsection{\texorpdfstring{Category }{}\texttt{luatex.pdfliteral.mode}}
%
% \begin{quote}
% \begin{tabular}{@{}>{\ttfamily}l>{\ttfamily}l@{}}
%    luatex.pdfliteral.mode.setorigin & 0\\
%    luatex.pdfliteral.mode.page & 1\\
%    luatex.pdfliteral.mode.direct & 2\\
%  \end{tabular}
%  \end{quote}
%
%
% \hypersetup{bookmarksopenlevel=1}
%
% \StopEventually{
% }
%
% \section{Implementation}
%
%    \begin{macrocode}
%<*package>
%    \end{macrocode}
%
% \subsection{Reload check and package identification}
%    Reload check, especially if the package is not used with \LaTeX.
%    \begin{macrocode}
\begingroup\catcode61\catcode48\catcode32=10\relax%
  \catcode13=5 % ^^M
  \endlinechar=13 %
  \catcode35=6 % #
  \catcode39=12 % '
  \catcode44=12 % ,
  \catcode45=12 % -
  \catcode46=12 % .
  \catcode58=12 % :
  \catcode64=11 % @
  \catcode123=1 % {
  \catcode125=2 % }
  \expandafter\let\expandafter\x\csname ver@magicnum.sty\endcsname
  \ifx\x\relax % plain-TeX, first loading
  \else
    \def\empty{}%
    \ifx\x\empty % LaTeX, first loading,
      % variable is initialized, but \ProvidesPackage not yet seen
    \else
      \expandafter\ifx\csname PackageInfo\endcsname\relax
        \def\x#1#2{%
          \immediate\write-1{Package #1 Info: #2.}%
        }%
      \else
        \def\x#1#2{\PackageInfo{#1}{#2, stopped}}%
      \fi
      \x{magicnum}{The package is already loaded}%
      \aftergroup\endinput
    \fi
  \fi
\endgroup%
%    \end{macrocode}
%    Package identification:
%    \begin{macrocode}
\begingroup\catcode61\catcode48\catcode32=10\relax%
  \catcode13=5 % ^^M
  \endlinechar=13 %
  \catcode35=6 % #
  \catcode39=12 % '
  \catcode40=12 % (
  \catcode41=12 % )
  \catcode44=12 % ,
  \catcode45=12 % -
  \catcode46=12 % .
  \catcode47=12 % /
  \catcode58=12 % :
  \catcode64=11 % @
  \catcode91=12 % [
  \catcode93=12 % ]
  \catcode123=1 % {
  \catcode125=2 % }
  \expandafter\ifx\csname ProvidesPackage\endcsname\relax
    \def\x#1#2#3[#4]{\endgroup
      \immediate\write-1{Package: #3 #4}%
      \xdef#1{#4}%
    }%
  \else
    \def\x#1#2[#3]{\endgroup
      #2[{#3}]%
      \ifx#1\@undefined
        \xdef#1{#3}%
      \fi
      \ifx#1\relax
        \xdef#1{#3}%
      \fi
    }%
  \fi
\expandafter\x\csname ver@magicnum.sty\endcsname
\ProvidesPackage{magicnum}%
  [2011/04/10 v1.4 Magic numbers (HO)]%
%    \end{macrocode}
%
% \subsection{Catcodes}
%
%    \begin{macrocode}
\begingroup\catcode61\catcode48\catcode32=10\relax%
  \catcode13=5 % ^^M
  \endlinechar=13 %
  \catcode123=1 % {
  \catcode125=2 % }
  \catcode64=11 % @
  \def\x{\endgroup
    \expandafter\edef\csname magicnum@AtEnd\endcsname{%
      \endlinechar=\the\endlinechar\relax
      \catcode13=\the\catcode13\relax
      \catcode32=\the\catcode32\relax
      \catcode35=\the\catcode35\relax
      \catcode61=\the\catcode61\relax
      \catcode64=\the\catcode64\relax
      \catcode123=\the\catcode123\relax
      \catcode125=\the\catcode125\relax
    }%
  }%
\x\catcode61\catcode48\catcode32=10\relax%
\catcode13=5 % ^^M
\endlinechar=13 %
\catcode35=6 % #
\catcode64=11 % @
\catcode123=1 % {
\catcode125=2 % }
\def\TMP@EnsureCode#1#2{%
  \edef\magicnum@AtEnd{%
    \magicnum@AtEnd
    \catcode#1=\the\catcode#1\relax
  }%
  \catcode#1=#2\relax
}
\TMP@EnsureCode{34}{12}% "
\TMP@EnsureCode{39}{12}% '
\TMP@EnsureCode{40}{12}% (
\TMP@EnsureCode{41}{12}% )
\TMP@EnsureCode{42}{12}% *
\TMP@EnsureCode{44}{12}% ,
\TMP@EnsureCode{45}{12}% -
\TMP@EnsureCode{46}{12}% .
\TMP@EnsureCode{47}{12}% /
\TMP@EnsureCode{58}{12}% :
\TMP@EnsureCode{60}{12}% <
\TMP@EnsureCode{62}{12}% >
\TMP@EnsureCode{91}{12}% [
\TMP@EnsureCode{93}{12}% ]
\edef\magicnum@AtEnd{\magicnum@AtEnd\noexpand\endinput}
%    \end{macrocode}
%
% \subsection{Check for previous definition}
%
%    \begin{macrocode}
\begingroup\expandafter\expandafter\expandafter\endgroup
\expandafter\ifx\csname newcommand\endcsname\relax
  \expandafter\ifx\csname magicnum\endcsname\relax
  \else
    \input infwarerr.sty\relax
    \@PackageError{magicnum}{%
      \string\magicnum\space is already defined%
    }\@ehc
  \fi
\else
  \newcommand*{\magicnum}{}%
\fi
%    \end{macrocode}
%
% \subsection{Without \LuaTeX}
%
%    \begin{macrocode}
\begingroup\expandafter\expandafter\expandafter\endgroup
\expandafter\ifx\csname directlua\endcsname\relax
%    \end{macrocode}
%
%    \begin{macro}{\magicnum}
%    \begin{macrocode}
  \begingroup\expandafter\expandafter\expandafter\endgroup
  \expandafter\ifx\csname ifcsname\endcsname\relax
    \def\magicnum#1{%
      \expandafter\ifx\csname MG@#1\endcsname\relax
      \else
        \csname MG@#1\endcsname
      \fi
    }%
  \else
    \begingroup
      \edef\x{\endgroup
        \def\noexpand\magicnum##1{%
          \expandafter\noexpand\csname
          ifcsname\endcsname MG@##1\noexpand\endcsname
            \noexpand\csname MG@##1%
                 \noexpand\expandafter\noexpand\endcsname
          \expandafter\noexpand\csname fi\endcsname
        }%
      }%
    \x
  \fi
%    \end{macrocode}
%    \end{macro}
%
%    \begin{macrocode}
\else
%    \end{macrocode}
%
% \subsection{With \LuaTeX}
%
%    \begin{macrocode}
  \begingroup\expandafter\expandafter\expandafter\endgroup
  \expandafter\ifx\csname RequirePackage\endcsname\relax
    \input ifluatex.sty\relax
    \input infwarerr.sty\relax
  \else
    \RequirePackage{ifluatex}[2010/03/01]%
    \RequirePackage{infwarerr}[2010/04/08]%
  \fi
%    \end{macrocode}
%
%    \begin{macro}{\magicnum@directlua}
%    \begin{macrocode}
  \ifnum\luatexversion<36 %
    \def\magicnum@directlua{\directlua0 }%
  \else
    \let\magicnum@directlua\directlua
  \fi
%    \end{macrocode}
%    \end{macro}
%    \begin{macrocode}
  \magicnum@directlua{%
    require("oberdiek.magicnum")%
  }%
  \begingroup
    \def\x{2011/04/10 v1.4}%
    \def\StripPrefix#1>{}%
    \edef\x{\expandafter\StripPrefix\meaning\x}%
    \edef\y{%
      \magicnum@directlua{%
        if oberdiek.magicnum.getversion then %
          oberdiek.magicnum.getversion()%
        end%
      }%
    }%
    \ifx\x\y
    \else
      \@PackageError{magicnum}{%
        Wrong version of lua module.\MessageBreak
        Package version: \x\MessageBreak
        Lua module: \y
      }\@ehc
    \fi
  \endgroup
%    \end{macrocode}
%    \begin{macro}{\luaescapestring}
%    \begin{macrocode}
  \begingroup
    \expandafter\ifx\csname luaescapestring\endcsname\relax
      \directlua{%
        if tex.enableprimitives then %
          tex.enableprimitives('magicnum@', {'luaescapestring'})%
        end%
      }%
      \global\let\luaescapestring\magicnum@luaescapestring
    \fi
    \expandafter\ifx\csname luaescapestring\endcsname\relax
      \escapechar=92 %
      \@PackageError{magicnum}{%
        Missing \string\luaescapestring
      }\@ehc
    \fi
  \endgroup
%    \end{macrocode}
%    \end{macro}
%    \begin{macro}{\magicnum}
%    \begin{macrocode}
  \def\magicnum#1{%
    \magicnum@directlua{%
      oberdiek.magicnum.get("\luaescapestring{#1}")%
    }%
  }%
%    \end{macrocode}
%    \end{macro}
%
%    \begin{macrocode}
  \expandafter\magicnum@AtEnd
\fi%
%</package>
%    \end{macrocode}
%
% \subsection{Data}
%
% \subsubsection{Plain data}
%
%    \begin{macrocode}
%<*data>
tex.catcode
  escape = 0
  begingroup = 1
  endgroup = 2
  math = 3
  align = 4
  eol = 5
  parameter = 6
  superscript = 7
  subscript = 8
  ignore = 9
  space = 10
  letter = 11
  other = 12
  active = 13
  comment = 14
  invalid = 15
etex.grouptype
  bottomlevel = 0
  simple = 1
  hbox = 2
  adjustedhbox = 3
  vbox = 4
  align = 5
  noalign = 6
  output = 8
  math = 9
  disc = 10
  insert = 11
  vcenter = 12
  mathchoice = 13
  semisimple = 14
  mathshift = 15
  mathleft = 16
etex.iftype
  none = 0
  char = 1
  cat = 2
  num = 3
  dim = 4
  odd = 5
  vmode = 6
  hmode = 7
  mmode = 8
  inner = 9
  void = 10
  hbox = 11
  vbox = 12
  x = 13
  eof = 14
  true = 15
  false = 16
  case = 17
  defined = 18
  csname = 19
  fontchar = 20
etex.nodetype
  none = -1
  char = 0
  hlist = 1
  vlist = 2
  rule = 3
  ins = 4
  mark = 5
  adjust = 6
  ligature = 7
  disc = 8
  whatsit = 9
  math = 10
  glue = 11
  kern = 12
  penalty = 13
  unset = 14
  maths = 15
etex.interactionmode
  batch = 0
  nonstop = 1
  scroll = 2
  errorstop = 3
luatex.pdfliteral.mode
  setorigin = 0
  page = 1
  direct = 2
%</data>
%    \end{macrocode}
%
% \subsubsection{Data for \TeX}
%
%    \begin{macrocode}
%<*package>
%    \end{macrocode}
%    \begin{macro}{\magicnum@add}
%    \begin{macrocode}
\begingroup\expandafter\expandafter\expandafter\endgroup
\expandafter\ifx\csname detokenize\endcsname\relax
  \def\magicnum@add#1#2#3{%
    \expandafter\magicnum@@add
        \csname MG@#1.#2\expandafter\endcsname
        \csname MG@#1.#3\endcsname
       {#3}{#2}%
  }%
  \def\magicnum@@add#1#2#3#4{%
    \def#1{#3}%
    \def#2{#4}%
    \edef#1{%
      \expandafter\strip@prefix\meaning#1%
    }%
    \edef#2{%
      \expandafter\strip@prefix\meaning#2%
    }%
  }%
  \expandafter\ifx\csname strip@prefix\endcsname\relax
    \def\strip@prefix#1->{}%
  \fi
\else
  \def\magicnum@add#1#2#3{%
    \expandafter\edef\csname MG@#1.#2\endcsname{%
      \detokenize{#3}%
    }%
    \expandafter\edef\csname MG@#1.#3\endcsname{%
      \detokenize{#2}%
    }%
  }%
\fi
%    \end{macrocode}
%    \end{macro}
%    \begin{macrocode}
\magicnum@add{tex.catcode}{escape}{0}
\magicnum@add{tex.catcode}{begingroup}{1}
\magicnum@add{tex.catcode}{endgroup}{2}
\magicnum@add{tex.catcode}{math}{3}
\magicnum@add{tex.catcode}{align}{4}
\magicnum@add{tex.catcode}{eol}{5}
\magicnum@add{tex.catcode}{parameter}{6}
\magicnum@add{tex.catcode}{superscript}{7}
\magicnum@add{tex.catcode}{subscript}{8}
\magicnum@add{tex.catcode}{ignore}{9}
\magicnum@add{tex.catcode}{space}{10}
\magicnum@add{tex.catcode}{letter}{11}
\magicnum@add{tex.catcode}{other}{12}
\magicnum@add{tex.catcode}{active}{13}
\magicnum@add{tex.catcode}{comment}{14}
\magicnum@add{tex.catcode}{invalid}{15}
\magicnum@add{etex.grouptype}{bottomlevel}{0}
\magicnum@add{etex.grouptype}{simple}{1}
\magicnum@add{etex.grouptype}{hbox}{2}
\magicnum@add{etex.grouptype}{adjustedhbox}{3}
\magicnum@add{etex.grouptype}{vbox}{4}
\magicnum@add{etex.grouptype}{align}{5}
\magicnum@add{etex.grouptype}{noalign}{6}
\magicnum@add{etex.grouptype}{output}{8}
\magicnum@add{etex.grouptype}{math}{9}
\magicnum@add{etex.grouptype}{disc}{10}
\magicnum@add{etex.grouptype}{insert}{11}
\magicnum@add{etex.grouptype}{vcenter}{12}
\magicnum@add{etex.grouptype}{mathchoice}{13}
\magicnum@add{etex.grouptype}{semisimple}{14}
\magicnum@add{etex.grouptype}{mathshift}{15}
\magicnum@add{etex.grouptype}{mathleft}{16}
\magicnum@add{etex.iftype}{none}{0}
\magicnum@add{etex.iftype}{char}{1}
\magicnum@add{etex.iftype}{cat}{2}
\magicnum@add{etex.iftype}{num}{3}
\magicnum@add{etex.iftype}{dim}{4}
\magicnum@add{etex.iftype}{odd}{5}
\magicnum@add{etex.iftype}{vmode}{6}
\magicnum@add{etex.iftype}{hmode}{7}
\magicnum@add{etex.iftype}{mmode}{8}
\magicnum@add{etex.iftype}{inner}{9}
\magicnum@add{etex.iftype}{void}{10}
\magicnum@add{etex.iftype}{hbox}{11}
\magicnum@add{etex.iftype}{vbox}{12}
\magicnum@add{etex.iftype}{x}{13}
\magicnum@add{etex.iftype}{eof}{14}
\magicnum@add{etex.iftype}{true}{15}
\magicnum@add{etex.iftype}{false}{16}
\magicnum@add{etex.iftype}{case}{17}
\magicnum@add{etex.iftype}{defined}{18}
\magicnum@add{etex.iftype}{csname}{19}
\magicnum@add{etex.iftype}{fontchar}{20}
\magicnum@add{etex.nodetype}{none}{-1}
\magicnum@add{etex.nodetype}{char}{0}
\magicnum@add{etex.nodetype}{hlist}{1}
\magicnum@add{etex.nodetype}{vlist}{2}
\magicnum@add{etex.nodetype}{rule}{3}
\magicnum@add{etex.nodetype}{ins}{4}
\magicnum@add{etex.nodetype}{mark}{5}
\magicnum@add{etex.nodetype}{adjust}{6}
\magicnum@add{etex.nodetype}{ligature}{7}
\magicnum@add{etex.nodetype}{disc}{8}
\magicnum@add{etex.nodetype}{whatsit}{9}
\magicnum@add{etex.nodetype}{math}{10}
\magicnum@add{etex.nodetype}{glue}{11}
\magicnum@add{etex.nodetype}{kern}{12}
\magicnum@add{etex.nodetype}{penalty}{13}
\magicnum@add{etex.nodetype}{unset}{14}
\magicnum@add{etex.nodetype}{maths}{15}
\magicnum@add{etex.interactionmode}{batch}{0}
\magicnum@add{etex.interactionmode}{nonstop}{1}
\magicnum@add{etex.interactionmode}{scroll}{2}
\magicnum@add{etex.interactionmode}{errorstop}{3}
\magicnum@add{luatex.pdfliteral.mode}{setorigin}{0}
\magicnum@add{luatex.pdfliteral.mode}{page}{1}
\magicnum@add{luatex.pdfliteral.mode}{direct}{2}
%    \end{macrocode}
%    \begin{macrocode}
\magicnum@AtEnd%
%</package>
%    \end{macrocode}
%
% \subsubsection{Lua module}
%
%    \begin{macrocode}
%<*lua>
%    \end{macrocode}
%    \begin{macrocode}
module("oberdiek.magicnum", package.seeall)
%    \end{macrocode}
%    \begin{macrocode}
function getversion()
  tex.write("2011/04/10 v1.4")
end
%    \end{macrocode}
%    \begin{macrocode}
local data = {
  ["tex.catcode"] = {
    [0] = "escape",
    [1] = "begingroup",
    [2] = "endgroup",
    [3] = "math",
    [4] = "align",
    [5] = "eol",
    [6] = "parameter",
    [7] = "superscript",
    [8] = "subscript",
    [9] = "ignore",
    [10] = "space",
    [11] = "letter",
    [12] = "other",
    [13] = "active",
    [14] = "comment",
    [15] = "invalid",
    ["active"] = 13,
    ["align"] = 4,
    ["begingroup"] = 1,
    ["comment"] = 14,
    ["endgroup"] = 2,
    ["eol"] = 5,
    ["escape"] = 0,
    ["ignore"] = 9,
    ["invalid"] = 15,
    ["letter"] = 11,
    ["math"] = 3,
    ["other"] = 12,
    ["parameter"] = 6,
    ["space"] = 10,
    ["subscript"] = 8,
    ["superscript"] = 7
  },
  ["etex.grouptype"] = {
    [0] = "bottomlevel",
    [1] = "simple",
    [2] = "hbox",
    [3] = "adjustedhbox",
    [4] = "vbox",
    [5] = "align",
    [6] = "noalign",
    [8] = "output",
    [9] = "math",
    [10] = "disc",
    [11] = "insert",
    [12] = "vcenter",
    [13] = "mathchoice",
    [14] = "semisimple",
    [15] = "mathshift",
    [16] = "mathleft",
    ["adjustedhbox"] = 3,
    ["align"] = 5,
    ["bottomlevel"] = 0,
    ["disc"] = 10,
    ["hbox"] = 2,
    ["insert"] = 11,
    ["math"] = 9,
    ["mathchoice"] = 13,
    ["mathleft"] = 16,
    ["mathshift"] = 15,
    ["noalign"] = 6,
    ["output"] = 8,
    ["semisimple"] = 14,
    ["simple"] = 1,
    ["vbox"] = 4,
    ["vcenter"] = 12
  },
  ["etex.iftype"] = {
    [0] = "none",
    [1] = "char",
    [2] = "cat",
    [3] = "num",
    [4] = "dim",
    [5] = "odd",
    [6] = "vmode",
    [7] = "hmode",
    [8] = "mmode",
    [9] = "inner",
    [10] = "void",
    [11] = "hbox",
    [12] = "vbox",
    [13] = "x",
    [14] = "eof",
    [15] = "true",
    [16] = "false",
    [17] = "case",
    [18] = "defined",
    [19] = "csname",
    [20] = "fontchar",
    ["case"] = 17,
    ["cat"] = 2,
    ["char"] = 1,
    ["csname"] = 19,
    ["defined"] = 18,
    ["dim"] = 4,
    ["eof"] = 14,
    ["false"] = 16,
    ["fontchar"] = 20,
    ["hbox"] = 11,
    ["hmode"] = 7,
    ["inner"] = 9,
    ["mmode"] = 8,
    ["none"] = 0,
    ["num"] = 3,
    ["odd"] = 5,
    ["true"] = 15,
    ["vbox"] = 12,
    ["vmode"] = 6,
    ["void"] = 10,
    ["x"] = 13
  },
  ["etex.nodetype"] = {
    [-1] = "none",
    [0] = "char",
    [1] = "hlist",
    [2] = "vlist",
    [3] = "rule",
    [4] = "ins",
    [5] = "mark",
    [6] = "adjust",
    [7] = "ligature",
    [8] = "disc",
    [9] = "whatsit",
    [10] = "math",
    [11] = "glue",
    [12] = "kern",
    [13] = "penalty",
    [14] = "unset",
    [15] = "maths",
    ["adjust"] = 6,
    ["char"] = 0,
    ["disc"] = 8,
    ["glue"] = 11,
    ["hlist"] = 1,
    ["ins"] = 4,
    ["kern"] = 12,
    ["ligature"] = 7,
    ["mark"] = 5,
    ["math"] = 10,
    ["maths"] = 15,
    ["none"] = -1,
    ["penalty"] = 13,
    ["rule"] = 3,
    ["unset"] = 14,
    ["vlist"] = 2,
    ["whatsit"] = 9
  },
  ["etex.interactionmode"] = {
    [0] = "batch",
    [1] = "nonstop",
    [2] = "scroll",
    [3] = "errorstop",
    ["batch"] = 0,
    ["errorstop"] = 3,
    ["nonstop"] = 1,
    ["scroll"] = 2
  },
  ["luatex.pdfliteral.mode"] = {
    [0] = "setorigin",
    [1] = "page",
    [2] = "direct",
    ["direct"] = 2,
    ["page"] = 1,
    ["setorigin"] = 0
  }
}
%    \end{macrocode}
%    \begin{macrocode}
function get(name)
  local startpos, endpos, category, entry =
      string.find(name, "^(%a[%a%d%.]*)%.(-?[%a%d]+)$")
  if not entry then
    return
  end
  local node = data[category]
  if not node then
    return
  end
  local num = tonumber(entry)
  local value
  if num then
    value = node[num]
    if not value then
      return
    end
  else
    value = node[entry]
    if not value then
      return
    end
    value = "" .. value
  end
  tex.write(value)
end
%    \end{macrocode}
%
%    \begin{macrocode}
%</lua>
%    \end{macrocode}
%
% \section{Test}
%
% \subsection{Catcode checks for loading}
%
%    \begin{macrocode}
%<*test1>
%    \end{macrocode}
%    \begin{macrocode}
\catcode`\{=1 %
\catcode`\}=2 %
\catcode`\#=6 %
\catcode`\@=11 %
\expandafter\ifx\csname count@\endcsname\relax
  \countdef\count@=255 %
\fi
\expandafter\ifx\csname @gobble\endcsname\relax
  \long\def\@gobble#1{}%
\fi
\expandafter\ifx\csname @firstofone\endcsname\relax
  \long\def\@firstofone#1{#1}%
\fi
\expandafter\ifx\csname loop\endcsname\relax
  \expandafter\@firstofone
\else
  \expandafter\@gobble
\fi
{%
  \def\loop#1\repeat{%
    \def\body{#1}%
    \iterate
  }%
  \def\iterate{%
    \body
      \let\next\iterate
    \else
      \let\next\relax
    \fi
    \next
  }%
  \let\repeat=\fi
}%
\def\RestoreCatcodes{}
\count@=0 %
\loop
  \edef\RestoreCatcodes{%
    \RestoreCatcodes
    \catcode\the\count@=\the\catcode\count@\relax
  }%
\ifnum\count@<255 %
  \advance\count@ 1 %
\repeat

\def\RangeCatcodeInvalid#1#2{%
  \count@=#1\relax
  \loop
    \catcode\count@=15 %
  \ifnum\count@<#2\relax
    \advance\count@ 1 %
  \repeat
}
\def\RangeCatcodeCheck#1#2#3{%
  \count@=#1\relax
  \loop
    \ifnum#3=\catcode\count@
    \else
      \errmessage{%
        Character \the\count@\space
        with wrong catcode \the\catcode\count@\space
        instead of \number#3%
      }%
    \fi
  \ifnum\count@<#2\relax
    \advance\count@ 1 %
  \repeat
}
\def\space{ }
\expandafter\ifx\csname LoadCommand\endcsname\relax
  \def\LoadCommand{\input magicnum.sty\relax}%
\fi
\def\Test{%
  \RangeCatcodeInvalid{0}{47}%
  \RangeCatcodeInvalid{58}{64}%
  \RangeCatcodeInvalid{91}{96}%
  \RangeCatcodeInvalid{123}{255}%
  \catcode`\@=12 %
  \catcode`\\=0 %
  \catcode`\%=14 %
  \LoadCommand
  \RangeCatcodeCheck{0}{36}{15}%
  \RangeCatcodeCheck{37}{37}{14}%
  \RangeCatcodeCheck{38}{47}{15}%
  \RangeCatcodeCheck{48}{57}{12}%
  \RangeCatcodeCheck{58}{63}{15}%
  \RangeCatcodeCheck{64}{64}{12}%
  \RangeCatcodeCheck{65}{90}{11}%
  \RangeCatcodeCheck{91}{91}{15}%
  \RangeCatcodeCheck{92}{92}{0}%
  \RangeCatcodeCheck{93}{96}{15}%
  \RangeCatcodeCheck{97}{122}{11}%
  \RangeCatcodeCheck{123}{255}{15}%
  \RestoreCatcodes
}
\Test
\csname @@end\endcsname
\end
%    \end{macrocode}
%    \begin{macrocode}
%</test1>
%    \end{macrocode}
%
% \subsection{Test data}
%
%    \begin{macrocode}
%<*testplain>
\input magicnum.sty\relax
\def\Test#1#2{%
  \edef\result{\magicnum{#1}}%
  \edef\expect{#2}%
  \edef\expect{\expandafter\stripprefix\meaning\expect}%
  \ifx\result\expect
  \else
    \errmessage{%
      Failed: [#1] % hash-ok
      returns [\result] instead of [\expect]%
    }%
  \fi
}
\def\stripprefix#1->{}
%</testplain>
%    \end{macrocode}
%    \begin{macrocode}
%<*testlatex>
\NeedsTeXFormat{LaTeX2e}
\documentclass{minimal}
\usepackage{magicnum}[2011/04/10]
\usepackage{qstest}
\IncludeTests{*}
\LogTests{log}{*}{*}
\newcommand*{\Test}[2]{%
  \Expect*{\magicnum{#1}}{#2}%
}
\begin{qstest}{magicnum}{magicnum}
%</testlatex>
%    \end{macrocode}
%    \begin{macrocode}
%<*testdata>
\Test{tex.catcode.escape}{0}
\Test{tex.catcode.invalid}{15}
\Test{tex.catcode.unknown}{}
\Test{tex.catcode.0}{escape}
\Test{tex.catcode.15}{invalid}
\Test{etex.iftype.true}{15}
\Test{etex.iftype.false}{16}
\Test{etex.iftype.15}{true}
\Test{etex.iftype.16}{false}
\Test{etex.nodetype.none}{-1}
\Test{etex.nodetype.-1}{none}
\Test{luatex.pdfliteral.mode.direct}{2}
\Test{luatex.pdfliteral.mode.1}{page}
\Test{}{}
\Test{unknown}{}
\Test{unknown.foo.bar}{}
\Test{unknown.foo.4}{}
%</testdata>
%    \end{macrocode}
%    \begin{macrocode}
%<*testplain>
\csname @@end\endcsname
\end
%</testplain>
%<*testlatex>
\end{qstest}
\csname @@end\endcsname
%</testlatex>
%    \end{macrocode}
%
% \subsection{Small test for \hologo{iniTeX}}
%
%    \begin{macrocode}
%<*test4>
\catcode`\{=1
\catcode`\}=2
\catcode`\#=6
\input magicnum.sty\relax
\edef\x{\magicnum{tex.catcode.15}}
\edef\y{invalid}
\def\Strip#1>{}
\edef\y{\expandafter\Strip\meaning\y}
\ifx\x\y
  \immediate\write16{Ok}%
\else
  \errmessage{\x<>\y}%
\fi
\csname @@end\endcsname\end
%</test4>
%    \end{macrocode}
%
% \section{Installation}
%
% \subsection{Download}
%
% \paragraph{Package.} This package is available on
% CTAN\footnote{\url{ftp://ftp.ctan.org/tex-archive/}}:
% \begin{description}
% \item[\CTAN{macros/latex/contrib/oberdiek/magicnum.dtx}] The source file.
% \item[\CTAN{macros/latex/contrib/oberdiek/magicnum.pdf}] Documentation.
% \end{description}
%
%
% \paragraph{Bundle.} All the packages of the bundle `oberdiek'
% are also available in a TDS compliant ZIP archive. There
% the packages are already unpacked and the documentation files
% are generated. The files and directories obey the TDS standard.
% \begin{description}
% \item[\CTAN{install/macros/latex/contrib/oberdiek.tds.zip}]
% \end{description}
% \emph{TDS} refers to the standard ``A Directory Structure
% for \TeX\ Files'' (\CTAN{tds/tds.pdf}). Directories
% with \xfile{texmf} in their name are usually organized this way.
%
% \subsection{Bundle installation}
%
% \paragraph{Unpacking.} Unpack the \xfile{oberdiek.tds.zip} in the
% TDS tree (also known as \xfile{texmf} tree) of your choice.
% Example (linux):
% \begin{quote}
%   |unzip oberdiek.tds.zip -d ~/texmf|
% \end{quote}
%
% \paragraph{Script installation.}
% Check the directory \xfile{TDS:scripts/oberdiek/} for
% scripts that need further installation steps.
% Package \xpackage{attachfile2} comes with the Perl script
% \xfile{pdfatfi.pl} that should be installed in such a way
% that it can be called as \texttt{pdfatfi}.
% Example (linux):
% \begin{quote}
%   |chmod +x scripts/oberdiek/pdfatfi.pl|\\
%   |cp scripts/oberdiek/pdfatfi.pl /usr/local/bin/|
% \end{quote}
%
% \subsection{Package installation}
%
% \paragraph{Unpacking.} The \xfile{.dtx} file is a self-extracting
% \docstrip\ archive. The files are extracted by running the
% \xfile{.dtx} through \plainTeX:
% \begin{quote}
%   \verb|tex magicnum.dtx|
% \end{quote}
%
% \paragraph{TDS.} Now the different files must be moved into
% the different directories in your installation TDS tree
% (also known as \xfile{texmf} tree):
% \begin{quote}
% \def\t{^^A
% \begin{tabular}{@{}>{\ttfamily}l@{ $\rightarrow$ }>{\ttfamily}l@{}}
%   magicnum.sty & tex/generic/oberdiek/magicnum.sty\\
%   magicnum.lua & scripts/oberdiek/magicnum.lua\\
%   oberdiek.magicnum.lua & scripts/oberdiek/oberdiek.magicnum.lua\\
%   magicnum.pdf & doc/latex/oberdiek/magicnum.pdf\\
%   magicnum.txt & doc/latex/oberdiek/magicnum.txt\\
%   test/magicnum-test1.tex & doc/latex/oberdiek/test/magicnum-test1.tex\\
%   test/magicnum-test2.tex & doc/latex/oberdiek/test/magicnum-test2.tex\\
%   test/magicnum-test3.tex & doc/latex/oberdiek/test/magicnum-test3.tex\\
%   test/magicnum-test4.tex & doc/latex/oberdiek/test/magicnum-test4.tex\\
%   magicnum.dtx & source/latex/oberdiek/magicnum.dtx\\
% \end{tabular}^^A
% }^^A
% \sbox0{\t}^^A
% \ifdim\wd0>\linewidth
%   \begingroup
%     \advance\linewidth by\leftmargin
%     \advance\linewidth by\rightmargin
%   \edef\x{\endgroup
%     \def\noexpand\lw{\the\linewidth}^^A
%   }\x
%   \def\lwbox{^^A
%     \leavevmode
%     \hbox to \linewidth{^^A
%       \kern-\leftmargin\relax
%       \hss
%       \usebox0
%       \hss
%       \kern-\rightmargin\relax
%     }^^A
%   }^^A
%   \ifdim\wd0>\lw
%     \sbox0{\small\t}^^A
%     \ifdim\wd0>\linewidth
%       \ifdim\wd0>\lw
%         \sbox0{\footnotesize\t}^^A
%         \ifdim\wd0>\linewidth
%           \ifdim\wd0>\lw
%             \sbox0{\scriptsize\t}^^A
%             \ifdim\wd0>\linewidth
%               \ifdim\wd0>\lw
%                 \sbox0{\tiny\t}^^A
%                 \ifdim\wd0>\linewidth
%                   \lwbox
%                 \else
%                   \usebox0
%                 \fi
%               \else
%                 \lwbox
%               \fi
%             \else
%               \usebox0
%             \fi
%           \else
%             \lwbox
%           \fi
%         \else
%           \usebox0
%         \fi
%       \else
%         \lwbox
%       \fi
%     \else
%       \usebox0
%     \fi
%   \else
%     \lwbox
%   \fi
% \else
%   \usebox0
% \fi
% \end{quote}
% If you have a \xfile{docstrip.cfg} that configures and enables \docstrip's
% TDS installing feature, then some files can already be in the right
% place, see the documentation of \docstrip.
%
% \subsection{Refresh file name databases}
%
% If your \TeX~distribution
% (\teTeX, \mikTeX, \dots) relies on file name databases, you must refresh
% these. For example, \teTeX\ users run \verb|texhash| or
% \verb|mktexlsr|.
%
% \subsection{Some details for the interested}
%
% \paragraph{Attached source.}
%
% The PDF documentation on CTAN also includes the
% \xfile{.dtx} source file. It can be extracted by
% AcrobatReader 6 or higher. Another option is \textsf{pdftk},
% e.g. unpack the file into the current directory:
% \begin{quote}
%   \verb|pdftk magicnum.pdf unpack_files output .|
% \end{quote}
%
% \paragraph{Unpacking with \LaTeX.}
% The \xfile{.dtx} chooses its action depending on the format:
% \begin{description}
% \item[\plainTeX:] Run \docstrip\ and extract the files.
% \item[\LaTeX:] Generate the documentation.
% \end{description}
% If you insist on using \LaTeX\ for \docstrip\ (really,
% \docstrip\ does not need \LaTeX), then inform the autodetect routine
% about your intention:
% \begin{quote}
%   \verb|latex \let\install=y% \iffalse meta-comment
%
% File: magicnum.dtx
% Version: 2011/04/10 v1.4
% Info: Magic numbers
%
% Copyright (C) 2007, 2009-2011 by
%    Heiko Oberdiek <heiko.oberdiek at googlemail.com>
%
% This work may be distributed and/or modified under the
% conditions of the LaTeX Project Public License, either
% version 1.3c of this license or (at your option) any later
% version. This version of this license is in
%    http://www.latex-project.org/lppl/lppl-1-3c.txt
% and the latest version of this license is in
%    http://www.latex-project.org/lppl.txt
% and version 1.3 or later is part of all distributions of
% LaTeX version 2005/12/01 or later.
%
% This work has the LPPL maintenance status "maintained".
%
% This Current Maintainer of this work is Heiko Oberdiek.
%
% The Base Interpreter refers to any `TeX-Format',
% because some files are installed in TDS:tex/generic//.
%
% This work consists of the main source file magicnum.dtx
% and the derived files
%    magicnum.sty, magicnum.pdf, magicnum.ins, magicnum.drv, magicnum.txt,
%    magicnum-test1.tex, magicnum-test2.tex, magicnum-test3.tex,
%    magicnum-test4.tex, magicnum.lua, oberdiek.magicnum.lua.
%
% Distribution:
%    CTAN:macros/latex/contrib/oberdiek/magicnum.dtx
%    CTAN:macros/latex/contrib/oberdiek/magicnum.pdf
%
% Unpacking:
%    (a) If magicnum.ins is present:
%           tex magicnum.ins
%    (b) Without magicnum.ins:
%           tex magicnum.dtx
%    (c) If you insist on using LaTeX
%           latex \let\install=y\input{magicnum.dtx}
%        (quote the arguments according to the demands of your shell)
%
% Documentation:
%    (a) If magicnum.drv is present:
%           latex magicnum.drv
%    (b) Without magicnum.drv:
%           latex magicnum.dtx; ...
%    The class ltxdoc loads the configuration file ltxdoc.cfg
%    if available. Here you can specify further options, e.g.
%    use A4 as paper format:
%       \PassOptionsToClass{a4paper}{article}
%
%    Programm calls to get the documentation (example):
%       pdflatex magicnum.dtx
%       makeindex -s gind.ist magicnum.idx
%       pdflatex magicnum.dtx
%       makeindex -s gind.ist magicnum.idx
%       pdflatex magicnum.dtx
%
% Installation:
%    TDS:tex/generic/oberdiek/magicnum.sty
%    TDS:scripts/oberdiek/magicnum.lua
%    TDS:scripts/oberdiek/oberdiek.magicnum.lua
%    TDS:doc/latex/oberdiek/magicnum.pdf
%    TDS:doc/latex/oberdiek/magicnum.txt
%    TDS:doc/latex/oberdiek/test/magicnum-test1.tex
%    TDS:doc/latex/oberdiek/test/magicnum-test2.tex
%    TDS:doc/latex/oberdiek/test/magicnum-test3.tex
%    TDS:doc/latex/oberdiek/test/magicnum-test4.tex
%    TDS:source/latex/oberdiek/magicnum.dtx
%
%<*ignore>
\begingroup
  \catcode123=1 %
  \catcode125=2 %
  \def\x{LaTeX2e}%
\expandafter\endgroup
\ifcase 0\ifx\install y1\fi\expandafter
         \ifx\csname processbatchFile\endcsname\relax\else1\fi
         \ifx\fmtname\x\else 1\fi\relax
\else\csname fi\endcsname
%</ignore>
%<*install>
\input docstrip.tex
\Msg{************************************************************************}
\Msg{* Installation}
\Msg{* Package: magicnum 2011/04/10 v1.4 Magic numbers (HO)}
\Msg{************************************************************************}

\keepsilent
\askforoverwritefalse

\let\MetaPrefix\relax
\preamble

This is a generated file.

Project: magicnum
Version: 2011/04/10 v1.4

Copyright (C) 2007, 2009-2011 by
   Heiko Oberdiek <heiko.oberdiek at googlemail.com>

This work may be distributed and/or modified under the
conditions of the LaTeX Project Public License, either
version 1.3c of this license or (at your option) any later
version. This version of this license is in
   http://www.latex-project.org/lppl/lppl-1-3c.txt
and the latest version of this license is in
   http://www.latex-project.org/lppl.txt
and version 1.3 or later is part of all distributions of
LaTeX version 2005/12/01 or later.

This work has the LPPL maintenance status "maintained".

This Current Maintainer of this work is Heiko Oberdiek.

The Base Interpreter refers to any `TeX-Format',
because some files are installed in TDS:tex/generic//.

This work consists of the main source file magicnum.dtx
and the derived files
   magicnum.sty, magicnum.pdf, magicnum.ins, magicnum.drv, magicnum.txt,
   magicnum-test1.tex, magicnum-test2.tex, magicnum-test3.tex,
   magicnum-test4.tex, magicnum.lua, oberdiek.magicnum.lua.

\endpreamble
\let\MetaPrefix\DoubleperCent

\generate{%
  \file{magicnum.ins}{\from{magicnum.dtx}{install}}%
  \file{magicnum.drv}{\from{magicnum.dtx}{driver}}%
  \usedir{tex/generic/oberdiek}%
  \file{magicnum.sty}{\from{magicnum.dtx}{package}}%
  \usedir{doc/latex/oberdiek/test}%
  \file{magicnum-test1.tex}{\from{magicnum.dtx}{test1}}%
  \file{magicnum-test2.tex}{\from{magicnum.dtx}{testplain,testdata}}%
  \file{magicnum-test3.tex}{\from{magicnum.dtx}{testlatex,testdata}}%
  \file{magicnum-test4.tex}{\from{magicnum.dtx}{test4}}%
  \nopreamble
  \nopostamble
  \usedir{doc/latex/oberdiek}%
  \file{magicnum.txt}{\from{magicnum.dtx}{data}}%
  \usedir{source/latex/oberdiek/catalogue}%
  \file{magicnum.xml}{\from{magicnum.dtx}{catalogue}}%
}
\def\MetaPrefix{-- }
\def\defaultpostamble{%
  \MetaPrefix^^J%
  \MetaPrefix\space End of File `\outFileName'.%
}
\def\currentpostamble{\defaultpostamble}%
\generate{%
  \usedir{scripts/oberdiek}%
  \file{magicnum.lua}{\from{magicnum.dtx}{lua}}%
  \file{oberdiek.magicnum.lua}{\from{magicnum.dtx}{lua}}%
}

\catcode32=13\relax% active space
\let =\space%
\Msg{************************************************************************}
\Msg{*}
\Msg{* To finish the installation you have to move the following}
\Msg{* file into a directory searched by TeX:}
\Msg{*}
\Msg{*     magicnum.sty}
\Msg{*}
\Msg{* And install the following script files:}
\Msg{*}
\Msg{*     magicnum.lua, oberdiek.magicnum.lua}
\Msg{*}
\Msg{* To produce the documentation run the file `magicnum.drv'}
\Msg{* through LaTeX.}
\Msg{*}
\Msg{* Happy TeXing!}
\Msg{*}
\Msg{************************************************************************}

\endbatchfile
%</install>
%<*ignore>
\fi
%</ignore>
%<*driver>
\NeedsTeXFormat{LaTeX2e}
\ProvidesFile{magicnum.drv}%
  [2011/04/10 v1.4 Magic numbers (HO)]%
\documentclass{ltxdoc}
\usepackage{holtxdoc}[2011/11/22]
\usepackage{array}
\begin{document}
  \DocInput{magicnum.dtx}%
\end{document}
%</driver>
% \fi
%
% \CheckSum{755}
%
% \CharacterTable
%  {Upper-case    \A\B\C\D\E\F\G\H\I\J\K\L\M\N\O\P\Q\R\S\T\U\V\W\X\Y\Z
%   Lower-case    \a\b\c\d\e\f\g\h\i\j\k\l\m\n\o\p\q\r\s\t\u\v\w\x\y\z
%   Digits        \0\1\2\3\4\5\6\7\8\9
%   Exclamation   \!     Double quote  \"     Hash (number) \#
%   Dollar        \$     Percent       \%     Ampersand     \&
%   Acute accent  \'     Left paren    \(     Right paren   \)
%   Asterisk      \*     Plus          \+     Comma         \,
%   Minus         \-     Point         \.     Solidus       \/
%   Colon         \:     Semicolon     \;     Less than     \<
%   Equals        \=     Greater than  \>     Question mark \?
%   Commercial at \@     Left bracket  \[     Backslash     \\
%   Right bracket \]     Circumflex    \^     Underscore    \_
%   Grave accent  \`     Left brace    \{     Vertical bar  \|
%   Right brace   \}     Tilde         \~}
%
% \GetFileInfo{magicnum.drv}
%
% \title{The \xpackage{magicnum} package}
% \date{2011/04/10 v1.4}
% \author{Heiko Oberdiek\\\xemail{heiko.oberdiek at googlemail.com}}
%
% \maketitle
%
% \begin{abstract}
% This packages allows to access magic numbers by a hierarchical
% name system.
% \end{abstract}
%
% \tableofcontents
%
% \hypersetup{bookmarksopenlevel=2}
% \section{Documentation}
%
% \subsection{Introduction}
%
% Especially since \eTeX\ there are many integer values
% with special meanings, such as catcodes, group types, \dots
% Package \xpackage{etex}, enabled by options, defines
% macros in the user namespace for these values.
%
% This package goes another approach for storing the names and values.
% \begin{itemize}
% \item If \LuaTeX\ is available, they
% are stored in Lua tables.
% \item Without \LuaTeX\ they are remembered using internal
% macros.
% \end{itemize}
%
% \subsection{User interface}
%
% The integer values and names are organized in a hierarchical
% scheme of categories with the property names as leaves.
% Example: \eTeX's \cs{currentgrouplevel} reports |2| for a
% group caused by \cs{hbox}. This package has choosen to organize
% the group types in a main category |etex| and its subcategory
% |grouptype|:
% \begin{quote}
%   |etex.grouptype.hbox| = |2|
% \end{quote}
% The property name |hbox| in category |etex.grouptype| has value |2|.
% Dots are used to separate components.
%
% If you want to have the value, the access key is constructed by
% the category with all its components and the property name.
% For the opposite the value is used instead of the property name.
%
% Values are always integers (including negative numbers).
%
% \subsubsection{\cs{magicnum}}
%
% \begin{declcs}{magicnum} \M{access key}
% \end{declcs}
% Macro \cs{magicnum} expects an access key as argument and
% expands to the requested data. The macro is always expandable.
% In case of errors the expansion result is empty.
%
% The same macro is also used for getting a property name.
% In this case the property name part in the access key is
% replaced by the value.
%
% The catcodes
% of the resulting numbers and strings follow \TeX's tradition of
% \cs{string}, \cs{meaning}, \dots: The space has catcode 10
% (|tex.catcode.space|) and the other characters have catcode
% 12 (|tex.catcode.other|).
%
% Examples:
% \begin{quote}
%   |\magicnum{etex.grouptype.hbox}| $\Rightarrow$ |2|\\
%   |\magicnum{tex.catcode.14}| $\Rightarrow$ |comment|\\
%   |\magicnum{tex.catcode.undefined}| $\Rightarrow$ $\emptyset$
% \end{quote}
%
% \subsubsection{Properties}
%
% \begin{itemize}
% \item The components of a category are either subcategories or
%       key value pairs, but not both.
% \item The full specified property names are unique and thus
%       has one integer value exactly.
% \item Also the values inside a category are unique.
%       This condition is a prerequisite for the reverse mapping
%       of \cs{magicnum}.
% \item All names start with a letter. Only letters or digits
%       may follow.
% \end{itemize}
%
% \subsection{Data}
%
%  \subsubsection{\texorpdfstring{Category }{}\texttt{tex.catcode}}
%
% \begin{quote}
% \begin{tabular}{@{}>{\ttfamily}l>{\ttfamily}l@{}}
%    tex.catcode.escape & 0\\
%    tex.catcode.begingroup & 1\\
%    tex.catcode.endgroup & 2\\
%    tex.catcode.math & 3\\
%    tex.catcode.align & 4\\
%    tex.catcode.eol & 5\\
%    tex.catcode.parameter & 6\\
%    tex.catcode.superscript & 7\\
%    tex.catcode.subscript & 8\\
%    tex.catcode.ignore & 9\\
%    tex.catcode.space & 10\\
%    tex.catcode.letter & 11\\
%    tex.catcode.other & 12\\
%    tex.catcode.active & 13\\
%    tex.catcode.comment & 14\\
%    tex.catcode.invalid & 15\\
%  \end{tabular}
%  \end{quote}
%
%  \subsubsection{\texorpdfstring{Category }{}\texttt{etex.grouptype}}
%
% \begin{quote}
% \begin{tabular}{@{}>{\ttfamily}l>{\ttfamily}l@{}}
%    etex.grouptype.bottomlevel & 0\\
%    etex.grouptype.simple & 1\\
%    etex.grouptype.hbox & 2\\
%    etex.grouptype.adjustedhbox & 3\\
%    etex.grouptype.vbox & 4\\
%    etex.grouptype.align & 5\\
%    etex.grouptype.noalign & 6\\
%    etex.grouptype.output & 8\\
%    etex.grouptype.math & 9\\
%    etex.grouptype.disc & 10\\
%    etex.grouptype.insert & 11\\
%    etex.grouptype.vcenter & 12\\
%    etex.grouptype.mathchoice & 13\\
%    etex.grouptype.semisimple & 14\\
%    etex.grouptype.mathshift & 15\\
%    etex.grouptype.mathleft & 16\\
%  \end{tabular}
%  \end{quote}
%
%  \subsubsection{\texorpdfstring{Category }{}\texttt{etex.iftype}}
%
% \begin{quote}
% \begin{tabular}{@{}>{\ttfamily}l>{\ttfamily}l@{}}
%    etex.iftype.none & 0\\
%    etex.iftype.char & 1\\
%    etex.iftype.cat & 2\\
%    etex.iftype.num & 3\\
%    etex.iftype.dim & 4\\
%    etex.iftype.odd & 5\\
%    etex.iftype.vmode & 6\\
%    etex.iftype.hmode & 7\\
%    etex.iftype.mmode & 8\\
%    etex.iftype.inner & 9\\
%    etex.iftype.void & 10\\
%    etex.iftype.hbox & 11\\
%    etex.iftype.vbox & 12\\
%    etex.iftype.x & 13\\
%    etex.iftype.eof & 14\\
%    etex.iftype.true & 15\\
%    etex.iftype.false & 16\\
%    etex.iftype.case & 17\\
%    etex.iftype.defined & 18\\
%    etex.iftype.csname & 19\\
%    etex.iftype.fontchar & 20\\
%  \end{tabular}
%  \end{quote}
%
%  \subsubsection{\texorpdfstring{Category }{}\texttt{etex.nodetype}}
%
% \begin{quote}
% \begin{tabular}{@{}>{\ttfamily}l>{\ttfamily}l@{}}
%    etex.nodetype.none & -1\\
%    etex.nodetype.char & 0\\
%    etex.nodetype.hlist & 1\\
%    etex.nodetype.vlist & 2\\
%    etex.nodetype.rule & 3\\
%    etex.nodetype.ins & 4\\
%    etex.nodetype.mark & 5\\
%    etex.nodetype.adjust & 6\\
%    etex.nodetype.ligature & 7\\
%    etex.nodetype.disc & 8\\
%    etex.nodetype.whatsit & 9\\
%    etex.nodetype.math & 10\\
%    etex.nodetype.glue & 11\\
%    etex.nodetype.kern & 12\\
%    etex.nodetype.penalty & 13\\
%    etex.nodetype.unset & 14\\
%    etex.nodetype.maths & 15\\
%  \end{tabular}
%  \end{quote}
%
%  \subsubsection{\texorpdfstring{Category }{}\texttt{etex.interactionmode}}
%
% \begin{quote}
% \begin{tabular}{@{}>{\ttfamily}l>{\ttfamily}l@{}}
%    etex.interactionmode.batch & 0\\
%    etex.interactionmode.nonstop & 1\\
%    etex.interactionmode.scroll & 2\\
%    etex.interactionmode.errorstop & 3\\
%  \end{tabular}
%  \end{quote}
%
%  \subsubsection{\texorpdfstring{Category }{}\texttt{luatex.pdfliteral.mode}}
%
% \begin{quote}
% \begin{tabular}{@{}>{\ttfamily}l>{\ttfamily}l@{}}
%    luatex.pdfliteral.mode.setorigin & 0\\
%    luatex.pdfliteral.mode.page & 1\\
%    luatex.pdfliteral.mode.direct & 2\\
%  \end{tabular}
%  \end{quote}
%
%
% \hypersetup{bookmarksopenlevel=1}
%
% \StopEventually{
% }
%
% \section{Implementation}
%
%    \begin{macrocode}
%<*package>
%    \end{macrocode}
%
% \subsection{Reload check and package identification}
%    Reload check, especially if the package is not used with \LaTeX.
%    \begin{macrocode}
\begingroup\catcode61\catcode48\catcode32=10\relax%
  \catcode13=5 % ^^M
  \endlinechar=13 %
  \catcode35=6 % #
  \catcode39=12 % '
  \catcode44=12 % ,
  \catcode45=12 % -
  \catcode46=12 % .
  \catcode58=12 % :
  \catcode64=11 % @
  \catcode123=1 % {
  \catcode125=2 % }
  \expandafter\let\expandafter\x\csname ver@magicnum.sty\endcsname
  \ifx\x\relax % plain-TeX, first loading
  \else
    \def\empty{}%
    \ifx\x\empty % LaTeX, first loading,
      % variable is initialized, but \ProvidesPackage not yet seen
    \else
      \expandafter\ifx\csname PackageInfo\endcsname\relax
        \def\x#1#2{%
          \immediate\write-1{Package #1 Info: #2.}%
        }%
      \else
        \def\x#1#2{\PackageInfo{#1}{#2, stopped}}%
      \fi
      \x{magicnum}{The package is already loaded}%
      \aftergroup\endinput
    \fi
  \fi
\endgroup%
%    \end{macrocode}
%    Package identification:
%    \begin{macrocode}
\begingroup\catcode61\catcode48\catcode32=10\relax%
  \catcode13=5 % ^^M
  \endlinechar=13 %
  \catcode35=6 % #
  \catcode39=12 % '
  \catcode40=12 % (
  \catcode41=12 % )
  \catcode44=12 % ,
  \catcode45=12 % -
  \catcode46=12 % .
  \catcode47=12 % /
  \catcode58=12 % :
  \catcode64=11 % @
  \catcode91=12 % [
  \catcode93=12 % ]
  \catcode123=1 % {
  \catcode125=2 % }
  \expandafter\ifx\csname ProvidesPackage\endcsname\relax
    \def\x#1#2#3[#4]{\endgroup
      \immediate\write-1{Package: #3 #4}%
      \xdef#1{#4}%
    }%
  \else
    \def\x#1#2[#3]{\endgroup
      #2[{#3}]%
      \ifx#1\@undefined
        \xdef#1{#3}%
      \fi
      \ifx#1\relax
        \xdef#1{#3}%
      \fi
    }%
  \fi
\expandafter\x\csname ver@magicnum.sty\endcsname
\ProvidesPackage{magicnum}%
  [2011/04/10 v1.4 Magic numbers (HO)]%
%    \end{macrocode}
%
% \subsection{Catcodes}
%
%    \begin{macrocode}
\begingroup\catcode61\catcode48\catcode32=10\relax%
  \catcode13=5 % ^^M
  \endlinechar=13 %
  \catcode123=1 % {
  \catcode125=2 % }
  \catcode64=11 % @
  \def\x{\endgroup
    \expandafter\edef\csname magicnum@AtEnd\endcsname{%
      \endlinechar=\the\endlinechar\relax
      \catcode13=\the\catcode13\relax
      \catcode32=\the\catcode32\relax
      \catcode35=\the\catcode35\relax
      \catcode61=\the\catcode61\relax
      \catcode64=\the\catcode64\relax
      \catcode123=\the\catcode123\relax
      \catcode125=\the\catcode125\relax
    }%
  }%
\x\catcode61\catcode48\catcode32=10\relax%
\catcode13=5 % ^^M
\endlinechar=13 %
\catcode35=6 % #
\catcode64=11 % @
\catcode123=1 % {
\catcode125=2 % }
\def\TMP@EnsureCode#1#2{%
  \edef\magicnum@AtEnd{%
    \magicnum@AtEnd
    \catcode#1=\the\catcode#1\relax
  }%
  \catcode#1=#2\relax
}
\TMP@EnsureCode{34}{12}% "
\TMP@EnsureCode{39}{12}% '
\TMP@EnsureCode{40}{12}% (
\TMP@EnsureCode{41}{12}% )
\TMP@EnsureCode{42}{12}% *
\TMP@EnsureCode{44}{12}% ,
\TMP@EnsureCode{45}{12}% -
\TMP@EnsureCode{46}{12}% .
\TMP@EnsureCode{47}{12}% /
\TMP@EnsureCode{58}{12}% :
\TMP@EnsureCode{60}{12}% <
\TMP@EnsureCode{62}{12}% >
\TMP@EnsureCode{91}{12}% [
\TMP@EnsureCode{93}{12}% ]
\edef\magicnum@AtEnd{\magicnum@AtEnd\noexpand\endinput}
%    \end{macrocode}
%
% \subsection{Check for previous definition}
%
%    \begin{macrocode}
\begingroup\expandafter\expandafter\expandafter\endgroup
\expandafter\ifx\csname newcommand\endcsname\relax
  \expandafter\ifx\csname magicnum\endcsname\relax
  \else
    \input infwarerr.sty\relax
    \@PackageError{magicnum}{%
      \string\magicnum\space is already defined%
    }\@ehc
  \fi
\else
  \newcommand*{\magicnum}{}%
\fi
%    \end{macrocode}
%
% \subsection{Without \LuaTeX}
%
%    \begin{macrocode}
\begingroup\expandafter\expandafter\expandafter\endgroup
\expandafter\ifx\csname directlua\endcsname\relax
%    \end{macrocode}
%
%    \begin{macro}{\magicnum}
%    \begin{macrocode}
  \begingroup\expandafter\expandafter\expandafter\endgroup
  \expandafter\ifx\csname ifcsname\endcsname\relax
    \def\magicnum#1{%
      \expandafter\ifx\csname MG@#1\endcsname\relax
      \else
        \csname MG@#1\endcsname
      \fi
    }%
  \else
    \begingroup
      \edef\x{\endgroup
        \def\noexpand\magicnum##1{%
          \expandafter\noexpand\csname
          ifcsname\endcsname MG@##1\noexpand\endcsname
            \noexpand\csname MG@##1%
                 \noexpand\expandafter\noexpand\endcsname
          \expandafter\noexpand\csname fi\endcsname
        }%
      }%
    \x
  \fi
%    \end{macrocode}
%    \end{macro}
%
%    \begin{macrocode}
\else
%    \end{macrocode}
%
% \subsection{With \LuaTeX}
%
%    \begin{macrocode}
  \begingroup\expandafter\expandafter\expandafter\endgroup
  \expandafter\ifx\csname RequirePackage\endcsname\relax
    \input ifluatex.sty\relax
    \input infwarerr.sty\relax
  \else
    \RequirePackage{ifluatex}[2010/03/01]%
    \RequirePackage{infwarerr}[2010/04/08]%
  \fi
%    \end{macrocode}
%
%    \begin{macro}{\magicnum@directlua}
%    \begin{macrocode}
  \ifnum\luatexversion<36 %
    \def\magicnum@directlua{\directlua0 }%
  \else
    \let\magicnum@directlua\directlua
  \fi
%    \end{macrocode}
%    \end{macro}
%    \begin{macrocode}
  \magicnum@directlua{%
    require("oberdiek.magicnum")%
  }%
  \begingroup
    \def\x{2011/04/10 v1.4}%
    \def\StripPrefix#1>{}%
    \edef\x{\expandafter\StripPrefix\meaning\x}%
    \edef\y{%
      \magicnum@directlua{%
        if oberdiek.magicnum.getversion then %
          oberdiek.magicnum.getversion()%
        end%
      }%
    }%
    \ifx\x\y
    \else
      \@PackageError{magicnum}{%
        Wrong version of lua module.\MessageBreak
        Package version: \x\MessageBreak
        Lua module: \y
      }\@ehc
    \fi
  \endgroup
%    \end{macrocode}
%    \begin{macro}{\luaescapestring}
%    \begin{macrocode}
  \begingroup
    \expandafter\ifx\csname luaescapestring\endcsname\relax
      \directlua{%
        if tex.enableprimitives then %
          tex.enableprimitives('magicnum@', {'luaescapestring'})%
        end%
      }%
      \global\let\luaescapestring\magicnum@luaescapestring
    \fi
    \expandafter\ifx\csname luaescapestring\endcsname\relax
      \escapechar=92 %
      \@PackageError{magicnum}{%
        Missing \string\luaescapestring
      }\@ehc
    \fi
  \endgroup
%    \end{macrocode}
%    \end{macro}
%    \begin{macro}{\magicnum}
%    \begin{macrocode}
  \def\magicnum#1{%
    \magicnum@directlua{%
      oberdiek.magicnum.get("\luaescapestring{#1}")%
    }%
  }%
%    \end{macrocode}
%    \end{macro}
%
%    \begin{macrocode}
  \expandafter\magicnum@AtEnd
\fi%
%</package>
%    \end{macrocode}
%
% \subsection{Data}
%
% \subsubsection{Plain data}
%
%    \begin{macrocode}
%<*data>
tex.catcode
  escape = 0
  begingroup = 1
  endgroup = 2
  math = 3
  align = 4
  eol = 5
  parameter = 6
  superscript = 7
  subscript = 8
  ignore = 9
  space = 10
  letter = 11
  other = 12
  active = 13
  comment = 14
  invalid = 15
etex.grouptype
  bottomlevel = 0
  simple = 1
  hbox = 2
  adjustedhbox = 3
  vbox = 4
  align = 5
  noalign = 6
  output = 8
  math = 9
  disc = 10
  insert = 11
  vcenter = 12
  mathchoice = 13
  semisimple = 14
  mathshift = 15
  mathleft = 16
etex.iftype
  none = 0
  char = 1
  cat = 2
  num = 3
  dim = 4
  odd = 5
  vmode = 6
  hmode = 7
  mmode = 8
  inner = 9
  void = 10
  hbox = 11
  vbox = 12
  x = 13
  eof = 14
  true = 15
  false = 16
  case = 17
  defined = 18
  csname = 19
  fontchar = 20
etex.nodetype
  none = -1
  char = 0
  hlist = 1
  vlist = 2
  rule = 3
  ins = 4
  mark = 5
  adjust = 6
  ligature = 7
  disc = 8
  whatsit = 9
  math = 10
  glue = 11
  kern = 12
  penalty = 13
  unset = 14
  maths = 15
etex.interactionmode
  batch = 0
  nonstop = 1
  scroll = 2
  errorstop = 3
luatex.pdfliteral.mode
  setorigin = 0
  page = 1
  direct = 2
%</data>
%    \end{macrocode}
%
% \subsubsection{Data for \TeX}
%
%    \begin{macrocode}
%<*package>
%    \end{macrocode}
%    \begin{macro}{\magicnum@add}
%    \begin{macrocode}
\begingroup\expandafter\expandafter\expandafter\endgroup
\expandafter\ifx\csname detokenize\endcsname\relax
  \def\magicnum@add#1#2#3{%
    \expandafter\magicnum@@add
        \csname MG@#1.#2\expandafter\endcsname
        \csname MG@#1.#3\endcsname
       {#3}{#2}%
  }%
  \def\magicnum@@add#1#2#3#4{%
    \def#1{#3}%
    \def#2{#4}%
    \edef#1{%
      \expandafter\strip@prefix\meaning#1%
    }%
    \edef#2{%
      \expandafter\strip@prefix\meaning#2%
    }%
  }%
  \expandafter\ifx\csname strip@prefix\endcsname\relax
    \def\strip@prefix#1->{}%
  \fi
\else
  \def\magicnum@add#1#2#3{%
    \expandafter\edef\csname MG@#1.#2\endcsname{%
      \detokenize{#3}%
    }%
    \expandafter\edef\csname MG@#1.#3\endcsname{%
      \detokenize{#2}%
    }%
  }%
\fi
%    \end{macrocode}
%    \end{macro}
%    \begin{macrocode}
\magicnum@add{tex.catcode}{escape}{0}
\magicnum@add{tex.catcode}{begingroup}{1}
\magicnum@add{tex.catcode}{endgroup}{2}
\magicnum@add{tex.catcode}{math}{3}
\magicnum@add{tex.catcode}{align}{4}
\magicnum@add{tex.catcode}{eol}{5}
\magicnum@add{tex.catcode}{parameter}{6}
\magicnum@add{tex.catcode}{superscript}{7}
\magicnum@add{tex.catcode}{subscript}{8}
\magicnum@add{tex.catcode}{ignore}{9}
\magicnum@add{tex.catcode}{space}{10}
\magicnum@add{tex.catcode}{letter}{11}
\magicnum@add{tex.catcode}{other}{12}
\magicnum@add{tex.catcode}{active}{13}
\magicnum@add{tex.catcode}{comment}{14}
\magicnum@add{tex.catcode}{invalid}{15}
\magicnum@add{etex.grouptype}{bottomlevel}{0}
\magicnum@add{etex.grouptype}{simple}{1}
\magicnum@add{etex.grouptype}{hbox}{2}
\magicnum@add{etex.grouptype}{adjustedhbox}{3}
\magicnum@add{etex.grouptype}{vbox}{4}
\magicnum@add{etex.grouptype}{align}{5}
\magicnum@add{etex.grouptype}{noalign}{6}
\magicnum@add{etex.grouptype}{output}{8}
\magicnum@add{etex.grouptype}{math}{9}
\magicnum@add{etex.grouptype}{disc}{10}
\magicnum@add{etex.grouptype}{insert}{11}
\magicnum@add{etex.grouptype}{vcenter}{12}
\magicnum@add{etex.grouptype}{mathchoice}{13}
\magicnum@add{etex.grouptype}{semisimple}{14}
\magicnum@add{etex.grouptype}{mathshift}{15}
\magicnum@add{etex.grouptype}{mathleft}{16}
\magicnum@add{etex.iftype}{none}{0}
\magicnum@add{etex.iftype}{char}{1}
\magicnum@add{etex.iftype}{cat}{2}
\magicnum@add{etex.iftype}{num}{3}
\magicnum@add{etex.iftype}{dim}{4}
\magicnum@add{etex.iftype}{odd}{5}
\magicnum@add{etex.iftype}{vmode}{6}
\magicnum@add{etex.iftype}{hmode}{7}
\magicnum@add{etex.iftype}{mmode}{8}
\magicnum@add{etex.iftype}{inner}{9}
\magicnum@add{etex.iftype}{void}{10}
\magicnum@add{etex.iftype}{hbox}{11}
\magicnum@add{etex.iftype}{vbox}{12}
\magicnum@add{etex.iftype}{x}{13}
\magicnum@add{etex.iftype}{eof}{14}
\magicnum@add{etex.iftype}{true}{15}
\magicnum@add{etex.iftype}{false}{16}
\magicnum@add{etex.iftype}{case}{17}
\magicnum@add{etex.iftype}{defined}{18}
\magicnum@add{etex.iftype}{csname}{19}
\magicnum@add{etex.iftype}{fontchar}{20}
\magicnum@add{etex.nodetype}{none}{-1}
\magicnum@add{etex.nodetype}{char}{0}
\magicnum@add{etex.nodetype}{hlist}{1}
\magicnum@add{etex.nodetype}{vlist}{2}
\magicnum@add{etex.nodetype}{rule}{3}
\magicnum@add{etex.nodetype}{ins}{4}
\magicnum@add{etex.nodetype}{mark}{5}
\magicnum@add{etex.nodetype}{adjust}{6}
\magicnum@add{etex.nodetype}{ligature}{7}
\magicnum@add{etex.nodetype}{disc}{8}
\magicnum@add{etex.nodetype}{whatsit}{9}
\magicnum@add{etex.nodetype}{math}{10}
\magicnum@add{etex.nodetype}{glue}{11}
\magicnum@add{etex.nodetype}{kern}{12}
\magicnum@add{etex.nodetype}{penalty}{13}
\magicnum@add{etex.nodetype}{unset}{14}
\magicnum@add{etex.nodetype}{maths}{15}
\magicnum@add{etex.interactionmode}{batch}{0}
\magicnum@add{etex.interactionmode}{nonstop}{1}
\magicnum@add{etex.interactionmode}{scroll}{2}
\magicnum@add{etex.interactionmode}{errorstop}{3}
\magicnum@add{luatex.pdfliteral.mode}{setorigin}{0}
\magicnum@add{luatex.pdfliteral.mode}{page}{1}
\magicnum@add{luatex.pdfliteral.mode}{direct}{2}
%    \end{macrocode}
%    \begin{macrocode}
\magicnum@AtEnd%
%</package>
%    \end{macrocode}
%
% \subsubsection{Lua module}
%
%    \begin{macrocode}
%<*lua>
%    \end{macrocode}
%    \begin{macrocode}
module("oberdiek.magicnum", package.seeall)
%    \end{macrocode}
%    \begin{macrocode}
function getversion()
  tex.write("2011/04/10 v1.4")
end
%    \end{macrocode}
%    \begin{macrocode}
local data = {
  ["tex.catcode"] = {
    [0] = "escape",
    [1] = "begingroup",
    [2] = "endgroup",
    [3] = "math",
    [4] = "align",
    [5] = "eol",
    [6] = "parameter",
    [7] = "superscript",
    [8] = "subscript",
    [9] = "ignore",
    [10] = "space",
    [11] = "letter",
    [12] = "other",
    [13] = "active",
    [14] = "comment",
    [15] = "invalid",
    ["active"] = 13,
    ["align"] = 4,
    ["begingroup"] = 1,
    ["comment"] = 14,
    ["endgroup"] = 2,
    ["eol"] = 5,
    ["escape"] = 0,
    ["ignore"] = 9,
    ["invalid"] = 15,
    ["letter"] = 11,
    ["math"] = 3,
    ["other"] = 12,
    ["parameter"] = 6,
    ["space"] = 10,
    ["subscript"] = 8,
    ["superscript"] = 7
  },
  ["etex.grouptype"] = {
    [0] = "bottomlevel",
    [1] = "simple",
    [2] = "hbox",
    [3] = "adjustedhbox",
    [4] = "vbox",
    [5] = "align",
    [6] = "noalign",
    [8] = "output",
    [9] = "math",
    [10] = "disc",
    [11] = "insert",
    [12] = "vcenter",
    [13] = "mathchoice",
    [14] = "semisimple",
    [15] = "mathshift",
    [16] = "mathleft",
    ["adjustedhbox"] = 3,
    ["align"] = 5,
    ["bottomlevel"] = 0,
    ["disc"] = 10,
    ["hbox"] = 2,
    ["insert"] = 11,
    ["math"] = 9,
    ["mathchoice"] = 13,
    ["mathleft"] = 16,
    ["mathshift"] = 15,
    ["noalign"] = 6,
    ["output"] = 8,
    ["semisimple"] = 14,
    ["simple"] = 1,
    ["vbox"] = 4,
    ["vcenter"] = 12
  },
  ["etex.iftype"] = {
    [0] = "none",
    [1] = "char",
    [2] = "cat",
    [3] = "num",
    [4] = "dim",
    [5] = "odd",
    [6] = "vmode",
    [7] = "hmode",
    [8] = "mmode",
    [9] = "inner",
    [10] = "void",
    [11] = "hbox",
    [12] = "vbox",
    [13] = "x",
    [14] = "eof",
    [15] = "true",
    [16] = "false",
    [17] = "case",
    [18] = "defined",
    [19] = "csname",
    [20] = "fontchar",
    ["case"] = 17,
    ["cat"] = 2,
    ["char"] = 1,
    ["csname"] = 19,
    ["defined"] = 18,
    ["dim"] = 4,
    ["eof"] = 14,
    ["false"] = 16,
    ["fontchar"] = 20,
    ["hbox"] = 11,
    ["hmode"] = 7,
    ["inner"] = 9,
    ["mmode"] = 8,
    ["none"] = 0,
    ["num"] = 3,
    ["odd"] = 5,
    ["true"] = 15,
    ["vbox"] = 12,
    ["vmode"] = 6,
    ["void"] = 10,
    ["x"] = 13
  },
  ["etex.nodetype"] = {
    [-1] = "none",
    [0] = "char",
    [1] = "hlist",
    [2] = "vlist",
    [3] = "rule",
    [4] = "ins",
    [5] = "mark",
    [6] = "adjust",
    [7] = "ligature",
    [8] = "disc",
    [9] = "whatsit",
    [10] = "math",
    [11] = "glue",
    [12] = "kern",
    [13] = "penalty",
    [14] = "unset",
    [15] = "maths",
    ["adjust"] = 6,
    ["char"] = 0,
    ["disc"] = 8,
    ["glue"] = 11,
    ["hlist"] = 1,
    ["ins"] = 4,
    ["kern"] = 12,
    ["ligature"] = 7,
    ["mark"] = 5,
    ["math"] = 10,
    ["maths"] = 15,
    ["none"] = -1,
    ["penalty"] = 13,
    ["rule"] = 3,
    ["unset"] = 14,
    ["vlist"] = 2,
    ["whatsit"] = 9
  },
  ["etex.interactionmode"] = {
    [0] = "batch",
    [1] = "nonstop",
    [2] = "scroll",
    [3] = "errorstop",
    ["batch"] = 0,
    ["errorstop"] = 3,
    ["nonstop"] = 1,
    ["scroll"] = 2
  },
  ["luatex.pdfliteral.mode"] = {
    [0] = "setorigin",
    [1] = "page",
    [2] = "direct",
    ["direct"] = 2,
    ["page"] = 1,
    ["setorigin"] = 0
  }
}
%    \end{macrocode}
%    \begin{macrocode}
function get(name)
  local startpos, endpos, category, entry =
      string.find(name, "^(%a[%a%d%.]*)%.(-?[%a%d]+)$")
  if not entry then
    return
  end
  local node = data[category]
  if not node then
    return
  end
  local num = tonumber(entry)
  local value
  if num then
    value = node[num]
    if not value then
      return
    end
  else
    value = node[entry]
    if not value then
      return
    end
    value = "" .. value
  end
  tex.write(value)
end
%    \end{macrocode}
%
%    \begin{macrocode}
%</lua>
%    \end{macrocode}
%
% \section{Test}
%
% \subsection{Catcode checks for loading}
%
%    \begin{macrocode}
%<*test1>
%    \end{macrocode}
%    \begin{macrocode}
\catcode`\{=1 %
\catcode`\}=2 %
\catcode`\#=6 %
\catcode`\@=11 %
\expandafter\ifx\csname count@\endcsname\relax
  \countdef\count@=255 %
\fi
\expandafter\ifx\csname @gobble\endcsname\relax
  \long\def\@gobble#1{}%
\fi
\expandafter\ifx\csname @firstofone\endcsname\relax
  \long\def\@firstofone#1{#1}%
\fi
\expandafter\ifx\csname loop\endcsname\relax
  \expandafter\@firstofone
\else
  \expandafter\@gobble
\fi
{%
  \def\loop#1\repeat{%
    \def\body{#1}%
    \iterate
  }%
  \def\iterate{%
    \body
      \let\next\iterate
    \else
      \let\next\relax
    \fi
    \next
  }%
  \let\repeat=\fi
}%
\def\RestoreCatcodes{}
\count@=0 %
\loop
  \edef\RestoreCatcodes{%
    \RestoreCatcodes
    \catcode\the\count@=\the\catcode\count@\relax
  }%
\ifnum\count@<255 %
  \advance\count@ 1 %
\repeat

\def\RangeCatcodeInvalid#1#2{%
  \count@=#1\relax
  \loop
    \catcode\count@=15 %
  \ifnum\count@<#2\relax
    \advance\count@ 1 %
  \repeat
}
\def\RangeCatcodeCheck#1#2#3{%
  \count@=#1\relax
  \loop
    \ifnum#3=\catcode\count@
    \else
      \errmessage{%
        Character \the\count@\space
        with wrong catcode \the\catcode\count@\space
        instead of \number#3%
      }%
    \fi
  \ifnum\count@<#2\relax
    \advance\count@ 1 %
  \repeat
}
\def\space{ }
\expandafter\ifx\csname LoadCommand\endcsname\relax
  \def\LoadCommand{\input magicnum.sty\relax}%
\fi
\def\Test{%
  \RangeCatcodeInvalid{0}{47}%
  \RangeCatcodeInvalid{58}{64}%
  \RangeCatcodeInvalid{91}{96}%
  \RangeCatcodeInvalid{123}{255}%
  \catcode`\@=12 %
  \catcode`\\=0 %
  \catcode`\%=14 %
  \LoadCommand
  \RangeCatcodeCheck{0}{36}{15}%
  \RangeCatcodeCheck{37}{37}{14}%
  \RangeCatcodeCheck{38}{47}{15}%
  \RangeCatcodeCheck{48}{57}{12}%
  \RangeCatcodeCheck{58}{63}{15}%
  \RangeCatcodeCheck{64}{64}{12}%
  \RangeCatcodeCheck{65}{90}{11}%
  \RangeCatcodeCheck{91}{91}{15}%
  \RangeCatcodeCheck{92}{92}{0}%
  \RangeCatcodeCheck{93}{96}{15}%
  \RangeCatcodeCheck{97}{122}{11}%
  \RangeCatcodeCheck{123}{255}{15}%
  \RestoreCatcodes
}
\Test
\csname @@end\endcsname
\end
%    \end{macrocode}
%    \begin{macrocode}
%</test1>
%    \end{macrocode}
%
% \subsection{Test data}
%
%    \begin{macrocode}
%<*testplain>
\input magicnum.sty\relax
\def\Test#1#2{%
  \edef\result{\magicnum{#1}}%
  \edef\expect{#2}%
  \edef\expect{\expandafter\stripprefix\meaning\expect}%
  \ifx\result\expect
  \else
    \errmessage{%
      Failed: [#1] % hash-ok
      returns [\result] instead of [\expect]%
    }%
  \fi
}
\def\stripprefix#1->{}
%</testplain>
%    \end{macrocode}
%    \begin{macrocode}
%<*testlatex>
\NeedsTeXFormat{LaTeX2e}
\documentclass{minimal}
\usepackage{magicnum}[2011/04/10]
\usepackage{qstest}
\IncludeTests{*}
\LogTests{log}{*}{*}
\newcommand*{\Test}[2]{%
  \Expect*{\magicnum{#1}}{#2}%
}
\begin{qstest}{magicnum}{magicnum}
%</testlatex>
%    \end{macrocode}
%    \begin{macrocode}
%<*testdata>
\Test{tex.catcode.escape}{0}
\Test{tex.catcode.invalid}{15}
\Test{tex.catcode.unknown}{}
\Test{tex.catcode.0}{escape}
\Test{tex.catcode.15}{invalid}
\Test{etex.iftype.true}{15}
\Test{etex.iftype.false}{16}
\Test{etex.iftype.15}{true}
\Test{etex.iftype.16}{false}
\Test{etex.nodetype.none}{-1}
\Test{etex.nodetype.-1}{none}
\Test{luatex.pdfliteral.mode.direct}{2}
\Test{luatex.pdfliteral.mode.1}{page}
\Test{}{}
\Test{unknown}{}
\Test{unknown.foo.bar}{}
\Test{unknown.foo.4}{}
%</testdata>
%    \end{macrocode}
%    \begin{macrocode}
%<*testplain>
\csname @@end\endcsname
\end
%</testplain>
%<*testlatex>
\end{qstest}
\csname @@end\endcsname
%</testlatex>
%    \end{macrocode}
%
% \subsection{Small test for \hologo{iniTeX}}
%
%    \begin{macrocode}
%<*test4>
\catcode`\{=1
\catcode`\}=2
\catcode`\#=6
\input magicnum.sty\relax
\edef\x{\magicnum{tex.catcode.15}}
\edef\y{invalid}
\def\Strip#1>{}
\edef\y{\expandafter\Strip\meaning\y}
\ifx\x\y
  \immediate\write16{Ok}%
\else
  \errmessage{\x<>\y}%
\fi
\csname @@end\endcsname\end
%</test4>
%    \end{macrocode}
%
% \section{Installation}
%
% \subsection{Download}
%
% \paragraph{Package.} This package is available on
% CTAN\footnote{\url{ftp://ftp.ctan.org/tex-archive/}}:
% \begin{description}
% \item[\CTAN{macros/latex/contrib/oberdiek/magicnum.dtx}] The source file.
% \item[\CTAN{macros/latex/contrib/oberdiek/magicnum.pdf}] Documentation.
% \end{description}
%
%
% \paragraph{Bundle.} All the packages of the bundle `oberdiek'
% are also available in a TDS compliant ZIP archive. There
% the packages are already unpacked and the documentation files
% are generated. The files and directories obey the TDS standard.
% \begin{description}
% \item[\CTAN{install/macros/latex/contrib/oberdiek.tds.zip}]
% \end{description}
% \emph{TDS} refers to the standard ``A Directory Structure
% for \TeX\ Files'' (\CTAN{tds/tds.pdf}). Directories
% with \xfile{texmf} in their name are usually organized this way.
%
% \subsection{Bundle installation}
%
% \paragraph{Unpacking.} Unpack the \xfile{oberdiek.tds.zip} in the
% TDS tree (also known as \xfile{texmf} tree) of your choice.
% Example (linux):
% \begin{quote}
%   |unzip oberdiek.tds.zip -d ~/texmf|
% \end{quote}
%
% \paragraph{Script installation.}
% Check the directory \xfile{TDS:scripts/oberdiek/} for
% scripts that need further installation steps.
% Package \xpackage{attachfile2} comes with the Perl script
% \xfile{pdfatfi.pl} that should be installed in such a way
% that it can be called as \texttt{pdfatfi}.
% Example (linux):
% \begin{quote}
%   |chmod +x scripts/oberdiek/pdfatfi.pl|\\
%   |cp scripts/oberdiek/pdfatfi.pl /usr/local/bin/|
% \end{quote}
%
% \subsection{Package installation}
%
% \paragraph{Unpacking.} The \xfile{.dtx} file is a self-extracting
% \docstrip\ archive. The files are extracted by running the
% \xfile{.dtx} through \plainTeX:
% \begin{quote}
%   \verb|tex magicnum.dtx|
% \end{quote}
%
% \paragraph{TDS.} Now the different files must be moved into
% the different directories in your installation TDS tree
% (also known as \xfile{texmf} tree):
% \begin{quote}
% \def\t{^^A
% \begin{tabular}{@{}>{\ttfamily}l@{ $\rightarrow$ }>{\ttfamily}l@{}}
%   magicnum.sty & tex/generic/oberdiek/magicnum.sty\\
%   magicnum.lua & scripts/oberdiek/magicnum.lua\\
%   oberdiek.magicnum.lua & scripts/oberdiek/oberdiek.magicnum.lua\\
%   magicnum.pdf & doc/latex/oberdiek/magicnum.pdf\\
%   magicnum.txt & doc/latex/oberdiek/magicnum.txt\\
%   test/magicnum-test1.tex & doc/latex/oberdiek/test/magicnum-test1.tex\\
%   test/magicnum-test2.tex & doc/latex/oberdiek/test/magicnum-test2.tex\\
%   test/magicnum-test3.tex & doc/latex/oberdiek/test/magicnum-test3.tex\\
%   test/magicnum-test4.tex & doc/latex/oberdiek/test/magicnum-test4.tex\\
%   magicnum.dtx & source/latex/oberdiek/magicnum.dtx\\
% \end{tabular}^^A
% }^^A
% \sbox0{\t}^^A
% \ifdim\wd0>\linewidth
%   \begingroup
%     \advance\linewidth by\leftmargin
%     \advance\linewidth by\rightmargin
%   \edef\x{\endgroup
%     \def\noexpand\lw{\the\linewidth}^^A
%   }\x
%   \def\lwbox{^^A
%     \leavevmode
%     \hbox to \linewidth{^^A
%       \kern-\leftmargin\relax
%       \hss
%       \usebox0
%       \hss
%       \kern-\rightmargin\relax
%     }^^A
%   }^^A
%   \ifdim\wd0>\lw
%     \sbox0{\small\t}^^A
%     \ifdim\wd0>\linewidth
%       \ifdim\wd0>\lw
%         \sbox0{\footnotesize\t}^^A
%         \ifdim\wd0>\linewidth
%           \ifdim\wd0>\lw
%             \sbox0{\scriptsize\t}^^A
%             \ifdim\wd0>\linewidth
%               \ifdim\wd0>\lw
%                 \sbox0{\tiny\t}^^A
%                 \ifdim\wd0>\linewidth
%                   \lwbox
%                 \else
%                   \usebox0
%                 \fi
%               \else
%                 \lwbox
%               \fi
%             \else
%               \usebox0
%             \fi
%           \else
%             \lwbox
%           \fi
%         \else
%           \usebox0
%         \fi
%       \else
%         \lwbox
%       \fi
%     \else
%       \usebox0
%     \fi
%   \else
%     \lwbox
%   \fi
% \else
%   \usebox0
% \fi
% \end{quote}
% If you have a \xfile{docstrip.cfg} that configures and enables \docstrip's
% TDS installing feature, then some files can already be in the right
% place, see the documentation of \docstrip.
%
% \subsection{Refresh file name databases}
%
% If your \TeX~distribution
% (\teTeX, \mikTeX, \dots) relies on file name databases, you must refresh
% these. For example, \teTeX\ users run \verb|texhash| or
% \verb|mktexlsr|.
%
% \subsection{Some details for the interested}
%
% \paragraph{Attached source.}
%
% The PDF documentation on CTAN also includes the
% \xfile{.dtx} source file. It can be extracted by
% AcrobatReader 6 or higher. Another option is \textsf{pdftk},
% e.g. unpack the file into the current directory:
% \begin{quote}
%   \verb|pdftk magicnum.pdf unpack_files output .|
% \end{quote}
%
% \paragraph{Unpacking with \LaTeX.}
% The \xfile{.dtx} chooses its action depending on the format:
% \begin{description}
% \item[\plainTeX:] Run \docstrip\ and extract the files.
% \item[\LaTeX:] Generate the documentation.
% \end{description}
% If you insist on using \LaTeX\ for \docstrip\ (really,
% \docstrip\ does not need \LaTeX), then inform the autodetect routine
% about your intention:
% \begin{quote}
%   \verb|latex \let\install=y\input{magicnum.dtx}|
% \end{quote}
% Do not forget to quote the argument according to the demands
% of your shell.
%
% \paragraph{Generating the documentation.}
% You can use both the \xfile{.dtx} or the \xfile{.drv} to generate
% the documentation. The process can be configured by the
% configuration file \xfile{ltxdoc.cfg}. For instance, put this
% line into this file, if you want to have A4 as paper format:
% \begin{quote}
%   \verb|\PassOptionsToClass{a4paper}{article}|
% \end{quote}
% An example follows how to generate the
% documentation with pdf\LaTeX:
% \begin{quote}
%\begin{verbatim}
%pdflatex magicnum.dtx
%makeindex -s gind.ist magicnum.idx
%pdflatex magicnum.dtx
%makeindex -s gind.ist magicnum.idx
%pdflatex magicnum.dtx
%\end{verbatim}
% \end{quote}
%
% \section{Catalogue}
%
% The following XML file can be used as source for the
% \href{http://mirror.ctan.org/help/Catalogue/catalogue.html}{\TeX\ Catalogue}.
% The elements \texttt{caption} and \texttt{description} are imported
% from the original XML file from the Catalogue.
% The name of the XML file in the Catalogue is \xfile{magicnum.xml}.
%    \begin{macrocode}
%<*catalogue>
<?xml version='1.0' encoding='us-ascii'?>
<!DOCTYPE entry SYSTEM 'catalogue.dtd'>
<entry datestamp='$Date$' modifier='$Author$' id='magicnum'>
  <name>magicnum</name>
  <caption>Access TeX systems' "magic numbers".</caption>
  <authorref id='auth:oberdiek'/>
  <copyright owner='Heiko Oberdiek' year='2007,2009-2011'/>
  <license type='lppl1.3'/>
  <version number='1.4'/>
  <description>
    This package allows access to the various parameter values in
    TeX (catcode values), e-TeX (group, if and node types, and
    interaction mode), and LuaTeX (pdfliteral mode) by a hierarchical
    name system.
    <p/>
    The package is part of the <xref refid='oberdiek'>oberdiek</xref> bundle.
  </description>
  <documentation details='Package documentation'
      href='ctan:/macros/latex/contrib/oberdiek/magicnum.pdf'/>
  <ctan file='true' path='/macros/latex/contrib/oberdiek/magicnum.dtx'/>
  <miktex location='oberdiek'/>
  <texlive location='oberdiek'/>
  <install path='/macros/latex/contrib/oberdiek/oberdiek.tds.zip'/>
</entry>
%</catalogue>
%    \end{macrocode}
%
% \begin{History}
%   \begin{Version}{2007/12/12 v1.0}
%   \item
%     First public version.
%   \end{Version}
%   \begin{Version}{2009/04/10 v1.1}
%   \item
%     Adaptation to \LuaTeX\ 0.40.
%   \end{Version}
%   \begin{Version}{2010/03/09 v1.2}
%   \item
%     Adaptation to package \xpackage{luatex} 0.4.
%   \end{Version}
%   \begin{Version}{2011/03/24 v1.3}
%   \item
%     Catcode fixes.
%   \end{Version}
%   \begin{Version}{2011/04/10 v1.4}
%   \item
%     Compatibility for \hologo{iniTeX}.
%   \item
%     Dependency from package \xpackage{luatex} removed.
%   \item
%     Version check for lua module.
%   \end{Version}
% \end{History}
%
% \PrintIndex
%
% \Finale
\endinput
|
% \end{quote}
% Do not forget to quote the argument according to the demands
% of your shell.
%
% \paragraph{Generating the documentation.}
% You can use both the \xfile{.dtx} or the \xfile{.drv} to generate
% the documentation. The process can be configured by the
% configuration file \xfile{ltxdoc.cfg}. For instance, put this
% line into this file, if you want to have A4 as paper format:
% \begin{quote}
%   \verb|\PassOptionsToClass{a4paper}{article}|
% \end{quote}
% An example follows how to generate the
% documentation with pdf\LaTeX:
% \begin{quote}
%\begin{verbatim}
%pdflatex magicnum.dtx
%makeindex -s gind.ist magicnum.idx
%pdflatex magicnum.dtx
%makeindex -s gind.ist magicnum.idx
%pdflatex magicnum.dtx
%\end{verbatim}
% \end{quote}
%
% \section{Catalogue}
%
% The following XML file can be used as source for the
% \href{http://mirror.ctan.org/help/Catalogue/catalogue.html}{\TeX\ Catalogue}.
% The elements \texttt{caption} and \texttt{description} are imported
% from the original XML file from the Catalogue.
% The name of the XML file in the Catalogue is \xfile{magicnum.xml}.
%    \begin{macrocode}
%<*catalogue>
<?xml version='1.0' encoding='us-ascii'?>
<!DOCTYPE entry SYSTEM 'catalogue.dtd'>
<entry datestamp='$Date$' modifier='$Author$' id='magicnum'>
  <name>magicnum</name>
  <caption>Access TeX systems' "magic numbers".</caption>
  <authorref id='auth:oberdiek'/>
  <copyright owner='Heiko Oberdiek' year='2007,2009-2011'/>
  <license type='lppl1.3'/>
  <version number='1.4'/>
  <description>
    This package allows access to the various parameter values in
    TeX (catcode values), e-TeX (group, if and node types, and
    interaction mode), and LuaTeX (pdfliteral mode) by a hierarchical
    name system.
    <p/>
    The package is part of the <xref refid='oberdiek'>oberdiek</xref> bundle.
  </description>
  <documentation details='Package documentation'
      href='ctan:/macros/latex/contrib/oberdiek/magicnum.pdf'/>
  <ctan file='true' path='/macros/latex/contrib/oberdiek/magicnum.dtx'/>
  <miktex location='oberdiek'/>
  <texlive location='oberdiek'/>
  <install path='/macros/latex/contrib/oberdiek/oberdiek.tds.zip'/>
</entry>
%</catalogue>
%    \end{macrocode}
%
% \begin{History}
%   \begin{Version}{2007/12/12 v1.0}
%   \item
%     First public version.
%   \end{Version}
%   \begin{Version}{2009/04/10 v1.1}
%   \item
%     Adaptation to \LuaTeX\ 0.40.
%   \end{Version}
%   \begin{Version}{2010/03/09 v1.2}
%   \item
%     Adaptation to package \xpackage{luatex} 0.4.
%   \end{Version}
%   \begin{Version}{2011/03/24 v1.3}
%   \item
%     Catcode fixes.
%   \end{Version}
%   \begin{Version}{2011/04/10 v1.4}
%   \item
%     Compatibility for \hologo{iniTeX}.
%   \item
%     Dependency from package \xpackage{luatex} removed.
%   \item
%     Version check for lua module.
%   \end{Version}
% \end{History}
%
% \PrintIndex
%
% \Finale
\endinput

%        (quote the arguments according to the demands of your shell)
%
% Documentation:
%    (a) If magicnum.drv is present:
%           latex magicnum.drv
%    (b) Without magicnum.drv:
%           latex magicnum.dtx; ...
%    The class ltxdoc loads the configuration file ltxdoc.cfg
%    if available. Here you can specify further options, e.g.
%    use A4 as paper format:
%       \PassOptionsToClass{a4paper}{article}
%
%    Programm calls to get the documentation (example):
%       pdflatex magicnum.dtx
%       makeindex -s gind.ist magicnum.idx
%       pdflatex magicnum.dtx
%       makeindex -s gind.ist magicnum.idx
%       pdflatex magicnum.dtx
%
% Installation:
%    TDS:tex/generic/oberdiek/magicnum.sty
%    TDS:scripts/oberdiek/magicnum.lua
%    TDS:scripts/oberdiek/oberdiek.magicnum.lua
%    TDS:doc/latex/oberdiek/magicnum.pdf
%    TDS:doc/latex/oberdiek/magicnum.txt
%    TDS:doc/latex/oberdiek/test/magicnum-test1.tex
%    TDS:doc/latex/oberdiek/test/magicnum-test2.tex
%    TDS:doc/latex/oberdiek/test/magicnum-test3.tex
%    TDS:doc/latex/oberdiek/test/magicnum-test4.tex
%    TDS:source/latex/oberdiek/magicnum.dtx
%
%<*ignore>
\begingroup
  \catcode123=1 %
  \catcode125=2 %
  \def\x{LaTeX2e}%
\expandafter\endgroup
\ifcase 0\ifx\install y1\fi\expandafter
         \ifx\csname processbatchFile\endcsname\relax\else1\fi
         \ifx\fmtname\x\else 1\fi\relax
\else\csname fi\endcsname
%</ignore>
%<*install>
\input docstrip.tex
\Msg{************************************************************************}
\Msg{* Installation}
\Msg{* Package: magicnum 2011/04/10 v1.4 Magic numbers (HO)}
\Msg{************************************************************************}

\keepsilent
\askforoverwritefalse

\let\MetaPrefix\relax
\preamble

This is a generated file.

Project: magicnum
Version: 2011/04/10 v1.4

Copyright (C) 2007, 2009-2011 by
   Heiko Oberdiek <heiko.oberdiek at googlemail.com>

This work may be distributed and/or modified under the
conditions of the LaTeX Project Public License, either
version 1.3c of this license or (at your option) any later
version. This version of this license is in
   http://www.latex-project.org/lppl/lppl-1-3c.txt
and the latest version of this license is in
   http://www.latex-project.org/lppl.txt
and version 1.3 or later is part of all distributions of
LaTeX version 2005/12/01 or later.

This work has the LPPL maintenance status "maintained".

This Current Maintainer of this work is Heiko Oberdiek.

The Base Interpreter refers to any `TeX-Format',
because some files are installed in TDS:tex/generic//.

This work consists of the main source file magicnum.dtx
and the derived files
   magicnum.sty, magicnum.pdf, magicnum.ins, magicnum.drv, magicnum.txt,
   magicnum-test1.tex, magicnum-test2.tex, magicnum-test3.tex,
   magicnum-test4.tex, magicnum.lua, oberdiek.magicnum.lua.

\endpreamble
\let\MetaPrefix\DoubleperCent

\generate{%
  \file{magicnum.ins}{\from{magicnum.dtx}{install}}%
  \file{magicnum.drv}{\from{magicnum.dtx}{driver}}%
  \usedir{tex/generic/oberdiek}%
  \file{magicnum.sty}{\from{magicnum.dtx}{package}}%
  \usedir{doc/latex/oberdiek/test}%
  \file{magicnum-test1.tex}{\from{magicnum.dtx}{test1}}%
  \file{magicnum-test2.tex}{\from{magicnum.dtx}{testplain,testdata}}%
  \file{magicnum-test3.tex}{\from{magicnum.dtx}{testlatex,testdata}}%
  \file{magicnum-test4.tex}{\from{magicnum.dtx}{test4}}%
  \nopreamble
  \nopostamble
  \usedir{doc/latex/oberdiek}%
  \file{magicnum.txt}{\from{magicnum.dtx}{data}}%
  \usedir{source/latex/oberdiek/catalogue}%
  \file{magicnum.xml}{\from{magicnum.dtx}{catalogue}}%
}
\def\MetaPrefix{-- }
\def\defaultpostamble{%
  \MetaPrefix^^J%
  \MetaPrefix\space End of File `\outFileName'.%
}
\def\currentpostamble{\defaultpostamble}%
\generate{%
  \usedir{scripts/oberdiek}%
  \file{magicnum.lua}{\from{magicnum.dtx}{lua}}%
  \file{oberdiek.magicnum.lua}{\from{magicnum.dtx}{lua}}%
}

\catcode32=13\relax% active space
\let =\space%
\Msg{************************************************************************}
\Msg{*}
\Msg{* To finish the installation you have to move the following}
\Msg{* file into a directory searched by TeX:}
\Msg{*}
\Msg{*     magicnum.sty}
\Msg{*}
\Msg{* And install the following script files:}
\Msg{*}
\Msg{*     magicnum.lua, oberdiek.magicnum.lua}
\Msg{*}
\Msg{* To produce the documentation run the file `magicnum.drv'}
\Msg{* through LaTeX.}
\Msg{*}
\Msg{* Happy TeXing!}
\Msg{*}
\Msg{************************************************************************}

\endbatchfile
%</install>
%<*ignore>
\fi
%</ignore>
%<*driver>
\NeedsTeXFormat{LaTeX2e}
\ProvidesFile{magicnum.drv}%
  [2011/04/10 v1.4 Magic numbers (HO)]%
\documentclass{ltxdoc}
\usepackage{holtxdoc}[2011/11/22]
\usepackage{array}
\begin{document}
  \DocInput{magicnum.dtx}%
\end{document}
%</driver>
% \fi
%
% \CheckSum{755}
%
% \CharacterTable
%  {Upper-case    \A\B\C\D\E\F\G\H\I\J\K\L\M\N\O\P\Q\R\S\T\U\V\W\X\Y\Z
%   Lower-case    \a\b\c\d\e\f\g\h\i\j\k\l\m\n\o\p\q\r\s\t\u\v\w\x\y\z
%   Digits        \0\1\2\3\4\5\6\7\8\9
%   Exclamation   \!     Double quote  \"     Hash (number) \#
%   Dollar        \$     Percent       \%     Ampersand     \&
%   Acute accent  \'     Left paren    \(     Right paren   \)
%   Asterisk      \*     Plus          \+     Comma         \,
%   Minus         \-     Point         \.     Solidus       \/
%   Colon         \:     Semicolon     \;     Less than     \<
%   Equals        \=     Greater than  \>     Question mark \?
%   Commercial at \@     Left bracket  \[     Backslash     \\
%   Right bracket \]     Circumflex    \^     Underscore    \_
%   Grave accent  \`     Left brace    \{     Vertical bar  \|
%   Right brace   \}     Tilde         \~}
%
% \GetFileInfo{magicnum.drv}
%
% \title{The \xpackage{magicnum} package}
% \date{2011/04/10 v1.4}
% \author{Heiko Oberdiek\\\xemail{heiko.oberdiek at googlemail.com}}
%
% \maketitle
%
% \begin{abstract}
% This packages allows to access magic numbers by a hierarchical
% name system.
% \end{abstract}
%
% \tableofcontents
%
% \hypersetup{bookmarksopenlevel=2}
% \section{Documentation}
%
% \subsection{Introduction}
%
% Especially since \eTeX\ there are many integer values
% with special meanings, such as catcodes, group types, \dots
% Package \xpackage{etex}, enabled by options, defines
% macros in the user namespace for these values.
%
% This package goes another approach for storing the names and values.
% \begin{itemize}
% \item If \LuaTeX\ is available, they
% are stored in Lua tables.
% \item Without \LuaTeX\ they are remembered using internal
% macros.
% \end{itemize}
%
% \subsection{User interface}
%
% The integer values and names are organized in a hierarchical
% scheme of categories with the property names as leaves.
% Example: \eTeX's \cs{currentgrouplevel} reports |2| for a
% group caused by \cs{hbox}. This package has choosen to organize
% the group types in a main category |etex| and its subcategory
% |grouptype|:
% \begin{quote}
%   |etex.grouptype.hbox| = |2|
% \end{quote}
% The property name |hbox| in category |etex.grouptype| has value |2|.
% Dots are used to separate components.
%
% If you want to have the value, the access key is constructed by
% the category with all its components and the property name.
% For the opposite the value is used instead of the property name.
%
% Values are always integers (including negative numbers).
%
% \subsubsection{\cs{magicnum}}
%
% \begin{declcs}{magicnum} \M{access key}
% \end{declcs}
% Macro \cs{magicnum} expects an access key as argument and
% expands to the requested data. The macro is always expandable.
% In case of errors the expansion result is empty.
%
% The same macro is also used for getting a property name.
% In this case the property name part in the access key is
% replaced by the value.
%
% The catcodes
% of the resulting numbers and strings follow \TeX's tradition of
% \cs{string}, \cs{meaning}, \dots: The space has catcode 10
% (|tex.catcode.space|) and the other characters have catcode
% 12 (|tex.catcode.other|).
%
% Examples:
% \begin{quote}
%   |\magicnum{etex.grouptype.hbox}| $\Rightarrow$ |2|\\
%   |\magicnum{tex.catcode.14}| $\Rightarrow$ |comment|\\
%   |\magicnum{tex.catcode.undefined}| $\Rightarrow$ $\emptyset$
% \end{quote}
%
% \subsubsection{Properties}
%
% \begin{itemize}
% \item The components of a category are either subcategories or
%       key value pairs, but not both.
% \item The full specified property names are unique and thus
%       has one integer value exactly.
% \item Also the values inside a category are unique.
%       This condition is a prerequisite for the reverse mapping
%       of \cs{magicnum}.
% \item All names start with a letter. Only letters or digits
%       may follow.
% \end{itemize}
%
% \subsection{Data}
%
%  \subsubsection{\texorpdfstring{Category }{}\texttt{tex.catcode}}
%
% \begin{quote}
% \begin{tabular}{@{}>{\ttfamily}l>{\ttfamily}l@{}}
%    tex.catcode.escape & 0\\
%    tex.catcode.begingroup & 1\\
%    tex.catcode.endgroup & 2\\
%    tex.catcode.math & 3\\
%    tex.catcode.align & 4\\
%    tex.catcode.eol & 5\\
%    tex.catcode.parameter & 6\\
%    tex.catcode.superscript & 7\\
%    tex.catcode.subscript & 8\\
%    tex.catcode.ignore & 9\\
%    tex.catcode.space & 10\\
%    tex.catcode.letter & 11\\
%    tex.catcode.other & 12\\
%    tex.catcode.active & 13\\
%    tex.catcode.comment & 14\\
%    tex.catcode.invalid & 15\\
%  \end{tabular}
%  \end{quote}
%
%  \subsubsection{\texorpdfstring{Category }{}\texttt{etex.grouptype}}
%
% \begin{quote}
% \begin{tabular}{@{}>{\ttfamily}l>{\ttfamily}l@{}}
%    etex.grouptype.bottomlevel & 0\\
%    etex.grouptype.simple & 1\\
%    etex.grouptype.hbox & 2\\
%    etex.grouptype.adjustedhbox & 3\\
%    etex.grouptype.vbox & 4\\
%    etex.grouptype.align & 5\\
%    etex.grouptype.noalign & 6\\
%    etex.grouptype.output & 8\\
%    etex.grouptype.math & 9\\
%    etex.grouptype.disc & 10\\
%    etex.grouptype.insert & 11\\
%    etex.grouptype.vcenter & 12\\
%    etex.grouptype.mathchoice & 13\\
%    etex.grouptype.semisimple & 14\\
%    etex.grouptype.mathshift & 15\\
%    etex.grouptype.mathleft & 16\\
%  \end{tabular}
%  \end{quote}
%
%  \subsubsection{\texorpdfstring{Category }{}\texttt{etex.iftype}}
%
% \begin{quote}
% \begin{tabular}{@{}>{\ttfamily}l>{\ttfamily}l@{}}
%    etex.iftype.none & 0\\
%    etex.iftype.char & 1\\
%    etex.iftype.cat & 2\\
%    etex.iftype.num & 3\\
%    etex.iftype.dim & 4\\
%    etex.iftype.odd & 5\\
%    etex.iftype.vmode & 6\\
%    etex.iftype.hmode & 7\\
%    etex.iftype.mmode & 8\\
%    etex.iftype.inner & 9\\
%    etex.iftype.void & 10\\
%    etex.iftype.hbox & 11\\
%    etex.iftype.vbox & 12\\
%    etex.iftype.x & 13\\
%    etex.iftype.eof & 14\\
%    etex.iftype.true & 15\\
%    etex.iftype.false & 16\\
%    etex.iftype.case & 17\\
%    etex.iftype.defined & 18\\
%    etex.iftype.csname & 19\\
%    etex.iftype.fontchar & 20\\
%  \end{tabular}
%  \end{quote}
%
%  \subsubsection{\texorpdfstring{Category }{}\texttt{etex.nodetype}}
%
% \begin{quote}
% \begin{tabular}{@{}>{\ttfamily}l>{\ttfamily}l@{}}
%    etex.nodetype.none & -1\\
%    etex.nodetype.char & 0\\
%    etex.nodetype.hlist & 1\\
%    etex.nodetype.vlist & 2\\
%    etex.nodetype.rule & 3\\
%    etex.nodetype.ins & 4\\
%    etex.nodetype.mark & 5\\
%    etex.nodetype.adjust & 6\\
%    etex.nodetype.ligature & 7\\
%    etex.nodetype.disc & 8\\
%    etex.nodetype.whatsit & 9\\
%    etex.nodetype.math & 10\\
%    etex.nodetype.glue & 11\\
%    etex.nodetype.kern & 12\\
%    etex.nodetype.penalty & 13\\
%    etex.nodetype.unset & 14\\
%    etex.nodetype.maths & 15\\
%  \end{tabular}
%  \end{quote}
%
%  \subsubsection{\texorpdfstring{Category }{}\texttt{etex.interactionmode}}
%
% \begin{quote}
% \begin{tabular}{@{}>{\ttfamily}l>{\ttfamily}l@{}}
%    etex.interactionmode.batch & 0\\
%    etex.interactionmode.nonstop & 1\\
%    etex.interactionmode.scroll & 2\\
%    etex.interactionmode.errorstop & 3\\
%  \end{tabular}
%  \end{quote}
%
%  \subsubsection{\texorpdfstring{Category }{}\texttt{luatex.pdfliteral.mode}}
%
% \begin{quote}
% \begin{tabular}{@{}>{\ttfamily}l>{\ttfamily}l@{}}
%    luatex.pdfliteral.mode.setorigin & 0\\
%    luatex.pdfliteral.mode.page & 1\\
%    luatex.pdfliteral.mode.direct & 2\\
%  \end{tabular}
%  \end{quote}
%
%
% \hypersetup{bookmarksopenlevel=1}
%
% \StopEventually{
% }
%
% \section{Implementation}
%
%    \begin{macrocode}
%<*package>
%    \end{macrocode}
%
% \subsection{Reload check and package identification}
%    Reload check, especially if the package is not used with \LaTeX.
%    \begin{macrocode}
\begingroup\catcode61\catcode48\catcode32=10\relax%
  \catcode13=5 % ^^M
  \endlinechar=13 %
  \catcode35=6 % #
  \catcode39=12 % '
  \catcode44=12 % ,
  \catcode45=12 % -
  \catcode46=12 % .
  \catcode58=12 % :
  \catcode64=11 % @
  \catcode123=1 % {
  \catcode125=2 % }
  \expandafter\let\expandafter\x\csname ver@magicnum.sty\endcsname
  \ifx\x\relax % plain-TeX, first loading
  \else
    \def\empty{}%
    \ifx\x\empty % LaTeX, first loading,
      % variable is initialized, but \ProvidesPackage not yet seen
    \else
      \expandafter\ifx\csname PackageInfo\endcsname\relax
        \def\x#1#2{%
          \immediate\write-1{Package #1 Info: #2.}%
        }%
      \else
        \def\x#1#2{\PackageInfo{#1}{#2, stopped}}%
      \fi
      \x{magicnum}{The package is already loaded}%
      \aftergroup\endinput
    \fi
  \fi
\endgroup%
%    \end{macrocode}
%    Package identification:
%    \begin{macrocode}
\begingroup\catcode61\catcode48\catcode32=10\relax%
  \catcode13=5 % ^^M
  \endlinechar=13 %
  \catcode35=6 % #
  \catcode39=12 % '
  \catcode40=12 % (
  \catcode41=12 % )
  \catcode44=12 % ,
  \catcode45=12 % -
  \catcode46=12 % .
  \catcode47=12 % /
  \catcode58=12 % :
  \catcode64=11 % @
  \catcode91=12 % [
  \catcode93=12 % ]
  \catcode123=1 % {
  \catcode125=2 % }
  \expandafter\ifx\csname ProvidesPackage\endcsname\relax
    \def\x#1#2#3[#4]{\endgroup
      \immediate\write-1{Package: #3 #4}%
      \xdef#1{#4}%
    }%
  \else
    \def\x#1#2[#3]{\endgroup
      #2[{#3}]%
      \ifx#1\@undefined
        \xdef#1{#3}%
      \fi
      \ifx#1\relax
        \xdef#1{#3}%
      \fi
    }%
  \fi
\expandafter\x\csname ver@magicnum.sty\endcsname
\ProvidesPackage{magicnum}%
  [2011/04/10 v1.4 Magic numbers (HO)]%
%    \end{macrocode}
%
% \subsection{Catcodes}
%
%    \begin{macrocode}
\begingroup\catcode61\catcode48\catcode32=10\relax%
  \catcode13=5 % ^^M
  \endlinechar=13 %
  \catcode123=1 % {
  \catcode125=2 % }
  \catcode64=11 % @
  \def\x{\endgroup
    \expandafter\edef\csname magicnum@AtEnd\endcsname{%
      \endlinechar=\the\endlinechar\relax
      \catcode13=\the\catcode13\relax
      \catcode32=\the\catcode32\relax
      \catcode35=\the\catcode35\relax
      \catcode61=\the\catcode61\relax
      \catcode64=\the\catcode64\relax
      \catcode123=\the\catcode123\relax
      \catcode125=\the\catcode125\relax
    }%
  }%
\x\catcode61\catcode48\catcode32=10\relax%
\catcode13=5 % ^^M
\endlinechar=13 %
\catcode35=6 % #
\catcode64=11 % @
\catcode123=1 % {
\catcode125=2 % }
\def\TMP@EnsureCode#1#2{%
  \edef\magicnum@AtEnd{%
    \magicnum@AtEnd
    \catcode#1=\the\catcode#1\relax
  }%
  \catcode#1=#2\relax
}
\TMP@EnsureCode{34}{12}% "
\TMP@EnsureCode{39}{12}% '
\TMP@EnsureCode{40}{12}% (
\TMP@EnsureCode{41}{12}% )
\TMP@EnsureCode{42}{12}% *
\TMP@EnsureCode{44}{12}% ,
\TMP@EnsureCode{45}{12}% -
\TMP@EnsureCode{46}{12}% .
\TMP@EnsureCode{47}{12}% /
\TMP@EnsureCode{58}{12}% :
\TMP@EnsureCode{60}{12}% <
\TMP@EnsureCode{62}{12}% >
\TMP@EnsureCode{91}{12}% [
\TMP@EnsureCode{93}{12}% ]
\edef\magicnum@AtEnd{\magicnum@AtEnd\noexpand\endinput}
%    \end{macrocode}
%
% \subsection{Check for previous definition}
%
%    \begin{macrocode}
\begingroup\expandafter\expandafter\expandafter\endgroup
\expandafter\ifx\csname newcommand\endcsname\relax
  \expandafter\ifx\csname magicnum\endcsname\relax
  \else
    \input infwarerr.sty\relax
    \@PackageError{magicnum}{%
      \string\magicnum\space is already defined%
    }\@ehc
  \fi
\else
  \newcommand*{\magicnum}{}%
\fi
%    \end{macrocode}
%
% \subsection{Without \LuaTeX}
%
%    \begin{macrocode}
\begingroup\expandafter\expandafter\expandafter\endgroup
\expandafter\ifx\csname directlua\endcsname\relax
%    \end{macrocode}
%
%    \begin{macro}{\magicnum}
%    \begin{macrocode}
  \begingroup\expandafter\expandafter\expandafter\endgroup
  \expandafter\ifx\csname ifcsname\endcsname\relax
    \def\magicnum#1{%
      \expandafter\ifx\csname MG@#1\endcsname\relax
      \else
        \csname MG@#1\endcsname
      \fi
    }%
  \else
    \begingroup
      \edef\x{\endgroup
        \def\noexpand\magicnum##1{%
          \expandafter\noexpand\csname
          ifcsname\endcsname MG@##1\noexpand\endcsname
            \noexpand\csname MG@##1%
                 \noexpand\expandafter\noexpand\endcsname
          \expandafter\noexpand\csname fi\endcsname
        }%
      }%
    \x
  \fi
%    \end{macrocode}
%    \end{macro}
%
%    \begin{macrocode}
\else
%    \end{macrocode}
%
% \subsection{With \LuaTeX}
%
%    \begin{macrocode}
  \begingroup\expandafter\expandafter\expandafter\endgroup
  \expandafter\ifx\csname RequirePackage\endcsname\relax
    \input ifluatex.sty\relax
    \input infwarerr.sty\relax
  \else
    \RequirePackage{ifluatex}[2010/03/01]%
    \RequirePackage{infwarerr}[2010/04/08]%
  \fi
%    \end{macrocode}
%
%    \begin{macro}{\magicnum@directlua}
%    \begin{macrocode}
  \ifnum\luatexversion<36 %
    \def\magicnum@directlua{\directlua0 }%
  \else
    \let\magicnum@directlua\directlua
  \fi
%    \end{macrocode}
%    \end{macro}
%    \begin{macrocode}
  \magicnum@directlua{%
    require("oberdiek.magicnum")%
  }%
  \begingroup
    \def\x{2011/04/10 v1.4}%
    \def\StripPrefix#1>{}%
    \edef\x{\expandafter\StripPrefix\meaning\x}%
    \edef\y{%
      \magicnum@directlua{%
        if oberdiek.magicnum.getversion then %
          oberdiek.magicnum.getversion()%
        end%
      }%
    }%
    \ifx\x\y
    \else
      \@PackageError{magicnum}{%
        Wrong version of lua module.\MessageBreak
        Package version: \x\MessageBreak
        Lua module: \y
      }\@ehc
    \fi
  \endgroup
%    \end{macrocode}
%    \begin{macro}{\luaescapestring}
%    \begin{macrocode}
  \begingroup
    \expandafter\ifx\csname luaescapestring\endcsname\relax
      \directlua{%
        if tex.enableprimitives then %
          tex.enableprimitives('magicnum@', {'luaescapestring'})%
        end%
      }%
      \global\let\luaescapestring\magicnum@luaescapestring
    \fi
    \expandafter\ifx\csname luaescapestring\endcsname\relax
      \escapechar=92 %
      \@PackageError{magicnum}{%
        Missing \string\luaescapestring
      }\@ehc
    \fi
  \endgroup
%    \end{macrocode}
%    \end{macro}
%    \begin{macro}{\magicnum}
%    \begin{macrocode}
  \def\magicnum#1{%
    \magicnum@directlua{%
      oberdiek.magicnum.get("\luaescapestring{#1}")%
    }%
  }%
%    \end{macrocode}
%    \end{macro}
%
%    \begin{macrocode}
  \expandafter\magicnum@AtEnd
\fi%
%</package>
%    \end{macrocode}
%
% \subsection{Data}
%
% \subsubsection{Plain data}
%
%    \begin{macrocode}
%<*data>
tex.catcode
  escape = 0
  begingroup = 1
  endgroup = 2
  math = 3
  align = 4
  eol = 5
  parameter = 6
  superscript = 7
  subscript = 8
  ignore = 9
  space = 10
  letter = 11
  other = 12
  active = 13
  comment = 14
  invalid = 15
etex.grouptype
  bottomlevel = 0
  simple = 1
  hbox = 2
  adjustedhbox = 3
  vbox = 4
  align = 5
  noalign = 6
  output = 8
  math = 9
  disc = 10
  insert = 11
  vcenter = 12
  mathchoice = 13
  semisimple = 14
  mathshift = 15
  mathleft = 16
etex.iftype
  none = 0
  char = 1
  cat = 2
  num = 3
  dim = 4
  odd = 5
  vmode = 6
  hmode = 7
  mmode = 8
  inner = 9
  void = 10
  hbox = 11
  vbox = 12
  x = 13
  eof = 14
  true = 15
  false = 16
  case = 17
  defined = 18
  csname = 19
  fontchar = 20
etex.nodetype
  none = -1
  char = 0
  hlist = 1
  vlist = 2
  rule = 3
  ins = 4
  mark = 5
  adjust = 6
  ligature = 7
  disc = 8
  whatsit = 9
  math = 10
  glue = 11
  kern = 12
  penalty = 13
  unset = 14
  maths = 15
etex.interactionmode
  batch = 0
  nonstop = 1
  scroll = 2
  errorstop = 3
luatex.pdfliteral.mode
  setorigin = 0
  page = 1
  direct = 2
%</data>
%    \end{macrocode}
%
% \subsubsection{Data for \TeX}
%
%    \begin{macrocode}
%<*package>
%    \end{macrocode}
%    \begin{macro}{\magicnum@add}
%    \begin{macrocode}
\begingroup\expandafter\expandafter\expandafter\endgroup
\expandafter\ifx\csname detokenize\endcsname\relax
  \def\magicnum@add#1#2#3{%
    \expandafter\magicnum@@add
        \csname MG@#1.#2\expandafter\endcsname
        \csname MG@#1.#3\endcsname
       {#3}{#2}%
  }%
  \def\magicnum@@add#1#2#3#4{%
    \def#1{#3}%
    \def#2{#4}%
    \edef#1{%
      \expandafter\strip@prefix\meaning#1%
    }%
    \edef#2{%
      \expandafter\strip@prefix\meaning#2%
    }%
  }%
  \expandafter\ifx\csname strip@prefix\endcsname\relax
    \def\strip@prefix#1->{}%
  \fi
\else
  \def\magicnum@add#1#2#3{%
    \expandafter\edef\csname MG@#1.#2\endcsname{%
      \detokenize{#3}%
    }%
    \expandafter\edef\csname MG@#1.#3\endcsname{%
      \detokenize{#2}%
    }%
  }%
\fi
%    \end{macrocode}
%    \end{macro}
%    \begin{macrocode}
\magicnum@add{tex.catcode}{escape}{0}
\magicnum@add{tex.catcode}{begingroup}{1}
\magicnum@add{tex.catcode}{endgroup}{2}
\magicnum@add{tex.catcode}{math}{3}
\magicnum@add{tex.catcode}{align}{4}
\magicnum@add{tex.catcode}{eol}{5}
\magicnum@add{tex.catcode}{parameter}{6}
\magicnum@add{tex.catcode}{superscript}{7}
\magicnum@add{tex.catcode}{subscript}{8}
\magicnum@add{tex.catcode}{ignore}{9}
\magicnum@add{tex.catcode}{space}{10}
\magicnum@add{tex.catcode}{letter}{11}
\magicnum@add{tex.catcode}{other}{12}
\magicnum@add{tex.catcode}{active}{13}
\magicnum@add{tex.catcode}{comment}{14}
\magicnum@add{tex.catcode}{invalid}{15}
\magicnum@add{etex.grouptype}{bottomlevel}{0}
\magicnum@add{etex.grouptype}{simple}{1}
\magicnum@add{etex.grouptype}{hbox}{2}
\magicnum@add{etex.grouptype}{adjustedhbox}{3}
\magicnum@add{etex.grouptype}{vbox}{4}
\magicnum@add{etex.grouptype}{align}{5}
\magicnum@add{etex.grouptype}{noalign}{6}
\magicnum@add{etex.grouptype}{output}{8}
\magicnum@add{etex.grouptype}{math}{9}
\magicnum@add{etex.grouptype}{disc}{10}
\magicnum@add{etex.grouptype}{insert}{11}
\magicnum@add{etex.grouptype}{vcenter}{12}
\magicnum@add{etex.grouptype}{mathchoice}{13}
\magicnum@add{etex.grouptype}{semisimple}{14}
\magicnum@add{etex.grouptype}{mathshift}{15}
\magicnum@add{etex.grouptype}{mathleft}{16}
\magicnum@add{etex.iftype}{none}{0}
\magicnum@add{etex.iftype}{char}{1}
\magicnum@add{etex.iftype}{cat}{2}
\magicnum@add{etex.iftype}{num}{3}
\magicnum@add{etex.iftype}{dim}{4}
\magicnum@add{etex.iftype}{odd}{5}
\magicnum@add{etex.iftype}{vmode}{6}
\magicnum@add{etex.iftype}{hmode}{7}
\magicnum@add{etex.iftype}{mmode}{8}
\magicnum@add{etex.iftype}{inner}{9}
\magicnum@add{etex.iftype}{void}{10}
\magicnum@add{etex.iftype}{hbox}{11}
\magicnum@add{etex.iftype}{vbox}{12}
\magicnum@add{etex.iftype}{x}{13}
\magicnum@add{etex.iftype}{eof}{14}
\magicnum@add{etex.iftype}{true}{15}
\magicnum@add{etex.iftype}{false}{16}
\magicnum@add{etex.iftype}{case}{17}
\magicnum@add{etex.iftype}{defined}{18}
\magicnum@add{etex.iftype}{csname}{19}
\magicnum@add{etex.iftype}{fontchar}{20}
\magicnum@add{etex.nodetype}{none}{-1}
\magicnum@add{etex.nodetype}{char}{0}
\magicnum@add{etex.nodetype}{hlist}{1}
\magicnum@add{etex.nodetype}{vlist}{2}
\magicnum@add{etex.nodetype}{rule}{3}
\magicnum@add{etex.nodetype}{ins}{4}
\magicnum@add{etex.nodetype}{mark}{5}
\magicnum@add{etex.nodetype}{adjust}{6}
\magicnum@add{etex.nodetype}{ligature}{7}
\magicnum@add{etex.nodetype}{disc}{8}
\magicnum@add{etex.nodetype}{whatsit}{9}
\magicnum@add{etex.nodetype}{math}{10}
\magicnum@add{etex.nodetype}{glue}{11}
\magicnum@add{etex.nodetype}{kern}{12}
\magicnum@add{etex.nodetype}{penalty}{13}
\magicnum@add{etex.nodetype}{unset}{14}
\magicnum@add{etex.nodetype}{maths}{15}
\magicnum@add{etex.interactionmode}{batch}{0}
\magicnum@add{etex.interactionmode}{nonstop}{1}
\magicnum@add{etex.interactionmode}{scroll}{2}
\magicnum@add{etex.interactionmode}{errorstop}{3}
\magicnum@add{luatex.pdfliteral.mode}{setorigin}{0}
\magicnum@add{luatex.pdfliteral.mode}{page}{1}
\magicnum@add{luatex.pdfliteral.mode}{direct}{2}
%    \end{macrocode}
%    \begin{macrocode}
\magicnum@AtEnd%
%</package>
%    \end{macrocode}
%
% \subsubsection{Lua module}
%
%    \begin{macrocode}
%<*lua>
%    \end{macrocode}
%    \begin{macrocode}
module("oberdiek.magicnum", package.seeall)
%    \end{macrocode}
%    \begin{macrocode}
function getversion()
  tex.write("2011/04/10 v1.4")
end
%    \end{macrocode}
%    \begin{macrocode}
local data = {
  ["tex.catcode"] = {
    [0] = "escape",
    [1] = "begingroup",
    [2] = "endgroup",
    [3] = "math",
    [4] = "align",
    [5] = "eol",
    [6] = "parameter",
    [7] = "superscript",
    [8] = "subscript",
    [9] = "ignore",
    [10] = "space",
    [11] = "letter",
    [12] = "other",
    [13] = "active",
    [14] = "comment",
    [15] = "invalid",
    ["active"] = 13,
    ["align"] = 4,
    ["begingroup"] = 1,
    ["comment"] = 14,
    ["endgroup"] = 2,
    ["eol"] = 5,
    ["escape"] = 0,
    ["ignore"] = 9,
    ["invalid"] = 15,
    ["letter"] = 11,
    ["math"] = 3,
    ["other"] = 12,
    ["parameter"] = 6,
    ["space"] = 10,
    ["subscript"] = 8,
    ["superscript"] = 7
  },
  ["etex.grouptype"] = {
    [0] = "bottomlevel",
    [1] = "simple",
    [2] = "hbox",
    [3] = "adjustedhbox",
    [4] = "vbox",
    [5] = "align",
    [6] = "noalign",
    [8] = "output",
    [9] = "math",
    [10] = "disc",
    [11] = "insert",
    [12] = "vcenter",
    [13] = "mathchoice",
    [14] = "semisimple",
    [15] = "mathshift",
    [16] = "mathleft",
    ["adjustedhbox"] = 3,
    ["align"] = 5,
    ["bottomlevel"] = 0,
    ["disc"] = 10,
    ["hbox"] = 2,
    ["insert"] = 11,
    ["math"] = 9,
    ["mathchoice"] = 13,
    ["mathleft"] = 16,
    ["mathshift"] = 15,
    ["noalign"] = 6,
    ["output"] = 8,
    ["semisimple"] = 14,
    ["simple"] = 1,
    ["vbox"] = 4,
    ["vcenter"] = 12
  },
  ["etex.iftype"] = {
    [0] = "none",
    [1] = "char",
    [2] = "cat",
    [3] = "num",
    [4] = "dim",
    [5] = "odd",
    [6] = "vmode",
    [7] = "hmode",
    [8] = "mmode",
    [9] = "inner",
    [10] = "void",
    [11] = "hbox",
    [12] = "vbox",
    [13] = "x",
    [14] = "eof",
    [15] = "true",
    [16] = "false",
    [17] = "case",
    [18] = "defined",
    [19] = "csname",
    [20] = "fontchar",
    ["case"] = 17,
    ["cat"] = 2,
    ["char"] = 1,
    ["csname"] = 19,
    ["defined"] = 18,
    ["dim"] = 4,
    ["eof"] = 14,
    ["false"] = 16,
    ["fontchar"] = 20,
    ["hbox"] = 11,
    ["hmode"] = 7,
    ["inner"] = 9,
    ["mmode"] = 8,
    ["none"] = 0,
    ["num"] = 3,
    ["odd"] = 5,
    ["true"] = 15,
    ["vbox"] = 12,
    ["vmode"] = 6,
    ["void"] = 10,
    ["x"] = 13
  },
  ["etex.nodetype"] = {
    [-1] = "none",
    [0] = "char",
    [1] = "hlist",
    [2] = "vlist",
    [3] = "rule",
    [4] = "ins",
    [5] = "mark",
    [6] = "adjust",
    [7] = "ligature",
    [8] = "disc",
    [9] = "whatsit",
    [10] = "math",
    [11] = "glue",
    [12] = "kern",
    [13] = "penalty",
    [14] = "unset",
    [15] = "maths",
    ["adjust"] = 6,
    ["char"] = 0,
    ["disc"] = 8,
    ["glue"] = 11,
    ["hlist"] = 1,
    ["ins"] = 4,
    ["kern"] = 12,
    ["ligature"] = 7,
    ["mark"] = 5,
    ["math"] = 10,
    ["maths"] = 15,
    ["none"] = -1,
    ["penalty"] = 13,
    ["rule"] = 3,
    ["unset"] = 14,
    ["vlist"] = 2,
    ["whatsit"] = 9
  },
  ["etex.interactionmode"] = {
    [0] = "batch",
    [1] = "nonstop",
    [2] = "scroll",
    [3] = "errorstop",
    ["batch"] = 0,
    ["errorstop"] = 3,
    ["nonstop"] = 1,
    ["scroll"] = 2
  },
  ["luatex.pdfliteral.mode"] = {
    [0] = "setorigin",
    [1] = "page",
    [2] = "direct",
    ["direct"] = 2,
    ["page"] = 1,
    ["setorigin"] = 0
  }
}
%    \end{macrocode}
%    \begin{macrocode}
function get(name)
  local startpos, endpos, category, entry =
      string.find(name, "^(%a[%a%d%.]*)%.(-?[%a%d]+)$")
  if not entry then
    return
  end
  local node = data[category]
  if not node then
    return
  end
  local num = tonumber(entry)
  local value
  if num then
    value = node[num]
    if not value then
      return
    end
  else
    value = node[entry]
    if not value then
      return
    end
    value = "" .. value
  end
  tex.write(value)
end
%    \end{macrocode}
%
%    \begin{macrocode}
%</lua>
%    \end{macrocode}
%
% \section{Test}
%
% \subsection{Catcode checks for loading}
%
%    \begin{macrocode}
%<*test1>
%    \end{macrocode}
%    \begin{macrocode}
\catcode`\{=1 %
\catcode`\}=2 %
\catcode`\#=6 %
\catcode`\@=11 %
\expandafter\ifx\csname count@\endcsname\relax
  \countdef\count@=255 %
\fi
\expandafter\ifx\csname @gobble\endcsname\relax
  \long\def\@gobble#1{}%
\fi
\expandafter\ifx\csname @firstofone\endcsname\relax
  \long\def\@firstofone#1{#1}%
\fi
\expandafter\ifx\csname loop\endcsname\relax
  \expandafter\@firstofone
\else
  \expandafter\@gobble
\fi
{%
  \def\loop#1\repeat{%
    \def\body{#1}%
    \iterate
  }%
  \def\iterate{%
    \body
      \let\next\iterate
    \else
      \let\next\relax
    \fi
    \next
  }%
  \let\repeat=\fi
}%
\def\RestoreCatcodes{}
\count@=0 %
\loop
  \edef\RestoreCatcodes{%
    \RestoreCatcodes
    \catcode\the\count@=\the\catcode\count@\relax
  }%
\ifnum\count@<255 %
  \advance\count@ 1 %
\repeat

\def\RangeCatcodeInvalid#1#2{%
  \count@=#1\relax
  \loop
    \catcode\count@=15 %
  \ifnum\count@<#2\relax
    \advance\count@ 1 %
  \repeat
}
\def\RangeCatcodeCheck#1#2#3{%
  \count@=#1\relax
  \loop
    \ifnum#3=\catcode\count@
    \else
      \errmessage{%
        Character \the\count@\space
        with wrong catcode \the\catcode\count@\space
        instead of \number#3%
      }%
    \fi
  \ifnum\count@<#2\relax
    \advance\count@ 1 %
  \repeat
}
\def\space{ }
\expandafter\ifx\csname LoadCommand\endcsname\relax
  \def\LoadCommand{\input magicnum.sty\relax}%
\fi
\def\Test{%
  \RangeCatcodeInvalid{0}{47}%
  \RangeCatcodeInvalid{58}{64}%
  \RangeCatcodeInvalid{91}{96}%
  \RangeCatcodeInvalid{123}{255}%
  \catcode`\@=12 %
  \catcode`\\=0 %
  \catcode`\%=14 %
  \LoadCommand
  \RangeCatcodeCheck{0}{36}{15}%
  \RangeCatcodeCheck{37}{37}{14}%
  \RangeCatcodeCheck{38}{47}{15}%
  \RangeCatcodeCheck{48}{57}{12}%
  \RangeCatcodeCheck{58}{63}{15}%
  \RangeCatcodeCheck{64}{64}{12}%
  \RangeCatcodeCheck{65}{90}{11}%
  \RangeCatcodeCheck{91}{91}{15}%
  \RangeCatcodeCheck{92}{92}{0}%
  \RangeCatcodeCheck{93}{96}{15}%
  \RangeCatcodeCheck{97}{122}{11}%
  \RangeCatcodeCheck{123}{255}{15}%
  \RestoreCatcodes
}
\Test
\csname @@end\endcsname
\end
%    \end{macrocode}
%    \begin{macrocode}
%</test1>
%    \end{macrocode}
%
% \subsection{Test data}
%
%    \begin{macrocode}
%<*testplain>
\input magicnum.sty\relax
\def\Test#1#2{%
  \edef\result{\magicnum{#1}}%
  \edef\expect{#2}%
  \edef\expect{\expandafter\stripprefix\meaning\expect}%
  \ifx\result\expect
  \else
    \errmessage{%
      Failed: [#1] % hash-ok
      returns [\result] instead of [\expect]%
    }%
  \fi
}
\def\stripprefix#1->{}
%</testplain>
%    \end{macrocode}
%    \begin{macrocode}
%<*testlatex>
\NeedsTeXFormat{LaTeX2e}
\documentclass{minimal}
\usepackage{magicnum}[2011/04/10]
\usepackage{qstest}
\IncludeTests{*}
\LogTests{log}{*}{*}
\newcommand*{\Test}[2]{%
  \Expect*{\magicnum{#1}}{#2}%
}
\begin{qstest}{magicnum}{magicnum}
%</testlatex>
%    \end{macrocode}
%    \begin{macrocode}
%<*testdata>
\Test{tex.catcode.escape}{0}
\Test{tex.catcode.invalid}{15}
\Test{tex.catcode.unknown}{}
\Test{tex.catcode.0}{escape}
\Test{tex.catcode.15}{invalid}
\Test{etex.iftype.true}{15}
\Test{etex.iftype.false}{16}
\Test{etex.iftype.15}{true}
\Test{etex.iftype.16}{false}
\Test{etex.nodetype.none}{-1}
\Test{etex.nodetype.-1}{none}
\Test{luatex.pdfliteral.mode.direct}{2}
\Test{luatex.pdfliteral.mode.1}{page}
\Test{}{}
\Test{unknown}{}
\Test{unknown.foo.bar}{}
\Test{unknown.foo.4}{}
%</testdata>
%    \end{macrocode}
%    \begin{macrocode}
%<*testplain>
\csname @@end\endcsname
\end
%</testplain>
%<*testlatex>
\end{qstest}
\csname @@end\endcsname
%</testlatex>
%    \end{macrocode}
%
% \subsection{Small test for \hologo{iniTeX}}
%
%    \begin{macrocode}
%<*test4>
\catcode`\{=1
\catcode`\}=2
\catcode`\#=6
\input magicnum.sty\relax
\edef\x{\magicnum{tex.catcode.15}}
\edef\y{invalid}
\def\Strip#1>{}
\edef\y{\expandafter\Strip\meaning\y}
\ifx\x\y
  \immediate\write16{Ok}%
\else
  \errmessage{\x<>\y}%
\fi
\csname @@end\endcsname\end
%</test4>
%    \end{macrocode}
%
% \section{Installation}
%
% \subsection{Download}
%
% \paragraph{Package.} This package is available on
% CTAN\footnote{\url{ftp://ftp.ctan.org/tex-archive/}}:
% \begin{description}
% \item[\CTAN{macros/latex/contrib/oberdiek/magicnum.dtx}] The source file.
% \item[\CTAN{macros/latex/contrib/oberdiek/magicnum.pdf}] Documentation.
% \end{description}
%
%
% \paragraph{Bundle.} All the packages of the bundle `oberdiek'
% are also available in a TDS compliant ZIP archive. There
% the packages are already unpacked and the documentation files
% are generated. The files and directories obey the TDS standard.
% \begin{description}
% \item[\CTAN{install/macros/latex/contrib/oberdiek.tds.zip}]
% \end{description}
% \emph{TDS} refers to the standard ``A Directory Structure
% for \TeX\ Files'' (\CTAN{tds/tds.pdf}). Directories
% with \xfile{texmf} in their name are usually organized this way.
%
% \subsection{Bundle installation}
%
% \paragraph{Unpacking.} Unpack the \xfile{oberdiek.tds.zip} in the
% TDS tree (also known as \xfile{texmf} tree) of your choice.
% Example (linux):
% \begin{quote}
%   |unzip oberdiek.tds.zip -d ~/texmf|
% \end{quote}
%
% \paragraph{Script installation.}
% Check the directory \xfile{TDS:scripts/oberdiek/} for
% scripts that need further installation steps.
% Package \xpackage{attachfile2} comes with the Perl script
% \xfile{pdfatfi.pl} that should be installed in such a way
% that it can be called as \texttt{pdfatfi}.
% Example (linux):
% \begin{quote}
%   |chmod +x scripts/oberdiek/pdfatfi.pl|\\
%   |cp scripts/oberdiek/pdfatfi.pl /usr/local/bin/|
% \end{quote}
%
% \subsection{Package installation}
%
% \paragraph{Unpacking.} The \xfile{.dtx} file is a self-extracting
% \docstrip\ archive. The files are extracted by running the
% \xfile{.dtx} through \plainTeX:
% \begin{quote}
%   \verb|tex magicnum.dtx|
% \end{quote}
%
% \paragraph{TDS.} Now the different files must be moved into
% the different directories in your installation TDS tree
% (also known as \xfile{texmf} tree):
% \begin{quote}
% \def\t{^^A
% \begin{tabular}{@{}>{\ttfamily}l@{ $\rightarrow$ }>{\ttfamily}l@{}}
%   magicnum.sty & tex/generic/oberdiek/magicnum.sty\\
%   magicnum.lua & scripts/oberdiek/magicnum.lua\\
%   oberdiek.magicnum.lua & scripts/oberdiek/oberdiek.magicnum.lua\\
%   magicnum.pdf & doc/latex/oberdiek/magicnum.pdf\\
%   magicnum.txt & doc/latex/oberdiek/magicnum.txt\\
%   test/magicnum-test1.tex & doc/latex/oberdiek/test/magicnum-test1.tex\\
%   test/magicnum-test2.tex & doc/latex/oberdiek/test/magicnum-test2.tex\\
%   test/magicnum-test3.tex & doc/latex/oberdiek/test/magicnum-test3.tex\\
%   test/magicnum-test4.tex & doc/latex/oberdiek/test/magicnum-test4.tex\\
%   magicnum.dtx & source/latex/oberdiek/magicnum.dtx\\
% \end{tabular}^^A
% }^^A
% \sbox0{\t}^^A
% \ifdim\wd0>\linewidth
%   \begingroup
%     \advance\linewidth by\leftmargin
%     \advance\linewidth by\rightmargin
%   \edef\x{\endgroup
%     \def\noexpand\lw{\the\linewidth}^^A
%   }\x
%   \def\lwbox{^^A
%     \leavevmode
%     \hbox to \linewidth{^^A
%       \kern-\leftmargin\relax
%       \hss
%       \usebox0
%       \hss
%       \kern-\rightmargin\relax
%     }^^A
%   }^^A
%   \ifdim\wd0>\lw
%     \sbox0{\small\t}^^A
%     \ifdim\wd0>\linewidth
%       \ifdim\wd0>\lw
%         \sbox0{\footnotesize\t}^^A
%         \ifdim\wd0>\linewidth
%           \ifdim\wd0>\lw
%             \sbox0{\scriptsize\t}^^A
%             \ifdim\wd0>\linewidth
%               \ifdim\wd0>\lw
%                 \sbox0{\tiny\t}^^A
%                 \ifdim\wd0>\linewidth
%                   \lwbox
%                 \else
%                   \usebox0
%                 \fi
%               \else
%                 \lwbox
%               \fi
%             \else
%               \usebox0
%             \fi
%           \else
%             \lwbox
%           \fi
%         \else
%           \usebox0
%         \fi
%       \else
%         \lwbox
%       \fi
%     \else
%       \usebox0
%     \fi
%   \else
%     \lwbox
%   \fi
% \else
%   \usebox0
% \fi
% \end{quote}
% If you have a \xfile{docstrip.cfg} that configures and enables \docstrip's
% TDS installing feature, then some files can already be in the right
% place, see the documentation of \docstrip.
%
% \subsection{Refresh file name databases}
%
% If your \TeX~distribution
% (\teTeX, \mikTeX, \dots) relies on file name databases, you must refresh
% these. For example, \teTeX\ users run \verb|texhash| or
% \verb|mktexlsr|.
%
% \subsection{Some details for the interested}
%
% \paragraph{Attached source.}
%
% The PDF documentation on CTAN also includes the
% \xfile{.dtx} source file. It can be extracted by
% AcrobatReader 6 or higher. Another option is \textsf{pdftk},
% e.g. unpack the file into the current directory:
% \begin{quote}
%   \verb|pdftk magicnum.pdf unpack_files output .|
% \end{quote}
%
% \paragraph{Unpacking with \LaTeX.}
% The \xfile{.dtx} chooses its action depending on the format:
% \begin{description}
% \item[\plainTeX:] Run \docstrip\ and extract the files.
% \item[\LaTeX:] Generate the documentation.
% \end{description}
% If you insist on using \LaTeX\ for \docstrip\ (really,
% \docstrip\ does not need \LaTeX), then inform the autodetect routine
% about your intention:
% \begin{quote}
%   \verb|latex \let\install=y% \iffalse meta-comment
%
% File: magicnum.dtx
% Version: 2011/04/10 v1.4
% Info: Magic numbers
%
% Copyright (C) 2007, 2009-2011 by
%    Heiko Oberdiek <heiko.oberdiek at googlemail.com>
%
% This work may be distributed and/or modified under the
% conditions of the LaTeX Project Public License, either
% version 1.3c of this license or (at your option) any later
% version. This version of this license is in
%    http://www.latex-project.org/lppl/lppl-1-3c.txt
% and the latest version of this license is in
%    http://www.latex-project.org/lppl.txt
% and version 1.3 or later is part of all distributions of
% LaTeX version 2005/12/01 or later.
%
% This work has the LPPL maintenance status "maintained".
%
% This Current Maintainer of this work is Heiko Oberdiek.
%
% The Base Interpreter refers to any `TeX-Format',
% because some files are installed in TDS:tex/generic//.
%
% This work consists of the main source file magicnum.dtx
% and the derived files
%    magicnum.sty, magicnum.pdf, magicnum.ins, magicnum.drv, magicnum.txt,
%    magicnum-test1.tex, magicnum-test2.tex, magicnum-test3.tex,
%    magicnum-test4.tex, magicnum.lua, oberdiek.magicnum.lua.
%
% Distribution:
%    CTAN:macros/latex/contrib/oberdiek/magicnum.dtx
%    CTAN:macros/latex/contrib/oberdiek/magicnum.pdf
%
% Unpacking:
%    (a) If magicnum.ins is present:
%           tex magicnum.ins
%    (b) Without magicnum.ins:
%           tex magicnum.dtx
%    (c) If you insist on using LaTeX
%           latex \let\install=y% \iffalse meta-comment
%
% File: magicnum.dtx
% Version: 2011/04/10 v1.4
% Info: Magic numbers
%
% Copyright (C) 2007, 2009-2011 by
%    Heiko Oberdiek <heiko.oberdiek at googlemail.com>
%
% This work may be distributed and/or modified under the
% conditions of the LaTeX Project Public License, either
% version 1.3c of this license or (at your option) any later
% version. This version of this license is in
%    http://www.latex-project.org/lppl/lppl-1-3c.txt
% and the latest version of this license is in
%    http://www.latex-project.org/lppl.txt
% and version 1.3 or later is part of all distributions of
% LaTeX version 2005/12/01 or later.
%
% This work has the LPPL maintenance status "maintained".
%
% This Current Maintainer of this work is Heiko Oberdiek.
%
% The Base Interpreter refers to any `TeX-Format',
% because some files are installed in TDS:tex/generic//.
%
% This work consists of the main source file magicnum.dtx
% and the derived files
%    magicnum.sty, magicnum.pdf, magicnum.ins, magicnum.drv, magicnum.txt,
%    magicnum-test1.tex, magicnum-test2.tex, magicnum-test3.tex,
%    magicnum-test4.tex, magicnum.lua, oberdiek.magicnum.lua.
%
% Distribution:
%    CTAN:macros/latex/contrib/oberdiek/magicnum.dtx
%    CTAN:macros/latex/contrib/oberdiek/magicnum.pdf
%
% Unpacking:
%    (a) If magicnum.ins is present:
%           tex magicnum.ins
%    (b) Without magicnum.ins:
%           tex magicnum.dtx
%    (c) If you insist on using LaTeX
%           latex \let\install=y\input{magicnum.dtx}
%        (quote the arguments according to the demands of your shell)
%
% Documentation:
%    (a) If magicnum.drv is present:
%           latex magicnum.drv
%    (b) Without magicnum.drv:
%           latex magicnum.dtx; ...
%    The class ltxdoc loads the configuration file ltxdoc.cfg
%    if available. Here you can specify further options, e.g.
%    use A4 as paper format:
%       \PassOptionsToClass{a4paper}{article}
%
%    Programm calls to get the documentation (example):
%       pdflatex magicnum.dtx
%       makeindex -s gind.ist magicnum.idx
%       pdflatex magicnum.dtx
%       makeindex -s gind.ist magicnum.idx
%       pdflatex magicnum.dtx
%
% Installation:
%    TDS:tex/generic/oberdiek/magicnum.sty
%    TDS:scripts/oberdiek/magicnum.lua
%    TDS:scripts/oberdiek/oberdiek.magicnum.lua
%    TDS:doc/latex/oberdiek/magicnum.pdf
%    TDS:doc/latex/oberdiek/magicnum.txt
%    TDS:doc/latex/oberdiek/test/magicnum-test1.tex
%    TDS:doc/latex/oberdiek/test/magicnum-test2.tex
%    TDS:doc/latex/oberdiek/test/magicnum-test3.tex
%    TDS:doc/latex/oberdiek/test/magicnum-test4.tex
%    TDS:source/latex/oberdiek/magicnum.dtx
%
%<*ignore>
\begingroup
  \catcode123=1 %
  \catcode125=2 %
  \def\x{LaTeX2e}%
\expandafter\endgroup
\ifcase 0\ifx\install y1\fi\expandafter
         \ifx\csname processbatchFile\endcsname\relax\else1\fi
         \ifx\fmtname\x\else 1\fi\relax
\else\csname fi\endcsname
%</ignore>
%<*install>
\input docstrip.tex
\Msg{************************************************************************}
\Msg{* Installation}
\Msg{* Package: magicnum 2011/04/10 v1.4 Magic numbers (HO)}
\Msg{************************************************************************}

\keepsilent
\askforoverwritefalse

\let\MetaPrefix\relax
\preamble

This is a generated file.

Project: magicnum
Version: 2011/04/10 v1.4

Copyright (C) 2007, 2009-2011 by
   Heiko Oberdiek <heiko.oberdiek at googlemail.com>

This work may be distributed and/or modified under the
conditions of the LaTeX Project Public License, either
version 1.3c of this license or (at your option) any later
version. This version of this license is in
   http://www.latex-project.org/lppl/lppl-1-3c.txt
and the latest version of this license is in
   http://www.latex-project.org/lppl.txt
and version 1.3 or later is part of all distributions of
LaTeX version 2005/12/01 or later.

This work has the LPPL maintenance status "maintained".

This Current Maintainer of this work is Heiko Oberdiek.

The Base Interpreter refers to any `TeX-Format',
because some files are installed in TDS:tex/generic//.

This work consists of the main source file magicnum.dtx
and the derived files
   magicnum.sty, magicnum.pdf, magicnum.ins, magicnum.drv, magicnum.txt,
   magicnum-test1.tex, magicnum-test2.tex, magicnum-test3.tex,
   magicnum-test4.tex, magicnum.lua, oberdiek.magicnum.lua.

\endpreamble
\let\MetaPrefix\DoubleperCent

\generate{%
  \file{magicnum.ins}{\from{magicnum.dtx}{install}}%
  \file{magicnum.drv}{\from{magicnum.dtx}{driver}}%
  \usedir{tex/generic/oberdiek}%
  \file{magicnum.sty}{\from{magicnum.dtx}{package}}%
  \usedir{doc/latex/oberdiek/test}%
  \file{magicnum-test1.tex}{\from{magicnum.dtx}{test1}}%
  \file{magicnum-test2.tex}{\from{magicnum.dtx}{testplain,testdata}}%
  \file{magicnum-test3.tex}{\from{magicnum.dtx}{testlatex,testdata}}%
  \file{magicnum-test4.tex}{\from{magicnum.dtx}{test4}}%
  \nopreamble
  \nopostamble
  \usedir{doc/latex/oberdiek}%
  \file{magicnum.txt}{\from{magicnum.dtx}{data}}%
  \usedir{source/latex/oberdiek/catalogue}%
  \file{magicnum.xml}{\from{magicnum.dtx}{catalogue}}%
}
\def\MetaPrefix{-- }
\def\defaultpostamble{%
  \MetaPrefix^^J%
  \MetaPrefix\space End of File `\outFileName'.%
}
\def\currentpostamble{\defaultpostamble}%
\generate{%
  \usedir{scripts/oberdiek}%
  \file{magicnum.lua}{\from{magicnum.dtx}{lua}}%
  \file{oberdiek.magicnum.lua}{\from{magicnum.dtx}{lua}}%
}

\catcode32=13\relax% active space
\let =\space%
\Msg{************************************************************************}
\Msg{*}
\Msg{* To finish the installation you have to move the following}
\Msg{* file into a directory searched by TeX:}
\Msg{*}
\Msg{*     magicnum.sty}
\Msg{*}
\Msg{* And install the following script files:}
\Msg{*}
\Msg{*     magicnum.lua, oberdiek.magicnum.lua}
\Msg{*}
\Msg{* To produce the documentation run the file `magicnum.drv'}
\Msg{* through LaTeX.}
\Msg{*}
\Msg{* Happy TeXing!}
\Msg{*}
\Msg{************************************************************************}

\endbatchfile
%</install>
%<*ignore>
\fi
%</ignore>
%<*driver>
\NeedsTeXFormat{LaTeX2e}
\ProvidesFile{magicnum.drv}%
  [2011/04/10 v1.4 Magic numbers (HO)]%
\documentclass{ltxdoc}
\usepackage{holtxdoc}[2011/11/22]
\usepackage{array}
\begin{document}
  \DocInput{magicnum.dtx}%
\end{document}
%</driver>
% \fi
%
% \CheckSum{755}
%
% \CharacterTable
%  {Upper-case    \A\B\C\D\E\F\G\H\I\J\K\L\M\N\O\P\Q\R\S\T\U\V\W\X\Y\Z
%   Lower-case    \a\b\c\d\e\f\g\h\i\j\k\l\m\n\o\p\q\r\s\t\u\v\w\x\y\z
%   Digits        \0\1\2\3\4\5\6\7\8\9
%   Exclamation   \!     Double quote  \"     Hash (number) \#
%   Dollar        \$     Percent       \%     Ampersand     \&
%   Acute accent  \'     Left paren    \(     Right paren   \)
%   Asterisk      \*     Plus          \+     Comma         \,
%   Minus         \-     Point         \.     Solidus       \/
%   Colon         \:     Semicolon     \;     Less than     \<
%   Equals        \=     Greater than  \>     Question mark \?
%   Commercial at \@     Left bracket  \[     Backslash     \\
%   Right bracket \]     Circumflex    \^     Underscore    \_
%   Grave accent  \`     Left brace    \{     Vertical bar  \|
%   Right brace   \}     Tilde         \~}
%
% \GetFileInfo{magicnum.drv}
%
% \title{The \xpackage{magicnum} package}
% \date{2011/04/10 v1.4}
% \author{Heiko Oberdiek\\\xemail{heiko.oberdiek at googlemail.com}}
%
% \maketitle
%
% \begin{abstract}
% This packages allows to access magic numbers by a hierarchical
% name system.
% \end{abstract}
%
% \tableofcontents
%
% \hypersetup{bookmarksopenlevel=2}
% \section{Documentation}
%
% \subsection{Introduction}
%
% Especially since \eTeX\ there are many integer values
% with special meanings, such as catcodes, group types, \dots
% Package \xpackage{etex}, enabled by options, defines
% macros in the user namespace for these values.
%
% This package goes another approach for storing the names and values.
% \begin{itemize}
% \item If \LuaTeX\ is available, they
% are stored in Lua tables.
% \item Without \LuaTeX\ they are remembered using internal
% macros.
% \end{itemize}
%
% \subsection{User interface}
%
% The integer values and names are organized in a hierarchical
% scheme of categories with the property names as leaves.
% Example: \eTeX's \cs{currentgrouplevel} reports |2| for a
% group caused by \cs{hbox}. This package has choosen to organize
% the group types in a main category |etex| and its subcategory
% |grouptype|:
% \begin{quote}
%   |etex.grouptype.hbox| = |2|
% \end{quote}
% The property name |hbox| in category |etex.grouptype| has value |2|.
% Dots are used to separate components.
%
% If you want to have the value, the access key is constructed by
% the category with all its components and the property name.
% For the opposite the value is used instead of the property name.
%
% Values are always integers (including negative numbers).
%
% \subsubsection{\cs{magicnum}}
%
% \begin{declcs}{magicnum} \M{access key}
% \end{declcs}
% Macro \cs{magicnum} expects an access key as argument and
% expands to the requested data. The macro is always expandable.
% In case of errors the expansion result is empty.
%
% The same macro is also used for getting a property name.
% In this case the property name part in the access key is
% replaced by the value.
%
% The catcodes
% of the resulting numbers and strings follow \TeX's tradition of
% \cs{string}, \cs{meaning}, \dots: The space has catcode 10
% (|tex.catcode.space|) and the other characters have catcode
% 12 (|tex.catcode.other|).
%
% Examples:
% \begin{quote}
%   |\magicnum{etex.grouptype.hbox}| $\Rightarrow$ |2|\\
%   |\magicnum{tex.catcode.14}| $\Rightarrow$ |comment|\\
%   |\magicnum{tex.catcode.undefined}| $\Rightarrow$ $\emptyset$
% \end{quote}
%
% \subsubsection{Properties}
%
% \begin{itemize}
% \item The components of a category are either subcategories or
%       key value pairs, but not both.
% \item The full specified property names are unique and thus
%       has one integer value exactly.
% \item Also the values inside a category are unique.
%       This condition is a prerequisite for the reverse mapping
%       of \cs{magicnum}.
% \item All names start with a letter. Only letters or digits
%       may follow.
% \end{itemize}
%
% \subsection{Data}
%
%  \subsubsection{\texorpdfstring{Category }{}\texttt{tex.catcode}}
%
% \begin{quote}
% \begin{tabular}{@{}>{\ttfamily}l>{\ttfamily}l@{}}
%    tex.catcode.escape & 0\\
%    tex.catcode.begingroup & 1\\
%    tex.catcode.endgroup & 2\\
%    tex.catcode.math & 3\\
%    tex.catcode.align & 4\\
%    tex.catcode.eol & 5\\
%    tex.catcode.parameter & 6\\
%    tex.catcode.superscript & 7\\
%    tex.catcode.subscript & 8\\
%    tex.catcode.ignore & 9\\
%    tex.catcode.space & 10\\
%    tex.catcode.letter & 11\\
%    tex.catcode.other & 12\\
%    tex.catcode.active & 13\\
%    tex.catcode.comment & 14\\
%    tex.catcode.invalid & 15\\
%  \end{tabular}
%  \end{quote}
%
%  \subsubsection{\texorpdfstring{Category }{}\texttt{etex.grouptype}}
%
% \begin{quote}
% \begin{tabular}{@{}>{\ttfamily}l>{\ttfamily}l@{}}
%    etex.grouptype.bottomlevel & 0\\
%    etex.grouptype.simple & 1\\
%    etex.grouptype.hbox & 2\\
%    etex.grouptype.adjustedhbox & 3\\
%    etex.grouptype.vbox & 4\\
%    etex.grouptype.align & 5\\
%    etex.grouptype.noalign & 6\\
%    etex.grouptype.output & 8\\
%    etex.grouptype.math & 9\\
%    etex.grouptype.disc & 10\\
%    etex.grouptype.insert & 11\\
%    etex.grouptype.vcenter & 12\\
%    etex.grouptype.mathchoice & 13\\
%    etex.grouptype.semisimple & 14\\
%    etex.grouptype.mathshift & 15\\
%    etex.grouptype.mathleft & 16\\
%  \end{tabular}
%  \end{quote}
%
%  \subsubsection{\texorpdfstring{Category }{}\texttt{etex.iftype}}
%
% \begin{quote}
% \begin{tabular}{@{}>{\ttfamily}l>{\ttfamily}l@{}}
%    etex.iftype.none & 0\\
%    etex.iftype.char & 1\\
%    etex.iftype.cat & 2\\
%    etex.iftype.num & 3\\
%    etex.iftype.dim & 4\\
%    etex.iftype.odd & 5\\
%    etex.iftype.vmode & 6\\
%    etex.iftype.hmode & 7\\
%    etex.iftype.mmode & 8\\
%    etex.iftype.inner & 9\\
%    etex.iftype.void & 10\\
%    etex.iftype.hbox & 11\\
%    etex.iftype.vbox & 12\\
%    etex.iftype.x & 13\\
%    etex.iftype.eof & 14\\
%    etex.iftype.true & 15\\
%    etex.iftype.false & 16\\
%    etex.iftype.case & 17\\
%    etex.iftype.defined & 18\\
%    etex.iftype.csname & 19\\
%    etex.iftype.fontchar & 20\\
%  \end{tabular}
%  \end{quote}
%
%  \subsubsection{\texorpdfstring{Category }{}\texttt{etex.nodetype}}
%
% \begin{quote}
% \begin{tabular}{@{}>{\ttfamily}l>{\ttfamily}l@{}}
%    etex.nodetype.none & -1\\
%    etex.nodetype.char & 0\\
%    etex.nodetype.hlist & 1\\
%    etex.nodetype.vlist & 2\\
%    etex.nodetype.rule & 3\\
%    etex.nodetype.ins & 4\\
%    etex.nodetype.mark & 5\\
%    etex.nodetype.adjust & 6\\
%    etex.nodetype.ligature & 7\\
%    etex.nodetype.disc & 8\\
%    etex.nodetype.whatsit & 9\\
%    etex.nodetype.math & 10\\
%    etex.nodetype.glue & 11\\
%    etex.nodetype.kern & 12\\
%    etex.nodetype.penalty & 13\\
%    etex.nodetype.unset & 14\\
%    etex.nodetype.maths & 15\\
%  \end{tabular}
%  \end{quote}
%
%  \subsubsection{\texorpdfstring{Category }{}\texttt{etex.interactionmode}}
%
% \begin{quote}
% \begin{tabular}{@{}>{\ttfamily}l>{\ttfamily}l@{}}
%    etex.interactionmode.batch & 0\\
%    etex.interactionmode.nonstop & 1\\
%    etex.interactionmode.scroll & 2\\
%    etex.interactionmode.errorstop & 3\\
%  \end{tabular}
%  \end{quote}
%
%  \subsubsection{\texorpdfstring{Category }{}\texttt{luatex.pdfliteral.mode}}
%
% \begin{quote}
% \begin{tabular}{@{}>{\ttfamily}l>{\ttfamily}l@{}}
%    luatex.pdfliteral.mode.setorigin & 0\\
%    luatex.pdfliteral.mode.page & 1\\
%    luatex.pdfliteral.mode.direct & 2\\
%  \end{tabular}
%  \end{quote}
%
%
% \hypersetup{bookmarksopenlevel=1}
%
% \StopEventually{
% }
%
% \section{Implementation}
%
%    \begin{macrocode}
%<*package>
%    \end{macrocode}
%
% \subsection{Reload check and package identification}
%    Reload check, especially if the package is not used with \LaTeX.
%    \begin{macrocode}
\begingroup\catcode61\catcode48\catcode32=10\relax%
  \catcode13=5 % ^^M
  \endlinechar=13 %
  \catcode35=6 % #
  \catcode39=12 % '
  \catcode44=12 % ,
  \catcode45=12 % -
  \catcode46=12 % .
  \catcode58=12 % :
  \catcode64=11 % @
  \catcode123=1 % {
  \catcode125=2 % }
  \expandafter\let\expandafter\x\csname ver@magicnum.sty\endcsname
  \ifx\x\relax % plain-TeX, first loading
  \else
    \def\empty{}%
    \ifx\x\empty % LaTeX, first loading,
      % variable is initialized, but \ProvidesPackage not yet seen
    \else
      \expandafter\ifx\csname PackageInfo\endcsname\relax
        \def\x#1#2{%
          \immediate\write-1{Package #1 Info: #2.}%
        }%
      \else
        \def\x#1#2{\PackageInfo{#1}{#2, stopped}}%
      \fi
      \x{magicnum}{The package is already loaded}%
      \aftergroup\endinput
    \fi
  \fi
\endgroup%
%    \end{macrocode}
%    Package identification:
%    \begin{macrocode}
\begingroup\catcode61\catcode48\catcode32=10\relax%
  \catcode13=5 % ^^M
  \endlinechar=13 %
  \catcode35=6 % #
  \catcode39=12 % '
  \catcode40=12 % (
  \catcode41=12 % )
  \catcode44=12 % ,
  \catcode45=12 % -
  \catcode46=12 % .
  \catcode47=12 % /
  \catcode58=12 % :
  \catcode64=11 % @
  \catcode91=12 % [
  \catcode93=12 % ]
  \catcode123=1 % {
  \catcode125=2 % }
  \expandafter\ifx\csname ProvidesPackage\endcsname\relax
    \def\x#1#2#3[#4]{\endgroup
      \immediate\write-1{Package: #3 #4}%
      \xdef#1{#4}%
    }%
  \else
    \def\x#1#2[#3]{\endgroup
      #2[{#3}]%
      \ifx#1\@undefined
        \xdef#1{#3}%
      \fi
      \ifx#1\relax
        \xdef#1{#3}%
      \fi
    }%
  \fi
\expandafter\x\csname ver@magicnum.sty\endcsname
\ProvidesPackage{magicnum}%
  [2011/04/10 v1.4 Magic numbers (HO)]%
%    \end{macrocode}
%
% \subsection{Catcodes}
%
%    \begin{macrocode}
\begingroup\catcode61\catcode48\catcode32=10\relax%
  \catcode13=5 % ^^M
  \endlinechar=13 %
  \catcode123=1 % {
  \catcode125=2 % }
  \catcode64=11 % @
  \def\x{\endgroup
    \expandafter\edef\csname magicnum@AtEnd\endcsname{%
      \endlinechar=\the\endlinechar\relax
      \catcode13=\the\catcode13\relax
      \catcode32=\the\catcode32\relax
      \catcode35=\the\catcode35\relax
      \catcode61=\the\catcode61\relax
      \catcode64=\the\catcode64\relax
      \catcode123=\the\catcode123\relax
      \catcode125=\the\catcode125\relax
    }%
  }%
\x\catcode61\catcode48\catcode32=10\relax%
\catcode13=5 % ^^M
\endlinechar=13 %
\catcode35=6 % #
\catcode64=11 % @
\catcode123=1 % {
\catcode125=2 % }
\def\TMP@EnsureCode#1#2{%
  \edef\magicnum@AtEnd{%
    \magicnum@AtEnd
    \catcode#1=\the\catcode#1\relax
  }%
  \catcode#1=#2\relax
}
\TMP@EnsureCode{34}{12}% "
\TMP@EnsureCode{39}{12}% '
\TMP@EnsureCode{40}{12}% (
\TMP@EnsureCode{41}{12}% )
\TMP@EnsureCode{42}{12}% *
\TMP@EnsureCode{44}{12}% ,
\TMP@EnsureCode{45}{12}% -
\TMP@EnsureCode{46}{12}% .
\TMP@EnsureCode{47}{12}% /
\TMP@EnsureCode{58}{12}% :
\TMP@EnsureCode{60}{12}% <
\TMP@EnsureCode{62}{12}% >
\TMP@EnsureCode{91}{12}% [
\TMP@EnsureCode{93}{12}% ]
\edef\magicnum@AtEnd{\magicnum@AtEnd\noexpand\endinput}
%    \end{macrocode}
%
% \subsection{Check for previous definition}
%
%    \begin{macrocode}
\begingroup\expandafter\expandafter\expandafter\endgroup
\expandafter\ifx\csname newcommand\endcsname\relax
  \expandafter\ifx\csname magicnum\endcsname\relax
  \else
    \input infwarerr.sty\relax
    \@PackageError{magicnum}{%
      \string\magicnum\space is already defined%
    }\@ehc
  \fi
\else
  \newcommand*{\magicnum}{}%
\fi
%    \end{macrocode}
%
% \subsection{Without \LuaTeX}
%
%    \begin{macrocode}
\begingroup\expandafter\expandafter\expandafter\endgroup
\expandafter\ifx\csname directlua\endcsname\relax
%    \end{macrocode}
%
%    \begin{macro}{\magicnum}
%    \begin{macrocode}
  \begingroup\expandafter\expandafter\expandafter\endgroup
  \expandafter\ifx\csname ifcsname\endcsname\relax
    \def\magicnum#1{%
      \expandafter\ifx\csname MG@#1\endcsname\relax
      \else
        \csname MG@#1\endcsname
      \fi
    }%
  \else
    \begingroup
      \edef\x{\endgroup
        \def\noexpand\magicnum##1{%
          \expandafter\noexpand\csname
          ifcsname\endcsname MG@##1\noexpand\endcsname
            \noexpand\csname MG@##1%
                 \noexpand\expandafter\noexpand\endcsname
          \expandafter\noexpand\csname fi\endcsname
        }%
      }%
    \x
  \fi
%    \end{macrocode}
%    \end{macro}
%
%    \begin{macrocode}
\else
%    \end{macrocode}
%
% \subsection{With \LuaTeX}
%
%    \begin{macrocode}
  \begingroup\expandafter\expandafter\expandafter\endgroup
  \expandafter\ifx\csname RequirePackage\endcsname\relax
    \input ifluatex.sty\relax
    \input infwarerr.sty\relax
  \else
    \RequirePackage{ifluatex}[2010/03/01]%
    \RequirePackage{infwarerr}[2010/04/08]%
  \fi
%    \end{macrocode}
%
%    \begin{macro}{\magicnum@directlua}
%    \begin{macrocode}
  \ifnum\luatexversion<36 %
    \def\magicnum@directlua{\directlua0 }%
  \else
    \let\magicnum@directlua\directlua
  \fi
%    \end{macrocode}
%    \end{macro}
%    \begin{macrocode}
  \magicnum@directlua{%
    require("oberdiek.magicnum")%
  }%
  \begingroup
    \def\x{2011/04/10 v1.4}%
    \def\StripPrefix#1>{}%
    \edef\x{\expandafter\StripPrefix\meaning\x}%
    \edef\y{%
      \magicnum@directlua{%
        if oberdiek.magicnum.getversion then %
          oberdiek.magicnum.getversion()%
        end%
      }%
    }%
    \ifx\x\y
    \else
      \@PackageError{magicnum}{%
        Wrong version of lua module.\MessageBreak
        Package version: \x\MessageBreak
        Lua module: \y
      }\@ehc
    \fi
  \endgroup
%    \end{macrocode}
%    \begin{macro}{\luaescapestring}
%    \begin{macrocode}
  \begingroup
    \expandafter\ifx\csname luaescapestring\endcsname\relax
      \directlua{%
        if tex.enableprimitives then %
          tex.enableprimitives('magicnum@', {'luaescapestring'})%
        end%
      }%
      \global\let\luaescapestring\magicnum@luaescapestring
    \fi
    \expandafter\ifx\csname luaescapestring\endcsname\relax
      \escapechar=92 %
      \@PackageError{magicnum}{%
        Missing \string\luaescapestring
      }\@ehc
    \fi
  \endgroup
%    \end{macrocode}
%    \end{macro}
%    \begin{macro}{\magicnum}
%    \begin{macrocode}
  \def\magicnum#1{%
    \magicnum@directlua{%
      oberdiek.magicnum.get("\luaescapestring{#1}")%
    }%
  }%
%    \end{macrocode}
%    \end{macro}
%
%    \begin{macrocode}
  \expandafter\magicnum@AtEnd
\fi%
%</package>
%    \end{macrocode}
%
% \subsection{Data}
%
% \subsubsection{Plain data}
%
%    \begin{macrocode}
%<*data>
tex.catcode
  escape = 0
  begingroup = 1
  endgroup = 2
  math = 3
  align = 4
  eol = 5
  parameter = 6
  superscript = 7
  subscript = 8
  ignore = 9
  space = 10
  letter = 11
  other = 12
  active = 13
  comment = 14
  invalid = 15
etex.grouptype
  bottomlevel = 0
  simple = 1
  hbox = 2
  adjustedhbox = 3
  vbox = 4
  align = 5
  noalign = 6
  output = 8
  math = 9
  disc = 10
  insert = 11
  vcenter = 12
  mathchoice = 13
  semisimple = 14
  mathshift = 15
  mathleft = 16
etex.iftype
  none = 0
  char = 1
  cat = 2
  num = 3
  dim = 4
  odd = 5
  vmode = 6
  hmode = 7
  mmode = 8
  inner = 9
  void = 10
  hbox = 11
  vbox = 12
  x = 13
  eof = 14
  true = 15
  false = 16
  case = 17
  defined = 18
  csname = 19
  fontchar = 20
etex.nodetype
  none = -1
  char = 0
  hlist = 1
  vlist = 2
  rule = 3
  ins = 4
  mark = 5
  adjust = 6
  ligature = 7
  disc = 8
  whatsit = 9
  math = 10
  glue = 11
  kern = 12
  penalty = 13
  unset = 14
  maths = 15
etex.interactionmode
  batch = 0
  nonstop = 1
  scroll = 2
  errorstop = 3
luatex.pdfliteral.mode
  setorigin = 0
  page = 1
  direct = 2
%</data>
%    \end{macrocode}
%
% \subsubsection{Data for \TeX}
%
%    \begin{macrocode}
%<*package>
%    \end{macrocode}
%    \begin{macro}{\magicnum@add}
%    \begin{macrocode}
\begingroup\expandafter\expandafter\expandafter\endgroup
\expandafter\ifx\csname detokenize\endcsname\relax
  \def\magicnum@add#1#2#3{%
    \expandafter\magicnum@@add
        \csname MG@#1.#2\expandafter\endcsname
        \csname MG@#1.#3\endcsname
       {#3}{#2}%
  }%
  \def\magicnum@@add#1#2#3#4{%
    \def#1{#3}%
    \def#2{#4}%
    \edef#1{%
      \expandafter\strip@prefix\meaning#1%
    }%
    \edef#2{%
      \expandafter\strip@prefix\meaning#2%
    }%
  }%
  \expandafter\ifx\csname strip@prefix\endcsname\relax
    \def\strip@prefix#1->{}%
  \fi
\else
  \def\magicnum@add#1#2#3{%
    \expandafter\edef\csname MG@#1.#2\endcsname{%
      \detokenize{#3}%
    }%
    \expandafter\edef\csname MG@#1.#3\endcsname{%
      \detokenize{#2}%
    }%
  }%
\fi
%    \end{macrocode}
%    \end{macro}
%    \begin{macrocode}
\magicnum@add{tex.catcode}{escape}{0}
\magicnum@add{tex.catcode}{begingroup}{1}
\magicnum@add{tex.catcode}{endgroup}{2}
\magicnum@add{tex.catcode}{math}{3}
\magicnum@add{tex.catcode}{align}{4}
\magicnum@add{tex.catcode}{eol}{5}
\magicnum@add{tex.catcode}{parameter}{6}
\magicnum@add{tex.catcode}{superscript}{7}
\magicnum@add{tex.catcode}{subscript}{8}
\magicnum@add{tex.catcode}{ignore}{9}
\magicnum@add{tex.catcode}{space}{10}
\magicnum@add{tex.catcode}{letter}{11}
\magicnum@add{tex.catcode}{other}{12}
\magicnum@add{tex.catcode}{active}{13}
\magicnum@add{tex.catcode}{comment}{14}
\magicnum@add{tex.catcode}{invalid}{15}
\magicnum@add{etex.grouptype}{bottomlevel}{0}
\magicnum@add{etex.grouptype}{simple}{1}
\magicnum@add{etex.grouptype}{hbox}{2}
\magicnum@add{etex.grouptype}{adjustedhbox}{3}
\magicnum@add{etex.grouptype}{vbox}{4}
\magicnum@add{etex.grouptype}{align}{5}
\magicnum@add{etex.grouptype}{noalign}{6}
\magicnum@add{etex.grouptype}{output}{8}
\magicnum@add{etex.grouptype}{math}{9}
\magicnum@add{etex.grouptype}{disc}{10}
\magicnum@add{etex.grouptype}{insert}{11}
\magicnum@add{etex.grouptype}{vcenter}{12}
\magicnum@add{etex.grouptype}{mathchoice}{13}
\magicnum@add{etex.grouptype}{semisimple}{14}
\magicnum@add{etex.grouptype}{mathshift}{15}
\magicnum@add{etex.grouptype}{mathleft}{16}
\magicnum@add{etex.iftype}{none}{0}
\magicnum@add{etex.iftype}{char}{1}
\magicnum@add{etex.iftype}{cat}{2}
\magicnum@add{etex.iftype}{num}{3}
\magicnum@add{etex.iftype}{dim}{4}
\magicnum@add{etex.iftype}{odd}{5}
\magicnum@add{etex.iftype}{vmode}{6}
\magicnum@add{etex.iftype}{hmode}{7}
\magicnum@add{etex.iftype}{mmode}{8}
\magicnum@add{etex.iftype}{inner}{9}
\magicnum@add{etex.iftype}{void}{10}
\magicnum@add{etex.iftype}{hbox}{11}
\magicnum@add{etex.iftype}{vbox}{12}
\magicnum@add{etex.iftype}{x}{13}
\magicnum@add{etex.iftype}{eof}{14}
\magicnum@add{etex.iftype}{true}{15}
\magicnum@add{etex.iftype}{false}{16}
\magicnum@add{etex.iftype}{case}{17}
\magicnum@add{etex.iftype}{defined}{18}
\magicnum@add{etex.iftype}{csname}{19}
\magicnum@add{etex.iftype}{fontchar}{20}
\magicnum@add{etex.nodetype}{none}{-1}
\magicnum@add{etex.nodetype}{char}{0}
\magicnum@add{etex.nodetype}{hlist}{1}
\magicnum@add{etex.nodetype}{vlist}{2}
\magicnum@add{etex.nodetype}{rule}{3}
\magicnum@add{etex.nodetype}{ins}{4}
\magicnum@add{etex.nodetype}{mark}{5}
\magicnum@add{etex.nodetype}{adjust}{6}
\magicnum@add{etex.nodetype}{ligature}{7}
\magicnum@add{etex.nodetype}{disc}{8}
\magicnum@add{etex.nodetype}{whatsit}{9}
\magicnum@add{etex.nodetype}{math}{10}
\magicnum@add{etex.nodetype}{glue}{11}
\magicnum@add{etex.nodetype}{kern}{12}
\magicnum@add{etex.nodetype}{penalty}{13}
\magicnum@add{etex.nodetype}{unset}{14}
\magicnum@add{etex.nodetype}{maths}{15}
\magicnum@add{etex.interactionmode}{batch}{0}
\magicnum@add{etex.interactionmode}{nonstop}{1}
\magicnum@add{etex.interactionmode}{scroll}{2}
\magicnum@add{etex.interactionmode}{errorstop}{3}
\magicnum@add{luatex.pdfliteral.mode}{setorigin}{0}
\magicnum@add{luatex.pdfliteral.mode}{page}{1}
\magicnum@add{luatex.pdfliteral.mode}{direct}{2}
%    \end{macrocode}
%    \begin{macrocode}
\magicnum@AtEnd%
%</package>
%    \end{macrocode}
%
% \subsubsection{Lua module}
%
%    \begin{macrocode}
%<*lua>
%    \end{macrocode}
%    \begin{macrocode}
module("oberdiek.magicnum", package.seeall)
%    \end{macrocode}
%    \begin{macrocode}
function getversion()
  tex.write("2011/04/10 v1.4")
end
%    \end{macrocode}
%    \begin{macrocode}
local data = {
  ["tex.catcode"] = {
    [0] = "escape",
    [1] = "begingroup",
    [2] = "endgroup",
    [3] = "math",
    [4] = "align",
    [5] = "eol",
    [6] = "parameter",
    [7] = "superscript",
    [8] = "subscript",
    [9] = "ignore",
    [10] = "space",
    [11] = "letter",
    [12] = "other",
    [13] = "active",
    [14] = "comment",
    [15] = "invalid",
    ["active"] = 13,
    ["align"] = 4,
    ["begingroup"] = 1,
    ["comment"] = 14,
    ["endgroup"] = 2,
    ["eol"] = 5,
    ["escape"] = 0,
    ["ignore"] = 9,
    ["invalid"] = 15,
    ["letter"] = 11,
    ["math"] = 3,
    ["other"] = 12,
    ["parameter"] = 6,
    ["space"] = 10,
    ["subscript"] = 8,
    ["superscript"] = 7
  },
  ["etex.grouptype"] = {
    [0] = "bottomlevel",
    [1] = "simple",
    [2] = "hbox",
    [3] = "adjustedhbox",
    [4] = "vbox",
    [5] = "align",
    [6] = "noalign",
    [8] = "output",
    [9] = "math",
    [10] = "disc",
    [11] = "insert",
    [12] = "vcenter",
    [13] = "mathchoice",
    [14] = "semisimple",
    [15] = "mathshift",
    [16] = "mathleft",
    ["adjustedhbox"] = 3,
    ["align"] = 5,
    ["bottomlevel"] = 0,
    ["disc"] = 10,
    ["hbox"] = 2,
    ["insert"] = 11,
    ["math"] = 9,
    ["mathchoice"] = 13,
    ["mathleft"] = 16,
    ["mathshift"] = 15,
    ["noalign"] = 6,
    ["output"] = 8,
    ["semisimple"] = 14,
    ["simple"] = 1,
    ["vbox"] = 4,
    ["vcenter"] = 12
  },
  ["etex.iftype"] = {
    [0] = "none",
    [1] = "char",
    [2] = "cat",
    [3] = "num",
    [4] = "dim",
    [5] = "odd",
    [6] = "vmode",
    [7] = "hmode",
    [8] = "mmode",
    [9] = "inner",
    [10] = "void",
    [11] = "hbox",
    [12] = "vbox",
    [13] = "x",
    [14] = "eof",
    [15] = "true",
    [16] = "false",
    [17] = "case",
    [18] = "defined",
    [19] = "csname",
    [20] = "fontchar",
    ["case"] = 17,
    ["cat"] = 2,
    ["char"] = 1,
    ["csname"] = 19,
    ["defined"] = 18,
    ["dim"] = 4,
    ["eof"] = 14,
    ["false"] = 16,
    ["fontchar"] = 20,
    ["hbox"] = 11,
    ["hmode"] = 7,
    ["inner"] = 9,
    ["mmode"] = 8,
    ["none"] = 0,
    ["num"] = 3,
    ["odd"] = 5,
    ["true"] = 15,
    ["vbox"] = 12,
    ["vmode"] = 6,
    ["void"] = 10,
    ["x"] = 13
  },
  ["etex.nodetype"] = {
    [-1] = "none",
    [0] = "char",
    [1] = "hlist",
    [2] = "vlist",
    [3] = "rule",
    [4] = "ins",
    [5] = "mark",
    [6] = "adjust",
    [7] = "ligature",
    [8] = "disc",
    [9] = "whatsit",
    [10] = "math",
    [11] = "glue",
    [12] = "kern",
    [13] = "penalty",
    [14] = "unset",
    [15] = "maths",
    ["adjust"] = 6,
    ["char"] = 0,
    ["disc"] = 8,
    ["glue"] = 11,
    ["hlist"] = 1,
    ["ins"] = 4,
    ["kern"] = 12,
    ["ligature"] = 7,
    ["mark"] = 5,
    ["math"] = 10,
    ["maths"] = 15,
    ["none"] = -1,
    ["penalty"] = 13,
    ["rule"] = 3,
    ["unset"] = 14,
    ["vlist"] = 2,
    ["whatsit"] = 9
  },
  ["etex.interactionmode"] = {
    [0] = "batch",
    [1] = "nonstop",
    [2] = "scroll",
    [3] = "errorstop",
    ["batch"] = 0,
    ["errorstop"] = 3,
    ["nonstop"] = 1,
    ["scroll"] = 2
  },
  ["luatex.pdfliteral.mode"] = {
    [0] = "setorigin",
    [1] = "page",
    [2] = "direct",
    ["direct"] = 2,
    ["page"] = 1,
    ["setorigin"] = 0
  }
}
%    \end{macrocode}
%    \begin{macrocode}
function get(name)
  local startpos, endpos, category, entry =
      string.find(name, "^(%a[%a%d%.]*)%.(-?[%a%d]+)$")
  if not entry then
    return
  end
  local node = data[category]
  if not node then
    return
  end
  local num = tonumber(entry)
  local value
  if num then
    value = node[num]
    if not value then
      return
    end
  else
    value = node[entry]
    if not value then
      return
    end
    value = "" .. value
  end
  tex.write(value)
end
%    \end{macrocode}
%
%    \begin{macrocode}
%</lua>
%    \end{macrocode}
%
% \section{Test}
%
% \subsection{Catcode checks for loading}
%
%    \begin{macrocode}
%<*test1>
%    \end{macrocode}
%    \begin{macrocode}
\catcode`\{=1 %
\catcode`\}=2 %
\catcode`\#=6 %
\catcode`\@=11 %
\expandafter\ifx\csname count@\endcsname\relax
  \countdef\count@=255 %
\fi
\expandafter\ifx\csname @gobble\endcsname\relax
  \long\def\@gobble#1{}%
\fi
\expandafter\ifx\csname @firstofone\endcsname\relax
  \long\def\@firstofone#1{#1}%
\fi
\expandafter\ifx\csname loop\endcsname\relax
  \expandafter\@firstofone
\else
  \expandafter\@gobble
\fi
{%
  \def\loop#1\repeat{%
    \def\body{#1}%
    \iterate
  }%
  \def\iterate{%
    \body
      \let\next\iterate
    \else
      \let\next\relax
    \fi
    \next
  }%
  \let\repeat=\fi
}%
\def\RestoreCatcodes{}
\count@=0 %
\loop
  \edef\RestoreCatcodes{%
    \RestoreCatcodes
    \catcode\the\count@=\the\catcode\count@\relax
  }%
\ifnum\count@<255 %
  \advance\count@ 1 %
\repeat

\def\RangeCatcodeInvalid#1#2{%
  \count@=#1\relax
  \loop
    \catcode\count@=15 %
  \ifnum\count@<#2\relax
    \advance\count@ 1 %
  \repeat
}
\def\RangeCatcodeCheck#1#2#3{%
  \count@=#1\relax
  \loop
    \ifnum#3=\catcode\count@
    \else
      \errmessage{%
        Character \the\count@\space
        with wrong catcode \the\catcode\count@\space
        instead of \number#3%
      }%
    \fi
  \ifnum\count@<#2\relax
    \advance\count@ 1 %
  \repeat
}
\def\space{ }
\expandafter\ifx\csname LoadCommand\endcsname\relax
  \def\LoadCommand{\input magicnum.sty\relax}%
\fi
\def\Test{%
  \RangeCatcodeInvalid{0}{47}%
  \RangeCatcodeInvalid{58}{64}%
  \RangeCatcodeInvalid{91}{96}%
  \RangeCatcodeInvalid{123}{255}%
  \catcode`\@=12 %
  \catcode`\\=0 %
  \catcode`\%=14 %
  \LoadCommand
  \RangeCatcodeCheck{0}{36}{15}%
  \RangeCatcodeCheck{37}{37}{14}%
  \RangeCatcodeCheck{38}{47}{15}%
  \RangeCatcodeCheck{48}{57}{12}%
  \RangeCatcodeCheck{58}{63}{15}%
  \RangeCatcodeCheck{64}{64}{12}%
  \RangeCatcodeCheck{65}{90}{11}%
  \RangeCatcodeCheck{91}{91}{15}%
  \RangeCatcodeCheck{92}{92}{0}%
  \RangeCatcodeCheck{93}{96}{15}%
  \RangeCatcodeCheck{97}{122}{11}%
  \RangeCatcodeCheck{123}{255}{15}%
  \RestoreCatcodes
}
\Test
\csname @@end\endcsname
\end
%    \end{macrocode}
%    \begin{macrocode}
%</test1>
%    \end{macrocode}
%
% \subsection{Test data}
%
%    \begin{macrocode}
%<*testplain>
\input magicnum.sty\relax
\def\Test#1#2{%
  \edef\result{\magicnum{#1}}%
  \edef\expect{#2}%
  \edef\expect{\expandafter\stripprefix\meaning\expect}%
  \ifx\result\expect
  \else
    \errmessage{%
      Failed: [#1] % hash-ok
      returns [\result] instead of [\expect]%
    }%
  \fi
}
\def\stripprefix#1->{}
%</testplain>
%    \end{macrocode}
%    \begin{macrocode}
%<*testlatex>
\NeedsTeXFormat{LaTeX2e}
\documentclass{minimal}
\usepackage{magicnum}[2011/04/10]
\usepackage{qstest}
\IncludeTests{*}
\LogTests{log}{*}{*}
\newcommand*{\Test}[2]{%
  \Expect*{\magicnum{#1}}{#2}%
}
\begin{qstest}{magicnum}{magicnum}
%</testlatex>
%    \end{macrocode}
%    \begin{macrocode}
%<*testdata>
\Test{tex.catcode.escape}{0}
\Test{tex.catcode.invalid}{15}
\Test{tex.catcode.unknown}{}
\Test{tex.catcode.0}{escape}
\Test{tex.catcode.15}{invalid}
\Test{etex.iftype.true}{15}
\Test{etex.iftype.false}{16}
\Test{etex.iftype.15}{true}
\Test{etex.iftype.16}{false}
\Test{etex.nodetype.none}{-1}
\Test{etex.nodetype.-1}{none}
\Test{luatex.pdfliteral.mode.direct}{2}
\Test{luatex.pdfliteral.mode.1}{page}
\Test{}{}
\Test{unknown}{}
\Test{unknown.foo.bar}{}
\Test{unknown.foo.4}{}
%</testdata>
%    \end{macrocode}
%    \begin{macrocode}
%<*testplain>
\csname @@end\endcsname
\end
%</testplain>
%<*testlatex>
\end{qstest}
\csname @@end\endcsname
%</testlatex>
%    \end{macrocode}
%
% \subsection{Small test for \hologo{iniTeX}}
%
%    \begin{macrocode}
%<*test4>
\catcode`\{=1
\catcode`\}=2
\catcode`\#=6
\input magicnum.sty\relax
\edef\x{\magicnum{tex.catcode.15}}
\edef\y{invalid}
\def\Strip#1>{}
\edef\y{\expandafter\Strip\meaning\y}
\ifx\x\y
  \immediate\write16{Ok}%
\else
  \errmessage{\x<>\y}%
\fi
\csname @@end\endcsname\end
%</test4>
%    \end{macrocode}
%
% \section{Installation}
%
% \subsection{Download}
%
% \paragraph{Package.} This package is available on
% CTAN\footnote{\url{ftp://ftp.ctan.org/tex-archive/}}:
% \begin{description}
% \item[\CTAN{macros/latex/contrib/oberdiek/magicnum.dtx}] The source file.
% \item[\CTAN{macros/latex/contrib/oberdiek/magicnum.pdf}] Documentation.
% \end{description}
%
%
% \paragraph{Bundle.} All the packages of the bundle `oberdiek'
% are also available in a TDS compliant ZIP archive. There
% the packages are already unpacked and the documentation files
% are generated. The files and directories obey the TDS standard.
% \begin{description}
% \item[\CTAN{install/macros/latex/contrib/oberdiek.tds.zip}]
% \end{description}
% \emph{TDS} refers to the standard ``A Directory Structure
% for \TeX\ Files'' (\CTAN{tds/tds.pdf}). Directories
% with \xfile{texmf} in their name are usually organized this way.
%
% \subsection{Bundle installation}
%
% \paragraph{Unpacking.} Unpack the \xfile{oberdiek.tds.zip} in the
% TDS tree (also known as \xfile{texmf} tree) of your choice.
% Example (linux):
% \begin{quote}
%   |unzip oberdiek.tds.zip -d ~/texmf|
% \end{quote}
%
% \paragraph{Script installation.}
% Check the directory \xfile{TDS:scripts/oberdiek/} for
% scripts that need further installation steps.
% Package \xpackage{attachfile2} comes with the Perl script
% \xfile{pdfatfi.pl} that should be installed in such a way
% that it can be called as \texttt{pdfatfi}.
% Example (linux):
% \begin{quote}
%   |chmod +x scripts/oberdiek/pdfatfi.pl|\\
%   |cp scripts/oberdiek/pdfatfi.pl /usr/local/bin/|
% \end{quote}
%
% \subsection{Package installation}
%
% \paragraph{Unpacking.} The \xfile{.dtx} file is a self-extracting
% \docstrip\ archive. The files are extracted by running the
% \xfile{.dtx} through \plainTeX:
% \begin{quote}
%   \verb|tex magicnum.dtx|
% \end{quote}
%
% \paragraph{TDS.} Now the different files must be moved into
% the different directories in your installation TDS tree
% (also known as \xfile{texmf} tree):
% \begin{quote}
% \def\t{^^A
% \begin{tabular}{@{}>{\ttfamily}l@{ $\rightarrow$ }>{\ttfamily}l@{}}
%   magicnum.sty & tex/generic/oberdiek/magicnum.sty\\
%   magicnum.lua & scripts/oberdiek/magicnum.lua\\
%   oberdiek.magicnum.lua & scripts/oberdiek/oberdiek.magicnum.lua\\
%   magicnum.pdf & doc/latex/oberdiek/magicnum.pdf\\
%   magicnum.txt & doc/latex/oberdiek/magicnum.txt\\
%   test/magicnum-test1.tex & doc/latex/oberdiek/test/magicnum-test1.tex\\
%   test/magicnum-test2.tex & doc/latex/oberdiek/test/magicnum-test2.tex\\
%   test/magicnum-test3.tex & doc/latex/oberdiek/test/magicnum-test3.tex\\
%   test/magicnum-test4.tex & doc/latex/oberdiek/test/magicnum-test4.tex\\
%   magicnum.dtx & source/latex/oberdiek/magicnum.dtx\\
% \end{tabular}^^A
% }^^A
% \sbox0{\t}^^A
% \ifdim\wd0>\linewidth
%   \begingroup
%     \advance\linewidth by\leftmargin
%     \advance\linewidth by\rightmargin
%   \edef\x{\endgroup
%     \def\noexpand\lw{\the\linewidth}^^A
%   }\x
%   \def\lwbox{^^A
%     \leavevmode
%     \hbox to \linewidth{^^A
%       \kern-\leftmargin\relax
%       \hss
%       \usebox0
%       \hss
%       \kern-\rightmargin\relax
%     }^^A
%   }^^A
%   \ifdim\wd0>\lw
%     \sbox0{\small\t}^^A
%     \ifdim\wd0>\linewidth
%       \ifdim\wd0>\lw
%         \sbox0{\footnotesize\t}^^A
%         \ifdim\wd0>\linewidth
%           \ifdim\wd0>\lw
%             \sbox0{\scriptsize\t}^^A
%             \ifdim\wd0>\linewidth
%               \ifdim\wd0>\lw
%                 \sbox0{\tiny\t}^^A
%                 \ifdim\wd0>\linewidth
%                   \lwbox
%                 \else
%                   \usebox0
%                 \fi
%               \else
%                 \lwbox
%               \fi
%             \else
%               \usebox0
%             \fi
%           \else
%             \lwbox
%           \fi
%         \else
%           \usebox0
%         \fi
%       \else
%         \lwbox
%       \fi
%     \else
%       \usebox0
%     \fi
%   \else
%     \lwbox
%   \fi
% \else
%   \usebox0
% \fi
% \end{quote}
% If you have a \xfile{docstrip.cfg} that configures and enables \docstrip's
% TDS installing feature, then some files can already be in the right
% place, see the documentation of \docstrip.
%
% \subsection{Refresh file name databases}
%
% If your \TeX~distribution
% (\teTeX, \mikTeX, \dots) relies on file name databases, you must refresh
% these. For example, \teTeX\ users run \verb|texhash| or
% \verb|mktexlsr|.
%
% \subsection{Some details for the interested}
%
% \paragraph{Attached source.}
%
% The PDF documentation on CTAN also includes the
% \xfile{.dtx} source file. It can be extracted by
% AcrobatReader 6 or higher. Another option is \textsf{pdftk},
% e.g. unpack the file into the current directory:
% \begin{quote}
%   \verb|pdftk magicnum.pdf unpack_files output .|
% \end{quote}
%
% \paragraph{Unpacking with \LaTeX.}
% The \xfile{.dtx} chooses its action depending on the format:
% \begin{description}
% \item[\plainTeX:] Run \docstrip\ and extract the files.
% \item[\LaTeX:] Generate the documentation.
% \end{description}
% If you insist on using \LaTeX\ for \docstrip\ (really,
% \docstrip\ does not need \LaTeX), then inform the autodetect routine
% about your intention:
% \begin{quote}
%   \verb|latex \let\install=y\input{magicnum.dtx}|
% \end{quote}
% Do not forget to quote the argument according to the demands
% of your shell.
%
% \paragraph{Generating the documentation.}
% You can use both the \xfile{.dtx} or the \xfile{.drv} to generate
% the documentation. The process can be configured by the
% configuration file \xfile{ltxdoc.cfg}. For instance, put this
% line into this file, if you want to have A4 as paper format:
% \begin{quote}
%   \verb|\PassOptionsToClass{a4paper}{article}|
% \end{quote}
% An example follows how to generate the
% documentation with pdf\LaTeX:
% \begin{quote}
%\begin{verbatim}
%pdflatex magicnum.dtx
%makeindex -s gind.ist magicnum.idx
%pdflatex magicnum.dtx
%makeindex -s gind.ist magicnum.idx
%pdflatex magicnum.dtx
%\end{verbatim}
% \end{quote}
%
% \section{Catalogue}
%
% The following XML file can be used as source for the
% \href{http://mirror.ctan.org/help/Catalogue/catalogue.html}{\TeX\ Catalogue}.
% The elements \texttt{caption} and \texttt{description} are imported
% from the original XML file from the Catalogue.
% The name of the XML file in the Catalogue is \xfile{magicnum.xml}.
%    \begin{macrocode}
%<*catalogue>
<?xml version='1.0' encoding='us-ascii'?>
<!DOCTYPE entry SYSTEM 'catalogue.dtd'>
<entry datestamp='$Date$' modifier='$Author$' id='magicnum'>
  <name>magicnum</name>
  <caption>Access TeX systems' "magic numbers".</caption>
  <authorref id='auth:oberdiek'/>
  <copyright owner='Heiko Oberdiek' year='2007,2009-2011'/>
  <license type='lppl1.3'/>
  <version number='1.4'/>
  <description>
    This package allows access to the various parameter values in
    TeX (catcode values), e-TeX (group, if and node types, and
    interaction mode), and LuaTeX (pdfliteral mode) by a hierarchical
    name system.
    <p/>
    The package is part of the <xref refid='oberdiek'>oberdiek</xref> bundle.
  </description>
  <documentation details='Package documentation'
      href='ctan:/macros/latex/contrib/oberdiek/magicnum.pdf'/>
  <ctan file='true' path='/macros/latex/contrib/oberdiek/magicnum.dtx'/>
  <miktex location='oberdiek'/>
  <texlive location='oberdiek'/>
  <install path='/macros/latex/contrib/oberdiek/oberdiek.tds.zip'/>
</entry>
%</catalogue>
%    \end{macrocode}
%
% \begin{History}
%   \begin{Version}{2007/12/12 v1.0}
%   \item
%     First public version.
%   \end{Version}
%   \begin{Version}{2009/04/10 v1.1}
%   \item
%     Adaptation to \LuaTeX\ 0.40.
%   \end{Version}
%   \begin{Version}{2010/03/09 v1.2}
%   \item
%     Adaptation to package \xpackage{luatex} 0.4.
%   \end{Version}
%   \begin{Version}{2011/03/24 v1.3}
%   \item
%     Catcode fixes.
%   \end{Version}
%   \begin{Version}{2011/04/10 v1.4}
%   \item
%     Compatibility for \hologo{iniTeX}.
%   \item
%     Dependency from package \xpackage{luatex} removed.
%   \item
%     Version check for lua module.
%   \end{Version}
% \end{History}
%
% \PrintIndex
%
% \Finale
\endinput

%        (quote the arguments according to the demands of your shell)
%
% Documentation:
%    (a) If magicnum.drv is present:
%           latex magicnum.drv
%    (b) Without magicnum.drv:
%           latex magicnum.dtx; ...
%    The class ltxdoc loads the configuration file ltxdoc.cfg
%    if available. Here you can specify further options, e.g.
%    use A4 as paper format:
%       \PassOptionsToClass{a4paper}{article}
%
%    Programm calls to get the documentation (example):
%       pdflatex magicnum.dtx
%       makeindex -s gind.ist magicnum.idx
%       pdflatex magicnum.dtx
%       makeindex -s gind.ist magicnum.idx
%       pdflatex magicnum.dtx
%
% Installation:
%    TDS:tex/generic/oberdiek/magicnum.sty
%    TDS:scripts/oberdiek/magicnum.lua
%    TDS:scripts/oberdiek/oberdiek.magicnum.lua
%    TDS:doc/latex/oberdiek/magicnum.pdf
%    TDS:doc/latex/oberdiek/magicnum.txt
%    TDS:doc/latex/oberdiek/test/magicnum-test1.tex
%    TDS:doc/latex/oberdiek/test/magicnum-test2.tex
%    TDS:doc/latex/oberdiek/test/magicnum-test3.tex
%    TDS:doc/latex/oberdiek/test/magicnum-test4.tex
%    TDS:source/latex/oberdiek/magicnum.dtx
%
%<*ignore>
\begingroup
  \catcode123=1 %
  \catcode125=2 %
  \def\x{LaTeX2e}%
\expandafter\endgroup
\ifcase 0\ifx\install y1\fi\expandafter
         \ifx\csname processbatchFile\endcsname\relax\else1\fi
         \ifx\fmtname\x\else 1\fi\relax
\else\csname fi\endcsname
%</ignore>
%<*install>
\input docstrip.tex
\Msg{************************************************************************}
\Msg{* Installation}
\Msg{* Package: magicnum 2011/04/10 v1.4 Magic numbers (HO)}
\Msg{************************************************************************}

\keepsilent
\askforoverwritefalse

\let\MetaPrefix\relax
\preamble

This is a generated file.

Project: magicnum
Version: 2011/04/10 v1.4

Copyright (C) 2007, 2009-2011 by
   Heiko Oberdiek <heiko.oberdiek at googlemail.com>

This work may be distributed and/or modified under the
conditions of the LaTeX Project Public License, either
version 1.3c of this license or (at your option) any later
version. This version of this license is in
   http://www.latex-project.org/lppl/lppl-1-3c.txt
and the latest version of this license is in
   http://www.latex-project.org/lppl.txt
and version 1.3 or later is part of all distributions of
LaTeX version 2005/12/01 or later.

This work has the LPPL maintenance status "maintained".

This Current Maintainer of this work is Heiko Oberdiek.

The Base Interpreter refers to any `TeX-Format',
because some files are installed in TDS:tex/generic//.

This work consists of the main source file magicnum.dtx
and the derived files
   magicnum.sty, magicnum.pdf, magicnum.ins, magicnum.drv, magicnum.txt,
   magicnum-test1.tex, magicnum-test2.tex, magicnum-test3.tex,
   magicnum-test4.tex, magicnum.lua, oberdiek.magicnum.lua.

\endpreamble
\let\MetaPrefix\DoubleperCent

\generate{%
  \file{magicnum.ins}{\from{magicnum.dtx}{install}}%
  \file{magicnum.drv}{\from{magicnum.dtx}{driver}}%
  \usedir{tex/generic/oberdiek}%
  \file{magicnum.sty}{\from{magicnum.dtx}{package}}%
  \usedir{doc/latex/oberdiek/test}%
  \file{magicnum-test1.tex}{\from{magicnum.dtx}{test1}}%
  \file{magicnum-test2.tex}{\from{magicnum.dtx}{testplain,testdata}}%
  \file{magicnum-test3.tex}{\from{magicnum.dtx}{testlatex,testdata}}%
  \file{magicnum-test4.tex}{\from{magicnum.dtx}{test4}}%
  \nopreamble
  \nopostamble
  \usedir{doc/latex/oberdiek}%
  \file{magicnum.txt}{\from{magicnum.dtx}{data}}%
  \usedir{source/latex/oberdiek/catalogue}%
  \file{magicnum.xml}{\from{magicnum.dtx}{catalogue}}%
}
\def\MetaPrefix{-- }
\def\defaultpostamble{%
  \MetaPrefix^^J%
  \MetaPrefix\space End of File `\outFileName'.%
}
\def\currentpostamble{\defaultpostamble}%
\generate{%
  \usedir{scripts/oberdiek}%
  \file{magicnum.lua}{\from{magicnum.dtx}{lua}}%
  \file{oberdiek.magicnum.lua}{\from{magicnum.dtx}{lua}}%
}

\catcode32=13\relax% active space
\let =\space%
\Msg{************************************************************************}
\Msg{*}
\Msg{* To finish the installation you have to move the following}
\Msg{* file into a directory searched by TeX:}
\Msg{*}
\Msg{*     magicnum.sty}
\Msg{*}
\Msg{* And install the following script files:}
\Msg{*}
\Msg{*     magicnum.lua, oberdiek.magicnum.lua}
\Msg{*}
\Msg{* To produce the documentation run the file `magicnum.drv'}
\Msg{* through LaTeX.}
\Msg{*}
\Msg{* Happy TeXing!}
\Msg{*}
\Msg{************************************************************************}

\endbatchfile
%</install>
%<*ignore>
\fi
%</ignore>
%<*driver>
\NeedsTeXFormat{LaTeX2e}
\ProvidesFile{magicnum.drv}%
  [2011/04/10 v1.4 Magic numbers (HO)]%
\documentclass{ltxdoc}
\usepackage{holtxdoc}[2011/11/22]
\usepackage{array}
\begin{document}
  \DocInput{magicnum.dtx}%
\end{document}
%</driver>
% \fi
%
% \CheckSum{755}
%
% \CharacterTable
%  {Upper-case    \A\B\C\D\E\F\G\H\I\J\K\L\M\N\O\P\Q\R\S\T\U\V\W\X\Y\Z
%   Lower-case    \a\b\c\d\e\f\g\h\i\j\k\l\m\n\o\p\q\r\s\t\u\v\w\x\y\z
%   Digits        \0\1\2\3\4\5\6\7\8\9
%   Exclamation   \!     Double quote  \"     Hash (number) \#
%   Dollar        \$     Percent       \%     Ampersand     \&
%   Acute accent  \'     Left paren    \(     Right paren   \)
%   Asterisk      \*     Plus          \+     Comma         \,
%   Minus         \-     Point         \.     Solidus       \/
%   Colon         \:     Semicolon     \;     Less than     \<
%   Equals        \=     Greater than  \>     Question mark \?
%   Commercial at \@     Left bracket  \[     Backslash     \\
%   Right bracket \]     Circumflex    \^     Underscore    \_
%   Grave accent  \`     Left brace    \{     Vertical bar  \|
%   Right brace   \}     Tilde         \~}
%
% \GetFileInfo{magicnum.drv}
%
% \title{The \xpackage{magicnum} package}
% \date{2011/04/10 v1.4}
% \author{Heiko Oberdiek\\\xemail{heiko.oberdiek at googlemail.com}}
%
% \maketitle
%
% \begin{abstract}
% This packages allows to access magic numbers by a hierarchical
% name system.
% \end{abstract}
%
% \tableofcontents
%
% \hypersetup{bookmarksopenlevel=2}
% \section{Documentation}
%
% \subsection{Introduction}
%
% Especially since \eTeX\ there are many integer values
% with special meanings, such as catcodes, group types, \dots
% Package \xpackage{etex}, enabled by options, defines
% macros in the user namespace for these values.
%
% This package goes another approach for storing the names and values.
% \begin{itemize}
% \item If \LuaTeX\ is available, they
% are stored in Lua tables.
% \item Without \LuaTeX\ they are remembered using internal
% macros.
% \end{itemize}
%
% \subsection{User interface}
%
% The integer values and names are organized in a hierarchical
% scheme of categories with the property names as leaves.
% Example: \eTeX's \cs{currentgrouplevel} reports |2| for a
% group caused by \cs{hbox}. This package has choosen to organize
% the group types in a main category |etex| and its subcategory
% |grouptype|:
% \begin{quote}
%   |etex.grouptype.hbox| = |2|
% \end{quote}
% The property name |hbox| in category |etex.grouptype| has value |2|.
% Dots are used to separate components.
%
% If you want to have the value, the access key is constructed by
% the category with all its components and the property name.
% For the opposite the value is used instead of the property name.
%
% Values are always integers (including negative numbers).
%
% \subsubsection{\cs{magicnum}}
%
% \begin{declcs}{magicnum} \M{access key}
% \end{declcs}
% Macro \cs{magicnum} expects an access key as argument and
% expands to the requested data. The macro is always expandable.
% In case of errors the expansion result is empty.
%
% The same macro is also used for getting a property name.
% In this case the property name part in the access key is
% replaced by the value.
%
% The catcodes
% of the resulting numbers and strings follow \TeX's tradition of
% \cs{string}, \cs{meaning}, \dots: The space has catcode 10
% (|tex.catcode.space|) and the other characters have catcode
% 12 (|tex.catcode.other|).
%
% Examples:
% \begin{quote}
%   |\magicnum{etex.grouptype.hbox}| $\Rightarrow$ |2|\\
%   |\magicnum{tex.catcode.14}| $\Rightarrow$ |comment|\\
%   |\magicnum{tex.catcode.undefined}| $\Rightarrow$ $\emptyset$
% \end{quote}
%
% \subsubsection{Properties}
%
% \begin{itemize}
% \item The components of a category are either subcategories or
%       key value pairs, but not both.
% \item The full specified property names are unique and thus
%       has one integer value exactly.
% \item Also the values inside a category are unique.
%       This condition is a prerequisite for the reverse mapping
%       of \cs{magicnum}.
% \item All names start with a letter. Only letters or digits
%       may follow.
% \end{itemize}
%
% \subsection{Data}
%
%  \subsubsection{\texorpdfstring{Category }{}\texttt{tex.catcode}}
%
% \begin{quote}
% \begin{tabular}{@{}>{\ttfamily}l>{\ttfamily}l@{}}
%    tex.catcode.escape & 0\\
%    tex.catcode.begingroup & 1\\
%    tex.catcode.endgroup & 2\\
%    tex.catcode.math & 3\\
%    tex.catcode.align & 4\\
%    tex.catcode.eol & 5\\
%    tex.catcode.parameter & 6\\
%    tex.catcode.superscript & 7\\
%    tex.catcode.subscript & 8\\
%    tex.catcode.ignore & 9\\
%    tex.catcode.space & 10\\
%    tex.catcode.letter & 11\\
%    tex.catcode.other & 12\\
%    tex.catcode.active & 13\\
%    tex.catcode.comment & 14\\
%    tex.catcode.invalid & 15\\
%  \end{tabular}
%  \end{quote}
%
%  \subsubsection{\texorpdfstring{Category }{}\texttt{etex.grouptype}}
%
% \begin{quote}
% \begin{tabular}{@{}>{\ttfamily}l>{\ttfamily}l@{}}
%    etex.grouptype.bottomlevel & 0\\
%    etex.grouptype.simple & 1\\
%    etex.grouptype.hbox & 2\\
%    etex.grouptype.adjustedhbox & 3\\
%    etex.grouptype.vbox & 4\\
%    etex.grouptype.align & 5\\
%    etex.grouptype.noalign & 6\\
%    etex.grouptype.output & 8\\
%    etex.grouptype.math & 9\\
%    etex.grouptype.disc & 10\\
%    etex.grouptype.insert & 11\\
%    etex.grouptype.vcenter & 12\\
%    etex.grouptype.mathchoice & 13\\
%    etex.grouptype.semisimple & 14\\
%    etex.grouptype.mathshift & 15\\
%    etex.grouptype.mathleft & 16\\
%  \end{tabular}
%  \end{quote}
%
%  \subsubsection{\texorpdfstring{Category }{}\texttt{etex.iftype}}
%
% \begin{quote}
% \begin{tabular}{@{}>{\ttfamily}l>{\ttfamily}l@{}}
%    etex.iftype.none & 0\\
%    etex.iftype.char & 1\\
%    etex.iftype.cat & 2\\
%    etex.iftype.num & 3\\
%    etex.iftype.dim & 4\\
%    etex.iftype.odd & 5\\
%    etex.iftype.vmode & 6\\
%    etex.iftype.hmode & 7\\
%    etex.iftype.mmode & 8\\
%    etex.iftype.inner & 9\\
%    etex.iftype.void & 10\\
%    etex.iftype.hbox & 11\\
%    etex.iftype.vbox & 12\\
%    etex.iftype.x & 13\\
%    etex.iftype.eof & 14\\
%    etex.iftype.true & 15\\
%    etex.iftype.false & 16\\
%    etex.iftype.case & 17\\
%    etex.iftype.defined & 18\\
%    etex.iftype.csname & 19\\
%    etex.iftype.fontchar & 20\\
%  \end{tabular}
%  \end{quote}
%
%  \subsubsection{\texorpdfstring{Category }{}\texttt{etex.nodetype}}
%
% \begin{quote}
% \begin{tabular}{@{}>{\ttfamily}l>{\ttfamily}l@{}}
%    etex.nodetype.none & -1\\
%    etex.nodetype.char & 0\\
%    etex.nodetype.hlist & 1\\
%    etex.nodetype.vlist & 2\\
%    etex.nodetype.rule & 3\\
%    etex.nodetype.ins & 4\\
%    etex.nodetype.mark & 5\\
%    etex.nodetype.adjust & 6\\
%    etex.nodetype.ligature & 7\\
%    etex.nodetype.disc & 8\\
%    etex.nodetype.whatsit & 9\\
%    etex.nodetype.math & 10\\
%    etex.nodetype.glue & 11\\
%    etex.nodetype.kern & 12\\
%    etex.nodetype.penalty & 13\\
%    etex.nodetype.unset & 14\\
%    etex.nodetype.maths & 15\\
%  \end{tabular}
%  \end{quote}
%
%  \subsubsection{\texorpdfstring{Category }{}\texttt{etex.interactionmode}}
%
% \begin{quote}
% \begin{tabular}{@{}>{\ttfamily}l>{\ttfamily}l@{}}
%    etex.interactionmode.batch & 0\\
%    etex.interactionmode.nonstop & 1\\
%    etex.interactionmode.scroll & 2\\
%    etex.interactionmode.errorstop & 3\\
%  \end{tabular}
%  \end{quote}
%
%  \subsubsection{\texorpdfstring{Category }{}\texttt{luatex.pdfliteral.mode}}
%
% \begin{quote}
% \begin{tabular}{@{}>{\ttfamily}l>{\ttfamily}l@{}}
%    luatex.pdfliteral.mode.setorigin & 0\\
%    luatex.pdfliteral.mode.page & 1\\
%    luatex.pdfliteral.mode.direct & 2\\
%  \end{tabular}
%  \end{quote}
%
%
% \hypersetup{bookmarksopenlevel=1}
%
% \StopEventually{
% }
%
% \section{Implementation}
%
%    \begin{macrocode}
%<*package>
%    \end{macrocode}
%
% \subsection{Reload check and package identification}
%    Reload check, especially if the package is not used with \LaTeX.
%    \begin{macrocode}
\begingroup\catcode61\catcode48\catcode32=10\relax%
  \catcode13=5 % ^^M
  \endlinechar=13 %
  \catcode35=6 % #
  \catcode39=12 % '
  \catcode44=12 % ,
  \catcode45=12 % -
  \catcode46=12 % .
  \catcode58=12 % :
  \catcode64=11 % @
  \catcode123=1 % {
  \catcode125=2 % }
  \expandafter\let\expandafter\x\csname ver@magicnum.sty\endcsname
  \ifx\x\relax % plain-TeX, first loading
  \else
    \def\empty{}%
    \ifx\x\empty % LaTeX, first loading,
      % variable is initialized, but \ProvidesPackage not yet seen
    \else
      \expandafter\ifx\csname PackageInfo\endcsname\relax
        \def\x#1#2{%
          \immediate\write-1{Package #1 Info: #2.}%
        }%
      \else
        \def\x#1#2{\PackageInfo{#1}{#2, stopped}}%
      \fi
      \x{magicnum}{The package is already loaded}%
      \aftergroup\endinput
    \fi
  \fi
\endgroup%
%    \end{macrocode}
%    Package identification:
%    \begin{macrocode}
\begingroup\catcode61\catcode48\catcode32=10\relax%
  \catcode13=5 % ^^M
  \endlinechar=13 %
  \catcode35=6 % #
  \catcode39=12 % '
  \catcode40=12 % (
  \catcode41=12 % )
  \catcode44=12 % ,
  \catcode45=12 % -
  \catcode46=12 % .
  \catcode47=12 % /
  \catcode58=12 % :
  \catcode64=11 % @
  \catcode91=12 % [
  \catcode93=12 % ]
  \catcode123=1 % {
  \catcode125=2 % }
  \expandafter\ifx\csname ProvidesPackage\endcsname\relax
    \def\x#1#2#3[#4]{\endgroup
      \immediate\write-1{Package: #3 #4}%
      \xdef#1{#4}%
    }%
  \else
    \def\x#1#2[#3]{\endgroup
      #2[{#3}]%
      \ifx#1\@undefined
        \xdef#1{#3}%
      \fi
      \ifx#1\relax
        \xdef#1{#3}%
      \fi
    }%
  \fi
\expandafter\x\csname ver@magicnum.sty\endcsname
\ProvidesPackage{magicnum}%
  [2011/04/10 v1.4 Magic numbers (HO)]%
%    \end{macrocode}
%
% \subsection{Catcodes}
%
%    \begin{macrocode}
\begingroup\catcode61\catcode48\catcode32=10\relax%
  \catcode13=5 % ^^M
  \endlinechar=13 %
  \catcode123=1 % {
  \catcode125=2 % }
  \catcode64=11 % @
  \def\x{\endgroup
    \expandafter\edef\csname magicnum@AtEnd\endcsname{%
      \endlinechar=\the\endlinechar\relax
      \catcode13=\the\catcode13\relax
      \catcode32=\the\catcode32\relax
      \catcode35=\the\catcode35\relax
      \catcode61=\the\catcode61\relax
      \catcode64=\the\catcode64\relax
      \catcode123=\the\catcode123\relax
      \catcode125=\the\catcode125\relax
    }%
  }%
\x\catcode61\catcode48\catcode32=10\relax%
\catcode13=5 % ^^M
\endlinechar=13 %
\catcode35=6 % #
\catcode64=11 % @
\catcode123=1 % {
\catcode125=2 % }
\def\TMP@EnsureCode#1#2{%
  \edef\magicnum@AtEnd{%
    \magicnum@AtEnd
    \catcode#1=\the\catcode#1\relax
  }%
  \catcode#1=#2\relax
}
\TMP@EnsureCode{34}{12}% "
\TMP@EnsureCode{39}{12}% '
\TMP@EnsureCode{40}{12}% (
\TMP@EnsureCode{41}{12}% )
\TMP@EnsureCode{42}{12}% *
\TMP@EnsureCode{44}{12}% ,
\TMP@EnsureCode{45}{12}% -
\TMP@EnsureCode{46}{12}% .
\TMP@EnsureCode{47}{12}% /
\TMP@EnsureCode{58}{12}% :
\TMP@EnsureCode{60}{12}% <
\TMP@EnsureCode{62}{12}% >
\TMP@EnsureCode{91}{12}% [
\TMP@EnsureCode{93}{12}% ]
\edef\magicnum@AtEnd{\magicnum@AtEnd\noexpand\endinput}
%    \end{macrocode}
%
% \subsection{Check for previous definition}
%
%    \begin{macrocode}
\begingroup\expandafter\expandafter\expandafter\endgroup
\expandafter\ifx\csname newcommand\endcsname\relax
  \expandafter\ifx\csname magicnum\endcsname\relax
  \else
    \input infwarerr.sty\relax
    \@PackageError{magicnum}{%
      \string\magicnum\space is already defined%
    }\@ehc
  \fi
\else
  \newcommand*{\magicnum}{}%
\fi
%    \end{macrocode}
%
% \subsection{Without \LuaTeX}
%
%    \begin{macrocode}
\begingroup\expandafter\expandafter\expandafter\endgroup
\expandafter\ifx\csname directlua\endcsname\relax
%    \end{macrocode}
%
%    \begin{macro}{\magicnum}
%    \begin{macrocode}
  \begingroup\expandafter\expandafter\expandafter\endgroup
  \expandafter\ifx\csname ifcsname\endcsname\relax
    \def\magicnum#1{%
      \expandafter\ifx\csname MG@#1\endcsname\relax
      \else
        \csname MG@#1\endcsname
      \fi
    }%
  \else
    \begingroup
      \edef\x{\endgroup
        \def\noexpand\magicnum##1{%
          \expandafter\noexpand\csname
          ifcsname\endcsname MG@##1\noexpand\endcsname
            \noexpand\csname MG@##1%
                 \noexpand\expandafter\noexpand\endcsname
          \expandafter\noexpand\csname fi\endcsname
        }%
      }%
    \x
  \fi
%    \end{macrocode}
%    \end{macro}
%
%    \begin{macrocode}
\else
%    \end{macrocode}
%
% \subsection{With \LuaTeX}
%
%    \begin{macrocode}
  \begingroup\expandafter\expandafter\expandafter\endgroup
  \expandafter\ifx\csname RequirePackage\endcsname\relax
    \input ifluatex.sty\relax
    \input infwarerr.sty\relax
  \else
    \RequirePackage{ifluatex}[2010/03/01]%
    \RequirePackage{infwarerr}[2010/04/08]%
  \fi
%    \end{macrocode}
%
%    \begin{macro}{\magicnum@directlua}
%    \begin{macrocode}
  \ifnum\luatexversion<36 %
    \def\magicnum@directlua{\directlua0 }%
  \else
    \let\magicnum@directlua\directlua
  \fi
%    \end{macrocode}
%    \end{macro}
%    \begin{macrocode}
  \magicnum@directlua{%
    require("oberdiek.magicnum")%
  }%
  \begingroup
    \def\x{2011/04/10 v1.4}%
    \def\StripPrefix#1>{}%
    \edef\x{\expandafter\StripPrefix\meaning\x}%
    \edef\y{%
      \magicnum@directlua{%
        if oberdiek.magicnum.getversion then %
          oberdiek.magicnum.getversion()%
        end%
      }%
    }%
    \ifx\x\y
    \else
      \@PackageError{magicnum}{%
        Wrong version of lua module.\MessageBreak
        Package version: \x\MessageBreak
        Lua module: \y
      }\@ehc
    \fi
  \endgroup
%    \end{macrocode}
%    \begin{macro}{\luaescapestring}
%    \begin{macrocode}
  \begingroup
    \expandafter\ifx\csname luaescapestring\endcsname\relax
      \directlua{%
        if tex.enableprimitives then %
          tex.enableprimitives('magicnum@', {'luaescapestring'})%
        end%
      }%
      \global\let\luaescapestring\magicnum@luaescapestring
    \fi
    \expandafter\ifx\csname luaescapestring\endcsname\relax
      \escapechar=92 %
      \@PackageError{magicnum}{%
        Missing \string\luaescapestring
      }\@ehc
    \fi
  \endgroup
%    \end{macrocode}
%    \end{macro}
%    \begin{macro}{\magicnum}
%    \begin{macrocode}
  \def\magicnum#1{%
    \magicnum@directlua{%
      oberdiek.magicnum.get("\luaescapestring{#1}")%
    }%
  }%
%    \end{macrocode}
%    \end{macro}
%
%    \begin{macrocode}
  \expandafter\magicnum@AtEnd
\fi%
%</package>
%    \end{macrocode}
%
% \subsection{Data}
%
% \subsubsection{Plain data}
%
%    \begin{macrocode}
%<*data>
tex.catcode
  escape = 0
  begingroup = 1
  endgroup = 2
  math = 3
  align = 4
  eol = 5
  parameter = 6
  superscript = 7
  subscript = 8
  ignore = 9
  space = 10
  letter = 11
  other = 12
  active = 13
  comment = 14
  invalid = 15
etex.grouptype
  bottomlevel = 0
  simple = 1
  hbox = 2
  adjustedhbox = 3
  vbox = 4
  align = 5
  noalign = 6
  output = 8
  math = 9
  disc = 10
  insert = 11
  vcenter = 12
  mathchoice = 13
  semisimple = 14
  mathshift = 15
  mathleft = 16
etex.iftype
  none = 0
  char = 1
  cat = 2
  num = 3
  dim = 4
  odd = 5
  vmode = 6
  hmode = 7
  mmode = 8
  inner = 9
  void = 10
  hbox = 11
  vbox = 12
  x = 13
  eof = 14
  true = 15
  false = 16
  case = 17
  defined = 18
  csname = 19
  fontchar = 20
etex.nodetype
  none = -1
  char = 0
  hlist = 1
  vlist = 2
  rule = 3
  ins = 4
  mark = 5
  adjust = 6
  ligature = 7
  disc = 8
  whatsit = 9
  math = 10
  glue = 11
  kern = 12
  penalty = 13
  unset = 14
  maths = 15
etex.interactionmode
  batch = 0
  nonstop = 1
  scroll = 2
  errorstop = 3
luatex.pdfliteral.mode
  setorigin = 0
  page = 1
  direct = 2
%</data>
%    \end{macrocode}
%
% \subsubsection{Data for \TeX}
%
%    \begin{macrocode}
%<*package>
%    \end{macrocode}
%    \begin{macro}{\magicnum@add}
%    \begin{macrocode}
\begingroup\expandafter\expandafter\expandafter\endgroup
\expandafter\ifx\csname detokenize\endcsname\relax
  \def\magicnum@add#1#2#3{%
    \expandafter\magicnum@@add
        \csname MG@#1.#2\expandafter\endcsname
        \csname MG@#1.#3\endcsname
       {#3}{#2}%
  }%
  \def\magicnum@@add#1#2#3#4{%
    \def#1{#3}%
    \def#2{#4}%
    \edef#1{%
      \expandafter\strip@prefix\meaning#1%
    }%
    \edef#2{%
      \expandafter\strip@prefix\meaning#2%
    }%
  }%
  \expandafter\ifx\csname strip@prefix\endcsname\relax
    \def\strip@prefix#1->{}%
  \fi
\else
  \def\magicnum@add#1#2#3{%
    \expandafter\edef\csname MG@#1.#2\endcsname{%
      \detokenize{#3}%
    }%
    \expandafter\edef\csname MG@#1.#3\endcsname{%
      \detokenize{#2}%
    }%
  }%
\fi
%    \end{macrocode}
%    \end{macro}
%    \begin{macrocode}
\magicnum@add{tex.catcode}{escape}{0}
\magicnum@add{tex.catcode}{begingroup}{1}
\magicnum@add{tex.catcode}{endgroup}{2}
\magicnum@add{tex.catcode}{math}{3}
\magicnum@add{tex.catcode}{align}{4}
\magicnum@add{tex.catcode}{eol}{5}
\magicnum@add{tex.catcode}{parameter}{6}
\magicnum@add{tex.catcode}{superscript}{7}
\magicnum@add{tex.catcode}{subscript}{8}
\magicnum@add{tex.catcode}{ignore}{9}
\magicnum@add{tex.catcode}{space}{10}
\magicnum@add{tex.catcode}{letter}{11}
\magicnum@add{tex.catcode}{other}{12}
\magicnum@add{tex.catcode}{active}{13}
\magicnum@add{tex.catcode}{comment}{14}
\magicnum@add{tex.catcode}{invalid}{15}
\magicnum@add{etex.grouptype}{bottomlevel}{0}
\magicnum@add{etex.grouptype}{simple}{1}
\magicnum@add{etex.grouptype}{hbox}{2}
\magicnum@add{etex.grouptype}{adjustedhbox}{3}
\magicnum@add{etex.grouptype}{vbox}{4}
\magicnum@add{etex.grouptype}{align}{5}
\magicnum@add{etex.grouptype}{noalign}{6}
\magicnum@add{etex.grouptype}{output}{8}
\magicnum@add{etex.grouptype}{math}{9}
\magicnum@add{etex.grouptype}{disc}{10}
\magicnum@add{etex.grouptype}{insert}{11}
\magicnum@add{etex.grouptype}{vcenter}{12}
\magicnum@add{etex.grouptype}{mathchoice}{13}
\magicnum@add{etex.grouptype}{semisimple}{14}
\magicnum@add{etex.grouptype}{mathshift}{15}
\magicnum@add{etex.grouptype}{mathleft}{16}
\magicnum@add{etex.iftype}{none}{0}
\magicnum@add{etex.iftype}{char}{1}
\magicnum@add{etex.iftype}{cat}{2}
\magicnum@add{etex.iftype}{num}{3}
\magicnum@add{etex.iftype}{dim}{4}
\magicnum@add{etex.iftype}{odd}{5}
\magicnum@add{etex.iftype}{vmode}{6}
\magicnum@add{etex.iftype}{hmode}{7}
\magicnum@add{etex.iftype}{mmode}{8}
\magicnum@add{etex.iftype}{inner}{9}
\magicnum@add{etex.iftype}{void}{10}
\magicnum@add{etex.iftype}{hbox}{11}
\magicnum@add{etex.iftype}{vbox}{12}
\magicnum@add{etex.iftype}{x}{13}
\magicnum@add{etex.iftype}{eof}{14}
\magicnum@add{etex.iftype}{true}{15}
\magicnum@add{etex.iftype}{false}{16}
\magicnum@add{etex.iftype}{case}{17}
\magicnum@add{etex.iftype}{defined}{18}
\magicnum@add{etex.iftype}{csname}{19}
\magicnum@add{etex.iftype}{fontchar}{20}
\magicnum@add{etex.nodetype}{none}{-1}
\magicnum@add{etex.nodetype}{char}{0}
\magicnum@add{etex.nodetype}{hlist}{1}
\magicnum@add{etex.nodetype}{vlist}{2}
\magicnum@add{etex.nodetype}{rule}{3}
\magicnum@add{etex.nodetype}{ins}{4}
\magicnum@add{etex.nodetype}{mark}{5}
\magicnum@add{etex.nodetype}{adjust}{6}
\magicnum@add{etex.nodetype}{ligature}{7}
\magicnum@add{etex.nodetype}{disc}{8}
\magicnum@add{etex.nodetype}{whatsit}{9}
\magicnum@add{etex.nodetype}{math}{10}
\magicnum@add{etex.nodetype}{glue}{11}
\magicnum@add{etex.nodetype}{kern}{12}
\magicnum@add{etex.nodetype}{penalty}{13}
\magicnum@add{etex.nodetype}{unset}{14}
\magicnum@add{etex.nodetype}{maths}{15}
\magicnum@add{etex.interactionmode}{batch}{0}
\magicnum@add{etex.interactionmode}{nonstop}{1}
\magicnum@add{etex.interactionmode}{scroll}{2}
\magicnum@add{etex.interactionmode}{errorstop}{3}
\magicnum@add{luatex.pdfliteral.mode}{setorigin}{0}
\magicnum@add{luatex.pdfliteral.mode}{page}{1}
\magicnum@add{luatex.pdfliteral.mode}{direct}{2}
%    \end{macrocode}
%    \begin{macrocode}
\magicnum@AtEnd%
%</package>
%    \end{macrocode}
%
% \subsubsection{Lua module}
%
%    \begin{macrocode}
%<*lua>
%    \end{macrocode}
%    \begin{macrocode}
module("oberdiek.magicnum", package.seeall)
%    \end{macrocode}
%    \begin{macrocode}
function getversion()
  tex.write("2011/04/10 v1.4")
end
%    \end{macrocode}
%    \begin{macrocode}
local data = {
  ["tex.catcode"] = {
    [0] = "escape",
    [1] = "begingroup",
    [2] = "endgroup",
    [3] = "math",
    [4] = "align",
    [5] = "eol",
    [6] = "parameter",
    [7] = "superscript",
    [8] = "subscript",
    [9] = "ignore",
    [10] = "space",
    [11] = "letter",
    [12] = "other",
    [13] = "active",
    [14] = "comment",
    [15] = "invalid",
    ["active"] = 13,
    ["align"] = 4,
    ["begingroup"] = 1,
    ["comment"] = 14,
    ["endgroup"] = 2,
    ["eol"] = 5,
    ["escape"] = 0,
    ["ignore"] = 9,
    ["invalid"] = 15,
    ["letter"] = 11,
    ["math"] = 3,
    ["other"] = 12,
    ["parameter"] = 6,
    ["space"] = 10,
    ["subscript"] = 8,
    ["superscript"] = 7
  },
  ["etex.grouptype"] = {
    [0] = "bottomlevel",
    [1] = "simple",
    [2] = "hbox",
    [3] = "adjustedhbox",
    [4] = "vbox",
    [5] = "align",
    [6] = "noalign",
    [8] = "output",
    [9] = "math",
    [10] = "disc",
    [11] = "insert",
    [12] = "vcenter",
    [13] = "mathchoice",
    [14] = "semisimple",
    [15] = "mathshift",
    [16] = "mathleft",
    ["adjustedhbox"] = 3,
    ["align"] = 5,
    ["bottomlevel"] = 0,
    ["disc"] = 10,
    ["hbox"] = 2,
    ["insert"] = 11,
    ["math"] = 9,
    ["mathchoice"] = 13,
    ["mathleft"] = 16,
    ["mathshift"] = 15,
    ["noalign"] = 6,
    ["output"] = 8,
    ["semisimple"] = 14,
    ["simple"] = 1,
    ["vbox"] = 4,
    ["vcenter"] = 12
  },
  ["etex.iftype"] = {
    [0] = "none",
    [1] = "char",
    [2] = "cat",
    [3] = "num",
    [4] = "dim",
    [5] = "odd",
    [6] = "vmode",
    [7] = "hmode",
    [8] = "mmode",
    [9] = "inner",
    [10] = "void",
    [11] = "hbox",
    [12] = "vbox",
    [13] = "x",
    [14] = "eof",
    [15] = "true",
    [16] = "false",
    [17] = "case",
    [18] = "defined",
    [19] = "csname",
    [20] = "fontchar",
    ["case"] = 17,
    ["cat"] = 2,
    ["char"] = 1,
    ["csname"] = 19,
    ["defined"] = 18,
    ["dim"] = 4,
    ["eof"] = 14,
    ["false"] = 16,
    ["fontchar"] = 20,
    ["hbox"] = 11,
    ["hmode"] = 7,
    ["inner"] = 9,
    ["mmode"] = 8,
    ["none"] = 0,
    ["num"] = 3,
    ["odd"] = 5,
    ["true"] = 15,
    ["vbox"] = 12,
    ["vmode"] = 6,
    ["void"] = 10,
    ["x"] = 13
  },
  ["etex.nodetype"] = {
    [-1] = "none",
    [0] = "char",
    [1] = "hlist",
    [2] = "vlist",
    [3] = "rule",
    [4] = "ins",
    [5] = "mark",
    [6] = "adjust",
    [7] = "ligature",
    [8] = "disc",
    [9] = "whatsit",
    [10] = "math",
    [11] = "glue",
    [12] = "kern",
    [13] = "penalty",
    [14] = "unset",
    [15] = "maths",
    ["adjust"] = 6,
    ["char"] = 0,
    ["disc"] = 8,
    ["glue"] = 11,
    ["hlist"] = 1,
    ["ins"] = 4,
    ["kern"] = 12,
    ["ligature"] = 7,
    ["mark"] = 5,
    ["math"] = 10,
    ["maths"] = 15,
    ["none"] = -1,
    ["penalty"] = 13,
    ["rule"] = 3,
    ["unset"] = 14,
    ["vlist"] = 2,
    ["whatsit"] = 9
  },
  ["etex.interactionmode"] = {
    [0] = "batch",
    [1] = "nonstop",
    [2] = "scroll",
    [3] = "errorstop",
    ["batch"] = 0,
    ["errorstop"] = 3,
    ["nonstop"] = 1,
    ["scroll"] = 2
  },
  ["luatex.pdfliteral.mode"] = {
    [0] = "setorigin",
    [1] = "page",
    [2] = "direct",
    ["direct"] = 2,
    ["page"] = 1,
    ["setorigin"] = 0
  }
}
%    \end{macrocode}
%    \begin{macrocode}
function get(name)
  local startpos, endpos, category, entry =
      string.find(name, "^(%a[%a%d%.]*)%.(-?[%a%d]+)$")
  if not entry then
    return
  end
  local node = data[category]
  if not node then
    return
  end
  local num = tonumber(entry)
  local value
  if num then
    value = node[num]
    if not value then
      return
    end
  else
    value = node[entry]
    if not value then
      return
    end
    value = "" .. value
  end
  tex.write(value)
end
%    \end{macrocode}
%
%    \begin{macrocode}
%</lua>
%    \end{macrocode}
%
% \section{Test}
%
% \subsection{Catcode checks for loading}
%
%    \begin{macrocode}
%<*test1>
%    \end{macrocode}
%    \begin{macrocode}
\catcode`\{=1 %
\catcode`\}=2 %
\catcode`\#=6 %
\catcode`\@=11 %
\expandafter\ifx\csname count@\endcsname\relax
  \countdef\count@=255 %
\fi
\expandafter\ifx\csname @gobble\endcsname\relax
  \long\def\@gobble#1{}%
\fi
\expandafter\ifx\csname @firstofone\endcsname\relax
  \long\def\@firstofone#1{#1}%
\fi
\expandafter\ifx\csname loop\endcsname\relax
  \expandafter\@firstofone
\else
  \expandafter\@gobble
\fi
{%
  \def\loop#1\repeat{%
    \def\body{#1}%
    \iterate
  }%
  \def\iterate{%
    \body
      \let\next\iterate
    \else
      \let\next\relax
    \fi
    \next
  }%
  \let\repeat=\fi
}%
\def\RestoreCatcodes{}
\count@=0 %
\loop
  \edef\RestoreCatcodes{%
    \RestoreCatcodes
    \catcode\the\count@=\the\catcode\count@\relax
  }%
\ifnum\count@<255 %
  \advance\count@ 1 %
\repeat

\def\RangeCatcodeInvalid#1#2{%
  \count@=#1\relax
  \loop
    \catcode\count@=15 %
  \ifnum\count@<#2\relax
    \advance\count@ 1 %
  \repeat
}
\def\RangeCatcodeCheck#1#2#3{%
  \count@=#1\relax
  \loop
    \ifnum#3=\catcode\count@
    \else
      \errmessage{%
        Character \the\count@\space
        with wrong catcode \the\catcode\count@\space
        instead of \number#3%
      }%
    \fi
  \ifnum\count@<#2\relax
    \advance\count@ 1 %
  \repeat
}
\def\space{ }
\expandafter\ifx\csname LoadCommand\endcsname\relax
  \def\LoadCommand{\input magicnum.sty\relax}%
\fi
\def\Test{%
  \RangeCatcodeInvalid{0}{47}%
  \RangeCatcodeInvalid{58}{64}%
  \RangeCatcodeInvalid{91}{96}%
  \RangeCatcodeInvalid{123}{255}%
  \catcode`\@=12 %
  \catcode`\\=0 %
  \catcode`\%=14 %
  \LoadCommand
  \RangeCatcodeCheck{0}{36}{15}%
  \RangeCatcodeCheck{37}{37}{14}%
  \RangeCatcodeCheck{38}{47}{15}%
  \RangeCatcodeCheck{48}{57}{12}%
  \RangeCatcodeCheck{58}{63}{15}%
  \RangeCatcodeCheck{64}{64}{12}%
  \RangeCatcodeCheck{65}{90}{11}%
  \RangeCatcodeCheck{91}{91}{15}%
  \RangeCatcodeCheck{92}{92}{0}%
  \RangeCatcodeCheck{93}{96}{15}%
  \RangeCatcodeCheck{97}{122}{11}%
  \RangeCatcodeCheck{123}{255}{15}%
  \RestoreCatcodes
}
\Test
\csname @@end\endcsname
\end
%    \end{macrocode}
%    \begin{macrocode}
%</test1>
%    \end{macrocode}
%
% \subsection{Test data}
%
%    \begin{macrocode}
%<*testplain>
\input magicnum.sty\relax
\def\Test#1#2{%
  \edef\result{\magicnum{#1}}%
  \edef\expect{#2}%
  \edef\expect{\expandafter\stripprefix\meaning\expect}%
  \ifx\result\expect
  \else
    \errmessage{%
      Failed: [#1] % hash-ok
      returns [\result] instead of [\expect]%
    }%
  \fi
}
\def\stripprefix#1->{}
%</testplain>
%    \end{macrocode}
%    \begin{macrocode}
%<*testlatex>
\NeedsTeXFormat{LaTeX2e}
\documentclass{minimal}
\usepackage{magicnum}[2011/04/10]
\usepackage{qstest}
\IncludeTests{*}
\LogTests{log}{*}{*}
\newcommand*{\Test}[2]{%
  \Expect*{\magicnum{#1}}{#2}%
}
\begin{qstest}{magicnum}{magicnum}
%</testlatex>
%    \end{macrocode}
%    \begin{macrocode}
%<*testdata>
\Test{tex.catcode.escape}{0}
\Test{tex.catcode.invalid}{15}
\Test{tex.catcode.unknown}{}
\Test{tex.catcode.0}{escape}
\Test{tex.catcode.15}{invalid}
\Test{etex.iftype.true}{15}
\Test{etex.iftype.false}{16}
\Test{etex.iftype.15}{true}
\Test{etex.iftype.16}{false}
\Test{etex.nodetype.none}{-1}
\Test{etex.nodetype.-1}{none}
\Test{luatex.pdfliteral.mode.direct}{2}
\Test{luatex.pdfliteral.mode.1}{page}
\Test{}{}
\Test{unknown}{}
\Test{unknown.foo.bar}{}
\Test{unknown.foo.4}{}
%</testdata>
%    \end{macrocode}
%    \begin{macrocode}
%<*testplain>
\csname @@end\endcsname
\end
%</testplain>
%<*testlatex>
\end{qstest}
\csname @@end\endcsname
%</testlatex>
%    \end{macrocode}
%
% \subsection{Small test for \hologo{iniTeX}}
%
%    \begin{macrocode}
%<*test4>
\catcode`\{=1
\catcode`\}=2
\catcode`\#=6
\input magicnum.sty\relax
\edef\x{\magicnum{tex.catcode.15}}
\edef\y{invalid}
\def\Strip#1>{}
\edef\y{\expandafter\Strip\meaning\y}
\ifx\x\y
  \immediate\write16{Ok}%
\else
  \errmessage{\x<>\y}%
\fi
\csname @@end\endcsname\end
%</test4>
%    \end{macrocode}
%
% \section{Installation}
%
% \subsection{Download}
%
% \paragraph{Package.} This package is available on
% CTAN\footnote{\url{ftp://ftp.ctan.org/tex-archive/}}:
% \begin{description}
% \item[\CTAN{macros/latex/contrib/oberdiek/magicnum.dtx}] The source file.
% \item[\CTAN{macros/latex/contrib/oberdiek/magicnum.pdf}] Documentation.
% \end{description}
%
%
% \paragraph{Bundle.} All the packages of the bundle `oberdiek'
% are also available in a TDS compliant ZIP archive. There
% the packages are already unpacked and the documentation files
% are generated. The files and directories obey the TDS standard.
% \begin{description}
% \item[\CTAN{install/macros/latex/contrib/oberdiek.tds.zip}]
% \end{description}
% \emph{TDS} refers to the standard ``A Directory Structure
% for \TeX\ Files'' (\CTAN{tds/tds.pdf}). Directories
% with \xfile{texmf} in their name are usually organized this way.
%
% \subsection{Bundle installation}
%
% \paragraph{Unpacking.} Unpack the \xfile{oberdiek.tds.zip} in the
% TDS tree (also known as \xfile{texmf} tree) of your choice.
% Example (linux):
% \begin{quote}
%   |unzip oberdiek.tds.zip -d ~/texmf|
% \end{quote}
%
% \paragraph{Script installation.}
% Check the directory \xfile{TDS:scripts/oberdiek/} for
% scripts that need further installation steps.
% Package \xpackage{attachfile2} comes with the Perl script
% \xfile{pdfatfi.pl} that should be installed in such a way
% that it can be called as \texttt{pdfatfi}.
% Example (linux):
% \begin{quote}
%   |chmod +x scripts/oberdiek/pdfatfi.pl|\\
%   |cp scripts/oberdiek/pdfatfi.pl /usr/local/bin/|
% \end{quote}
%
% \subsection{Package installation}
%
% \paragraph{Unpacking.} The \xfile{.dtx} file is a self-extracting
% \docstrip\ archive. The files are extracted by running the
% \xfile{.dtx} through \plainTeX:
% \begin{quote}
%   \verb|tex magicnum.dtx|
% \end{quote}
%
% \paragraph{TDS.} Now the different files must be moved into
% the different directories in your installation TDS tree
% (also known as \xfile{texmf} tree):
% \begin{quote}
% \def\t{^^A
% \begin{tabular}{@{}>{\ttfamily}l@{ $\rightarrow$ }>{\ttfamily}l@{}}
%   magicnum.sty & tex/generic/oberdiek/magicnum.sty\\
%   magicnum.lua & scripts/oberdiek/magicnum.lua\\
%   oberdiek.magicnum.lua & scripts/oberdiek/oberdiek.magicnum.lua\\
%   magicnum.pdf & doc/latex/oberdiek/magicnum.pdf\\
%   magicnum.txt & doc/latex/oberdiek/magicnum.txt\\
%   test/magicnum-test1.tex & doc/latex/oberdiek/test/magicnum-test1.tex\\
%   test/magicnum-test2.tex & doc/latex/oberdiek/test/magicnum-test2.tex\\
%   test/magicnum-test3.tex & doc/latex/oberdiek/test/magicnum-test3.tex\\
%   test/magicnum-test4.tex & doc/latex/oberdiek/test/magicnum-test4.tex\\
%   magicnum.dtx & source/latex/oberdiek/magicnum.dtx\\
% \end{tabular}^^A
% }^^A
% \sbox0{\t}^^A
% \ifdim\wd0>\linewidth
%   \begingroup
%     \advance\linewidth by\leftmargin
%     \advance\linewidth by\rightmargin
%   \edef\x{\endgroup
%     \def\noexpand\lw{\the\linewidth}^^A
%   }\x
%   \def\lwbox{^^A
%     \leavevmode
%     \hbox to \linewidth{^^A
%       \kern-\leftmargin\relax
%       \hss
%       \usebox0
%       \hss
%       \kern-\rightmargin\relax
%     }^^A
%   }^^A
%   \ifdim\wd0>\lw
%     \sbox0{\small\t}^^A
%     \ifdim\wd0>\linewidth
%       \ifdim\wd0>\lw
%         \sbox0{\footnotesize\t}^^A
%         \ifdim\wd0>\linewidth
%           \ifdim\wd0>\lw
%             \sbox0{\scriptsize\t}^^A
%             \ifdim\wd0>\linewidth
%               \ifdim\wd0>\lw
%                 \sbox0{\tiny\t}^^A
%                 \ifdim\wd0>\linewidth
%                   \lwbox
%                 \else
%                   \usebox0
%                 \fi
%               \else
%                 \lwbox
%               \fi
%             \else
%               \usebox0
%             \fi
%           \else
%             \lwbox
%           \fi
%         \else
%           \usebox0
%         \fi
%       \else
%         \lwbox
%       \fi
%     \else
%       \usebox0
%     \fi
%   \else
%     \lwbox
%   \fi
% \else
%   \usebox0
% \fi
% \end{quote}
% If you have a \xfile{docstrip.cfg} that configures and enables \docstrip's
% TDS installing feature, then some files can already be in the right
% place, see the documentation of \docstrip.
%
% \subsection{Refresh file name databases}
%
% If your \TeX~distribution
% (\teTeX, \mikTeX, \dots) relies on file name databases, you must refresh
% these. For example, \teTeX\ users run \verb|texhash| or
% \verb|mktexlsr|.
%
% \subsection{Some details for the interested}
%
% \paragraph{Attached source.}
%
% The PDF documentation on CTAN also includes the
% \xfile{.dtx} source file. It can be extracted by
% AcrobatReader 6 or higher. Another option is \textsf{pdftk},
% e.g. unpack the file into the current directory:
% \begin{quote}
%   \verb|pdftk magicnum.pdf unpack_files output .|
% \end{quote}
%
% \paragraph{Unpacking with \LaTeX.}
% The \xfile{.dtx} chooses its action depending on the format:
% \begin{description}
% \item[\plainTeX:] Run \docstrip\ and extract the files.
% \item[\LaTeX:] Generate the documentation.
% \end{description}
% If you insist on using \LaTeX\ for \docstrip\ (really,
% \docstrip\ does not need \LaTeX), then inform the autodetect routine
% about your intention:
% \begin{quote}
%   \verb|latex \let\install=y% \iffalse meta-comment
%
% File: magicnum.dtx
% Version: 2011/04/10 v1.4
% Info: Magic numbers
%
% Copyright (C) 2007, 2009-2011 by
%    Heiko Oberdiek <heiko.oberdiek at googlemail.com>
%
% This work may be distributed and/or modified under the
% conditions of the LaTeX Project Public License, either
% version 1.3c of this license or (at your option) any later
% version. This version of this license is in
%    http://www.latex-project.org/lppl/lppl-1-3c.txt
% and the latest version of this license is in
%    http://www.latex-project.org/lppl.txt
% and version 1.3 or later is part of all distributions of
% LaTeX version 2005/12/01 or later.
%
% This work has the LPPL maintenance status "maintained".
%
% This Current Maintainer of this work is Heiko Oberdiek.
%
% The Base Interpreter refers to any `TeX-Format',
% because some files are installed in TDS:tex/generic//.
%
% This work consists of the main source file magicnum.dtx
% and the derived files
%    magicnum.sty, magicnum.pdf, magicnum.ins, magicnum.drv, magicnum.txt,
%    magicnum-test1.tex, magicnum-test2.tex, magicnum-test3.tex,
%    magicnum-test4.tex, magicnum.lua, oberdiek.magicnum.lua.
%
% Distribution:
%    CTAN:macros/latex/contrib/oberdiek/magicnum.dtx
%    CTAN:macros/latex/contrib/oberdiek/magicnum.pdf
%
% Unpacking:
%    (a) If magicnum.ins is present:
%           tex magicnum.ins
%    (b) Without magicnum.ins:
%           tex magicnum.dtx
%    (c) If you insist on using LaTeX
%           latex \let\install=y\input{magicnum.dtx}
%        (quote the arguments according to the demands of your shell)
%
% Documentation:
%    (a) If magicnum.drv is present:
%           latex magicnum.drv
%    (b) Without magicnum.drv:
%           latex magicnum.dtx; ...
%    The class ltxdoc loads the configuration file ltxdoc.cfg
%    if available. Here you can specify further options, e.g.
%    use A4 as paper format:
%       \PassOptionsToClass{a4paper}{article}
%
%    Programm calls to get the documentation (example):
%       pdflatex magicnum.dtx
%       makeindex -s gind.ist magicnum.idx
%       pdflatex magicnum.dtx
%       makeindex -s gind.ist magicnum.idx
%       pdflatex magicnum.dtx
%
% Installation:
%    TDS:tex/generic/oberdiek/magicnum.sty
%    TDS:scripts/oberdiek/magicnum.lua
%    TDS:scripts/oberdiek/oberdiek.magicnum.lua
%    TDS:doc/latex/oberdiek/magicnum.pdf
%    TDS:doc/latex/oberdiek/magicnum.txt
%    TDS:doc/latex/oberdiek/test/magicnum-test1.tex
%    TDS:doc/latex/oberdiek/test/magicnum-test2.tex
%    TDS:doc/latex/oberdiek/test/magicnum-test3.tex
%    TDS:doc/latex/oberdiek/test/magicnum-test4.tex
%    TDS:source/latex/oberdiek/magicnum.dtx
%
%<*ignore>
\begingroup
  \catcode123=1 %
  \catcode125=2 %
  \def\x{LaTeX2e}%
\expandafter\endgroup
\ifcase 0\ifx\install y1\fi\expandafter
         \ifx\csname processbatchFile\endcsname\relax\else1\fi
         \ifx\fmtname\x\else 1\fi\relax
\else\csname fi\endcsname
%</ignore>
%<*install>
\input docstrip.tex
\Msg{************************************************************************}
\Msg{* Installation}
\Msg{* Package: magicnum 2011/04/10 v1.4 Magic numbers (HO)}
\Msg{************************************************************************}

\keepsilent
\askforoverwritefalse

\let\MetaPrefix\relax
\preamble

This is a generated file.

Project: magicnum
Version: 2011/04/10 v1.4

Copyright (C) 2007, 2009-2011 by
   Heiko Oberdiek <heiko.oberdiek at googlemail.com>

This work may be distributed and/or modified under the
conditions of the LaTeX Project Public License, either
version 1.3c of this license or (at your option) any later
version. This version of this license is in
   http://www.latex-project.org/lppl/lppl-1-3c.txt
and the latest version of this license is in
   http://www.latex-project.org/lppl.txt
and version 1.3 or later is part of all distributions of
LaTeX version 2005/12/01 or later.

This work has the LPPL maintenance status "maintained".

This Current Maintainer of this work is Heiko Oberdiek.

The Base Interpreter refers to any `TeX-Format',
because some files are installed in TDS:tex/generic//.

This work consists of the main source file magicnum.dtx
and the derived files
   magicnum.sty, magicnum.pdf, magicnum.ins, magicnum.drv, magicnum.txt,
   magicnum-test1.tex, magicnum-test2.tex, magicnum-test3.tex,
   magicnum-test4.tex, magicnum.lua, oberdiek.magicnum.lua.

\endpreamble
\let\MetaPrefix\DoubleperCent

\generate{%
  \file{magicnum.ins}{\from{magicnum.dtx}{install}}%
  \file{magicnum.drv}{\from{magicnum.dtx}{driver}}%
  \usedir{tex/generic/oberdiek}%
  \file{magicnum.sty}{\from{magicnum.dtx}{package}}%
  \usedir{doc/latex/oberdiek/test}%
  \file{magicnum-test1.tex}{\from{magicnum.dtx}{test1}}%
  \file{magicnum-test2.tex}{\from{magicnum.dtx}{testplain,testdata}}%
  \file{magicnum-test3.tex}{\from{magicnum.dtx}{testlatex,testdata}}%
  \file{magicnum-test4.tex}{\from{magicnum.dtx}{test4}}%
  \nopreamble
  \nopostamble
  \usedir{doc/latex/oberdiek}%
  \file{magicnum.txt}{\from{magicnum.dtx}{data}}%
  \usedir{source/latex/oberdiek/catalogue}%
  \file{magicnum.xml}{\from{magicnum.dtx}{catalogue}}%
}
\def\MetaPrefix{-- }
\def\defaultpostamble{%
  \MetaPrefix^^J%
  \MetaPrefix\space End of File `\outFileName'.%
}
\def\currentpostamble{\defaultpostamble}%
\generate{%
  \usedir{scripts/oberdiek}%
  \file{magicnum.lua}{\from{magicnum.dtx}{lua}}%
  \file{oberdiek.magicnum.lua}{\from{magicnum.dtx}{lua}}%
}

\catcode32=13\relax% active space
\let =\space%
\Msg{************************************************************************}
\Msg{*}
\Msg{* To finish the installation you have to move the following}
\Msg{* file into a directory searched by TeX:}
\Msg{*}
\Msg{*     magicnum.sty}
\Msg{*}
\Msg{* And install the following script files:}
\Msg{*}
\Msg{*     magicnum.lua, oberdiek.magicnum.lua}
\Msg{*}
\Msg{* To produce the documentation run the file `magicnum.drv'}
\Msg{* through LaTeX.}
\Msg{*}
\Msg{* Happy TeXing!}
\Msg{*}
\Msg{************************************************************************}

\endbatchfile
%</install>
%<*ignore>
\fi
%</ignore>
%<*driver>
\NeedsTeXFormat{LaTeX2e}
\ProvidesFile{magicnum.drv}%
  [2011/04/10 v1.4 Magic numbers (HO)]%
\documentclass{ltxdoc}
\usepackage{holtxdoc}[2011/11/22]
\usepackage{array}
\begin{document}
  \DocInput{magicnum.dtx}%
\end{document}
%</driver>
% \fi
%
% \CheckSum{755}
%
% \CharacterTable
%  {Upper-case    \A\B\C\D\E\F\G\H\I\J\K\L\M\N\O\P\Q\R\S\T\U\V\W\X\Y\Z
%   Lower-case    \a\b\c\d\e\f\g\h\i\j\k\l\m\n\o\p\q\r\s\t\u\v\w\x\y\z
%   Digits        \0\1\2\3\4\5\6\7\8\9
%   Exclamation   \!     Double quote  \"     Hash (number) \#
%   Dollar        \$     Percent       \%     Ampersand     \&
%   Acute accent  \'     Left paren    \(     Right paren   \)
%   Asterisk      \*     Plus          \+     Comma         \,
%   Minus         \-     Point         \.     Solidus       \/
%   Colon         \:     Semicolon     \;     Less than     \<
%   Equals        \=     Greater than  \>     Question mark \?
%   Commercial at \@     Left bracket  \[     Backslash     \\
%   Right bracket \]     Circumflex    \^     Underscore    \_
%   Grave accent  \`     Left brace    \{     Vertical bar  \|
%   Right brace   \}     Tilde         \~}
%
% \GetFileInfo{magicnum.drv}
%
% \title{The \xpackage{magicnum} package}
% \date{2011/04/10 v1.4}
% \author{Heiko Oberdiek\\\xemail{heiko.oberdiek at googlemail.com}}
%
% \maketitle
%
% \begin{abstract}
% This packages allows to access magic numbers by a hierarchical
% name system.
% \end{abstract}
%
% \tableofcontents
%
% \hypersetup{bookmarksopenlevel=2}
% \section{Documentation}
%
% \subsection{Introduction}
%
% Especially since \eTeX\ there are many integer values
% with special meanings, such as catcodes, group types, \dots
% Package \xpackage{etex}, enabled by options, defines
% macros in the user namespace for these values.
%
% This package goes another approach for storing the names and values.
% \begin{itemize}
% \item If \LuaTeX\ is available, they
% are stored in Lua tables.
% \item Without \LuaTeX\ they are remembered using internal
% macros.
% \end{itemize}
%
% \subsection{User interface}
%
% The integer values and names are organized in a hierarchical
% scheme of categories with the property names as leaves.
% Example: \eTeX's \cs{currentgrouplevel} reports |2| for a
% group caused by \cs{hbox}. This package has choosen to organize
% the group types in a main category |etex| and its subcategory
% |grouptype|:
% \begin{quote}
%   |etex.grouptype.hbox| = |2|
% \end{quote}
% The property name |hbox| in category |etex.grouptype| has value |2|.
% Dots are used to separate components.
%
% If you want to have the value, the access key is constructed by
% the category with all its components and the property name.
% For the opposite the value is used instead of the property name.
%
% Values are always integers (including negative numbers).
%
% \subsubsection{\cs{magicnum}}
%
% \begin{declcs}{magicnum} \M{access key}
% \end{declcs}
% Macro \cs{magicnum} expects an access key as argument and
% expands to the requested data. The macro is always expandable.
% In case of errors the expansion result is empty.
%
% The same macro is also used for getting a property name.
% In this case the property name part in the access key is
% replaced by the value.
%
% The catcodes
% of the resulting numbers and strings follow \TeX's tradition of
% \cs{string}, \cs{meaning}, \dots: The space has catcode 10
% (|tex.catcode.space|) and the other characters have catcode
% 12 (|tex.catcode.other|).
%
% Examples:
% \begin{quote}
%   |\magicnum{etex.grouptype.hbox}| $\Rightarrow$ |2|\\
%   |\magicnum{tex.catcode.14}| $\Rightarrow$ |comment|\\
%   |\magicnum{tex.catcode.undefined}| $\Rightarrow$ $\emptyset$
% \end{quote}
%
% \subsubsection{Properties}
%
% \begin{itemize}
% \item The components of a category are either subcategories or
%       key value pairs, but not both.
% \item The full specified property names are unique and thus
%       has one integer value exactly.
% \item Also the values inside a category are unique.
%       This condition is a prerequisite for the reverse mapping
%       of \cs{magicnum}.
% \item All names start with a letter. Only letters or digits
%       may follow.
% \end{itemize}
%
% \subsection{Data}
%
%  \subsubsection{\texorpdfstring{Category }{}\texttt{tex.catcode}}
%
% \begin{quote}
% \begin{tabular}{@{}>{\ttfamily}l>{\ttfamily}l@{}}
%    tex.catcode.escape & 0\\
%    tex.catcode.begingroup & 1\\
%    tex.catcode.endgroup & 2\\
%    tex.catcode.math & 3\\
%    tex.catcode.align & 4\\
%    tex.catcode.eol & 5\\
%    tex.catcode.parameter & 6\\
%    tex.catcode.superscript & 7\\
%    tex.catcode.subscript & 8\\
%    tex.catcode.ignore & 9\\
%    tex.catcode.space & 10\\
%    tex.catcode.letter & 11\\
%    tex.catcode.other & 12\\
%    tex.catcode.active & 13\\
%    tex.catcode.comment & 14\\
%    tex.catcode.invalid & 15\\
%  \end{tabular}
%  \end{quote}
%
%  \subsubsection{\texorpdfstring{Category }{}\texttt{etex.grouptype}}
%
% \begin{quote}
% \begin{tabular}{@{}>{\ttfamily}l>{\ttfamily}l@{}}
%    etex.grouptype.bottomlevel & 0\\
%    etex.grouptype.simple & 1\\
%    etex.grouptype.hbox & 2\\
%    etex.grouptype.adjustedhbox & 3\\
%    etex.grouptype.vbox & 4\\
%    etex.grouptype.align & 5\\
%    etex.grouptype.noalign & 6\\
%    etex.grouptype.output & 8\\
%    etex.grouptype.math & 9\\
%    etex.grouptype.disc & 10\\
%    etex.grouptype.insert & 11\\
%    etex.grouptype.vcenter & 12\\
%    etex.grouptype.mathchoice & 13\\
%    etex.grouptype.semisimple & 14\\
%    etex.grouptype.mathshift & 15\\
%    etex.grouptype.mathleft & 16\\
%  \end{tabular}
%  \end{quote}
%
%  \subsubsection{\texorpdfstring{Category }{}\texttt{etex.iftype}}
%
% \begin{quote}
% \begin{tabular}{@{}>{\ttfamily}l>{\ttfamily}l@{}}
%    etex.iftype.none & 0\\
%    etex.iftype.char & 1\\
%    etex.iftype.cat & 2\\
%    etex.iftype.num & 3\\
%    etex.iftype.dim & 4\\
%    etex.iftype.odd & 5\\
%    etex.iftype.vmode & 6\\
%    etex.iftype.hmode & 7\\
%    etex.iftype.mmode & 8\\
%    etex.iftype.inner & 9\\
%    etex.iftype.void & 10\\
%    etex.iftype.hbox & 11\\
%    etex.iftype.vbox & 12\\
%    etex.iftype.x & 13\\
%    etex.iftype.eof & 14\\
%    etex.iftype.true & 15\\
%    etex.iftype.false & 16\\
%    etex.iftype.case & 17\\
%    etex.iftype.defined & 18\\
%    etex.iftype.csname & 19\\
%    etex.iftype.fontchar & 20\\
%  \end{tabular}
%  \end{quote}
%
%  \subsubsection{\texorpdfstring{Category }{}\texttt{etex.nodetype}}
%
% \begin{quote}
% \begin{tabular}{@{}>{\ttfamily}l>{\ttfamily}l@{}}
%    etex.nodetype.none & -1\\
%    etex.nodetype.char & 0\\
%    etex.nodetype.hlist & 1\\
%    etex.nodetype.vlist & 2\\
%    etex.nodetype.rule & 3\\
%    etex.nodetype.ins & 4\\
%    etex.nodetype.mark & 5\\
%    etex.nodetype.adjust & 6\\
%    etex.nodetype.ligature & 7\\
%    etex.nodetype.disc & 8\\
%    etex.nodetype.whatsit & 9\\
%    etex.nodetype.math & 10\\
%    etex.nodetype.glue & 11\\
%    etex.nodetype.kern & 12\\
%    etex.nodetype.penalty & 13\\
%    etex.nodetype.unset & 14\\
%    etex.nodetype.maths & 15\\
%  \end{tabular}
%  \end{quote}
%
%  \subsubsection{\texorpdfstring{Category }{}\texttt{etex.interactionmode}}
%
% \begin{quote}
% \begin{tabular}{@{}>{\ttfamily}l>{\ttfamily}l@{}}
%    etex.interactionmode.batch & 0\\
%    etex.interactionmode.nonstop & 1\\
%    etex.interactionmode.scroll & 2\\
%    etex.interactionmode.errorstop & 3\\
%  \end{tabular}
%  \end{quote}
%
%  \subsubsection{\texorpdfstring{Category }{}\texttt{luatex.pdfliteral.mode}}
%
% \begin{quote}
% \begin{tabular}{@{}>{\ttfamily}l>{\ttfamily}l@{}}
%    luatex.pdfliteral.mode.setorigin & 0\\
%    luatex.pdfliteral.mode.page & 1\\
%    luatex.pdfliteral.mode.direct & 2\\
%  \end{tabular}
%  \end{quote}
%
%
% \hypersetup{bookmarksopenlevel=1}
%
% \StopEventually{
% }
%
% \section{Implementation}
%
%    \begin{macrocode}
%<*package>
%    \end{macrocode}
%
% \subsection{Reload check and package identification}
%    Reload check, especially if the package is not used with \LaTeX.
%    \begin{macrocode}
\begingroup\catcode61\catcode48\catcode32=10\relax%
  \catcode13=5 % ^^M
  \endlinechar=13 %
  \catcode35=6 % #
  \catcode39=12 % '
  \catcode44=12 % ,
  \catcode45=12 % -
  \catcode46=12 % .
  \catcode58=12 % :
  \catcode64=11 % @
  \catcode123=1 % {
  \catcode125=2 % }
  \expandafter\let\expandafter\x\csname ver@magicnum.sty\endcsname
  \ifx\x\relax % plain-TeX, first loading
  \else
    \def\empty{}%
    \ifx\x\empty % LaTeX, first loading,
      % variable is initialized, but \ProvidesPackage not yet seen
    \else
      \expandafter\ifx\csname PackageInfo\endcsname\relax
        \def\x#1#2{%
          \immediate\write-1{Package #1 Info: #2.}%
        }%
      \else
        \def\x#1#2{\PackageInfo{#1}{#2, stopped}}%
      \fi
      \x{magicnum}{The package is already loaded}%
      \aftergroup\endinput
    \fi
  \fi
\endgroup%
%    \end{macrocode}
%    Package identification:
%    \begin{macrocode}
\begingroup\catcode61\catcode48\catcode32=10\relax%
  \catcode13=5 % ^^M
  \endlinechar=13 %
  \catcode35=6 % #
  \catcode39=12 % '
  \catcode40=12 % (
  \catcode41=12 % )
  \catcode44=12 % ,
  \catcode45=12 % -
  \catcode46=12 % .
  \catcode47=12 % /
  \catcode58=12 % :
  \catcode64=11 % @
  \catcode91=12 % [
  \catcode93=12 % ]
  \catcode123=1 % {
  \catcode125=2 % }
  \expandafter\ifx\csname ProvidesPackage\endcsname\relax
    \def\x#1#2#3[#4]{\endgroup
      \immediate\write-1{Package: #3 #4}%
      \xdef#1{#4}%
    }%
  \else
    \def\x#1#2[#3]{\endgroup
      #2[{#3}]%
      \ifx#1\@undefined
        \xdef#1{#3}%
      \fi
      \ifx#1\relax
        \xdef#1{#3}%
      \fi
    }%
  \fi
\expandafter\x\csname ver@magicnum.sty\endcsname
\ProvidesPackage{magicnum}%
  [2011/04/10 v1.4 Magic numbers (HO)]%
%    \end{macrocode}
%
% \subsection{Catcodes}
%
%    \begin{macrocode}
\begingroup\catcode61\catcode48\catcode32=10\relax%
  \catcode13=5 % ^^M
  \endlinechar=13 %
  \catcode123=1 % {
  \catcode125=2 % }
  \catcode64=11 % @
  \def\x{\endgroup
    \expandafter\edef\csname magicnum@AtEnd\endcsname{%
      \endlinechar=\the\endlinechar\relax
      \catcode13=\the\catcode13\relax
      \catcode32=\the\catcode32\relax
      \catcode35=\the\catcode35\relax
      \catcode61=\the\catcode61\relax
      \catcode64=\the\catcode64\relax
      \catcode123=\the\catcode123\relax
      \catcode125=\the\catcode125\relax
    }%
  }%
\x\catcode61\catcode48\catcode32=10\relax%
\catcode13=5 % ^^M
\endlinechar=13 %
\catcode35=6 % #
\catcode64=11 % @
\catcode123=1 % {
\catcode125=2 % }
\def\TMP@EnsureCode#1#2{%
  \edef\magicnum@AtEnd{%
    \magicnum@AtEnd
    \catcode#1=\the\catcode#1\relax
  }%
  \catcode#1=#2\relax
}
\TMP@EnsureCode{34}{12}% "
\TMP@EnsureCode{39}{12}% '
\TMP@EnsureCode{40}{12}% (
\TMP@EnsureCode{41}{12}% )
\TMP@EnsureCode{42}{12}% *
\TMP@EnsureCode{44}{12}% ,
\TMP@EnsureCode{45}{12}% -
\TMP@EnsureCode{46}{12}% .
\TMP@EnsureCode{47}{12}% /
\TMP@EnsureCode{58}{12}% :
\TMP@EnsureCode{60}{12}% <
\TMP@EnsureCode{62}{12}% >
\TMP@EnsureCode{91}{12}% [
\TMP@EnsureCode{93}{12}% ]
\edef\magicnum@AtEnd{\magicnum@AtEnd\noexpand\endinput}
%    \end{macrocode}
%
% \subsection{Check for previous definition}
%
%    \begin{macrocode}
\begingroup\expandafter\expandafter\expandafter\endgroup
\expandafter\ifx\csname newcommand\endcsname\relax
  \expandafter\ifx\csname magicnum\endcsname\relax
  \else
    \input infwarerr.sty\relax
    \@PackageError{magicnum}{%
      \string\magicnum\space is already defined%
    }\@ehc
  \fi
\else
  \newcommand*{\magicnum}{}%
\fi
%    \end{macrocode}
%
% \subsection{Without \LuaTeX}
%
%    \begin{macrocode}
\begingroup\expandafter\expandafter\expandafter\endgroup
\expandafter\ifx\csname directlua\endcsname\relax
%    \end{macrocode}
%
%    \begin{macro}{\magicnum}
%    \begin{macrocode}
  \begingroup\expandafter\expandafter\expandafter\endgroup
  \expandafter\ifx\csname ifcsname\endcsname\relax
    \def\magicnum#1{%
      \expandafter\ifx\csname MG@#1\endcsname\relax
      \else
        \csname MG@#1\endcsname
      \fi
    }%
  \else
    \begingroup
      \edef\x{\endgroup
        \def\noexpand\magicnum##1{%
          \expandafter\noexpand\csname
          ifcsname\endcsname MG@##1\noexpand\endcsname
            \noexpand\csname MG@##1%
                 \noexpand\expandafter\noexpand\endcsname
          \expandafter\noexpand\csname fi\endcsname
        }%
      }%
    \x
  \fi
%    \end{macrocode}
%    \end{macro}
%
%    \begin{macrocode}
\else
%    \end{macrocode}
%
% \subsection{With \LuaTeX}
%
%    \begin{macrocode}
  \begingroup\expandafter\expandafter\expandafter\endgroup
  \expandafter\ifx\csname RequirePackage\endcsname\relax
    \input ifluatex.sty\relax
    \input infwarerr.sty\relax
  \else
    \RequirePackage{ifluatex}[2010/03/01]%
    \RequirePackage{infwarerr}[2010/04/08]%
  \fi
%    \end{macrocode}
%
%    \begin{macro}{\magicnum@directlua}
%    \begin{macrocode}
  \ifnum\luatexversion<36 %
    \def\magicnum@directlua{\directlua0 }%
  \else
    \let\magicnum@directlua\directlua
  \fi
%    \end{macrocode}
%    \end{macro}
%    \begin{macrocode}
  \magicnum@directlua{%
    require("oberdiek.magicnum")%
  }%
  \begingroup
    \def\x{2011/04/10 v1.4}%
    \def\StripPrefix#1>{}%
    \edef\x{\expandafter\StripPrefix\meaning\x}%
    \edef\y{%
      \magicnum@directlua{%
        if oberdiek.magicnum.getversion then %
          oberdiek.magicnum.getversion()%
        end%
      }%
    }%
    \ifx\x\y
    \else
      \@PackageError{magicnum}{%
        Wrong version of lua module.\MessageBreak
        Package version: \x\MessageBreak
        Lua module: \y
      }\@ehc
    \fi
  \endgroup
%    \end{macrocode}
%    \begin{macro}{\luaescapestring}
%    \begin{macrocode}
  \begingroup
    \expandafter\ifx\csname luaescapestring\endcsname\relax
      \directlua{%
        if tex.enableprimitives then %
          tex.enableprimitives('magicnum@', {'luaescapestring'})%
        end%
      }%
      \global\let\luaescapestring\magicnum@luaescapestring
    \fi
    \expandafter\ifx\csname luaescapestring\endcsname\relax
      \escapechar=92 %
      \@PackageError{magicnum}{%
        Missing \string\luaescapestring
      }\@ehc
    \fi
  \endgroup
%    \end{macrocode}
%    \end{macro}
%    \begin{macro}{\magicnum}
%    \begin{macrocode}
  \def\magicnum#1{%
    \magicnum@directlua{%
      oberdiek.magicnum.get("\luaescapestring{#1}")%
    }%
  }%
%    \end{macrocode}
%    \end{macro}
%
%    \begin{macrocode}
  \expandafter\magicnum@AtEnd
\fi%
%</package>
%    \end{macrocode}
%
% \subsection{Data}
%
% \subsubsection{Plain data}
%
%    \begin{macrocode}
%<*data>
tex.catcode
  escape = 0
  begingroup = 1
  endgroup = 2
  math = 3
  align = 4
  eol = 5
  parameter = 6
  superscript = 7
  subscript = 8
  ignore = 9
  space = 10
  letter = 11
  other = 12
  active = 13
  comment = 14
  invalid = 15
etex.grouptype
  bottomlevel = 0
  simple = 1
  hbox = 2
  adjustedhbox = 3
  vbox = 4
  align = 5
  noalign = 6
  output = 8
  math = 9
  disc = 10
  insert = 11
  vcenter = 12
  mathchoice = 13
  semisimple = 14
  mathshift = 15
  mathleft = 16
etex.iftype
  none = 0
  char = 1
  cat = 2
  num = 3
  dim = 4
  odd = 5
  vmode = 6
  hmode = 7
  mmode = 8
  inner = 9
  void = 10
  hbox = 11
  vbox = 12
  x = 13
  eof = 14
  true = 15
  false = 16
  case = 17
  defined = 18
  csname = 19
  fontchar = 20
etex.nodetype
  none = -1
  char = 0
  hlist = 1
  vlist = 2
  rule = 3
  ins = 4
  mark = 5
  adjust = 6
  ligature = 7
  disc = 8
  whatsit = 9
  math = 10
  glue = 11
  kern = 12
  penalty = 13
  unset = 14
  maths = 15
etex.interactionmode
  batch = 0
  nonstop = 1
  scroll = 2
  errorstop = 3
luatex.pdfliteral.mode
  setorigin = 0
  page = 1
  direct = 2
%</data>
%    \end{macrocode}
%
% \subsubsection{Data for \TeX}
%
%    \begin{macrocode}
%<*package>
%    \end{macrocode}
%    \begin{macro}{\magicnum@add}
%    \begin{macrocode}
\begingroup\expandafter\expandafter\expandafter\endgroup
\expandafter\ifx\csname detokenize\endcsname\relax
  \def\magicnum@add#1#2#3{%
    \expandafter\magicnum@@add
        \csname MG@#1.#2\expandafter\endcsname
        \csname MG@#1.#3\endcsname
       {#3}{#2}%
  }%
  \def\magicnum@@add#1#2#3#4{%
    \def#1{#3}%
    \def#2{#4}%
    \edef#1{%
      \expandafter\strip@prefix\meaning#1%
    }%
    \edef#2{%
      \expandafter\strip@prefix\meaning#2%
    }%
  }%
  \expandafter\ifx\csname strip@prefix\endcsname\relax
    \def\strip@prefix#1->{}%
  \fi
\else
  \def\magicnum@add#1#2#3{%
    \expandafter\edef\csname MG@#1.#2\endcsname{%
      \detokenize{#3}%
    }%
    \expandafter\edef\csname MG@#1.#3\endcsname{%
      \detokenize{#2}%
    }%
  }%
\fi
%    \end{macrocode}
%    \end{macro}
%    \begin{macrocode}
\magicnum@add{tex.catcode}{escape}{0}
\magicnum@add{tex.catcode}{begingroup}{1}
\magicnum@add{tex.catcode}{endgroup}{2}
\magicnum@add{tex.catcode}{math}{3}
\magicnum@add{tex.catcode}{align}{4}
\magicnum@add{tex.catcode}{eol}{5}
\magicnum@add{tex.catcode}{parameter}{6}
\magicnum@add{tex.catcode}{superscript}{7}
\magicnum@add{tex.catcode}{subscript}{8}
\magicnum@add{tex.catcode}{ignore}{9}
\magicnum@add{tex.catcode}{space}{10}
\magicnum@add{tex.catcode}{letter}{11}
\magicnum@add{tex.catcode}{other}{12}
\magicnum@add{tex.catcode}{active}{13}
\magicnum@add{tex.catcode}{comment}{14}
\magicnum@add{tex.catcode}{invalid}{15}
\magicnum@add{etex.grouptype}{bottomlevel}{0}
\magicnum@add{etex.grouptype}{simple}{1}
\magicnum@add{etex.grouptype}{hbox}{2}
\magicnum@add{etex.grouptype}{adjustedhbox}{3}
\magicnum@add{etex.grouptype}{vbox}{4}
\magicnum@add{etex.grouptype}{align}{5}
\magicnum@add{etex.grouptype}{noalign}{6}
\magicnum@add{etex.grouptype}{output}{8}
\magicnum@add{etex.grouptype}{math}{9}
\magicnum@add{etex.grouptype}{disc}{10}
\magicnum@add{etex.grouptype}{insert}{11}
\magicnum@add{etex.grouptype}{vcenter}{12}
\magicnum@add{etex.grouptype}{mathchoice}{13}
\magicnum@add{etex.grouptype}{semisimple}{14}
\magicnum@add{etex.grouptype}{mathshift}{15}
\magicnum@add{etex.grouptype}{mathleft}{16}
\magicnum@add{etex.iftype}{none}{0}
\magicnum@add{etex.iftype}{char}{1}
\magicnum@add{etex.iftype}{cat}{2}
\magicnum@add{etex.iftype}{num}{3}
\magicnum@add{etex.iftype}{dim}{4}
\magicnum@add{etex.iftype}{odd}{5}
\magicnum@add{etex.iftype}{vmode}{6}
\magicnum@add{etex.iftype}{hmode}{7}
\magicnum@add{etex.iftype}{mmode}{8}
\magicnum@add{etex.iftype}{inner}{9}
\magicnum@add{etex.iftype}{void}{10}
\magicnum@add{etex.iftype}{hbox}{11}
\magicnum@add{etex.iftype}{vbox}{12}
\magicnum@add{etex.iftype}{x}{13}
\magicnum@add{etex.iftype}{eof}{14}
\magicnum@add{etex.iftype}{true}{15}
\magicnum@add{etex.iftype}{false}{16}
\magicnum@add{etex.iftype}{case}{17}
\magicnum@add{etex.iftype}{defined}{18}
\magicnum@add{etex.iftype}{csname}{19}
\magicnum@add{etex.iftype}{fontchar}{20}
\magicnum@add{etex.nodetype}{none}{-1}
\magicnum@add{etex.nodetype}{char}{0}
\magicnum@add{etex.nodetype}{hlist}{1}
\magicnum@add{etex.nodetype}{vlist}{2}
\magicnum@add{etex.nodetype}{rule}{3}
\magicnum@add{etex.nodetype}{ins}{4}
\magicnum@add{etex.nodetype}{mark}{5}
\magicnum@add{etex.nodetype}{adjust}{6}
\magicnum@add{etex.nodetype}{ligature}{7}
\magicnum@add{etex.nodetype}{disc}{8}
\magicnum@add{etex.nodetype}{whatsit}{9}
\magicnum@add{etex.nodetype}{math}{10}
\magicnum@add{etex.nodetype}{glue}{11}
\magicnum@add{etex.nodetype}{kern}{12}
\magicnum@add{etex.nodetype}{penalty}{13}
\magicnum@add{etex.nodetype}{unset}{14}
\magicnum@add{etex.nodetype}{maths}{15}
\magicnum@add{etex.interactionmode}{batch}{0}
\magicnum@add{etex.interactionmode}{nonstop}{1}
\magicnum@add{etex.interactionmode}{scroll}{2}
\magicnum@add{etex.interactionmode}{errorstop}{3}
\magicnum@add{luatex.pdfliteral.mode}{setorigin}{0}
\magicnum@add{luatex.pdfliteral.mode}{page}{1}
\magicnum@add{luatex.pdfliteral.mode}{direct}{2}
%    \end{macrocode}
%    \begin{macrocode}
\magicnum@AtEnd%
%</package>
%    \end{macrocode}
%
% \subsubsection{Lua module}
%
%    \begin{macrocode}
%<*lua>
%    \end{macrocode}
%    \begin{macrocode}
module("oberdiek.magicnum", package.seeall)
%    \end{macrocode}
%    \begin{macrocode}
function getversion()
  tex.write("2011/04/10 v1.4")
end
%    \end{macrocode}
%    \begin{macrocode}
local data = {
  ["tex.catcode"] = {
    [0] = "escape",
    [1] = "begingroup",
    [2] = "endgroup",
    [3] = "math",
    [4] = "align",
    [5] = "eol",
    [6] = "parameter",
    [7] = "superscript",
    [8] = "subscript",
    [9] = "ignore",
    [10] = "space",
    [11] = "letter",
    [12] = "other",
    [13] = "active",
    [14] = "comment",
    [15] = "invalid",
    ["active"] = 13,
    ["align"] = 4,
    ["begingroup"] = 1,
    ["comment"] = 14,
    ["endgroup"] = 2,
    ["eol"] = 5,
    ["escape"] = 0,
    ["ignore"] = 9,
    ["invalid"] = 15,
    ["letter"] = 11,
    ["math"] = 3,
    ["other"] = 12,
    ["parameter"] = 6,
    ["space"] = 10,
    ["subscript"] = 8,
    ["superscript"] = 7
  },
  ["etex.grouptype"] = {
    [0] = "bottomlevel",
    [1] = "simple",
    [2] = "hbox",
    [3] = "adjustedhbox",
    [4] = "vbox",
    [5] = "align",
    [6] = "noalign",
    [8] = "output",
    [9] = "math",
    [10] = "disc",
    [11] = "insert",
    [12] = "vcenter",
    [13] = "mathchoice",
    [14] = "semisimple",
    [15] = "mathshift",
    [16] = "mathleft",
    ["adjustedhbox"] = 3,
    ["align"] = 5,
    ["bottomlevel"] = 0,
    ["disc"] = 10,
    ["hbox"] = 2,
    ["insert"] = 11,
    ["math"] = 9,
    ["mathchoice"] = 13,
    ["mathleft"] = 16,
    ["mathshift"] = 15,
    ["noalign"] = 6,
    ["output"] = 8,
    ["semisimple"] = 14,
    ["simple"] = 1,
    ["vbox"] = 4,
    ["vcenter"] = 12
  },
  ["etex.iftype"] = {
    [0] = "none",
    [1] = "char",
    [2] = "cat",
    [3] = "num",
    [4] = "dim",
    [5] = "odd",
    [6] = "vmode",
    [7] = "hmode",
    [8] = "mmode",
    [9] = "inner",
    [10] = "void",
    [11] = "hbox",
    [12] = "vbox",
    [13] = "x",
    [14] = "eof",
    [15] = "true",
    [16] = "false",
    [17] = "case",
    [18] = "defined",
    [19] = "csname",
    [20] = "fontchar",
    ["case"] = 17,
    ["cat"] = 2,
    ["char"] = 1,
    ["csname"] = 19,
    ["defined"] = 18,
    ["dim"] = 4,
    ["eof"] = 14,
    ["false"] = 16,
    ["fontchar"] = 20,
    ["hbox"] = 11,
    ["hmode"] = 7,
    ["inner"] = 9,
    ["mmode"] = 8,
    ["none"] = 0,
    ["num"] = 3,
    ["odd"] = 5,
    ["true"] = 15,
    ["vbox"] = 12,
    ["vmode"] = 6,
    ["void"] = 10,
    ["x"] = 13
  },
  ["etex.nodetype"] = {
    [-1] = "none",
    [0] = "char",
    [1] = "hlist",
    [2] = "vlist",
    [3] = "rule",
    [4] = "ins",
    [5] = "mark",
    [6] = "adjust",
    [7] = "ligature",
    [8] = "disc",
    [9] = "whatsit",
    [10] = "math",
    [11] = "glue",
    [12] = "kern",
    [13] = "penalty",
    [14] = "unset",
    [15] = "maths",
    ["adjust"] = 6,
    ["char"] = 0,
    ["disc"] = 8,
    ["glue"] = 11,
    ["hlist"] = 1,
    ["ins"] = 4,
    ["kern"] = 12,
    ["ligature"] = 7,
    ["mark"] = 5,
    ["math"] = 10,
    ["maths"] = 15,
    ["none"] = -1,
    ["penalty"] = 13,
    ["rule"] = 3,
    ["unset"] = 14,
    ["vlist"] = 2,
    ["whatsit"] = 9
  },
  ["etex.interactionmode"] = {
    [0] = "batch",
    [1] = "nonstop",
    [2] = "scroll",
    [3] = "errorstop",
    ["batch"] = 0,
    ["errorstop"] = 3,
    ["nonstop"] = 1,
    ["scroll"] = 2
  },
  ["luatex.pdfliteral.mode"] = {
    [0] = "setorigin",
    [1] = "page",
    [2] = "direct",
    ["direct"] = 2,
    ["page"] = 1,
    ["setorigin"] = 0
  }
}
%    \end{macrocode}
%    \begin{macrocode}
function get(name)
  local startpos, endpos, category, entry =
      string.find(name, "^(%a[%a%d%.]*)%.(-?[%a%d]+)$")
  if not entry then
    return
  end
  local node = data[category]
  if not node then
    return
  end
  local num = tonumber(entry)
  local value
  if num then
    value = node[num]
    if not value then
      return
    end
  else
    value = node[entry]
    if not value then
      return
    end
    value = "" .. value
  end
  tex.write(value)
end
%    \end{macrocode}
%
%    \begin{macrocode}
%</lua>
%    \end{macrocode}
%
% \section{Test}
%
% \subsection{Catcode checks for loading}
%
%    \begin{macrocode}
%<*test1>
%    \end{macrocode}
%    \begin{macrocode}
\catcode`\{=1 %
\catcode`\}=2 %
\catcode`\#=6 %
\catcode`\@=11 %
\expandafter\ifx\csname count@\endcsname\relax
  \countdef\count@=255 %
\fi
\expandafter\ifx\csname @gobble\endcsname\relax
  \long\def\@gobble#1{}%
\fi
\expandafter\ifx\csname @firstofone\endcsname\relax
  \long\def\@firstofone#1{#1}%
\fi
\expandafter\ifx\csname loop\endcsname\relax
  \expandafter\@firstofone
\else
  \expandafter\@gobble
\fi
{%
  \def\loop#1\repeat{%
    \def\body{#1}%
    \iterate
  }%
  \def\iterate{%
    \body
      \let\next\iterate
    \else
      \let\next\relax
    \fi
    \next
  }%
  \let\repeat=\fi
}%
\def\RestoreCatcodes{}
\count@=0 %
\loop
  \edef\RestoreCatcodes{%
    \RestoreCatcodes
    \catcode\the\count@=\the\catcode\count@\relax
  }%
\ifnum\count@<255 %
  \advance\count@ 1 %
\repeat

\def\RangeCatcodeInvalid#1#2{%
  \count@=#1\relax
  \loop
    \catcode\count@=15 %
  \ifnum\count@<#2\relax
    \advance\count@ 1 %
  \repeat
}
\def\RangeCatcodeCheck#1#2#3{%
  \count@=#1\relax
  \loop
    \ifnum#3=\catcode\count@
    \else
      \errmessage{%
        Character \the\count@\space
        with wrong catcode \the\catcode\count@\space
        instead of \number#3%
      }%
    \fi
  \ifnum\count@<#2\relax
    \advance\count@ 1 %
  \repeat
}
\def\space{ }
\expandafter\ifx\csname LoadCommand\endcsname\relax
  \def\LoadCommand{\input magicnum.sty\relax}%
\fi
\def\Test{%
  \RangeCatcodeInvalid{0}{47}%
  \RangeCatcodeInvalid{58}{64}%
  \RangeCatcodeInvalid{91}{96}%
  \RangeCatcodeInvalid{123}{255}%
  \catcode`\@=12 %
  \catcode`\\=0 %
  \catcode`\%=14 %
  \LoadCommand
  \RangeCatcodeCheck{0}{36}{15}%
  \RangeCatcodeCheck{37}{37}{14}%
  \RangeCatcodeCheck{38}{47}{15}%
  \RangeCatcodeCheck{48}{57}{12}%
  \RangeCatcodeCheck{58}{63}{15}%
  \RangeCatcodeCheck{64}{64}{12}%
  \RangeCatcodeCheck{65}{90}{11}%
  \RangeCatcodeCheck{91}{91}{15}%
  \RangeCatcodeCheck{92}{92}{0}%
  \RangeCatcodeCheck{93}{96}{15}%
  \RangeCatcodeCheck{97}{122}{11}%
  \RangeCatcodeCheck{123}{255}{15}%
  \RestoreCatcodes
}
\Test
\csname @@end\endcsname
\end
%    \end{macrocode}
%    \begin{macrocode}
%</test1>
%    \end{macrocode}
%
% \subsection{Test data}
%
%    \begin{macrocode}
%<*testplain>
\input magicnum.sty\relax
\def\Test#1#2{%
  \edef\result{\magicnum{#1}}%
  \edef\expect{#2}%
  \edef\expect{\expandafter\stripprefix\meaning\expect}%
  \ifx\result\expect
  \else
    \errmessage{%
      Failed: [#1] % hash-ok
      returns [\result] instead of [\expect]%
    }%
  \fi
}
\def\stripprefix#1->{}
%</testplain>
%    \end{macrocode}
%    \begin{macrocode}
%<*testlatex>
\NeedsTeXFormat{LaTeX2e}
\documentclass{minimal}
\usepackage{magicnum}[2011/04/10]
\usepackage{qstest}
\IncludeTests{*}
\LogTests{log}{*}{*}
\newcommand*{\Test}[2]{%
  \Expect*{\magicnum{#1}}{#2}%
}
\begin{qstest}{magicnum}{magicnum}
%</testlatex>
%    \end{macrocode}
%    \begin{macrocode}
%<*testdata>
\Test{tex.catcode.escape}{0}
\Test{tex.catcode.invalid}{15}
\Test{tex.catcode.unknown}{}
\Test{tex.catcode.0}{escape}
\Test{tex.catcode.15}{invalid}
\Test{etex.iftype.true}{15}
\Test{etex.iftype.false}{16}
\Test{etex.iftype.15}{true}
\Test{etex.iftype.16}{false}
\Test{etex.nodetype.none}{-1}
\Test{etex.nodetype.-1}{none}
\Test{luatex.pdfliteral.mode.direct}{2}
\Test{luatex.pdfliteral.mode.1}{page}
\Test{}{}
\Test{unknown}{}
\Test{unknown.foo.bar}{}
\Test{unknown.foo.4}{}
%</testdata>
%    \end{macrocode}
%    \begin{macrocode}
%<*testplain>
\csname @@end\endcsname
\end
%</testplain>
%<*testlatex>
\end{qstest}
\csname @@end\endcsname
%</testlatex>
%    \end{macrocode}
%
% \subsection{Small test for \hologo{iniTeX}}
%
%    \begin{macrocode}
%<*test4>
\catcode`\{=1
\catcode`\}=2
\catcode`\#=6
\input magicnum.sty\relax
\edef\x{\magicnum{tex.catcode.15}}
\edef\y{invalid}
\def\Strip#1>{}
\edef\y{\expandafter\Strip\meaning\y}
\ifx\x\y
  \immediate\write16{Ok}%
\else
  \errmessage{\x<>\y}%
\fi
\csname @@end\endcsname\end
%</test4>
%    \end{macrocode}
%
% \section{Installation}
%
% \subsection{Download}
%
% \paragraph{Package.} This package is available on
% CTAN\footnote{\url{ftp://ftp.ctan.org/tex-archive/}}:
% \begin{description}
% \item[\CTAN{macros/latex/contrib/oberdiek/magicnum.dtx}] The source file.
% \item[\CTAN{macros/latex/contrib/oberdiek/magicnum.pdf}] Documentation.
% \end{description}
%
%
% \paragraph{Bundle.} All the packages of the bundle `oberdiek'
% are also available in a TDS compliant ZIP archive. There
% the packages are already unpacked and the documentation files
% are generated. The files and directories obey the TDS standard.
% \begin{description}
% \item[\CTAN{install/macros/latex/contrib/oberdiek.tds.zip}]
% \end{description}
% \emph{TDS} refers to the standard ``A Directory Structure
% for \TeX\ Files'' (\CTAN{tds/tds.pdf}). Directories
% with \xfile{texmf} in their name are usually organized this way.
%
% \subsection{Bundle installation}
%
% \paragraph{Unpacking.} Unpack the \xfile{oberdiek.tds.zip} in the
% TDS tree (also known as \xfile{texmf} tree) of your choice.
% Example (linux):
% \begin{quote}
%   |unzip oberdiek.tds.zip -d ~/texmf|
% \end{quote}
%
% \paragraph{Script installation.}
% Check the directory \xfile{TDS:scripts/oberdiek/} for
% scripts that need further installation steps.
% Package \xpackage{attachfile2} comes with the Perl script
% \xfile{pdfatfi.pl} that should be installed in such a way
% that it can be called as \texttt{pdfatfi}.
% Example (linux):
% \begin{quote}
%   |chmod +x scripts/oberdiek/pdfatfi.pl|\\
%   |cp scripts/oberdiek/pdfatfi.pl /usr/local/bin/|
% \end{quote}
%
% \subsection{Package installation}
%
% \paragraph{Unpacking.} The \xfile{.dtx} file is a self-extracting
% \docstrip\ archive. The files are extracted by running the
% \xfile{.dtx} through \plainTeX:
% \begin{quote}
%   \verb|tex magicnum.dtx|
% \end{quote}
%
% \paragraph{TDS.} Now the different files must be moved into
% the different directories in your installation TDS tree
% (also known as \xfile{texmf} tree):
% \begin{quote}
% \def\t{^^A
% \begin{tabular}{@{}>{\ttfamily}l@{ $\rightarrow$ }>{\ttfamily}l@{}}
%   magicnum.sty & tex/generic/oberdiek/magicnum.sty\\
%   magicnum.lua & scripts/oberdiek/magicnum.lua\\
%   oberdiek.magicnum.lua & scripts/oberdiek/oberdiek.magicnum.lua\\
%   magicnum.pdf & doc/latex/oberdiek/magicnum.pdf\\
%   magicnum.txt & doc/latex/oberdiek/magicnum.txt\\
%   test/magicnum-test1.tex & doc/latex/oberdiek/test/magicnum-test1.tex\\
%   test/magicnum-test2.tex & doc/latex/oberdiek/test/magicnum-test2.tex\\
%   test/magicnum-test3.tex & doc/latex/oberdiek/test/magicnum-test3.tex\\
%   test/magicnum-test4.tex & doc/latex/oberdiek/test/magicnum-test4.tex\\
%   magicnum.dtx & source/latex/oberdiek/magicnum.dtx\\
% \end{tabular}^^A
% }^^A
% \sbox0{\t}^^A
% \ifdim\wd0>\linewidth
%   \begingroup
%     \advance\linewidth by\leftmargin
%     \advance\linewidth by\rightmargin
%   \edef\x{\endgroup
%     \def\noexpand\lw{\the\linewidth}^^A
%   }\x
%   \def\lwbox{^^A
%     \leavevmode
%     \hbox to \linewidth{^^A
%       \kern-\leftmargin\relax
%       \hss
%       \usebox0
%       \hss
%       \kern-\rightmargin\relax
%     }^^A
%   }^^A
%   \ifdim\wd0>\lw
%     \sbox0{\small\t}^^A
%     \ifdim\wd0>\linewidth
%       \ifdim\wd0>\lw
%         \sbox0{\footnotesize\t}^^A
%         \ifdim\wd0>\linewidth
%           \ifdim\wd0>\lw
%             \sbox0{\scriptsize\t}^^A
%             \ifdim\wd0>\linewidth
%               \ifdim\wd0>\lw
%                 \sbox0{\tiny\t}^^A
%                 \ifdim\wd0>\linewidth
%                   \lwbox
%                 \else
%                   \usebox0
%                 \fi
%               \else
%                 \lwbox
%               \fi
%             \else
%               \usebox0
%             \fi
%           \else
%             \lwbox
%           \fi
%         \else
%           \usebox0
%         \fi
%       \else
%         \lwbox
%       \fi
%     \else
%       \usebox0
%     \fi
%   \else
%     \lwbox
%   \fi
% \else
%   \usebox0
% \fi
% \end{quote}
% If you have a \xfile{docstrip.cfg} that configures and enables \docstrip's
% TDS installing feature, then some files can already be in the right
% place, see the documentation of \docstrip.
%
% \subsection{Refresh file name databases}
%
% If your \TeX~distribution
% (\teTeX, \mikTeX, \dots) relies on file name databases, you must refresh
% these. For example, \teTeX\ users run \verb|texhash| or
% \verb|mktexlsr|.
%
% \subsection{Some details for the interested}
%
% \paragraph{Attached source.}
%
% The PDF documentation on CTAN also includes the
% \xfile{.dtx} source file. It can be extracted by
% AcrobatReader 6 or higher. Another option is \textsf{pdftk},
% e.g. unpack the file into the current directory:
% \begin{quote}
%   \verb|pdftk magicnum.pdf unpack_files output .|
% \end{quote}
%
% \paragraph{Unpacking with \LaTeX.}
% The \xfile{.dtx} chooses its action depending on the format:
% \begin{description}
% \item[\plainTeX:] Run \docstrip\ and extract the files.
% \item[\LaTeX:] Generate the documentation.
% \end{description}
% If you insist on using \LaTeX\ for \docstrip\ (really,
% \docstrip\ does not need \LaTeX), then inform the autodetect routine
% about your intention:
% \begin{quote}
%   \verb|latex \let\install=y\input{magicnum.dtx}|
% \end{quote}
% Do not forget to quote the argument according to the demands
% of your shell.
%
% \paragraph{Generating the documentation.}
% You can use both the \xfile{.dtx} or the \xfile{.drv} to generate
% the documentation. The process can be configured by the
% configuration file \xfile{ltxdoc.cfg}. For instance, put this
% line into this file, if you want to have A4 as paper format:
% \begin{quote}
%   \verb|\PassOptionsToClass{a4paper}{article}|
% \end{quote}
% An example follows how to generate the
% documentation with pdf\LaTeX:
% \begin{quote}
%\begin{verbatim}
%pdflatex magicnum.dtx
%makeindex -s gind.ist magicnum.idx
%pdflatex magicnum.dtx
%makeindex -s gind.ist magicnum.idx
%pdflatex magicnum.dtx
%\end{verbatim}
% \end{quote}
%
% \section{Catalogue}
%
% The following XML file can be used as source for the
% \href{http://mirror.ctan.org/help/Catalogue/catalogue.html}{\TeX\ Catalogue}.
% The elements \texttt{caption} and \texttt{description} are imported
% from the original XML file from the Catalogue.
% The name of the XML file in the Catalogue is \xfile{magicnum.xml}.
%    \begin{macrocode}
%<*catalogue>
<?xml version='1.0' encoding='us-ascii'?>
<!DOCTYPE entry SYSTEM 'catalogue.dtd'>
<entry datestamp='$Date$' modifier='$Author$' id='magicnum'>
  <name>magicnum</name>
  <caption>Access TeX systems' "magic numbers".</caption>
  <authorref id='auth:oberdiek'/>
  <copyright owner='Heiko Oberdiek' year='2007,2009-2011'/>
  <license type='lppl1.3'/>
  <version number='1.4'/>
  <description>
    This package allows access to the various parameter values in
    TeX (catcode values), e-TeX (group, if and node types, and
    interaction mode), and LuaTeX (pdfliteral mode) by a hierarchical
    name system.
    <p/>
    The package is part of the <xref refid='oberdiek'>oberdiek</xref> bundle.
  </description>
  <documentation details='Package documentation'
      href='ctan:/macros/latex/contrib/oberdiek/magicnum.pdf'/>
  <ctan file='true' path='/macros/latex/contrib/oberdiek/magicnum.dtx'/>
  <miktex location='oberdiek'/>
  <texlive location='oberdiek'/>
  <install path='/macros/latex/contrib/oberdiek/oberdiek.tds.zip'/>
</entry>
%</catalogue>
%    \end{macrocode}
%
% \begin{History}
%   \begin{Version}{2007/12/12 v1.0}
%   \item
%     First public version.
%   \end{Version}
%   \begin{Version}{2009/04/10 v1.1}
%   \item
%     Adaptation to \LuaTeX\ 0.40.
%   \end{Version}
%   \begin{Version}{2010/03/09 v1.2}
%   \item
%     Adaptation to package \xpackage{luatex} 0.4.
%   \end{Version}
%   \begin{Version}{2011/03/24 v1.3}
%   \item
%     Catcode fixes.
%   \end{Version}
%   \begin{Version}{2011/04/10 v1.4}
%   \item
%     Compatibility for \hologo{iniTeX}.
%   \item
%     Dependency from package \xpackage{luatex} removed.
%   \item
%     Version check for lua module.
%   \end{Version}
% \end{History}
%
% \PrintIndex
%
% \Finale
\endinput
|
% \end{quote}
% Do not forget to quote the argument according to the demands
% of your shell.
%
% \paragraph{Generating the documentation.}
% You can use both the \xfile{.dtx} or the \xfile{.drv} to generate
% the documentation. The process can be configured by the
% configuration file \xfile{ltxdoc.cfg}. For instance, put this
% line into this file, if you want to have A4 as paper format:
% \begin{quote}
%   \verb|\PassOptionsToClass{a4paper}{article}|
% \end{quote}
% An example follows how to generate the
% documentation with pdf\LaTeX:
% \begin{quote}
%\begin{verbatim}
%pdflatex magicnum.dtx
%makeindex -s gind.ist magicnum.idx
%pdflatex magicnum.dtx
%makeindex -s gind.ist magicnum.idx
%pdflatex magicnum.dtx
%\end{verbatim}
% \end{quote}
%
% \section{Catalogue}
%
% The following XML file can be used as source for the
% \href{http://mirror.ctan.org/help/Catalogue/catalogue.html}{\TeX\ Catalogue}.
% The elements \texttt{caption} and \texttt{description} are imported
% from the original XML file from the Catalogue.
% The name of the XML file in the Catalogue is \xfile{magicnum.xml}.
%    \begin{macrocode}
%<*catalogue>
<?xml version='1.0' encoding='us-ascii'?>
<!DOCTYPE entry SYSTEM 'catalogue.dtd'>
<entry datestamp='$Date$' modifier='$Author$' id='magicnum'>
  <name>magicnum</name>
  <caption>Access TeX systems' "magic numbers".</caption>
  <authorref id='auth:oberdiek'/>
  <copyright owner='Heiko Oberdiek' year='2007,2009-2011'/>
  <license type='lppl1.3'/>
  <version number='1.4'/>
  <description>
    This package allows access to the various parameter values in
    TeX (catcode values), e-TeX (group, if and node types, and
    interaction mode), and LuaTeX (pdfliteral mode) by a hierarchical
    name system.
    <p/>
    The package is part of the <xref refid='oberdiek'>oberdiek</xref> bundle.
  </description>
  <documentation details='Package documentation'
      href='ctan:/macros/latex/contrib/oberdiek/magicnum.pdf'/>
  <ctan file='true' path='/macros/latex/contrib/oberdiek/magicnum.dtx'/>
  <miktex location='oberdiek'/>
  <texlive location='oberdiek'/>
  <install path='/macros/latex/contrib/oberdiek/oberdiek.tds.zip'/>
</entry>
%</catalogue>
%    \end{macrocode}
%
% \begin{History}
%   \begin{Version}{2007/12/12 v1.0}
%   \item
%     First public version.
%   \end{Version}
%   \begin{Version}{2009/04/10 v1.1}
%   \item
%     Adaptation to \LuaTeX\ 0.40.
%   \end{Version}
%   \begin{Version}{2010/03/09 v1.2}
%   \item
%     Adaptation to package \xpackage{luatex} 0.4.
%   \end{Version}
%   \begin{Version}{2011/03/24 v1.3}
%   \item
%     Catcode fixes.
%   \end{Version}
%   \begin{Version}{2011/04/10 v1.4}
%   \item
%     Compatibility for \hologo{iniTeX}.
%   \item
%     Dependency from package \xpackage{luatex} removed.
%   \item
%     Version check for lua module.
%   \end{Version}
% \end{History}
%
% \PrintIndex
%
% \Finale
\endinput
|
% \end{quote}
% Do not forget to quote the argument according to the demands
% of your shell.
%
% \paragraph{Generating the documentation.}
% You can use both the \xfile{.dtx} or the \xfile{.drv} to generate
% the documentation. The process can be configured by the
% configuration file \xfile{ltxdoc.cfg}. For instance, put this
% line into this file, if you want to have A4 as paper format:
% \begin{quote}
%   \verb|\PassOptionsToClass{a4paper}{article}|
% \end{quote}
% An example follows how to generate the
% documentation with pdf\LaTeX:
% \begin{quote}
%\begin{verbatim}
%pdflatex magicnum.dtx
%makeindex -s gind.ist magicnum.idx
%pdflatex magicnum.dtx
%makeindex -s gind.ist magicnum.idx
%pdflatex magicnum.dtx
%\end{verbatim}
% \end{quote}
%
% \section{Catalogue}
%
% The following XML file can be used as source for the
% \href{http://mirror.ctan.org/help/Catalogue/catalogue.html}{\TeX\ Catalogue}.
% The elements \texttt{caption} and \texttt{description} are imported
% from the original XML file from the Catalogue.
% The name of the XML file in the Catalogue is \xfile{magicnum.xml}.
%    \begin{macrocode}
%<*catalogue>
<?xml version='1.0' encoding='us-ascii'?>
<!DOCTYPE entry SYSTEM 'catalogue.dtd'>
<entry datestamp='$Date$' modifier='$Author$' id='magicnum'>
  <name>magicnum</name>
  <caption>Access TeX systems' "magic numbers".</caption>
  <authorref id='auth:oberdiek'/>
  <copyright owner='Heiko Oberdiek' year='2007,2009-2011'/>
  <license type='lppl1.3'/>
  <version number='1.4'/>
  <description>
    This package allows access to the various parameter values in
    TeX (catcode values), e-TeX (group, if and node types, and
    interaction mode), and LuaTeX (pdfliteral mode) by a hierarchical
    name system.
    <p/>
    The package is part of the <xref refid='oberdiek'>oberdiek</xref> bundle.
  </description>
  <documentation details='Package documentation'
      href='ctan:/macros/latex/contrib/oberdiek/magicnum.pdf'/>
  <ctan file='true' path='/macros/latex/contrib/oberdiek/magicnum.dtx'/>
  <miktex location='oberdiek'/>
  <texlive location='oberdiek'/>
  <install path='/macros/latex/contrib/oberdiek/oberdiek.tds.zip'/>
</entry>
%</catalogue>
%    \end{macrocode}
%
% \begin{History}
%   \begin{Version}{2007/12/12 v1.0}
%   \item
%     First public version.
%   \end{Version}
%   \begin{Version}{2009/04/10 v1.1}
%   \item
%     Adaptation to \LuaTeX\ 0.40.
%   \end{Version}
%   \begin{Version}{2010/03/09 v1.2}
%   \item
%     Adaptation to package \xpackage{luatex} 0.4.
%   \end{Version}
%   \begin{Version}{2011/03/24 v1.3}
%   \item
%     Catcode fixes.
%   \end{Version}
%   \begin{Version}{2011/04/10 v1.4}
%   \item
%     Compatibility for \hologo{iniTeX}.
%   \item
%     Dependency from package \xpackage{luatex} removed.
%   \item
%     Version check for lua module.
%   \end{Version}
% \end{History}
%
% \PrintIndex
%
% \Finale
\endinput

%        (quote the arguments according to the demands of your shell)
%
% Documentation:
%    (a) If magicnum.drv is present:
%           latex magicnum.drv
%    (b) Without magicnum.drv:
%           latex magicnum.dtx; ...
%    The class ltxdoc loads the configuration file ltxdoc.cfg
%    if available. Here you can specify further options, e.g.
%    use A4 as paper format:
%       \PassOptionsToClass{a4paper}{article}
%
%    Programm calls to get the documentation (example):
%       pdflatex magicnum.dtx
%       makeindex -s gind.ist magicnum.idx
%       pdflatex magicnum.dtx
%       makeindex -s gind.ist magicnum.idx
%       pdflatex magicnum.dtx
%
% Installation:
%    TDS:tex/generic/oberdiek/magicnum.sty
%    TDS:scripts/oberdiek/magicnum.lua
%    TDS:scripts/oberdiek/oberdiek.magicnum.lua
%    TDS:doc/latex/oberdiek/magicnum.pdf
%    TDS:doc/latex/oberdiek/magicnum.txt
%    TDS:doc/latex/oberdiek/test/magicnum-test1.tex
%    TDS:doc/latex/oberdiek/test/magicnum-test2.tex
%    TDS:doc/latex/oberdiek/test/magicnum-test3.tex
%    TDS:doc/latex/oberdiek/test/magicnum-test4.tex
%    TDS:source/latex/oberdiek/magicnum.dtx
%
%<*ignore>
\begingroup
  \catcode123=1 %
  \catcode125=2 %
  \def\x{LaTeX2e}%
\expandafter\endgroup
\ifcase 0\ifx\install y1\fi\expandafter
         \ifx\csname processbatchFile\endcsname\relax\else1\fi
         \ifx\fmtname\x\else 1\fi\relax
\else\csname fi\endcsname
%</ignore>
%<*install>
\input docstrip.tex
\Msg{************************************************************************}
\Msg{* Installation}
\Msg{* Package: magicnum 2011/04/10 v1.4 Magic numbers (HO)}
\Msg{************************************************************************}

\keepsilent
\askforoverwritefalse

\let\MetaPrefix\relax
\preamble

This is a generated file.

Project: magicnum
Version: 2011/04/10 v1.4

Copyright (C) 2007, 2009-2011 by
   Heiko Oberdiek <heiko.oberdiek at googlemail.com>

This work may be distributed and/or modified under the
conditions of the LaTeX Project Public License, either
version 1.3c of this license or (at your option) any later
version. This version of this license is in
   http://www.latex-project.org/lppl/lppl-1-3c.txt
and the latest version of this license is in
   http://www.latex-project.org/lppl.txt
and version 1.3 or later is part of all distributions of
LaTeX version 2005/12/01 or later.

This work has the LPPL maintenance status "maintained".

This Current Maintainer of this work is Heiko Oberdiek.

The Base Interpreter refers to any `TeX-Format',
because some files are installed in TDS:tex/generic//.

This work consists of the main source file magicnum.dtx
and the derived files
   magicnum.sty, magicnum.pdf, magicnum.ins, magicnum.drv, magicnum.txt,
   magicnum-test1.tex, magicnum-test2.tex, magicnum-test3.tex,
   magicnum-test4.tex, magicnum.lua, oberdiek.magicnum.lua.

\endpreamble
\let\MetaPrefix\DoubleperCent

\generate{%
  \file{magicnum.ins}{\from{magicnum.dtx}{install}}%
  \file{magicnum.drv}{\from{magicnum.dtx}{driver}}%
  \usedir{tex/generic/oberdiek}%
  \file{magicnum.sty}{\from{magicnum.dtx}{package}}%
  \usedir{doc/latex/oberdiek/test}%
  \file{magicnum-test1.tex}{\from{magicnum.dtx}{test1}}%
  \file{magicnum-test2.tex}{\from{magicnum.dtx}{testplain,testdata}}%
  \file{magicnum-test3.tex}{\from{magicnum.dtx}{testlatex,testdata}}%
  \file{magicnum-test4.tex}{\from{magicnum.dtx}{test4}}%
  \nopreamble
  \nopostamble
  \usedir{doc/latex/oberdiek}%
  \file{magicnum.txt}{\from{magicnum.dtx}{data}}%
  \usedir{source/latex/oberdiek/catalogue}%
  \file{magicnum.xml}{\from{magicnum.dtx}{catalogue}}%
}
\def\MetaPrefix{-- }
\def\defaultpostamble{%
  \MetaPrefix^^J%
  \MetaPrefix\space End of File `\outFileName'.%
}
\def\currentpostamble{\defaultpostamble}%
\generate{%
  \usedir{scripts/oberdiek}%
  \file{magicnum.lua}{\from{magicnum.dtx}{lua}}%
  \file{oberdiek.magicnum.lua}{\from{magicnum.dtx}{lua}}%
}

\catcode32=13\relax% active space
\let =\space%
\Msg{************************************************************************}
\Msg{*}
\Msg{* To finish the installation you have to move the following}
\Msg{* file into a directory searched by TeX:}
\Msg{*}
\Msg{*     magicnum.sty}
\Msg{*}
\Msg{* And install the following script files:}
\Msg{*}
\Msg{*     magicnum.lua, oberdiek.magicnum.lua}
\Msg{*}
\Msg{* To produce the documentation run the file `magicnum.drv'}
\Msg{* through LaTeX.}
\Msg{*}
\Msg{* Happy TeXing!}
\Msg{*}
\Msg{************************************************************************}

\endbatchfile
%</install>
%<*ignore>
\fi
%</ignore>
%<*driver>
\NeedsTeXFormat{LaTeX2e}
\ProvidesFile{magicnum.drv}%
  [2011/04/10 v1.4 Magic numbers (HO)]%
\documentclass{ltxdoc}
\usepackage{holtxdoc}[2011/11/22]
\usepackage{array}
\begin{document}
  \DocInput{magicnum.dtx}%
\end{document}
%</driver>
% \fi
%
% \CheckSum{755}
%
% \CharacterTable
%  {Upper-case    \A\B\C\D\E\F\G\H\I\J\K\L\M\N\O\P\Q\R\S\T\U\V\W\X\Y\Z
%   Lower-case    \a\b\c\d\e\f\g\h\i\j\k\l\m\n\o\p\q\r\s\t\u\v\w\x\y\z
%   Digits        \0\1\2\3\4\5\6\7\8\9
%   Exclamation   \!     Double quote  \"     Hash (number) \#
%   Dollar        \$     Percent       \%     Ampersand     \&
%   Acute accent  \'     Left paren    \(     Right paren   \)
%   Asterisk      \*     Plus          \+     Comma         \,
%   Minus         \-     Point         \.     Solidus       \/
%   Colon         \:     Semicolon     \;     Less than     \<
%   Equals        \=     Greater than  \>     Question mark \?
%   Commercial at \@     Left bracket  \[     Backslash     \\
%   Right bracket \]     Circumflex    \^     Underscore    \_
%   Grave accent  \`     Left brace    \{     Vertical bar  \|
%   Right brace   \}     Tilde         \~}
%
% \GetFileInfo{magicnum.drv}
%
% \title{The \xpackage{magicnum} package}
% \date{2011/04/10 v1.4}
% \author{Heiko Oberdiek\\\xemail{heiko.oberdiek at googlemail.com}}
%
% \maketitle
%
% \begin{abstract}
% This packages allows to access magic numbers by a hierarchical
% name system.
% \end{abstract}
%
% \tableofcontents
%
% \hypersetup{bookmarksopenlevel=2}
% \section{Documentation}
%
% \subsection{Introduction}
%
% Especially since \eTeX\ there are many integer values
% with special meanings, such as catcodes, group types, \dots
% Package \xpackage{etex}, enabled by options, defines
% macros in the user namespace for these values.
%
% This package goes another approach for storing the names and values.
% \begin{itemize}
% \item If \LuaTeX\ is available, they
% are stored in Lua tables.
% \item Without \LuaTeX\ they are remembered using internal
% macros.
% \end{itemize}
%
% \subsection{User interface}
%
% The integer values and names are organized in a hierarchical
% scheme of categories with the property names as leaves.
% Example: \eTeX's \cs{currentgrouplevel} reports |2| for a
% group caused by \cs{hbox}. This package has choosen to organize
% the group types in a main category |etex| and its subcategory
% |grouptype|:
% \begin{quote}
%   |etex.grouptype.hbox| = |2|
% \end{quote}
% The property name |hbox| in category |etex.grouptype| has value |2|.
% Dots are used to separate components.
%
% If you want to have the value, the access key is constructed by
% the category with all its components and the property name.
% For the opposite the value is used instead of the property name.
%
% Values are always integers (including negative numbers).
%
% \subsubsection{\cs{magicnum}}
%
% \begin{declcs}{magicnum} \M{access key}
% \end{declcs}
% Macro \cs{magicnum} expects an access key as argument and
% expands to the requested data. The macro is always expandable.
% In case of errors the expansion result is empty.
%
% The same macro is also used for getting a property name.
% In this case the property name part in the access key is
% replaced by the value.
%
% The catcodes
% of the resulting numbers and strings follow \TeX's tradition of
% \cs{string}, \cs{meaning}, \dots: The space has catcode 10
% (|tex.catcode.space|) and the other characters have catcode
% 12 (|tex.catcode.other|).
%
% Examples:
% \begin{quote}
%   |\magicnum{etex.grouptype.hbox}| $\Rightarrow$ |2|\\
%   |\magicnum{tex.catcode.14}| $\Rightarrow$ |comment|\\
%   |\magicnum{tex.catcode.undefined}| $\Rightarrow$ $\emptyset$
% \end{quote}
%
% \subsubsection{Properties}
%
% \begin{itemize}
% \item The components of a category are either subcategories or
%       key value pairs, but not both.
% \item The full specified property names are unique and thus
%       has one integer value exactly.
% \item Also the values inside a category are unique.
%       This condition is a prerequisite for the reverse mapping
%       of \cs{magicnum}.
% \item All names start with a letter. Only letters or digits
%       may follow.
% \end{itemize}
%
% \subsection{Data}
%
%  \subsubsection{\texorpdfstring{Category }{}\texttt{tex.catcode}}
%
% \begin{quote}
% \begin{tabular}{@{}>{\ttfamily}l>{\ttfamily}l@{}}
%    tex.catcode.escape & 0\\
%    tex.catcode.begingroup & 1\\
%    tex.catcode.endgroup & 2\\
%    tex.catcode.math & 3\\
%    tex.catcode.align & 4\\
%    tex.catcode.eol & 5\\
%    tex.catcode.parameter & 6\\
%    tex.catcode.superscript & 7\\
%    tex.catcode.subscript & 8\\
%    tex.catcode.ignore & 9\\
%    tex.catcode.space & 10\\
%    tex.catcode.letter & 11\\
%    tex.catcode.other & 12\\
%    tex.catcode.active & 13\\
%    tex.catcode.comment & 14\\
%    tex.catcode.invalid & 15\\
%  \end{tabular}
%  \end{quote}
%
%  \subsubsection{\texorpdfstring{Category }{}\texttt{etex.grouptype}}
%
% \begin{quote}
% \begin{tabular}{@{}>{\ttfamily}l>{\ttfamily}l@{}}
%    etex.grouptype.bottomlevel & 0\\
%    etex.grouptype.simple & 1\\
%    etex.grouptype.hbox & 2\\
%    etex.grouptype.adjustedhbox & 3\\
%    etex.grouptype.vbox & 4\\
%    etex.grouptype.align & 5\\
%    etex.grouptype.noalign & 6\\
%    etex.grouptype.output & 8\\
%    etex.grouptype.math & 9\\
%    etex.grouptype.disc & 10\\
%    etex.grouptype.insert & 11\\
%    etex.grouptype.vcenter & 12\\
%    etex.grouptype.mathchoice & 13\\
%    etex.grouptype.semisimple & 14\\
%    etex.grouptype.mathshift & 15\\
%    etex.grouptype.mathleft & 16\\
%  \end{tabular}
%  \end{quote}
%
%  \subsubsection{\texorpdfstring{Category }{}\texttt{etex.iftype}}
%
% \begin{quote}
% \begin{tabular}{@{}>{\ttfamily}l>{\ttfamily}l@{}}
%    etex.iftype.none & 0\\
%    etex.iftype.char & 1\\
%    etex.iftype.cat & 2\\
%    etex.iftype.num & 3\\
%    etex.iftype.dim & 4\\
%    etex.iftype.odd & 5\\
%    etex.iftype.vmode & 6\\
%    etex.iftype.hmode & 7\\
%    etex.iftype.mmode & 8\\
%    etex.iftype.inner & 9\\
%    etex.iftype.void & 10\\
%    etex.iftype.hbox & 11\\
%    etex.iftype.vbox & 12\\
%    etex.iftype.x & 13\\
%    etex.iftype.eof & 14\\
%    etex.iftype.true & 15\\
%    etex.iftype.false & 16\\
%    etex.iftype.case & 17\\
%    etex.iftype.defined & 18\\
%    etex.iftype.csname & 19\\
%    etex.iftype.fontchar & 20\\
%  \end{tabular}
%  \end{quote}
%
%  \subsubsection{\texorpdfstring{Category }{}\texttt{etex.nodetype}}
%
% \begin{quote}
% \begin{tabular}{@{}>{\ttfamily}l>{\ttfamily}l@{}}
%    etex.nodetype.none & -1\\
%    etex.nodetype.char & 0\\
%    etex.nodetype.hlist & 1\\
%    etex.nodetype.vlist & 2\\
%    etex.nodetype.rule & 3\\
%    etex.nodetype.ins & 4\\
%    etex.nodetype.mark & 5\\
%    etex.nodetype.adjust & 6\\
%    etex.nodetype.ligature & 7\\
%    etex.nodetype.disc & 8\\
%    etex.nodetype.whatsit & 9\\
%    etex.nodetype.math & 10\\
%    etex.nodetype.glue & 11\\
%    etex.nodetype.kern & 12\\
%    etex.nodetype.penalty & 13\\
%    etex.nodetype.unset & 14\\
%    etex.nodetype.maths & 15\\
%  \end{tabular}
%  \end{quote}
%
%  \subsubsection{\texorpdfstring{Category }{}\texttt{etex.interactionmode}}
%
% \begin{quote}
% \begin{tabular}{@{}>{\ttfamily}l>{\ttfamily}l@{}}
%    etex.interactionmode.batch & 0\\
%    etex.interactionmode.nonstop & 1\\
%    etex.interactionmode.scroll & 2\\
%    etex.interactionmode.errorstop & 3\\
%  \end{tabular}
%  \end{quote}
%
%  \subsubsection{\texorpdfstring{Category }{}\texttt{luatex.pdfliteral.mode}}
%
% \begin{quote}
% \begin{tabular}{@{}>{\ttfamily}l>{\ttfamily}l@{}}
%    luatex.pdfliteral.mode.setorigin & 0\\
%    luatex.pdfliteral.mode.page & 1\\
%    luatex.pdfliteral.mode.direct & 2\\
%  \end{tabular}
%  \end{quote}
%
%
% \hypersetup{bookmarksopenlevel=1}
%
% \StopEventually{
% }
%
% \section{Implementation}
%
%    \begin{macrocode}
%<*package>
%    \end{macrocode}
%
% \subsection{Reload check and package identification}
%    Reload check, especially if the package is not used with \LaTeX.
%    \begin{macrocode}
\begingroup\catcode61\catcode48\catcode32=10\relax%
  \catcode13=5 % ^^M
  \endlinechar=13 %
  \catcode35=6 % #
  \catcode39=12 % '
  \catcode44=12 % ,
  \catcode45=12 % -
  \catcode46=12 % .
  \catcode58=12 % :
  \catcode64=11 % @
  \catcode123=1 % {
  \catcode125=2 % }
  \expandafter\let\expandafter\x\csname ver@magicnum.sty\endcsname
  \ifx\x\relax % plain-TeX, first loading
  \else
    \def\empty{}%
    \ifx\x\empty % LaTeX, first loading,
      % variable is initialized, but \ProvidesPackage not yet seen
    \else
      \expandafter\ifx\csname PackageInfo\endcsname\relax
        \def\x#1#2{%
          \immediate\write-1{Package #1 Info: #2.}%
        }%
      \else
        \def\x#1#2{\PackageInfo{#1}{#2, stopped}}%
      \fi
      \x{magicnum}{The package is already loaded}%
      \aftergroup\endinput
    \fi
  \fi
\endgroup%
%    \end{macrocode}
%    Package identification:
%    \begin{macrocode}
\begingroup\catcode61\catcode48\catcode32=10\relax%
  \catcode13=5 % ^^M
  \endlinechar=13 %
  \catcode35=6 % #
  \catcode39=12 % '
  \catcode40=12 % (
  \catcode41=12 % )
  \catcode44=12 % ,
  \catcode45=12 % -
  \catcode46=12 % .
  \catcode47=12 % /
  \catcode58=12 % :
  \catcode64=11 % @
  \catcode91=12 % [
  \catcode93=12 % ]
  \catcode123=1 % {
  \catcode125=2 % }
  \expandafter\ifx\csname ProvidesPackage\endcsname\relax
    \def\x#1#2#3[#4]{\endgroup
      \immediate\write-1{Package: #3 #4}%
      \xdef#1{#4}%
    }%
  \else
    \def\x#1#2[#3]{\endgroup
      #2[{#3}]%
      \ifx#1\@undefined
        \xdef#1{#3}%
      \fi
      \ifx#1\relax
        \xdef#1{#3}%
      \fi
    }%
  \fi
\expandafter\x\csname ver@magicnum.sty\endcsname
\ProvidesPackage{magicnum}%
  [2011/04/10 v1.4 Magic numbers (HO)]%
%    \end{macrocode}
%
% \subsection{Catcodes}
%
%    \begin{macrocode}
\begingroup\catcode61\catcode48\catcode32=10\relax%
  \catcode13=5 % ^^M
  \endlinechar=13 %
  \catcode123=1 % {
  \catcode125=2 % }
  \catcode64=11 % @
  \def\x{\endgroup
    \expandafter\edef\csname magicnum@AtEnd\endcsname{%
      \endlinechar=\the\endlinechar\relax
      \catcode13=\the\catcode13\relax
      \catcode32=\the\catcode32\relax
      \catcode35=\the\catcode35\relax
      \catcode61=\the\catcode61\relax
      \catcode64=\the\catcode64\relax
      \catcode123=\the\catcode123\relax
      \catcode125=\the\catcode125\relax
    }%
  }%
\x\catcode61\catcode48\catcode32=10\relax%
\catcode13=5 % ^^M
\endlinechar=13 %
\catcode35=6 % #
\catcode64=11 % @
\catcode123=1 % {
\catcode125=2 % }
\def\TMP@EnsureCode#1#2{%
  \edef\magicnum@AtEnd{%
    \magicnum@AtEnd
    \catcode#1=\the\catcode#1\relax
  }%
  \catcode#1=#2\relax
}
\TMP@EnsureCode{34}{12}% "
\TMP@EnsureCode{39}{12}% '
\TMP@EnsureCode{40}{12}% (
\TMP@EnsureCode{41}{12}% )
\TMP@EnsureCode{42}{12}% *
\TMP@EnsureCode{44}{12}% ,
\TMP@EnsureCode{45}{12}% -
\TMP@EnsureCode{46}{12}% .
\TMP@EnsureCode{47}{12}% /
\TMP@EnsureCode{58}{12}% :
\TMP@EnsureCode{60}{12}% <
\TMP@EnsureCode{62}{12}% >
\TMP@EnsureCode{91}{12}% [
\TMP@EnsureCode{93}{12}% ]
\edef\magicnum@AtEnd{\magicnum@AtEnd\noexpand\endinput}
%    \end{macrocode}
%
% \subsection{Check for previous definition}
%
%    \begin{macrocode}
\begingroup\expandafter\expandafter\expandafter\endgroup
\expandafter\ifx\csname newcommand\endcsname\relax
  \expandafter\ifx\csname magicnum\endcsname\relax
  \else
    \input infwarerr.sty\relax
    \@PackageError{magicnum}{%
      \string\magicnum\space is already defined%
    }\@ehc
  \fi
\else
  \newcommand*{\magicnum}{}%
\fi
%    \end{macrocode}
%
% \subsection{Without \LuaTeX}
%
%    \begin{macrocode}
\begingroup\expandafter\expandafter\expandafter\endgroup
\expandafter\ifx\csname directlua\endcsname\relax
%    \end{macrocode}
%
%    \begin{macro}{\magicnum}
%    \begin{macrocode}
  \begingroup\expandafter\expandafter\expandafter\endgroup
  \expandafter\ifx\csname ifcsname\endcsname\relax
    \def\magicnum#1{%
      \expandafter\ifx\csname MG@#1\endcsname\relax
      \else
        \csname MG@#1\endcsname
      \fi
    }%
  \else
    \begingroup
      \edef\x{\endgroup
        \def\noexpand\magicnum##1{%
          \expandafter\noexpand\csname
          ifcsname\endcsname MG@##1\noexpand\endcsname
            \noexpand\csname MG@##1%
                 \noexpand\expandafter\noexpand\endcsname
          \expandafter\noexpand\csname fi\endcsname
        }%
      }%
    \x
  \fi
%    \end{macrocode}
%    \end{macro}
%
%    \begin{macrocode}
\else
%    \end{macrocode}
%
% \subsection{With \LuaTeX}
%
%    \begin{macrocode}
  \begingroup\expandafter\expandafter\expandafter\endgroup
  \expandafter\ifx\csname RequirePackage\endcsname\relax
    \input ifluatex.sty\relax
    \input infwarerr.sty\relax
  \else
    \RequirePackage{ifluatex}[2010/03/01]%
    \RequirePackage{infwarerr}[2010/04/08]%
  \fi
%    \end{macrocode}
%
%    \begin{macro}{\magicnum@directlua}
%    \begin{macrocode}
  \ifnum\luatexversion<36 %
    \def\magicnum@directlua{\directlua0 }%
  \else
    \let\magicnum@directlua\directlua
  \fi
%    \end{macrocode}
%    \end{macro}
%    \begin{macrocode}
  \magicnum@directlua{%
    require("oberdiek.magicnum")%
  }%
  \begingroup
    \def\x{2011/04/10 v1.4}%
    \def\StripPrefix#1>{}%
    \edef\x{\expandafter\StripPrefix\meaning\x}%
    \edef\y{%
      \magicnum@directlua{%
        if oberdiek.magicnum.getversion then %
          oberdiek.magicnum.getversion()%
        end%
      }%
    }%
    \ifx\x\y
    \else
      \@PackageError{magicnum}{%
        Wrong version of lua module.\MessageBreak
        Package version: \x\MessageBreak
        Lua module: \y
      }\@ehc
    \fi
  \endgroup
%    \end{macrocode}
%    \begin{macro}{\luaescapestring}
%    \begin{macrocode}
  \begingroup
    \expandafter\ifx\csname luaescapestring\endcsname\relax
      \directlua{%
        if tex.enableprimitives then %
          tex.enableprimitives('magicnum@', {'luaescapestring'})%
        end%
      }%
      \global\let\luaescapestring\magicnum@luaescapestring
    \fi
    \expandafter\ifx\csname luaescapestring\endcsname\relax
      \escapechar=92 %
      \@PackageError{magicnum}{%
        Missing \string\luaescapestring
      }\@ehc
    \fi
  \endgroup
%    \end{macrocode}
%    \end{macro}
%    \begin{macro}{\magicnum}
%    \begin{macrocode}
  \def\magicnum#1{%
    \magicnum@directlua{%
      oberdiek.magicnum.get("\luaescapestring{#1}")%
    }%
  }%
%    \end{macrocode}
%    \end{macro}
%
%    \begin{macrocode}
  \expandafter\magicnum@AtEnd
\fi%
%</package>
%    \end{macrocode}
%
% \subsection{Data}
%
% \subsubsection{Plain data}
%
%    \begin{macrocode}
%<*data>
tex.catcode
  escape = 0
  begingroup = 1
  endgroup = 2
  math = 3
  align = 4
  eol = 5
  parameter = 6
  superscript = 7
  subscript = 8
  ignore = 9
  space = 10
  letter = 11
  other = 12
  active = 13
  comment = 14
  invalid = 15
etex.grouptype
  bottomlevel = 0
  simple = 1
  hbox = 2
  adjustedhbox = 3
  vbox = 4
  align = 5
  noalign = 6
  output = 8
  math = 9
  disc = 10
  insert = 11
  vcenter = 12
  mathchoice = 13
  semisimple = 14
  mathshift = 15
  mathleft = 16
etex.iftype
  none = 0
  char = 1
  cat = 2
  num = 3
  dim = 4
  odd = 5
  vmode = 6
  hmode = 7
  mmode = 8
  inner = 9
  void = 10
  hbox = 11
  vbox = 12
  x = 13
  eof = 14
  true = 15
  false = 16
  case = 17
  defined = 18
  csname = 19
  fontchar = 20
etex.nodetype
  none = -1
  char = 0
  hlist = 1
  vlist = 2
  rule = 3
  ins = 4
  mark = 5
  adjust = 6
  ligature = 7
  disc = 8
  whatsit = 9
  math = 10
  glue = 11
  kern = 12
  penalty = 13
  unset = 14
  maths = 15
etex.interactionmode
  batch = 0
  nonstop = 1
  scroll = 2
  errorstop = 3
luatex.pdfliteral.mode
  setorigin = 0
  page = 1
  direct = 2
%</data>
%    \end{macrocode}
%
% \subsubsection{Data for \TeX}
%
%    \begin{macrocode}
%<*package>
%    \end{macrocode}
%    \begin{macro}{\magicnum@add}
%    \begin{macrocode}
\begingroup\expandafter\expandafter\expandafter\endgroup
\expandafter\ifx\csname detokenize\endcsname\relax
  \def\magicnum@add#1#2#3{%
    \expandafter\magicnum@@add
        \csname MG@#1.#2\expandafter\endcsname
        \csname MG@#1.#3\endcsname
       {#3}{#2}%
  }%
  \def\magicnum@@add#1#2#3#4{%
    \def#1{#3}%
    \def#2{#4}%
    \edef#1{%
      \expandafter\strip@prefix\meaning#1%
    }%
    \edef#2{%
      \expandafter\strip@prefix\meaning#2%
    }%
  }%
  \expandafter\ifx\csname strip@prefix\endcsname\relax
    \def\strip@prefix#1->{}%
  \fi
\else
  \def\magicnum@add#1#2#3{%
    \expandafter\edef\csname MG@#1.#2\endcsname{%
      \detokenize{#3}%
    }%
    \expandafter\edef\csname MG@#1.#3\endcsname{%
      \detokenize{#2}%
    }%
  }%
\fi
%    \end{macrocode}
%    \end{macro}
%    \begin{macrocode}
\magicnum@add{tex.catcode}{escape}{0}
\magicnum@add{tex.catcode}{begingroup}{1}
\magicnum@add{tex.catcode}{endgroup}{2}
\magicnum@add{tex.catcode}{math}{3}
\magicnum@add{tex.catcode}{align}{4}
\magicnum@add{tex.catcode}{eol}{5}
\magicnum@add{tex.catcode}{parameter}{6}
\magicnum@add{tex.catcode}{superscript}{7}
\magicnum@add{tex.catcode}{subscript}{8}
\magicnum@add{tex.catcode}{ignore}{9}
\magicnum@add{tex.catcode}{space}{10}
\magicnum@add{tex.catcode}{letter}{11}
\magicnum@add{tex.catcode}{other}{12}
\magicnum@add{tex.catcode}{active}{13}
\magicnum@add{tex.catcode}{comment}{14}
\magicnum@add{tex.catcode}{invalid}{15}
\magicnum@add{etex.grouptype}{bottomlevel}{0}
\magicnum@add{etex.grouptype}{simple}{1}
\magicnum@add{etex.grouptype}{hbox}{2}
\magicnum@add{etex.grouptype}{adjustedhbox}{3}
\magicnum@add{etex.grouptype}{vbox}{4}
\magicnum@add{etex.grouptype}{align}{5}
\magicnum@add{etex.grouptype}{noalign}{6}
\magicnum@add{etex.grouptype}{output}{8}
\magicnum@add{etex.grouptype}{math}{9}
\magicnum@add{etex.grouptype}{disc}{10}
\magicnum@add{etex.grouptype}{insert}{11}
\magicnum@add{etex.grouptype}{vcenter}{12}
\magicnum@add{etex.grouptype}{mathchoice}{13}
\magicnum@add{etex.grouptype}{semisimple}{14}
\magicnum@add{etex.grouptype}{mathshift}{15}
\magicnum@add{etex.grouptype}{mathleft}{16}
\magicnum@add{etex.iftype}{none}{0}
\magicnum@add{etex.iftype}{char}{1}
\magicnum@add{etex.iftype}{cat}{2}
\magicnum@add{etex.iftype}{num}{3}
\magicnum@add{etex.iftype}{dim}{4}
\magicnum@add{etex.iftype}{odd}{5}
\magicnum@add{etex.iftype}{vmode}{6}
\magicnum@add{etex.iftype}{hmode}{7}
\magicnum@add{etex.iftype}{mmode}{8}
\magicnum@add{etex.iftype}{inner}{9}
\magicnum@add{etex.iftype}{void}{10}
\magicnum@add{etex.iftype}{hbox}{11}
\magicnum@add{etex.iftype}{vbox}{12}
\magicnum@add{etex.iftype}{x}{13}
\magicnum@add{etex.iftype}{eof}{14}
\magicnum@add{etex.iftype}{true}{15}
\magicnum@add{etex.iftype}{false}{16}
\magicnum@add{etex.iftype}{case}{17}
\magicnum@add{etex.iftype}{defined}{18}
\magicnum@add{etex.iftype}{csname}{19}
\magicnum@add{etex.iftype}{fontchar}{20}
\magicnum@add{etex.nodetype}{none}{-1}
\magicnum@add{etex.nodetype}{char}{0}
\magicnum@add{etex.nodetype}{hlist}{1}
\magicnum@add{etex.nodetype}{vlist}{2}
\magicnum@add{etex.nodetype}{rule}{3}
\magicnum@add{etex.nodetype}{ins}{4}
\magicnum@add{etex.nodetype}{mark}{5}
\magicnum@add{etex.nodetype}{adjust}{6}
\magicnum@add{etex.nodetype}{ligature}{7}
\magicnum@add{etex.nodetype}{disc}{8}
\magicnum@add{etex.nodetype}{whatsit}{9}
\magicnum@add{etex.nodetype}{math}{10}
\magicnum@add{etex.nodetype}{glue}{11}
\magicnum@add{etex.nodetype}{kern}{12}
\magicnum@add{etex.nodetype}{penalty}{13}
\magicnum@add{etex.nodetype}{unset}{14}
\magicnum@add{etex.nodetype}{maths}{15}
\magicnum@add{etex.interactionmode}{batch}{0}
\magicnum@add{etex.interactionmode}{nonstop}{1}
\magicnum@add{etex.interactionmode}{scroll}{2}
\magicnum@add{etex.interactionmode}{errorstop}{3}
\magicnum@add{luatex.pdfliteral.mode}{setorigin}{0}
\magicnum@add{luatex.pdfliteral.mode}{page}{1}
\magicnum@add{luatex.pdfliteral.mode}{direct}{2}
%    \end{macrocode}
%    \begin{macrocode}
\magicnum@AtEnd%
%</package>
%    \end{macrocode}
%
% \subsubsection{Lua module}
%
%    \begin{macrocode}
%<*lua>
%    \end{macrocode}
%    \begin{macrocode}
module("oberdiek.magicnum", package.seeall)
%    \end{macrocode}
%    \begin{macrocode}
function getversion()
  tex.write("2011/04/10 v1.4")
end
%    \end{macrocode}
%    \begin{macrocode}
local data = {
  ["tex.catcode"] = {
    [0] = "escape",
    [1] = "begingroup",
    [2] = "endgroup",
    [3] = "math",
    [4] = "align",
    [5] = "eol",
    [6] = "parameter",
    [7] = "superscript",
    [8] = "subscript",
    [9] = "ignore",
    [10] = "space",
    [11] = "letter",
    [12] = "other",
    [13] = "active",
    [14] = "comment",
    [15] = "invalid",
    ["active"] = 13,
    ["align"] = 4,
    ["begingroup"] = 1,
    ["comment"] = 14,
    ["endgroup"] = 2,
    ["eol"] = 5,
    ["escape"] = 0,
    ["ignore"] = 9,
    ["invalid"] = 15,
    ["letter"] = 11,
    ["math"] = 3,
    ["other"] = 12,
    ["parameter"] = 6,
    ["space"] = 10,
    ["subscript"] = 8,
    ["superscript"] = 7
  },
  ["etex.grouptype"] = {
    [0] = "bottomlevel",
    [1] = "simple",
    [2] = "hbox",
    [3] = "adjustedhbox",
    [4] = "vbox",
    [5] = "align",
    [6] = "noalign",
    [8] = "output",
    [9] = "math",
    [10] = "disc",
    [11] = "insert",
    [12] = "vcenter",
    [13] = "mathchoice",
    [14] = "semisimple",
    [15] = "mathshift",
    [16] = "mathleft",
    ["adjustedhbox"] = 3,
    ["align"] = 5,
    ["bottomlevel"] = 0,
    ["disc"] = 10,
    ["hbox"] = 2,
    ["insert"] = 11,
    ["math"] = 9,
    ["mathchoice"] = 13,
    ["mathleft"] = 16,
    ["mathshift"] = 15,
    ["noalign"] = 6,
    ["output"] = 8,
    ["semisimple"] = 14,
    ["simple"] = 1,
    ["vbox"] = 4,
    ["vcenter"] = 12
  },
  ["etex.iftype"] = {
    [0] = "none",
    [1] = "char",
    [2] = "cat",
    [3] = "num",
    [4] = "dim",
    [5] = "odd",
    [6] = "vmode",
    [7] = "hmode",
    [8] = "mmode",
    [9] = "inner",
    [10] = "void",
    [11] = "hbox",
    [12] = "vbox",
    [13] = "x",
    [14] = "eof",
    [15] = "true",
    [16] = "false",
    [17] = "case",
    [18] = "defined",
    [19] = "csname",
    [20] = "fontchar",
    ["case"] = 17,
    ["cat"] = 2,
    ["char"] = 1,
    ["csname"] = 19,
    ["defined"] = 18,
    ["dim"] = 4,
    ["eof"] = 14,
    ["false"] = 16,
    ["fontchar"] = 20,
    ["hbox"] = 11,
    ["hmode"] = 7,
    ["inner"] = 9,
    ["mmode"] = 8,
    ["none"] = 0,
    ["num"] = 3,
    ["odd"] = 5,
    ["true"] = 15,
    ["vbox"] = 12,
    ["vmode"] = 6,
    ["void"] = 10,
    ["x"] = 13
  },
  ["etex.nodetype"] = {
    [-1] = "none",
    [0] = "char",
    [1] = "hlist",
    [2] = "vlist",
    [3] = "rule",
    [4] = "ins",
    [5] = "mark",
    [6] = "adjust",
    [7] = "ligature",
    [8] = "disc",
    [9] = "whatsit",
    [10] = "math",
    [11] = "glue",
    [12] = "kern",
    [13] = "penalty",
    [14] = "unset",
    [15] = "maths",
    ["adjust"] = 6,
    ["char"] = 0,
    ["disc"] = 8,
    ["glue"] = 11,
    ["hlist"] = 1,
    ["ins"] = 4,
    ["kern"] = 12,
    ["ligature"] = 7,
    ["mark"] = 5,
    ["math"] = 10,
    ["maths"] = 15,
    ["none"] = -1,
    ["penalty"] = 13,
    ["rule"] = 3,
    ["unset"] = 14,
    ["vlist"] = 2,
    ["whatsit"] = 9
  },
  ["etex.interactionmode"] = {
    [0] = "batch",
    [1] = "nonstop",
    [2] = "scroll",
    [3] = "errorstop",
    ["batch"] = 0,
    ["errorstop"] = 3,
    ["nonstop"] = 1,
    ["scroll"] = 2
  },
  ["luatex.pdfliteral.mode"] = {
    [0] = "setorigin",
    [1] = "page",
    [2] = "direct",
    ["direct"] = 2,
    ["page"] = 1,
    ["setorigin"] = 0
  }
}
%    \end{macrocode}
%    \begin{macrocode}
function get(name)
  local startpos, endpos, category, entry =
      string.find(name, "^(%a[%a%d%.]*)%.(-?[%a%d]+)$")
  if not entry then
    return
  end
  local node = data[category]
  if not node then
    return
  end
  local num = tonumber(entry)
  local value
  if num then
    value = node[num]
    if not value then
      return
    end
  else
    value = node[entry]
    if not value then
      return
    end
    value = "" .. value
  end
  tex.write(value)
end
%    \end{macrocode}
%
%    \begin{macrocode}
%</lua>
%    \end{macrocode}
%
% \section{Test}
%
% \subsection{Catcode checks for loading}
%
%    \begin{macrocode}
%<*test1>
%    \end{macrocode}
%    \begin{macrocode}
\catcode`\{=1 %
\catcode`\}=2 %
\catcode`\#=6 %
\catcode`\@=11 %
\expandafter\ifx\csname count@\endcsname\relax
  \countdef\count@=255 %
\fi
\expandafter\ifx\csname @gobble\endcsname\relax
  \long\def\@gobble#1{}%
\fi
\expandafter\ifx\csname @firstofone\endcsname\relax
  \long\def\@firstofone#1{#1}%
\fi
\expandafter\ifx\csname loop\endcsname\relax
  \expandafter\@firstofone
\else
  \expandafter\@gobble
\fi
{%
  \def\loop#1\repeat{%
    \def\body{#1}%
    \iterate
  }%
  \def\iterate{%
    \body
      \let\next\iterate
    \else
      \let\next\relax
    \fi
    \next
  }%
  \let\repeat=\fi
}%
\def\RestoreCatcodes{}
\count@=0 %
\loop
  \edef\RestoreCatcodes{%
    \RestoreCatcodes
    \catcode\the\count@=\the\catcode\count@\relax
  }%
\ifnum\count@<255 %
  \advance\count@ 1 %
\repeat

\def\RangeCatcodeInvalid#1#2{%
  \count@=#1\relax
  \loop
    \catcode\count@=15 %
  \ifnum\count@<#2\relax
    \advance\count@ 1 %
  \repeat
}
\def\RangeCatcodeCheck#1#2#3{%
  \count@=#1\relax
  \loop
    \ifnum#3=\catcode\count@
    \else
      \errmessage{%
        Character \the\count@\space
        with wrong catcode \the\catcode\count@\space
        instead of \number#3%
      }%
    \fi
  \ifnum\count@<#2\relax
    \advance\count@ 1 %
  \repeat
}
\def\space{ }
\expandafter\ifx\csname LoadCommand\endcsname\relax
  \def\LoadCommand{\input magicnum.sty\relax}%
\fi
\def\Test{%
  \RangeCatcodeInvalid{0}{47}%
  \RangeCatcodeInvalid{58}{64}%
  \RangeCatcodeInvalid{91}{96}%
  \RangeCatcodeInvalid{123}{255}%
  \catcode`\@=12 %
  \catcode`\\=0 %
  \catcode`\%=14 %
  \LoadCommand
  \RangeCatcodeCheck{0}{36}{15}%
  \RangeCatcodeCheck{37}{37}{14}%
  \RangeCatcodeCheck{38}{47}{15}%
  \RangeCatcodeCheck{48}{57}{12}%
  \RangeCatcodeCheck{58}{63}{15}%
  \RangeCatcodeCheck{64}{64}{12}%
  \RangeCatcodeCheck{65}{90}{11}%
  \RangeCatcodeCheck{91}{91}{15}%
  \RangeCatcodeCheck{92}{92}{0}%
  \RangeCatcodeCheck{93}{96}{15}%
  \RangeCatcodeCheck{97}{122}{11}%
  \RangeCatcodeCheck{123}{255}{15}%
  \RestoreCatcodes
}
\Test
\csname @@end\endcsname
\end
%    \end{macrocode}
%    \begin{macrocode}
%</test1>
%    \end{macrocode}
%
% \subsection{Test data}
%
%    \begin{macrocode}
%<*testplain>
\input magicnum.sty\relax
\def\Test#1#2{%
  \edef\result{\magicnum{#1}}%
  \edef\expect{#2}%
  \edef\expect{\expandafter\stripprefix\meaning\expect}%
  \ifx\result\expect
  \else
    \errmessage{%
      Failed: [#1] % hash-ok
      returns [\result] instead of [\expect]%
    }%
  \fi
}
\def\stripprefix#1->{}
%</testplain>
%    \end{macrocode}
%    \begin{macrocode}
%<*testlatex>
\NeedsTeXFormat{LaTeX2e}
\documentclass{minimal}
\usepackage{magicnum}[2011/04/10]
\usepackage{qstest}
\IncludeTests{*}
\LogTests{log}{*}{*}
\newcommand*{\Test}[2]{%
  \Expect*{\magicnum{#1}}{#2}%
}
\begin{qstest}{magicnum}{magicnum}
%</testlatex>
%    \end{macrocode}
%    \begin{macrocode}
%<*testdata>
\Test{tex.catcode.escape}{0}
\Test{tex.catcode.invalid}{15}
\Test{tex.catcode.unknown}{}
\Test{tex.catcode.0}{escape}
\Test{tex.catcode.15}{invalid}
\Test{etex.iftype.true}{15}
\Test{etex.iftype.false}{16}
\Test{etex.iftype.15}{true}
\Test{etex.iftype.16}{false}
\Test{etex.nodetype.none}{-1}
\Test{etex.nodetype.-1}{none}
\Test{luatex.pdfliteral.mode.direct}{2}
\Test{luatex.pdfliteral.mode.1}{page}
\Test{}{}
\Test{unknown}{}
\Test{unknown.foo.bar}{}
\Test{unknown.foo.4}{}
%</testdata>
%    \end{macrocode}
%    \begin{macrocode}
%<*testplain>
\csname @@end\endcsname
\end
%</testplain>
%<*testlatex>
\end{qstest}
\csname @@end\endcsname
%</testlatex>
%    \end{macrocode}
%
% \subsection{Small test for \hologo{iniTeX}}
%
%    \begin{macrocode}
%<*test4>
\catcode`\{=1
\catcode`\}=2
\catcode`\#=6
\input magicnum.sty\relax
\edef\x{\magicnum{tex.catcode.15}}
\edef\y{invalid}
\def\Strip#1>{}
\edef\y{\expandafter\Strip\meaning\y}
\ifx\x\y
  \immediate\write16{Ok}%
\else
  \errmessage{\x<>\y}%
\fi
\csname @@end\endcsname\end
%</test4>
%    \end{macrocode}
%
% \section{Installation}
%
% \subsection{Download}
%
% \paragraph{Package.} This package is available on
% CTAN\footnote{\url{ftp://ftp.ctan.org/tex-archive/}}:
% \begin{description}
% \item[\CTAN{macros/latex/contrib/oberdiek/magicnum.dtx}] The source file.
% \item[\CTAN{macros/latex/contrib/oberdiek/magicnum.pdf}] Documentation.
% \end{description}
%
%
% \paragraph{Bundle.} All the packages of the bundle `oberdiek'
% are also available in a TDS compliant ZIP archive. There
% the packages are already unpacked and the documentation files
% are generated. The files and directories obey the TDS standard.
% \begin{description}
% \item[\CTAN{install/macros/latex/contrib/oberdiek.tds.zip}]
% \end{description}
% \emph{TDS} refers to the standard ``A Directory Structure
% for \TeX\ Files'' (\CTAN{tds/tds.pdf}). Directories
% with \xfile{texmf} in their name are usually organized this way.
%
% \subsection{Bundle installation}
%
% \paragraph{Unpacking.} Unpack the \xfile{oberdiek.tds.zip} in the
% TDS tree (also known as \xfile{texmf} tree) of your choice.
% Example (linux):
% \begin{quote}
%   |unzip oberdiek.tds.zip -d ~/texmf|
% \end{quote}
%
% \paragraph{Script installation.}
% Check the directory \xfile{TDS:scripts/oberdiek/} for
% scripts that need further installation steps.
% Package \xpackage{attachfile2} comes with the Perl script
% \xfile{pdfatfi.pl} that should be installed in such a way
% that it can be called as \texttt{pdfatfi}.
% Example (linux):
% \begin{quote}
%   |chmod +x scripts/oberdiek/pdfatfi.pl|\\
%   |cp scripts/oberdiek/pdfatfi.pl /usr/local/bin/|
% \end{quote}
%
% \subsection{Package installation}
%
% \paragraph{Unpacking.} The \xfile{.dtx} file is a self-extracting
% \docstrip\ archive. The files are extracted by running the
% \xfile{.dtx} through \plainTeX:
% \begin{quote}
%   \verb|tex magicnum.dtx|
% \end{quote}
%
% \paragraph{TDS.} Now the different files must be moved into
% the different directories in your installation TDS tree
% (also known as \xfile{texmf} tree):
% \begin{quote}
% \def\t{^^A
% \begin{tabular}{@{}>{\ttfamily}l@{ $\rightarrow$ }>{\ttfamily}l@{}}
%   magicnum.sty & tex/generic/oberdiek/magicnum.sty\\
%   magicnum.lua & scripts/oberdiek/magicnum.lua\\
%   oberdiek.magicnum.lua & scripts/oberdiek/oberdiek.magicnum.lua\\
%   magicnum.pdf & doc/latex/oberdiek/magicnum.pdf\\
%   magicnum.txt & doc/latex/oberdiek/magicnum.txt\\
%   test/magicnum-test1.tex & doc/latex/oberdiek/test/magicnum-test1.tex\\
%   test/magicnum-test2.tex & doc/latex/oberdiek/test/magicnum-test2.tex\\
%   test/magicnum-test3.tex & doc/latex/oberdiek/test/magicnum-test3.tex\\
%   test/magicnum-test4.tex & doc/latex/oberdiek/test/magicnum-test4.tex\\
%   magicnum.dtx & source/latex/oberdiek/magicnum.dtx\\
% \end{tabular}^^A
% }^^A
% \sbox0{\t}^^A
% \ifdim\wd0>\linewidth
%   \begingroup
%     \advance\linewidth by\leftmargin
%     \advance\linewidth by\rightmargin
%   \edef\x{\endgroup
%     \def\noexpand\lw{\the\linewidth}^^A
%   }\x
%   \def\lwbox{^^A
%     \leavevmode
%     \hbox to \linewidth{^^A
%       \kern-\leftmargin\relax
%       \hss
%       \usebox0
%       \hss
%       \kern-\rightmargin\relax
%     }^^A
%   }^^A
%   \ifdim\wd0>\lw
%     \sbox0{\small\t}^^A
%     \ifdim\wd0>\linewidth
%       \ifdim\wd0>\lw
%         \sbox0{\footnotesize\t}^^A
%         \ifdim\wd0>\linewidth
%           \ifdim\wd0>\lw
%             \sbox0{\scriptsize\t}^^A
%             \ifdim\wd0>\linewidth
%               \ifdim\wd0>\lw
%                 \sbox0{\tiny\t}^^A
%                 \ifdim\wd0>\linewidth
%                   \lwbox
%                 \else
%                   \usebox0
%                 \fi
%               \else
%                 \lwbox
%               \fi
%             \else
%               \usebox0
%             \fi
%           \else
%             \lwbox
%           \fi
%         \else
%           \usebox0
%         \fi
%       \else
%         \lwbox
%       \fi
%     \else
%       \usebox0
%     \fi
%   \else
%     \lwbox
%   \fi
% \else
%   \usebox0
% \fi
% \end{quote}
% If you have a \xfile{docstrip.cfg} that configures and enables \docstrip's
% TDS installing feature, then some files can already be in the right
% place, see the documentation of \docstrip.
%
% \subsection{Refresh file name databases}
%
% If your \TeX~distribution
% (\teTeX, \mikTeX, \dots) relies on file name databases, you must refresh
% these. For example, \teTeX\ users run \verb|texhash| or
% \verb|mktexlsr|.
%
% \subsection{Some details for the interested}
%
% \paragraph{Attached source.}
%
% The PDF documentation on CTAN also includes the
% \xfile{.dtx} source file. It can be extracted by
% AcrobatReader 6 or higher. Another option is \textsf{pdftk},
% e.g. unpack the file into the current directory:
% \begin{quote}
%   \verb|pdftk magicnum.pdf unpack_files output .|
% \end{quote}
%
% \paragraph{Unpacking with \LaTeX.}
% The \xfile{.dtx} chooses its action depending on the format:
% \begin{description}
% \item[\plainTeX:] Run \docstrip\ and extract the files.
% \item[\LaTeX:] Generate the documentation.
% \end{description}
% If you insist on using \LaTeX\ for \docstrip\ (really,
% \docstrip\ does not need \LaTeX), then inform the autodetect routine
% about your intention:
% \begin{quote}
%   \verb|latex \let\install=y% \iffalse meta-comment
%
% File: magicnum.dtx
% Version: 2011/04/10 v1.4
% Info: Magic numbers
%
% Copyright (C) 2007, 2009-2011 by
%    Heiko Oberdiek <heiko.oberdiek at googlemail.com>
%
% This work may be distributed and/or modified under the
% conditions of the LaTeX Project Public License, either
% version 1.3c of this license or (at your option) any later
% version. This version of this license is in
%    http://www.latex-project.org/lppl/lppl-1-3c.txt
% and the latest version of this license is in
%    http://www.latex-project.org/lppl.txt
% and version 1.3 or later is part of all distributions of
% LaTeX version 2005/12/01 or later.
%
% This work has the LPPL maintenance status "maintained".
%
% This Current Maintainer of this work is Heiko Oberdiek.
%
% The Base Interpreter refers to any `TeX-Format',
% because some files are installed in TDS:tex/generic//.
%
% This work consists of the main source file magicnum.dtx
% and the derived files
%    magicnum.sty, magicnum.pdf, magicnum.ins, magicnum.drv, magicnum.txt,
%    magicnum-test1.tex, magicnum-test2.tex, magicnum-test3.tex,
%    magicnum-test4.tex, magicnum.lua, oberdiek.magicnum.lua.
%
% Distribution:
%    CTAN:macros/latex/contrib/oberdiek/magicnum.dtx
%    CTAN:macros/latex/contrib/oberdiek/magicnum.pdf
%
% Unpacking:
%    (a) If magicnum.ins is present:
%           tex magicnum.ins
%    (b) Without magicnum.ins:
%           tex magicnum.dtx
%    (c) If you insist on using LaTeX
%           latex \let\install=y% \iffalse meta-comment
%
% File: magicnum.dtx
% Version: 2011/04/10 v1.4
% Info: Magic numbers
%
% Copyright (C) 2007, 2009-2011 by
%    Heiko Oberdiek <heiko.oberdiek at googlemail.com>
%
% This work may be distributed and/or modified under the
% conditions of the LaTeX Project Public License, either
% version 1.3c of this license or (at your option) any later
% version. This version of this license is in
%    http://www.latex-project.org/lppl/lppl-1-3c.txt
% and the latest version of this license is in
%    http://www.latex-project.org/lppl.txt
% and version 1.3 or later is part of all distributions of
% LaTeX version 2005/12/01 or later.
%
% This work has the LPPL maintenance status "maintained".
%
% This Current Maintainer of this work is Heiko Oberdiek.
%
% The Base Interpreter refers to any `TeX-Format',
% because some files are installed in TDS:tex/generic//.
%
% This work consists of the main source file magicnum.dtx
% and the derived files
%    magicnum.sty, magicnum.pdf, magicnum.ins, magicnum.drv, magicnum.txt,
%    magicnum-test1.tex, magicnum-test2.tex, magicnum-test3.tex,
%    magicnum-test4.tex, magicnum.lua, oberdiek.magicnum.lua.
%
% Distribution:
%    CTAN:macros/latex/contrib/oberdiek/magicnum.dtx
%    CTAN:macros/latex/contrib/oberdiek/magicnum.pdf
%
% Unpacking:
%    (a) If magicnum.ins is present:
%           tex magicnum.ins
%    (b) Without magicnum.ins:
%           tex magicnum.dtx
%    (c) If you insist on using LaTeX
%           latex \let\install=y% \iffalse meta-comment
%
% File: magicnum.dtx
% Version: 2011/04/10 v1.4
% Info: Magic numbers
%
% Copyright (C) 2007, 2009-2011 by
%    Heiko Oberdiek <heiko.oberdiek at googlemail.com>
%
% This work may be distributed and/or modified under the
% conditions of the LaTeX Project Public License, either
% version 1.3c of this license or (at your option) any later
% version. This version of this license is in
%    http://www.latex-project.org/lppl/lppl-1-3c.txt
% and the latest version of this license is in
%    http://www.latex-project.org/lppl.txt
% and version 1.3 or later is part of all distributions of
% LaTeX version 2005/12/01 or later.
%
% This work has the LPPL maintenance status "maintained".
%
% This Current Maintainer of this work is Heiko Oberdiek.
%
% The Base Interpreter refers to any `TeX-Format',
% because some files are installed in TDS:tex/generic//.
%
% This work consists of the main source file magicnum.dtx
% and the derived files
%    magicnum.sty, magicnum.pdf, magicnum.ins, magicnum.drv, magicnum.txt,
%    magicnum-test1.tex, magicnum-test2.tex, magicnum-test3.tex,
%    magicnum-test4.tex, magicnum.lua, oberdiek.magicnum.lua.
%
% Distribution:
%    CTAN:macros/latex/contrib/oberdiek/magicnum.dtx
%    CTAN:macros/latex/contrib/oberdiek/magicnum.pdf
%
% Unpacking:
%    (a) If magicnum.ins is present:
%           tex magicnum.ins
%    (b) Without magicnum.ins:
%           tex magicnum.dtx
%    (c) If you insist on using LaTeX
%           latex \let\install=y\input{magicnum.dtx}
%        (quote the arguments according to the demands of your shell)
%
% Documentation:
%    (a) If magicnum.drv is present:
%           latex magicnum.drv
%    (b) Without magicnum.drv:
%           latex magicnum.dtx; ...
%    The class ltxdoc loads the configuration file ltxdoc.cfg
%    if available. Here you can specify further options, e.g.
%    use A4 as paper format:
%       \PassOptionsToClass{a4paper}{article}
%
%    Programm calls to get the documentation (example):
%       pdflatex magicnum.dtx
%       makeindex -s gind.ist magicnum.idx
%       pdflatex magicnum.dtx
%       makeindex -s gind.ist magicnum.idx
%       pdflatex magicnum.dtx
%
% Installation:
%    TDS:tex/generic/oberdiek/magicnum.sty
%    TDS:scripts/oberdiek/magicnum.lua
%    TDS:scripts/oberdiek/oberdiek.magicnum.lua
%    TDS:doc/latex/oberdiek/magicnum.pdf
%    TDS:doc/latex/oberdiek/magicnum.txt
%    TDS:doc/latex/oberdiek/test/magicnum-test1.tex
%    TDS:doc/latex/oberdiek/test/magicnum-test2.tex
%    TDS:doc/latex/oberdiek/test/magicnum-test3.tex
%    TDS:doc/latex/oberdiek/test/magicnum-test4.tex
%    TDS:source/latex/oberdiek/magicnum.dtx
%
%<*ignore>
\begingroup
  \catcode123=1 %
  \catcode125=2 %
  \def\x{LaTeX2e}%
\expandafter\endgroup
\ifcase 0\ifx\install y1\fi\expandafter
         \ifx\csname processbatchFile\endcsname\relax\else1\fi
         \ifx\fmtname\x\else 1\fi\relax
\else\csname fi\endcsname
%</ignore>
%<*install>
\input docstrip.tex
\Msg{************************************************************************}
\Msg{* Installation}
\Msg{* Package: magicnum 2011/04/10 v1.4 Magic numbers (HO)}
\Msg{************************************************************************}

\keepsilent
\askforoverwritefalse

\let\MetaPrefix\relax
\preamble

This is a generated file.

Project: magicnum
Version: 2011/04/10 v1.4

Copyright (C) 2007, 2009-2011 by
   Heiko Oberdiek <heiko.oberdiek at googlemail.com>

This work may be distributed and/or modified under the
conditions of the LaTeX Project Public License, either
version 1.3c of this license or (at your option) any later
version. This version of this license is in
   http://www.latex-project.org/lppl/lppl-1-3c.txt
and the latest version of this license is in
   http://www.latex-project.org/lppl.txt
and version 1.3 or later is part of all distributions of
LaTeX version 2005/12/01 or later.

This work has the LPPL maintenance status "maintained".

This Current Maintainer of this work is Heiko Oberdiek.

The Base Interpreter refers to any `TeX-Format',
because some files are installed in TDS:tex/generic//.

This work consists of the main source file magicnum.dtx
and the derived files
   magicnum.sty, magicnum.pdf, magicnum.ins, magicnum.drv, magicnum.txt,
   magicnum-test1.tex, magicnum-test2.tex, magicnum-test3.tex,
   magicnum-test4.tex, magicnum.lua, oberdiek.magicnum.lua.

\endpreamble
\let\MetaPrefix\DoubleperCent

\generate{%
  \file{magicnum.ins}{\from{magicnum.dtx}{install}}%
  \file{magicnum.drv}{\from{magicnum.dtx}{driver}}%
  \usedir{tex/generic/oberdiek}%
  \file{magicnum.sty}{\from{magicnum.dtx}{package}}%
  \usedir{doc/latex/oberdiek/test}%
  \file{magicnum-test1.tex}{\from{magicnum.dtx}{test1}}%
  \file{magicnum-test2.tex}{\from{magicnum.dtx}{testplain,testdata}}%
  \file{magicnum-test3.tex}{\from{magicnum.dtx}{testlatex,testdata}}%
  \file{magicnum-test4.tex}{\from{magicnum.dtx}{test4}}%
  \nopreamble
  \nopostamble
  \usedir{doc/latex/oberdiek}%
  \file{magicnum.txt}{\from{magicnum.dtx}{data}}%
  \usedir{source/latex/oberdiek/catalogue}%
  \file{magicnum.xml}{\from{magicnum.dtx}{catalogue}}%
}
\def\MetaPrefix{-- }
\def\defaultpostamble{%
  \MetaPrefix^^J%
  \MetaPrefix\space End of File `\outFileName'.%
}
\def\currentpostamble{\defaultpostamble}%
\generate{%
  \usedir{scripts/oberdiek}%
  \file{magicnum.lua}{\from{magicnum.dtx}{lua}}%
  \file{oberdiek.magicnum.lua}{\from{magicnum.dtx}{lua}}%
}

\catcode32=13\relax% active space
\let =\space%
\Msg{************************************************************************}
\Msg{*}
\Msg{* To finish the installation you have to move the following}
\Msg{* file into a directory searched by TeX:}
\Msg{*}
\Msg{*     magicnum.sty}
\Msg{*}
\Msg{* And install the following script files:}
\Msg{*}
\Msg{*     magicnum.lua, oberdiek.magicnum.lua}
\Msg{*}
\Msg{* To produce the documentation run the file `magicnum.drv'}
\Msg{* through LaTeX.}
\Msg{*}
\Msg{* Happy TeXing!}
\Msg{*}
\Msg{************************************************************************}

\endbatchfile
%</install>
%<*ignore>
\fi
%</ignore>
%<*driver>
\NeedsTeXFormat{LaTeX2e}
\ProvidesFile{magicnum.drv}%
  [2011/04/10 v1.4 Magic numbers (HO)]%
\documentclass{ltxdoc}
\usepackage{holtxdoc}[2011/11/22]
\usepackage{array}
\begin{document}
  \DocInput{magicnum.dtx}%
\end{document}
%</driver>
% \fi
%
% \CheckSum{755}
%
% \CharacterTable
%  {Upper-case    \A\B\C\D\E\F\G\H\I\J\K\L\M\N\O\P\Q\R\S\T\U\V\W\X\Y\Z
%   Lower-case    \a\b\c\d\e\f\g\h\i\j\k\l\m\n\o\p\q\r\s\t\u\v\w\x\y\z
%   Digits        \0\1\2\3\4\5\6\7\8\9
%   Exclamation   \!     Double quote  \"     Hash (number) \#
%   Dollar        \$     Percent       \%     Ampersand     \&
%   Acute accent  \'     Left paren    \(     Right paren   \)
%   Asterisk      \*     Plus          \+     Comma         \,
%   Minus         \-     Point         \.     Solidus       \/
%   Colon         \:     Semicolon     \;     Less than     \<
%   Equals        \=     Greater than  \>     Question mark \?
%   Commercial at \@     Left bracket  \[     Backslash     \\
%   Right bracket \]     Circumflex    \^     Underscore    \_
%   Grave accent  \`     Left brace    \{     Vertical bar  \|
%   Right brace   \}     Tilde         \~}
%
% \GetFileInfo{magicnum.drv}
%
% \title{The \xpackage{magicnum} package}
% \date{2011/04/10 v1.4}
% \author{Heiko Oberdiek\\\xemail{heiko.oberdiek at googlemail.com}}
%
% \maketitle
%
% \begin{abstract}
% This packages allows to access magic numbers by a hierarchical
% name system.
% \end{abstract}
%
% \tableofcontents
%
% \hypersetup{bookmarksopenlevel=2}
% \section{Documentation}
%
% \subsection{Introduction}
%
% Especially since \eTeX\ there are many integer values
% with special meanings, such as catcodes, group types, \dots
% Package \xpackage{etex}, enabled by options, defines
% macros in the user namespace for these values.
%
% This package goes another approach for storing the names and values.
% \begin{itemize}
% \item If \LuaTeX\ is available, they
% are stored in Lua tables.
% \item Without \LuaTeX\ they are remembered using internal
% macros.
% \end{itemize}
%
% \subsection{User interface}
%
% The integer values and names are organized in a hierarchical
% scheme of categories with the property names as leaves.
% Example: \eTeX's \cs{currentgrouplevel} reports |2| for a
% group caused by \cs{hbox}. This package has choosen to organize
% the group types in a main category |etex| and its subcategory
% |grouptype|:
% \begin{quote}
%   |etex.grouptype.hbox| = |2|
% \end{quote}
% The property name |hbox| in category |etex.grouptype| has value |2|.
% Dots are used to separate components.
%
% If you want to have the value, the access key is constructed by
% the category with all its components and the property name.
% For the opposite the value is used instead of the property name.
%
% Values are always integers (including negative numbers).
%
% \subsubsection{\cs{magicnum}}
%
% \begin{declcs}{magicnum} \M{access key}
% \end{declcs}
% Macro \cs{magicnum} expects an access key as argument and
% expands to the requested data. The macro is always expandable.
% In case of errors the expansion result is empty.
%
% The same macro is also used for getting a property name.
% In this case the property name part in the access key is
% replaced by the value.
%
% The catcodes
% of the resulting numbers and strings follow \TeX's tradition of
% \cs{string}, \cs{meaning}, \dots: The space has catcode 10
% (|tex.catcode.space|) and the other characters have catcode
% 12 (|tex.catcode.other|).
%
% Examples:
% \begin{quote}
%   |\magicnum{etex.grouptype.hbox}| $\Rightarrow$ |2|\\
%   |\magicnum{tex.catcode.14}| $\Rightarrow$ |comment|\\
%   |\magicnum{tex.catcode.undefined}| $\Rightarrow$ $\emptyset$
% \end{quote}
%
% \subsubsection{Properties}
%
% \begin{itemize}
% \item The components of a category are either subcategories or
%       key value pairs, but not both.
% \item The full specified property names are unique and thus
%       has one integer value exactly.
% \item Also the values inside a category are unique.
%       This condition is a prerequisite for the reverse mapping
%       of \cs{magicnum}.
% \item All names start with a letter. Only letters or digits
%       may follow.
% \end{itemize}
%
% \subsection{Data}
%
%  \subsubsection{\texorpdfstring{Category }{}\texttt{tex.catcode}}
%
% \begin{quote}
% \begin{tabular}{@{}>{\ttfamily}l>{\ttfamily}l@{}}
%    tex.catcode.escape & 0\\
%    tex.catcode.begingroup & 1\\
%    tex.catcode.endgroup & 2\\
%    tex.catcode.math & 3\\
%    tex.catcode.align & 4\\
%    tex.catcode.eol & 5\\
%    tex.catcode.parameter & 6\\
%    tex.catcode.superscript & 7\\
%    tex.catcode.subscript & 8\\
%    tex.catcode.ignore & 9\\
%    tex.catcode.space & 10\\
%    tex.catcode.letter & 11\\
%    tex.catcode.other & 12\\
%    tex.catcode.active & 13\\
%    tex.catcode.comment & 14\\
%    tex.catcode.invalid & 15\\
%  \end{tabular}
%  \end{quote}
%
%  \subsubsection{\texorpdfstring{Category }{}\texttt{etex.grouptype}}
%
% \begin{quote}
% \begin{tabular}{@{}>{\ttfamily}l>{\ttfamily}l@{}}
%    etex.grouptype.bottomlevel & 0\\
%    etex.grouptype.simple & 1\\
%    etex.grouptype.hbox & 2\\
%    etex.grouptype.adjustedhbox & 3\\
%    etex.grouptype.vbox & 4\\
%    etex.grouptype.align & 5\\
%    etex.grouptype.noalign & 6\\
%    etex.grouptype.output & 8\\
%    etex.grouptype.math & 9\\
%    etex.grouptype.disc & 10\\
%    etex.grouptype.insert & 11\\
%    etex.grouptype.vcenter & 12\\
%    etex.grouptype.mathchoice & 13\\
%    etex.grouptype.semisimple & 14\\
%    etex.grouptype.mathshift & 15\\
%    etex.grouptype.mathleft & 16\\
%  \end{tabular}
%  \end{quote}
%
%  \subsubsection{\texorpdfstring{Category }{}\texttt{etex.iftype}}
%
% \begin{quote}
% \begin{tabular}{@{}>{\ttfamily}l>{\ttfamily}l@{}}
%    etex.iftype.none & 0\\
%    etex.iftype.char & 1\\
%    etex.iftype.cat & 2\\
%    etex.iftype.num & 3\\
%    etex.iftype.dim & 4\\
%    etex.iftype.odd & 5\\
%    etex.iftype.vmode & 6\\
%    etex.iftype.hmode & 7\\
%    etex.iftype.mmode & 8\\
%    etex.iftype.inner & 9\\
%    etex.iftype.void & 10\\
%    etex.iftype.hbox & 11\\
%    etex.iftype.vbox & 12\\
%    etex.iftype.x & 13\\
%    etex.iftype.eof & 14\\
%    etex.iftype.true & 15\\
%    etex.iftype.false & 16\\
%    etex.iftype.case & 17\\
%    etex.iftype.defined & 18\\
%    etex.iftype.csname & 19\\
%    etex.iftype.fontchar & 20\\
%  \end{tabular}
%  \end{quote}
%
%  \subsubsection{\texorpdfstring{Category }{}\texttt{etex.nodetype}}
%
% \begin{quote}
% \begin{tabular}{@{}>{\ttfamily}l>{\ttfamily}l@{}}
%    etex.nodetype.none & -1\\
%    etex.nodetype.char & 0\\
%    etex.nodetype.hlist & 1\\
%    etex.nodetype.vlist & 2\\
%    etex.nodetype.rule & 3\\
%    etex.nodetype.ins & 4\\
%    etex.nodetype.mark & 5\\
%    etex.nodetype.adjust & 6\\
%    etex.nodetype.ligature & 7\\
%    etex.nodetype.disc & 8\\
%    etex.nodetype.whatsit & 9\\
%    etex.nodetype.math & 10\\
%    etex.nodetype.glue & 11\\
%    etex.nodetype.kern & 12\\
%    etex.nodetype.penalty & 13\\
%    etex.nodetype.unset & 14\\
%    etex.nodetype.maths & 15\\
%  \end{tabular}
%  \end{quote}
%
%  \subsubsection{\texorpdfstring{Category }{}\texttt{etex.interactionmode}}
%
% \begin{quote}
% \begin{tabular}{@{}>{\ttfamily}l>{\ttfamily}l@{}}
%    etex.interactionmode.batch & 0\\
%    etex.interactionmode.nonstop & 1\\
%    etex.interactionmode.scroll & 2\\
%    etex.interactionmode.errorstop & 3\\
%  \end{tabular}
%  \end{quote}
%
%  \subsubsection{\texorpdfstring{Category }{}\texttt{luatex.pdfliteral.mode}}
%
% \begin{quote}
% \begin{tabular}{@{}>{\ttfamily}l>{\ttfamily}l@{}}
%    luatex.pdfliteral.mode.setorigin & 0\\
%    luatex.pdfliteral.mode.page & 1\\
%    luatex.pdfliteral.mode.direct & 2\\
%  \end{tabular}
%  \end{quote}
%
%
% \hypersetup{bookmarksopenlevel=1}
%
% \StopEventually{
% }
%
% \section{Implementation}
%
%    \begin{macrocode}
%<*package>
%    \end{macrocode}
%
% \subsection{Reload check and package identification}
%    Reload check, especially if the package is not used with \LaTeX.
%    \begin{macrocode}
\begingroup\catcode61\catcode48\catcode32=10\relax%
  \catcode13=5 % ^^M
  \endlinechar=13 %
  \catcode35=6 % #
  \catcode39=12 % '
  \catcode44=12 % ,
  \catcode45=12 % -
  \catcode46=12 % .
  \catcode58=12 % :
  \catcode64=11 % @
  \catcode123=1 % {
  \catcode125=2 % }
  \expandafter\let\expandafter\x\csname ver@magicnum.sty\endcsname
  \ifx\x\relax % plain-TeX, first loading
  \else
    \def\empty{}%
    \ifx\x\empty % LaTeX, first loading,
      % variable is initialized, but \ProvidesPackage not yet seen
    \else
      \expandafter\ifx\csname PackageInfo\endcsname\relax
        \def\x#1#2{%
          \immediate\write-1{Package #1 Info: #2.}%
        }%
      \else
        \def\x#1#2{\PackageInfo{#1}{#2, stopped}}%
      \fi
      \x{magicnum}{The package is already loaded}%
      \aftergroup\endinput
    \fi
  \fi
\endgroup%
%    \end{macrocode}
%    Package identification:
%    \begin{macrocode}
\begingroup\catcode61\catcode48\catcode32=10\relax%
  \catcode13=5 % ^^M
  \endlinechar=13 %
  \catcode35=6 % #
  \catcode39=12 % '
  \catcode40=12 % (
  \catcode41=12 % )
  \catcode44=12 % ,
  \catcode45=12 % -
  \catcode46=12 % .
  \catcode47=12 % /
  \catcode58=12 % :
  \catcode64=11 % @
  \catcode91=12 % [
  \catcode93=12 % ]
  \catcode123=1 % {
  \catcode125=2 % }
  \expandafter\ifx\csname ProvidesPackage\endcsname\relax
    \def\x#1#2#3[#4]{\endgroup
      \immediate\write-1{Package: #3 #4}%
      \xdef#1{#4}%
    }%
  \else
    \def\x#1#2[#3]{\endgroup
      #2[{#3}]%
      \ifx#1\@undefined
        \xdef#1{#3}%
      \fi
      \ifx#1\relax
        \xdef#1{#3}%
      \fi
    }%
  \fi
\expandafter\x\csname ver@magicnum.sty\endcsname
\ProvidesPackage{magicnum}%
  [2011/04/10 v1.4 Magic numbers (HO)]%
%    \end{macrocode}
%
% \subsection{Catcodes}
%
%    \begin{macrocode}
\begingroup\catcode61\catcode48\catcode32=10\relax%
  \catcode13=5 % ^^M
  \endlinechar=13 %
  \catcode123=1 % {
  \catcode125=2 % }
  \catcode64=11 % @
  \def\x{\endgroup
    \expandafter\edef\csname magicnum@AtEnd\endcsname{%
      \endlinechar=\the\endlinechar\relax
      \catcode13=\the\catcode13\relax
      \catcode32=\the\catcode32\relax
      \catcode35=\the\catcode35\relax
      \catcode61=\the\catcode61\relax
      \catcode64=\the\catcode64\relax
      \catcode123=\the\catcode123\relax
      \catcode125=\the\catcode125\relax
    }%
  }%
\x\catcode61\catcode48\catcode32=10\relax%
\catcode13=5 % ^^M
\endlinechar=13 %
\catcode35=6 % #
\catcode64=11 % @
\catcode123=1 % {
\catcode125=2 % }
\def\TMP@EnsureCode#1#2{%
  \edef\magicnum@AtEnd{%
    \magicnum@AtEnd
    \catcode#1=\the\catcode#1\relax
  }%
  \catcode#1=#2\relax
}
\TMP@EnsureCode{34}{12}% "
\TMP@EnsureCode{39}{12}% '
\TMP@EnsureCode{40}{12}% (
\TMP@EnsureCode{41}{12}% )
\TMP@EnsureCode{42}{12}% *
\TMP@EnsureCode{44}{12}% ,
\TMP@EnsureCode{45}{12}% -
\TMP@EnsureCode{46}{12}% .
\TMP@EnsureCode{47}{12}% /
\TMP@EnsureCode{58}{12}% :
\TMP@EnsureCode{60}{12}% <
\TMP@EnsureCode{62}{12}% >
\TMP@EnsureCode{91}{12}% [
\TMP@EnsureCode{93}{12}% ]
\edef\magicnum@AtEnd{\magicnum@AtEnd\noexpand\endinput}
%    \end{macrocode}
%
% \subsection{Check for previous definition}
%
%    \begin{macrocode}
\begingroup\expandafter\expandafter\expandafter\endgroup
\expandafter\ifx\csname newcommand\endcsname\relax
  \expandafter\ifx\csname magicnum\endcsname\relax
  \else
    \input infwarerr.sty\relax
    \@PackageError{magicnum}{%
      \string\magicnum\space is already defined%
    }\@ehc
  \fi
\else
  \newcommand*{\magicnum}{}%
\fi
%    \end{macrocode}
%
% \subsection{Without \LuaTeX}
%
%    \begin{macrocode}
\begingroup\expandafter\expandafter\expandafter\endgroup
\expandafter\ifx\csname directlua\endcsname\relax
%    \end{macrocode}
%
%    \begin{macro}{\magicnum}
%    \begin{macrocode}
  \begingroup\expandafter\expandafter\expandafter\endgroup
  \expandafter\ifx\csname ifcsname\endcsname\relax
    \def\magicnum#1{%
      \expandafter\ifx\csname MG@#1\endcsname\relax
      \else
        \csname MG@#1\endcsname
      \fi
    }%
  \else
    \begingroup
      \edef\x{\endgroup
        \def\noexpand\magicnum##1{%
          \expandafter\noexpand\csname
          ifcsname\endcsname MG@##1\noexpand\endcsname
            \noexpand\csname MG@##1%
                 \noexpand\expandafter\noexpand\endcsname
          \expandafter\noexpand\csname fi\endcsname
        }%
      }%
    \x
  \fi
%    \end{macrocode}
%    \end{macro}
%
%    \begin{macrocode}
\else
%    \end{macrocode}
%
% \subsection{With \LuaTeX}
%
%    \begin{macrocode}
  \begingroup\expandafter\expandafter\expandafter\endgroup
  \expandafter\ifx\csname RequirePackage\endcsname\relax
    \input ifluatex.sty\relax
    \input infwarerr.sty\relax
  \else
    \RequirePackage{ifluatex}[2010/03/01]%
    \RequirePackage{infwarerr}[2010/04/08]%
  \fi
%    \end{macrocode}
%
%    \begin{macro}{\magicnum@directlua}
%    \begin{macrocode}
  \ifnum\luatexversion<36 %
    \def\magicnum@directlua{\directlua0 }%
  \else
    \let\magicnum@directlua\directlua
  \fi
%    \end{macrocode}
%    \end{macro}
%    \begin{macrocode}
  \magicnum@directlua{%
    require("oberdiek.magicnum")%
  }%
  \begingroup
    \def\x{2011/04/10 v1.4}%
    \def\StripPrefix#1>{}%
    \edef\x{\expandafter\StripPrefix\meaning\x}%
    \edef\y{%
      \magicnum@directlua{%
        if oberdiek.magicnum.getversion then %
          oberdiek.magicnum.getversion()%
        end%
      }%
    }%
    \ifx\x\y
    \else
      \@PackageError{magicnum}{%
        Wrong version of lua module.\MessageBreak
        Package version: \x\MessageBreak
        Lua module: \y
      }\@ehc
    \fi
  \endgroup
%    \end{macrocode}
%    \begin{macro}{\luaescapestring}
%    \begin{macrocode}
  \begingroup
    \expandafter\ifx\csname luaescapestring\endcsname\relax
      \directlua{%
        if tex.enableprimitives then %
          tex.enableprimitives('magicnum@', {'luaescapestring'})%
        end%
      }%
      \global\let\luaescapestring\magicnum@luaescapestring
    \fi
    \expandafter\ifx\csname luaescapestring\endcsname\relax
      \escapechar=92 %
      \@PackageError{magicnum}{%
        Missing \string\luaescapestring
      }\@ehc
    \fi
  \endgroup
%    \end{macrocode}
%    \end{macro}
%    \begin{macro}{\magicnum}
%    \begin{macrocode}
  \def\magicnum#1{%
    \magicnum@directlua{%
      oberdiek.magicnum.get("\luaescapestring{#1}")%
    }%
  }%
%    \end{macrocode}
%    \end{macro}
%
%    \begin{macrocode}
  \expandafter\magicnum@AtEnd
\fi%
%</package>
%    \end{macrocode}
%
% \subsection{Data}
%
% \subsubsection{Plain data}
%
%    \begin{macrocode}
%<*data>
tex.catcode
  escape = 0
  begingroup = 1
  endgroup = 2
  math = 3
  align = 4
  eol = 5
  parameter = 6
  superscript = 7
  subscript = 8
  ignore = 9
  space = 10
  letter = 11
  other = 12
  active = 13
  comment = 14
  invalid = 15
etex.grouptype
  bottomlevel = 0
  simple = 1
  hbox = 2
  adjustedhbox = 3
  vbox = 4
  align = 5
  noalign = 6
  output = 8
  math = 9
  disc = 10
  insert = 11
  vcenter = 12
  mathchoice = 13
  semisimple = 14
  mathshift = 15
  mathleft = 16
etex.iftype
  none = 0
  char = 1
  cat = 2
  num = 3
  dim = 4
  odd = 5
  vmode = 6
  hmode = 7
  mmode = 8
  inner = 9
  void = 10
  hbox = 11
  vbox = 12
  x = 13
  eof = 14
  true = 15
  false = 16
  case = 17
  defined = 18
  csname = 19
  fontchar = 20
etex.nodetype
  none = -1
  char = 0
  hlist = 1
  vlist = 2
  rule = 3
  ins = 4
  mark = 5
  adjust = 6
  ligature = 7
  disc = 8
  whatsit = 9
  math = 10
  glue = 11
  kern = 12
  penalty = 13
  unset = 14
  maths = 15
etex.interactionmode
  batch = 0
  nonstop = 1
  scroll = 2
  errorstop = 3
luatex.pdfliteral.mode
  setorigin = 0
  page = 1
  direct = 2
%</data>
%    \end{macrocode}
%
% \subsubsection{Data for \TeX}
%
%    \begin{macrocode}
%<*package>
%    \end{macrocode}
%    \begin{macro}{\magicnum@add}
%    \begin{macrocode}
\begingroup\expandafter\expandafter\expandafter\endgroup
\expandafter\ifx\csname detokenize\endcsname\relax
  \def\magicnum@add#1#2#3{%
    \expandafter\magicnum@@add
        \csname MG@#1.#2\expandafter\endcsname
        \csname MG@#1.#3\endcsname
       {#3}{#2}%
  }%
  \def\magicnum@@add#1#2#3#4{%
    \def#1{#3}%
    \def#2{#4}%
    \edef#1{%
      \expandafter\strip@prefix\meaning#1%
    }%
    \edef#2{%
      \expandafter\strip@prefix\meaning#2%
    }%
  }%
  \expandafter\ifx\csname strip@prefix\endcsname\relax
    \def\strip@prefix#1->{}%
  \fi
\else
  \def\magicnum@add#1#2#3{%
    \expandafter\edef\csname MG@#1.#2\endcsname{%
      \detokenize{#3}%
    }%
    \expandafter\edef\csname MG@#1.#3\endcsname{%
      \detokenize{#2}%
    }%
  }%
\fi
%    \end{macrocode}
%    \end{macro}
%    \begin{macrocode}
\magicnum@add{tex.catcode}{escape}{0}
\magicnum@add{tex.catcode}{begingroup}{1}
\magicnum@add{tex.catcode}{endgroup}{2}
\magicnum@add{tex.catcode}{math}{3}
\magicnum@add{tex.catcode}{align}{4}
\magicnum@add{tex.catcode}{eol}{5}
\magicnum@add{tex.catcode}{parameter}{6}
\magicnum@add{tex.catcode}{superscript}{7}
\magicnum@add{tex.catcode}{subscript}{8}
\magicnum@add{tex.catcode}{ignore}{9}
\magicnum@add{tex.catcode}{space}{10}
\magicnum@add{tex.catcode}{letter}{11}
\magicnum@add{tex.catcode}{other}{12}
\magicnum@add{tex.catcode}{active}{13}
\magicnum@add{tex.catcode}{comment}{14}
\magicnum@add{tex.catcode}{invalid}{15}
\magicnum@add{etex.grouptype}{bottomlevel}{0}
\magicnum@add{etex.grouptype}{simple}{1}
\magicnum@add{etex.grouptype}{hbox}{2}
\magicnum@add{etex.grouptype}{adjustedhbox}{3}
\magicnum@add{etex.grouptype}{vbox}{4}
\magicnum@add{etex.grouptype}{align}{5}
\magicnum@add{etex.grouptype}{noalign}{6}
\magicnum@add{etex.grouptype}{output}{8}
\magicnum@add{etex.grouptype}{math}{9}
\magicnum@add{etex.grouptype}{disc}{10}
\magicnum@add{etex.grouptype}{insert}{11}
\magicnum@add{etex.grouptype}{vcenter}{12}
\magicnum@add{etex.grouptype}{mathchoice}{13}
\magicnum@add{etex.grouptype}{semisimple}{14}
\magicnum@add{etex.grouptype}{mathshift}{15}
\magicnum@add{etex.grouptype}{mathleft}{16}
\magicnum@add{etex.iftype}{none}{0}
\magicnum@add{etex.iftype}{char}{1}
\magicnum@add{etex.iftype}{cat}{2}
\magicnum@add{etex.iftype}{num}{3}
\magicnum@add{etex.iftype}{dim}{4}
\magicnum@add{etex.iftype}{odd}{5}
\magicnum@add{etex.iftype}{vmode}{6}
\magicnum@add{etex.iftype}{hmode}{7}
\magicnum@add{etex.iftype}{mmode}{8}
\magicnum@add{etex.iftype}{inner}{9}
\magicnum@add{etex.iftype}{void}{10}
\magicnum@add{etex.iftype}{hbox}{11}
\magicnum@add{etex.iftype}{vbox}{12}
\magicnum@add{etex.iftype}{x}{13}
\magicnum@add{etex.iftype}{eof}{14}
\magicnum@add{etex.iftype}{true}{15}
\magicnum@add{etex.iftype}{false}{16}
\magicnum@add{etex.iftype}{case}{17}
\magicnum@add{etex.iftype}{defined}{18}
\magicnum@add{etex.iftype}{csname}{19}
\magicnum@add{etex.iftype}{fontchar}{20}
\magicnum@add{etex.nodetype}{none}{-1}
\magicnum@add{etex.nodetype}{char}{0}
\magicnum@add{etex.nodetype}{hlist}{1}
\magicnum@add{etex.nodetype}{vlist}{2}
\magicnum@add{etex.nodetype}{rule}{3}
\magicnum@add{etex.nodetype}{ins}{4}
\magicnum@add{etex.nodetype}{mark}{5}
\magicnum@add{etex.nodetype}{adjust}{6}
\magicnum@add{etex.nodetype}{ligature}{7}
\magicnum@add{etex.nodetype}{disc}{8}
\magicnum@add{etex.nodetype}{whatsit}{9}
\magicnum@add{etex.nodetype}{math}{10}
\magicnum@add{etex.nodetype}{glue}{11}
\magicnum@add{etex.nodetype}{kern}{12}
\magicnum@add{etex.nodetype}{penalty}{13}
\magicnum@add{etex.nodetype}{unset}{14}
\magicnum@add{etex.nodetype}{maths}{15}
\magicnum@add{etex.interactionmode}{batch}{0}
\magicnum@add{etex.interactionmode}{nonstop}{1}
\magicnum@add{etex.interactionmode}{scroll}{2}
\magicnum@add{etex.interactionmode}{errorstop}{3}
\magicnum@add{luatex.pdfliteral.mode}{setorigin}{0}
\magicnum@add{luatex.pdfliteral.mode}{page}{1}
\magicnum@add{luatex.pdfliteral.mode}{direct}{2}
%    \end{macrocode}
%    \begin{macrocode}
\magicnum@AtEnd%
%</package>
%    \end{macrocode}
%
% \subsubsection{Lua module}
%
%    \begin{macrocode}
%<*lua>
%    \end{macrocode}
%    \begin{macrocode}
module("oberdiek.magicnum", package.seeall)
%    \end{macrocode}
%    \begin{macrocode}
function getversion()
  tex.write("2011/04/10 v1.4")
end
%    \end{macrocode}
%    \begin{macrocode}
local data = {
  ["tex.catcode"] = {
    [0] = "escape",
    [1] = "begingroup",
    [2] = "endgroup",
    [3] = "math",
    [4] = "align",
    [5] = "eol",
    [6] = "parameter",
    [7] = "superscript",
    [8] = "subscript",
    [9] = "ignore",
    [10] = "space",
    [11] = "letter",
    [12] = "other",
    [13] = "active",
    [14] = "comment",
    [15] = "invalid",
    ["active"] = 13,
    ["align"] = 4,
    ["begingroup"] = 1,
    ["comment"] = 14,
    ["endgroup"] = 2,
    ["eol"] = 5,
    ["escape"] = 0,
    ["ignore"] = 9,
    ["invalid"] = 15,
    ["letter"] = 11,
    ["math"] = 3,
    ["other"] = 12,
    ["parameter"] = 6,
    ["space"] = 10,
    ["subscript"] = 8,
    ["superscript"] = 7
  },
  ["etex.grouptype"] = {
    [0] = "bottomlevel",
    [1] = "simple",
    [2] = "hbox",
    [3] = "adjustedhbox",
    [4] = "vbox",
    [5] = "align",
    [6] = "noalign",
    [8] = "output",
    [9] = "math",
    [10] = "disc",
    [11] = "insert",
    [12] = "vcenter",
    [13] = "mathchoice",
    [14] = "semisimple",
    [15] = "mathshift",
    [16] = "mathleft",
    ["adjustedhbox"] = 3,
    ["align"] = 5,
    ["bottomlevel"] = 0,
    ["disc"] = 10,
    ["hbox"] = 2,
    ["insert"] = 11,
    ["math"] = 9,
    ["mathchoice"] = 13,
    ["mathleft"] = 16,
    ["mathshift"] = 15,
    ["noalign"] = 6,
    ["output"] = 8,
    ["semisimple"] = 14,
    ["simple"] = 1,
    ["vbox"] = 4,
    ["vcenter"] = 12
  },
  ["etex.iftype"] = {
    [0] = "none",
    [1] = "char",
    [2] = "cat",
    [3] = "num",
    [4] = "dim",
    [5] = "odd",
    [6] = "vmode",
    [7] = "hmode",
    [8] = "mmode",
    [9] = "inner",
    [10] = "void",
    [11] = "hbox",
    [12] = "vbox",
    [13] = "x",
    [14] = "eof",
    [15] = "true",
    [16] = "false",
    [17] = "case",
    [18] = "defined",
    [19] = "csname",
    [20] = "fontchar",
    ["case"] = 17,
    ["cat"] = 2,
    ["char"] = 1,
    ["csname"] = 19,
    ["defined"] = 18,
    ["dim"] = 4,
    ["eof"] = 14,
    ["false"] = 16,
    ["fontchar"] = 20,
    ["hbox"] = 11,
    ["hmode"] = 7,
    ["inner"] = 9,
    ["mmode"] = 8,
    ["none"] = 0,
    ["num"] = 3,
    ["odd"] = 5,
    ["true"] = 15,
    ["vbox"] = 12,
    ["vmode"] = 6,
    ["void"] = 10,
    ["x"] = 13
  },
  ["etex.nodetype"] = {
    [-1] = "none",
    [0] = "char",
    [1] = "hlist",
    [2] = "vlist",
    [3] = "rule",
    [4] = "ins",
    [5] = "mark",
    [6] = "adjust",
    [7] = "ligature",
    [8] = "disc",
    [9] = "whatsit",
    [10] = "math",
    [11] = "glue",
    [12] = "kern",
    [13] = "penalty",
    [14] = "unset",
    [15] = "maths",
    ["adjust"] = 6,
    ["char"] = 0,
    ["disc"] = 8,
    ["glue"] = 11,
    ["hlist"] = 1,
    ["ins"] = 4,
    ["kern"] = 12,
    ["ligature"] = 7,
    ["mark"] = 5,
    ["math"] = 10,
    ["maths"] = 15,
    ["none"] = -1,
    ["penalty"] = 13,
    ["rule"] = 3,
    ["unset"] = 14,
    ["vlist"] = 2,
    ["whatsit"] = 9
  },
  ["etex.interactionmode"] = {
    [0] = "batch",
    [1] = "nonstop",
    [2] = "scroll",
    [3] = "errorstop",
    ["batch"] = 0,
    ["errorstop"] = 3,
    ["nonstop"] = 1,
    ["scroll"] = 2
  },
  ["luatex.pdfliteral.mode"] = {
    [0] = "setorigin",
    [1] = "page",
    [2] = "direct",
    ["direct"] = 2,
    ["page"] = 1,
    ["setorigin"] = 0
  }
}
%    \end{macrocode}
%    \begin{macrocode}
function get(name)
  local startpos, endpos, category, entry =
      string.find(name, "^(%a[%a%d%.]*)%.(-?[%a%d]+)$")
  if not entry then
    return
  end
  local node = data[category]
  if not node then
    return
  end
  local num = tonumber(entry)
  local value
  if num then
    value = node[num]
    if not value then
      return
    end
  else
    value = node[entry]
    if not value then
      return
    end
    value = "" .. value
  end
  tex.write(value)
end
%    \end{macrocode}
%
%    \begin{macrocode}
%</lua>
%    \end{macrocode}
%
% \section{Test}
%
% \subsection{Catcode checks for loading}
%
%    \begin{macrocode}
%<*test1>
%    \end{macrocode}
%    \begin{macrocode}
\catcode`\{=1 %
\catcode`\}=2 %
\catcode`\#=6 %
\catcode`\@=11 %
\expandafter\ifx\csname count@\endcsname\relax
  \countdef\count@=255 %
\fi
\expandafter\ifx\csname @gobble\endcsname\relax
  \long\def\@gobble#1{}%
\fi
\expandafter\ifx\csname @firstofone\endcsname\relax
  \long\def\@firstofone#1{#1}%
\fi
\expandafter\ifx\csname loop\endcsname\relax
  \expandafter\@firstofone
\else
  \expandafter\@gobble
\fi
{%
  \def\loop#1\repeat{%
    \def\body{#1}%
    \iterate
  }%
  \def\iterate{%
    \body
      \let\next\iterate
    \else
      \let\next\relax
    \fi
    \next
  }%
  \let\repeat=\fi
}%
\def\RestoreCatcodes{}
\count@=0 %
\loop
  \edef\RestoreCatcodes{%
    \RestoreCatcodes
    \catcode\the\count@=\the\catcode\count@\relax
  }%
\ifnum\count@<255 %
  \advance\count@ 1 %
\repeat

\def\RangeCatcodeInvalid#1#2{%
  \count@=#1\relax
  \loop
    \catcode\count@=15 %
  \ifnum\count@<#2\relax
    \advance\count@ 1 %
  \repeat
}
\def\RangeCatcodeCheck#1#2#3{%
  \count@=#1\relax
  \loop
    \ifnum#3=\catcode\count@
    \else
      \errmessage{%
        Character \the\count@\space
        with wrong catcode \the\catcode\count@\space
        instead of \number#3%
      }%
    \fi
  \ifnum\count@<#2\relax
    \advance\count@ 1 %
  \repeat
}
\def\space{ }
\expandafter\ifx\csname LoadCommand\endcsname\relax
  \def\LoadCommand{\input magicnum.sty\relax}%
\fi
\def\Test{%
  \RangeCatcodeInvalid{0}{47}%
  \RangeCatcodeInvalid{58}{64}%
  \RangeCatcodeInvalid{91}{96}%
  \RangeCatcodeInvalid{123}{255}%
  \catcode`\@=12 %
  \catcode`\\=0 %
  \catcode`\%=14 %
  \LoadCommand
  \RangeCatcodeCheck{0}{36}{15}%
  \RangeCatcodeCheck{37}{37}{14}%
  \RangeCatcodeCheck{38}{47}{15}%
  \RangeCatcodeCheck{48}{57}{12}%
  \RangeCatcodeCheck{58}{63}{15}%
  \RangeCatcodeCheck{64}{64}{12}%
  \RangeCatcodeCheck{65}{90}{11}%
  \RangeCatcodeCheck{91}{91}{15}%
  \RangeCatcodeCheck{92}{92}{0}%
  \RangeCatcodeCheck{93}{96}{15}%
  \RangeCatcodeCheck{97}{122}{11}%
  \RangeCatcodeCheck{123}{255}{15}%
  \RestoreCatcodes
}
\Test
\csname @@end\endcsname
\end
%    \end{macrocode}
%    \begin{macrocode}
%</test1>
%    \end{macrocode}
%
% \subsection{Test data}
%
%    \begin{macrocode}
%<*testplain>
\input magicnum.sty\relax
\def\Test#1#2{%
  \edef\result{\magicnum{#1}}%
  \edef\expect{#2}%
  \edef\expect{\expandafter\stripprefix\meaning\expect}%
  \ifx\result\expect
  \else
    \errmessage{%
      Failed: [#1] % hash-ok
      returns [\result] instead of [\expect]%
    }%
  \fi
}
\def\stripprefix#1->{}
%</testplain>
%    \end{macrocode}
%    \begin{macrocode}
%<*testlatex>
\NeedsTeXFormat{LaTeX2e}
\documentclass{minimal}
\usepackage{magicnum}[2011/04/10]
\usepackage{qstest}
\IncludeTests{*}
\LogTests{log}{*}{*}
\newcommand*{\Test}[2]{%
  \Expect*{\magicnum{#1}}{#2}%
}
\begin{qstest}{magicnum}{magicnum}
%</testlatex>
%    \end{macrocode}
%    \begin{macrocode}
%<*testdata>
\Test{tex.catcode.escape}{0}
\Test{tex.catcode.invalid}{15}
\Test{tex.catcode.unknown}{}
\Test{tex.catcode.0}{escape}
\Test{tex.catcode.15}{invalid}
\Test{etex.iftype.true}{15}
\Test{etex.iftype.false}{16}
\Test{etex.iftype.15}{true}
\Test{etex.iftype.16}{false}
\Test{etex.nodetype.none}{-1}
\Test{etex.nodetype.-1}{none}
\Test{luatex.pdfliteral.mode.direct}{2}
\Test{luatex.pdfliteral.mode.1}{page}
\Test{}{}
\Test{unknown}{}
\Test{unknown.foo.bar}{}
\Test{unknown.foo.4}{}
%</testdata>
%    \end{macrocode}
%    \begin{macrocode}
%<*testplain>
\csname @@end\endcsname
\end
%</testplain>
%<*testlatex>
\end{qstest}
\csname @@end\endcsname
%</testlatex>
%    \end{macrocode}
%
% \subsection{Small test for \hologo{iniTeX}}
%
%    \begin{macrocode}
%<*test4>
\catcode`\{=1
\catcode`\}=2
\catcode`\#=6
\input magicnum.sty\relax
\edef\x{\magicnum{tex.catcode.15}}
\edef\y{invalid}
\def\Strip#1>{}
\edef\y{\expandafter\Strip\meaning\y}
\ifx\x\y
  \immediate\write16{Ok}%
\else
  \errmessage{\x<>\y}%
\fi
\csname @@end\endcsname\end
%</test4>
%    \end{macrocode}
%
% \section{Installation}
%
% \subsection{Download}
%
% \paragraph{Package.} This package is available on
% CTAN\footnote{\url{ftp://ftp.ctan.org/tex-archive/}}:
% \begin{description}
% \item[\CTAN{macros/latex/contrib/oberdiek/magicnum.dtx}] The source file.
% \item[\CTAN{macros/latex/contrib/oberdiek/magicnum.pdf}] Documentation.
% \end{description}
%
%
% \paragraph{Bundle.} All the packages of the bundle `oberdiek'
% are also available in a TDS compliant ZIP archive. There
% the packages are already unpacked and the documentation files
% are generated. The files and directories obey the TDS standard.
% \begin{description}
% \item[\CTAN{install/macros/latex/contrib/oberdiek.tds.zip}]
% \end{description}
% \emph{TDS} refers to the standard ``A Directory Structure
% for \TeX\ Files'' (\CTAN{tds/tds.pdf}). Directories
% with \xfile{texmf} in their name are usually organized this way.
%
% \subsection{Bundle installation}
%
% \paragraph{Unpacking.} Unpack the \xfile{oberdiek.tds.zip} in the
% TDS tree (also known as \xfile{texmf} tree) of your choice.
% Example (linux):
% \begin{quote}
%   |unzip oberdiek.tds.zip -d ~/texmf|
% \end{quote}
%
% \paragraph{Script installation.}
% Check the directory \xfile{TDS:scripts/oberdiek/} for
% scripts that need further installation steps.
% Package \xpackage{attachfile2} comes with the Perl script
% \xfile{pdfatfi.pl} that should be installed in such a way
% that it can be called as \texttt{pdfatfi}.
% Example (linux):
% \begin{quote}
%   |chmod +x scripts/oberdiek/pdfatfi.pl|\\
%   |cp scripts/oberdiek/pdfatfi.pl /usr/local/bin/|
% \end{quote}
%
% \subsection{Package installation}
%
% \paragraph{Unpacking.} The \xfile{.dtx} file is a self-extracting
% \docstrip\ archive. The files are extracted by running the
% \xfile{.dtx} through \plainTeX:
% \begin{quote}
%   \verb|tex magicnum.dtx|
% \end{quote}
%
% \paragraph{TDS.} Now the different files must be moved into
% the different directories in your installation TDS tree
% (also known as \xfile{texmf} tree):
% \begin{quote}
% \def\t{^^A
% \begin{tabular}{@{}>{\ttfamily}l@{ $\rightarrow$ }>{\ttfamily}l@{}}
%   magicnum.sty & tex/generic/oberdiek/magicnum.sty\\
%   magicnum.lua & scripts/oberdiek/magicnum.lua\\
%   oberdiek.magicnum.lua & scripts/oberdiek/oberdiek.magicnum.lua\\
%   magicnum.pdf & doc/latex/oberdiek/magicnum.pdf\\
%   magicnum.txt & doc/latex/oberdiek/magicnum.txt\\
%   test/magicnum-test1.tex & doc/latex/oberdiek/test/magicnum-test1.tex\\
%   test/magicnum-test2.tex & doc/latex/oberdiek/test/magicnum-test2.tex\\
%   test/magicnum-test3.tex & doc/latex/oberdiek/test/magicnum-test3.tex\\
%   test/magicnum-test4.tex & doc/latex/oberdiek/test/magicnum-test4.tex\\
%   magicnum.dtx & source/latex/oberdiek/magicnum.dtx\\
% \end{tabular}^^A
% }^^A
% \sbox0{\t}^^A
% \ifdim\wd0>\linewidth
%   \begingroup
%     \advance\linewidth by\leftmargin
%     \advance\linewidth by\rightmargin
%   \edef\x{\endgroup
%     \def\noexpand\lw{\the\linewidth}^^A
%   }\x
%   \def\lwbox{^^A
%     \leavevmode
%     \hbox to \linewidth{^^A
%       \kern-\leftmargin\relax
%       \hss
%       \usebox0
%       \hss
%       \kern-\rightmargin\relax
%     }^^A
%   }^^A
%   \ifdim\wd0>\lw
%     \sbox0{\small\t}^^A
%     \ifdim\wd0>\linewidth
%       \ifdim\wd0>\lw
%         \sbox0{\footnotesize\t}^^A
%         \ifdim\wd0>\linewidth
%           \ifdim\wd0>\lw
%             \sbox0{\scriptsize\t}^^A
%             \ifdim\wd0>\linewidth
%               \ifdim\wd0>\lw
%                 \sbox0{\tiny\t}^^A
%                 \ifdim\wd0>\linewidth
%                   \lwbox
%                 \else
%                   \usebox0
%                 \fi
%               \else
%                 \lwbox
%               \fi
%             \else
%               \usebox0
%             \fi
%           \else
%             \lwbox
%           \fi
%         \else
%           \usebox0
%         \fi
%       \else
%         \lwbox
%       \fi
%     \else
%       \usebox0
%     \fi
%   \else
%     \lwbox
%   \fi
% \else
%   \usebox0
% \fi
% \end{quote}
% If you have a \xfile{docstrip.cfg} that configures and enables \docstrip's
% TDS installing feature, then some files can already be in the right
% place, see the documentation of \docstrip.
%
% \subsection{Refresh file name databases}
%
% If your \TeX~distribution
% (\teTeX, \mikTeX, \dots) relies on file name databases, you must refresh
% these. For example, \teTeX\ users run \verb|texhash| or
% \verb|mktexlsr|.
%
% \subsection{Some details for the interested}
%
% \paragraph{Attached source.}
%
% The PDF documentation on CTAN also includes the
% \xfile{.dtx} source file. It can be extracted by
% AcrobatReader 6 or higher. Another option is \textsf{pdftk},
% e.g. unpack the file into the current directory:
% \begin{quote}
%   \verb|pdftk magicnum.pdf unpack_files output .|
% \end{quote}
%
% \paragraph{Unpacking with \LaTeX.}
% The \xfile{.dtx} chooses its action depending on the format:
% \begin{description}
% \item[\plainTeX:] Run \docstrip\ and extract the files.
% \item[\LaTeX:] Generate the documentation.
% \end{description}
% If you insist on using \LaTeX\ for \docstrip\ (really,
% \docstrip\ does not need \LaTeX), then inform the autodetect routine
% about your intention:
% \begin{quote}
%   \verb|latex \let\install=y\input{magicnum.dtx}|
% \end{quote}
% Do not forget to quote the argument according to the demands
% of your shell.
%
% \paragraph{Generating the documentation.}
% You can use both the \xfile{.dtx} or the \xfile{.drv} to generate
% the documentation. The process can be configured by the
% configuration file \xfile{ltxdoc.cfg}. For instance, put this
% line into this file, if you want to have A4 as paper format:
% \begin{quote}
%   \verb|\PassOptionsToClass{a4paper}{article}|
% \end{quote}
% An example follows how to generate the
% documentation with pdf\LaTeX:
% \begin{quote}
%\begin{verbatim}
%pdflatex magicnum.dtx
%makeindex -s gind.ist magicnum.idx
%pdflatex magicnum.dtx
%makeindex -s gind.ist magicnum.idx
%pdflatex magicnum.dtx
%\end{verbatim}
% \end{quote}
%
% \section{Catalogue}
%
% The following XML file can be used as source for the
% \href{http://mirror.ctan.org/help/Catalogue/catalogue.html}{\TeX\ Catalogue}.
% The elements \texttt{caption} and \texttt{description} are imported
% from the original XML file from the Catalogue.
% The name of the XML file in the Catalogue is \xfile{magicnum.xml}.
%    \begin{macrocode}
%<*catalogue>
<?xml version='1.0' encoding='us-ascii'?>
<!DOCTYPE entry SYSTEM 'catalogue.dtd'>
<entry datestamp='$Date$' modifier='$Author$' id='magicnum'>
  <name>magicnum</name>
  <caption>Access TeX systems' "magic numbers".</caption>
  <authorref id='auth:oberdiek'/>
  <copyright owner='Heiko Oberdiek' year='2007,2009-2011'/>
  <license type='lppl1.3'/>
  <version number='1.4'/>
  <description>
    This package allows access to the various parameter values in
    TeX (catcode values), e-TeX (group, if and node types, and
    interaction mode), and LuaTeX (pdfliteral mode) by a hierarchical
    name system.
    <p/>
    The package is part of the <xref refid='oberdiek'>oberdiek</xref> bundle.
  </description>
  <documentation details='Package documentation'
      href='ctan:/macros/latex/contrib/oberdiek/magicnum.pdf'/>
  <ctan file='true' path='/macros/latex/contrib/oberdiek/magicnum.dtx'/>
  <miktex location='oberdiek'/>
  <texlive location='oberdiek'/>
  <install path='/macros/latex/contrib/oberdiek/oberdiek.tds.zip'/>
</entry>
%</catalogue>
%    \end{macrocode}
%
% \begin{History}
%   \begin{Version}{2007/12/12 v1.0}
%   \item
%     First public version.
%   \end{Version}
%   \begin{Version}{2009/04/10 v1.1}
%   \item
%     Adaptation to \LuaTeX\ 0.40.
%   \end{Version}
%   \begin{Version}{2010/03/09 v1.2}
%   \item
%     Adaptation to package \xpackage{luatex} 0.4.
%   \end{Version}
%   \begin{Version}{2011/03/24 v1.3}
%   \item
%     Catcode fixes.
%   \end{Version}
%   \begin{Version}{2011/04/10 v1.4}
%   \item
%     Compatibility for \hologo{iniTeX}.
%   \item
%     Dependency from package \xpackage{luatex} removed.
%   \item
%     Version check for lua module.
%   \end{Version}
% \end{History}
%
% \PrintIndex
%
% \Finale
\endinput

%        (quote the arguments according to the demands of your shell)
%
% Documentation:
%    (a) If magicnum.drv is present:
%           latex magicnum.drv
%    (b) Without magicnum.drv:
%           latex magicnum.dtx; ...
%    The class ltxdoc loads the configuration file ltxdoc.cfg
%    if available. Here you can specify further options, e.g.
%    use A4 as paper format:
%       \PassOptionsToClass{a4paper}{article}
%
%    Programm calls to get the documentation (example):
%       pdflatex magicnum.dtx
%       makeindex -s gind.ist magicnum.idx
%       pdflatex magicnum.dtx
%       makeindex -s gind.ist magicnum.idx
%       pdflatex magicnum.dtx
%
% Installation:
%    TDS:tex/generic/oberdiek/magicnum.sty
%    TDS:scripts/oberdiek/magicnum.lua
%    TDS:scripts/oberdiek/oberdiek.magicnum.lua
%    TDS:doc/latex/oberdiek/magicnum.pdf
%    TDS:doc/latex/oberdiek/magicnum.txt
%    TDS:doc/latex/oberdiek/test/magicnum-test1.tex
%    TDS:doc/latex/oberdiek/test/magicnum-test2.tex
%    TDS:doc/latex/oberdiek/test/magicnum-test3.tex
%    TDS:doc/latex/oberdiek/test/magicnum-test4.tex
%    TDS:source/latex/oberdiek/magicnum.dtx
%
%<*ignore>
\begingroup
  \catcode123=1 %
  \catcode125=2 %
  \def\x{LaTeX2e}%
\expandafter\endgroup
\ifcase 0\ifx\install y1\fi\expandafter
         \ifx\csname processbatchFile\endcsname\relax\else1\fi
         \ifx\fmtname\x\else 1\fi\relax
\else\csname fi\endcsname
%</ignore>
%<*install>
\input docstrip.tex
\Msg{************************************************************************}
\Msg{* Installation}
\Msg{* Package: magicnum 2011/04/10 v1.4 Magic numbers (HO)}
\Msg{************************************************************************}

\keepsilent
\askforoverwritefalse

\let\MetaPrefix\relax
\preamble

This is a generated file.

Project: magicnum
Version: 2011/04/10 v1.4

Copyright (C) 2007, 2009-2011 by
   Heiko Oberdiek <heiko.oberdiek at googlemail.com>

This work may be distributed and/or modified under the
conditions of the LaTeX Project Public License, either
version 1.3c of this license or (at your option) any later
version. This version of this license is in
   http://www.latex-project.org/lppl/lppl-1-3c.txt
and the latest version of this license is in
   http://www.latex-project.org/lppl.txt
and version 1.3 or later is part of all distributions of
LaTeX version 2005/12/01 or later.

This work has the LPPL maintenance status "maintained".

This Current Maintainer of this work is Heiko Oberdiek.

The Base Interpreter refers to any `TeX-Format',
because some files are installed in TDS:tex/generic//.

This work consists of the main source file magicnum.dtx
and the derived files
   magicnum.sty, magicnum.pdf, magicnum.ins, magicnum.drv, magicnum.txt,
   magicnum-test1.tex, magicnum-test2.tex, magicnum-test3.tex,
   magicnum-test4.tex, magicnum.lua, oberdiek.magicnum.lua.

\endpreamble
\let\MetaPrefix\DoubleperCent

\generate{%
  \file{magicnum.ins}{\from{magicnum.dtx}{install}}%
  \file{magicnum.drv}{\from{magicnum.dtx}{driver}}%
  \usedir{tex/generic/oberdiek}%
  \file{magicnum.sty}{\from{magicnum.dtx}{package}}%
  \usedir{doc/latex/oberdiek/test}%
  \file{magicnum-test1.tex}{\from{magicnum.dtx}{test1}}%
  \file{magicnum-test2.tex}{\from{magicnum.dtx}{testplain,testdata}}%
  \file{magicnum-test3.tex}{\from{magicnum.dtx}{testlatex,testdata}}%
  \file{magicnum-test4.tex}{\from{magicnum.dtx}{test4}}%
  \nopreamble
  \nopostamble
  \usedir{doc/latex/oberdiek}%
  \file{magicnum.txt}{\from{magicnum.dtx}{data}}%
  \usedir{source/latex/oberdiek/catalogue}%
  \file{magicnum.xml}{\from{magicnum.dtx}{catalogue}}%
}
\def\MetaPrefix{-- }
\def\defaultpostamble{%
  \MetaPrefix^^J%
  \MetaPrefix\space End of File `\outFileName'.%
}
\def\currentpostamble{\defaultpostamble}%
\generate{%
  \usedir{scripts/oberdiek}%
  \file{magicnum.lua}{\from{magicnum.dtx}{lua}}%
  \file{oberdiek.magicnum.lua}{\from{magicnum.dtx}{lua}}%
}

\catcode32=13\relax% active space
\let =\space%
\Msg{************************************************************************}
\Msg{*}
\Msg{* To finish the installation you have to move the following}
\Msg{* file into a directory searched by TeX:}
\Msg{*}
\Msg{*     magicnum.sty}
\Msg{*}
\Msg{* And install the following script files:}
\Msg{*}
\Msg{*     magicnum.lua, oberdiek.magicnum.lua}
\Msg{*}
\Msg{* To produce the documentation run the file `magicnum.drv'}
\Msg{* through LaTeX.}
\Msg{*}
\Msg{* Happy TeXing!}
\Msg{*}
\Msg{************************************************************************}

\endbatchfile
%</install>
%<*ignore>
\fi
%</ignore>
%<*driver>
\NeedsTeXFormat{LaTeX2e}
\ProvidesFile{magicnum.drv}%
  [2011/04/10 v1.4 Magic numbers (HO)]%
\documentclass{ltxdoc}
\usepackage{holtxdoc}[2011/11/22]
\usepackage{array}
\begin{document}
  \DocInput{magicnum.dtx}%
\end{document}
%</driver>
% \fi
%
% \CheckSum{755}
%
% \CharacterTable
%  {Upper-case    \A\B\C\D\E\F\G\H\I\J\K\L\M\N\O\P\Q\R\S\T\U\V\W\X\Y\Z
%   Lower-case    \a\b\c\d\e\f\g\h\i\j\k\l\m\n\o\p\q\r\s\t\u\v\w\x\y\z
%   Digits        \0\1\2\3\4\5\6\7\8\9
%   Exclamation   \!     Double quote  \"     Hash (number) \#
%   Dollar        \$     Percent       \%     Ampersand     \&
%   Acute accent  \'     Left paren    \(     Right paren   \)
%   Asterisk      \*     Plus          \+     Comma         \,
%   Minus         \-     Point         \.     Solidus       \/
%   Colon         \:     Semicolon     \;     Less than     \<
%   Equals        \=     Greater than  \>     Question mark \?
%   Commercial at \@     Left bracket  \[     Backslash     \\
%   Right bracket \]     Circumflex    \^     Underscore    \_
%   Grave accent  \`     Left brace    \{     Vertical bar  \|
%   Right brace   \}     Tilde         \~}
%
% \GetFileInfo{magicnum.drv}
%
% \title{The \xpackage{magicnum} package}
% \date{2011/04/10 v1.4}
% \author{Heiko Oberdiek\\\xemail{heiko.oberdiek at googlemail.com}}
%
% \maketitle
%
% \begin{abstract}
% This packages allows to access magic numbers by a hierarchical
% name system.
% \end{abstract}
%
% \tableofcontents
%
% \hypersetup{bookmarksopenlevel=2}
% \section{Documentation}
%
% \subsection{Introduction}
%
% Especially since \eTeX\ there are many integer values
% with special meanings, such as catcodes, group types, \dots
% Package \xpackage{etex}, enabled by options, defines
% macros in the user namespace for these values.
%
% This package goes another approach for storing the names and values.
% \begin{itemize}
% \item If \LuaTeX\ is available, they
% are stored in Lua tables.
% \item Without \LuaTeX\ they are remembered using internal
% macros.
% \end{itemize}
%
% \subsection{User interface}
%
% The integer values and names are organized in a hierarchical
% scheme of categories with the property names as leaves.
% Example: \eTeX's \cs{currentgrouplevel} reports |2| for a
% group caused by \cs{hbox}. This package has choosen to organize
% the group types in a main category |etex| and its subcategory
% |grouptype|:
% \begin{quote}
%   |etex.grouptype.hbox| = |2|
% \end{quote}
% The property name |hbox| in category |etex.grouptype| has value |2|.
% Dots are used to separate components.
%
% If you want to have the value, the access key is constructed by
% the category with all its components and the property name.
% For the opposite the value is used instead of the property name.
%
% Values are always integers (including negative numbers).
%
% \subsubsection{\cs{magicnum}}
%
% \begin{declcs}{magicnum} \M{access key}
% \end{declcs}
% Macro \cs{magicnum} expects an access key as argument and
% expands to the requested data. The macro is always expandable.
% In case of errors the expansion result is empty.
%
% The same macro is also used for getting a property name.
% In this case the property name part in the access key is
% replaced by the value.
%
% The catcodes
% of the resulting numbers and strings follow \TeX's tradition of
% \cs{string}, \cs{meaning}, \dots: The space has catcode 10
% (|tex.catcode.space|) and the other characters have catcode
% 12 (|tex.catcode.other|).
%
% Examples:
% \begin{quote}
%   |\magicnum{etex.grouptype.hbox}| $\Rightarrow$ |2|\\
%   |\magicnum{tex.catcode.14}| $\Rightarrow$ |comment|\\
%   |\magicnum{tex.catcode.undefined}| $\Rightarrow$ $\emptyset$
% \end{quote}
%
% \subsubsection{Properties}
%
% \begin{itemize}
% \item The components of a category are either subcategories or
%       key value pairs, but not both.
% \item The full specified property names are unique and thus
%       has one integer value exactly.
% \item Also the values inside a category are unique.
%       This condition is a prerequisite for the reverse mapping
%       of \cs{magicnum}.
% \item All names start with a letter. Only letters or digits
%       may follow.
% \end{itemize}
%
% \subsection{Data}
%
%  \subsubsection{\texorpdfstring{Category }{}\texttt{tex.catcode}}
%
% \begin{quote}
% \begin{tabular}{@{}>{\ttfamily}l>{\ttfamily}l@{}}
%    tex.catcode.escape & 0\\
%    tex.catcode.begingroup & 1\\
%    tex.catcode.endgroup & 2\\
%    tex.catcode.math & 3\\
%    tex.catcode.align & 4\\
%    tex.catcode.eol & 5\\
%    tex.catcode.parameter & 6\\
%    tex.catcode.superscript & 7\\
%    tex.catcode.subscript & 8\\
%    tex.catcode.ignore & 9\\
%    tex.catcode.space & 10\\
%    tex.catcode.letter & 11\\
%    tex.catcode.other & 12\\
%    tex.catcode.active & 13\\
%    tex.catcode.comment & 14\\
%    tex.catcode.invalid & 15\\
%  \end{tabular}
%  \end{quote}
%
%  \subsubsection{\texorpdfstring{Category }{}\texttt{etex.grouptype}}
%
% \begin{quote}
% \begin{tabular}{@{}>{\ttfamily}l>{\ttfamily}l@{}}
%    etex.grouptype.bottomlevel & 0\\
%    etex.grouptype.simple & 1\\
%    etex.grouptype.hbox & 2\\
%    etex.grouptype.adjustedhbox & 3\\
%    etex.grouptype.vbox & 4\\
%    etex.grouptype.align & 5\\
%    etex.grouptype.noalign & 6\\
%    etex.grouptype.output & 8\\
%    etex.grouptype.math & 9\\
%    etex.grouptype.disc & 10\\
%    etex.grouptype.insert & 11\\
%    etex.grouptype.vcenter & 12\\
%    etex.grouptype.mathchoice & 13\\
%    etex.grouptype.semisimple & 14\\
%    etex.grouptype.mathshift & 15\\
%    etex.grouptype.mathleft & 16\\
%  \end{tabular}
%  \end{quote}
%
%  \subsubsection{\texorpdfstring{Category }{}\texttt{etex.iftype}}
%
% \begin{quote}
% \begin{tabular}{@{}>{\ttfamily}l>{\ttfamily}l@{}}
%    etex.iftype.none & 0\\
%    etex.iftype.char & 1\\
%    etex.iftype.cat & 2\\
%    etex.iftype.num & 3\\
%    etex.iftype.dim & 4\\
%    etex.iftype.odd & 5\\
%    etex.iftype.vmode & 6\\
%    etex.iftype.hmode & 7\\
%    etex.iftype.mmode & 8\\
%    etex.iftype.inner & 9\\
%    etex.iftype.void & 10\\
%    etex.iftype.hbox & 11\\
%    etex.iftype.vbox & 12\\
%    etex.iftype.x & 13\\
%    etex.iftype.eof & 14\\
%    etex.iftype.true & 15\\
%    etex.iftype.false & 16\\
%    etex.iftype.case & 17\\
%    etex.iftype.defined & 18\\
%    etex.iftype.csname & 19\\
%    etex.iftype.fontchar & 20\\
%  \end{tabular}
%  \end{quote}
%
%  \subsubsection{\texorpdfstring{Category }{}\texttt{etex.nodetype}}
%
% \begin{quote}
% \begin{tabular}{@{}>{\ttfamily}l>{\ttfamily}l@{}}
%    etex.nodetype.none & -1\\
%    etex.nodetype.char & 0\\
%    etex.nodetype.hlist & 1\\
%    etex.nodetype.vlist & 2\\
%    etex.nodetype.rule & 3\\
%    etex.nodetype.ins & 4\\
%    etex.nodetype.mark & 5\\
%    etex.nodetype.adjust & 6\\
%    etex.nodetype.ligature & 7\\
%    etex.nodetype.disc & 8\\
%    etex.nodetype.whatsit & 9\\
%    etex.nodetype.math & 10\\
%    etex.nodetype.glue & 11\\
%    etex.nodetype.kern & 12\\
%    etex.nodetype.penalty & 13\\
%    etex.nodetype.unset & 14\\
%    etex.nodetype.maths & 15\\
%  \end{tabular}
%  \end{quote}
%
%  \subsubsection{\texorpdfstring{Category }{}\texttt{etex.interactionmode}}
%
% \begin{quote}
% \begin{tabular}{@{}>{\ttfamily}l>{\ttfamily}l@{}}
%    etex.interactionmode.batch & 0\\
%    etex.interactionmode.nonstop & 1\\
%    etex.interactionmode.scroll & 2\\
%    etex.interactionmode.errorstop & 3\\
%  \end{tabular}
%  \end{quote}
%
%  \subsubsection{\texorpdfstring{Category }{}\texttt{luatex.pdfliteral.mode}}
%
% \begin{quote}
% \begin{tabular}{@{}>{\ttfamily}l>{\ttfamily}l@{}}
%    luatex.pdfliteral.mode.setorigin & 0\\
%    luatex.pdfliteral.mode.page & 1\\
%    luatex.pdfliteral.mode.direct & 2\\
%  \end{tabular}
%  \end{quote}
%
%
% \hypersetup{bookmarksopenlevel=1}
%
% \StopEventually{
% }
%
% \section{Implementation}
%
%    \begin{macrocode}
%<*package>
%    \end{macrocode}
%
% \subsection{Reload check and package identification}
%    Reload check, especially if the package is not used with \LaTeX.
%    \begin{macrocode}
\begingroup\catcode61\catcode48\catcode32=10\relax%
  \catcode13=5 % ^^M
  \endlinechar=13 %
  \catcode35=6 % #
  \catcode39=12 % '
  \catcode44=12 % ,
  \catcode45=12 % -
  \catcode46=12 % .
  \catcode58=12 % :
  \catcode64=11 % @
  \catcode123=1 % {
  \catcode125=2 % }
  \expandafter\let\expandafter\x\csname ver@magicnum.sty\endcsname
  \ifx\x\relax % plain-TeX, first loading
  \else
    \def\empty{}%
    \ifx\x\empty % LaTeX, first loading,
      % variable is initialized, but \ProvidesPackage not yet seen
    \else
      \expandafter\ifx\csname PackageInfo\endcsname\relax
        \def\x#1#2{%
          \immediate\write-1{Package #1 Info: #2.}%
        }%
      \else
        \def\x#1#2{\PackageInfo{#1}{#2, stopped}}%
      \fi
      \x{magicnum}{The package is already loaded}%
      \aftergroup\endinput
    \fi
  \fi
\endgroup%
%    \end{macrocode}
%    Package identification:
%    \begin{macrocode}
\begingroup\catcode61\catcode48\catcode32=10\relax%
  \catcode13=5 % ^^M
  \endlinechar=13 %
  \catcode35=6 % #
  \catcode39=12 % '
  \catcode40=12 % (
  \catcode41=12 % )
  \catcode44=12 % ,
  \catcode45=12 % -
  \catcode46=12 % .
  \catcode47=12 % /
  \catcode58=12 % :
  \catcode64=11 % @
  \catcode91=12 % [
  \catcode93=12 % ]
  \catcode123=1 % {
  \catcode125=2 % }
  \expandafter\ifx\csname ProvidesPackage\endcsname\relax
    \def\x#1#2#3[#4]{\endgroup
      \immediate\write-1{Package: #3 #4}%
      \xdef#1{#4}%
    }%
  \else
    \def\x#1#2[#3]{\endgroup
      #2[{#3}]%
      \ifx#1\@undefined
        \xdef#1{#3}%
      \fi
      \ifx#1\relax
        \xdef#1{#3}%
      \fi
    }%
  \fi
\expandafter\x\csname ver@magicnum.sty\endcsname
\ProvidesPackage{magicnum}%
  [2011/04/10 v1.4 Magic numbers (HO)]%
%    \end{macrocode}
%
% \subsection{Catcodes}
%
%    \begin{macrocode}
\begingroup\catcode61\catcode48\catcode32=10\relax%
  \catcode13=5 % ^^M
  \endlinechar=13 %
  \catcode123=1 % {
  \catcode125=2 % }
  \catcode64=11 % @
  \def\x{\endgroup
    \expandafter\edef\csname magicnum@AtEnd\endcsname{%
      \endlinechar=\the\endlinechar\relax
      \catcode13=\the\catcode13\relax
      \catcode32=\the\catcode32\relax
      \catcode35=\the\catcode35\relax
      \catcode61=\the\catcode61\relax
      \catcode64=\the\catcode64\relax
      \catcode123=\the\catcode123\relax
      \catcode125=\the\catcode125\relax
    }%
  }%
\x\catcode61\catcode48\catcode32=10\relax%
\catcode13=5 % ^^M
\endlinechar=13 %
\catcode35=6 % #
\catcode64=11 % @
\catcode123=1 % {
\catcode125=2 % }
\def\TMP@EnsureCode#1#2{%
  \edef\magicnum@AtEnd{%
    \magicnum@AtEnd
    \catcode#1=\the\catcode#1\relax
  }%
  \catcode#1=#2\relax
}
\TMP@EnsureCode{34}{12}% "
\TMP@EnsureCode{39}{12}% '
\TMP@EnsureCode{40}{12}% (
\TMP@EnsureCode{41}{12}% )
\TMP@EnsureCode{42}{12}% *
\TMP@EnsureCode{44}{12}% ,
\TMP@EnsureCode{45}{12}% -
\TMP@EnsureCode{46}{12}% .
\TMP@EnsureCode{47}{12}% /
\TMP@EnsureCode{58}{12}% :
\TMP@EnsureCode{60}{12}% <
\TMP@EnsureCode{62}{12}% >
\TMP@EnsureCode{91}{12}% [
\TMP@EnsureCode{93}{12}% ]
\edef\magicnum@AtEnd{\magicnum@AtEnd\noexpand\endinput}
%    \end{macrocode}
%
% \subsection{Check for previous definition}
%
%    \begin{macrocode}
\begingroup\expandafter\expandafter\expandafter\endgroup
\expandafter\ifx\csname newcommand\endcsname\relax
  \expandafter\ifx\csname magicnum\endcsname\relax
  \else
    \input infwarerr.sty\relax
    \@PackageError{magicnum}{%
      \string\magicnum\space is already defined%
    }\@ehc
  \fi
\else
  \newcommand*{\magicnum}{}%
\fi
%    \end{macrocode}
%
% \subsection{Without \LuaTeX}
%
%    \begin{macrocode}
\begingroup\expandafter\expandafter\expandafter\endgroup
\expandafter\ifx\csname directlua\endcsname\relax
%    \end{macrocode}
%
%    \begin{macro}{\magicnum}
%    \begin{macrocode}
  \begingroup\expandafter\expandafter\expandafter\endgroup
  \expandafter\ifx\csname ifcsname\endcsname\relax
    \def\magicnum#1{%
      \expandafter\ifx\csname MG@#1\endcsname\relax
      \else
        \csname MG@#1\endcsname
      \fi
    }%
  \else
    \begingroup
      \edef\x{\endgroup
        \def\noexpand\magicnum##1{%
          \expandafter\noexpand\csname
          ifcsname\endcsname MG@##1\noexpand\endcsname
            \noexpand\csname MG@##1%
                 \noexpand\expandafter\noexpand\endcsname
          \expandafter\noexpand\csname fi\endcsname
        }%
      }%
    \x
  \fi
%    \end{macrocode}
%    \end{macro}
%
%    \begin{macrocode}
\else
%    \end{macrocode}
%
% \subsection{With \LuaTeX}
%
%    \begin{macrocode}
  \begingroup\expandafter\expandafter\expandafter\endgroup
  \expandafter\ifx\csname RequirePackage\endcsname\relax
    \input ifluatex.sty\relax
    \input infwarerr.sty\relax
  \else
    \RequirePackage{ifluatex}[2010/03/01]%
    \RequirePackage{infwarerr}[2010/04/08]%
  \fi
%    \end{macrocode}
%
%    \begin{macro}{\magicnum@directlua}
%    \begin{macrocode}
  \ifnum\luatexversion<36 %
    \def\magicnum@directlua{\directlua0 }%
  \else
    \let\magicnum@directlua\directlua
  \fi
%    \end{macrocode}
%    \end{macro}
%    \begin{macrocode}
  \magicnum@directlua{%
    require("oberdiek.magicnum")%
  }%
  \begingroup
    \def\x{2011/04/10 v1.4}%
    \def\StripPrefix#1>{}%
    \edef\x{\expandafter\StripPrefix\meaning\x}%
    \edef\y{%
      \magicnum@directlua{%
        if oberdiek.magicnum.getversion then %
          oberdiek.magicnum.getversion()%
        end%
      }%
    }%
    \ifx\x\y
    \else
      \@PackageError{magicnum}{%
        Wrong version of lua module.\MessageBreak
        Package version: \x\MessageBreak
        Lua module: \y
      }\@ehc
    \fi
  \endgroup
%    \end{macrocode}
%    \begin{macro}{\luaescapestring}
%    \begin{macrocode}
  \begingroup
    \expandafter\ifx\csname luaescapestring\endcsname\relax
      \directlua{%
        if tex.enableprimitives then %
          tex.enableprimitives('magicnum@', {'luaescapestring'})%
        end%
      }%
      \global\let\luaescapestring\magicnum@luaescapestring
    \fi
    \expandafter\ifx\csname luaescapestring\endcsname\relax
      \escapechar=92 %
      \@PackageError{magicnum}{%
        Missing \string\luaescapestring
      }\@ehc
    \fi
  \endgroup
%    \end{macrocode}
%    \end{macro}
%    \begin{macro}{\magicnum}
%    \begin{macrocode}
  \def\magicnum#1{%
    \magicnum@directlua{%
      oberdiek.magicnum.get("\luaescapestring{#1}")%
    }%
  }%
%    \end{macrocode}
%    \end{macro}
%
%    \begin{macrocode}
  \expandafter\magicnum@AtEnd
\fi%
%</package>
%    \end{macrocode}
%
% \subsection{Data}
%
% \subsubsection{Plain data}
%
%    \begin{macrocode}
%<*data>
tex.catcode
  escape = 0
  begingroup = 1
  endgroup = 2
  math = 3
  align = 4
  eol = 5
  parameter = 6
  superscript = 7
  subscript = 8
  ignore = 9
  space = 10
  letter = 11
  other = 12
  active = 13
  comment = 14
  invalid = 15
etex.grouptype
  bottomlevel = 0
  simple = 1
  hbox = 2
  adjustedhbox = 3
  vbox = 4
  align = 5
  noalign = 6
  output = 8
  math = 9
  disc = 10
  insert = 11
  vcenter = 12
  mathchoice = 13
  semisimple = 14
  mathshift = 15
  mathleft = 16
etex.iftype
  none = 0
  char = 1
  cat = 2
  num = 3
  dim = 4
  odd = 5
  vmode = 6
  hmode = 7
  mmode = 8
  inner = 9
  void = 10
  hbox = 11
  vbox = 12
  x = 13
  eof = 14
  true = 15
  false = 16
  case = 17
  defined = 18
  csname = 19
  fontchar = 20
etex.nodetype
  none = -1
  char = 0
  hlist = 1
  vlist = 2
  rule = 3
  ins = 4
  mark = 5
  adjust = 6
  ligature = 7
  disc = 8
  whatsit = 9
  math = 10
  glue = 11
  kern = 12
  penalty = 13
  unset = 14
  maths = 15
etex.interactionmode
  batch = 0
  nonstop = 1
  scroll = 2
  errorstop = 3
luatex.pdfliteral.mode
  setorigin = 0
  page = 1
  direct = 2
%</data>
%    \end{macrocode}
%
% \subsubsection{Data for \TeX}
%
%    \begin{macrocode}
%<*package>
%    \end{macrocode}
%    \begin{macro}{\magicnum@add}
%    \begin{macrocode}
\begingroup\expandafter\expandafter\expandafter\endgroup
\expandafter\ifx\csname detokenize\endcsname\relax
  \def\magicnum@add#1#2#3{%
    \expandafter\magicnum@@add
        \csname MG@#1.#2\expandafter\endcsname
        \csname MG@#1.#3\endcsname
       {#3}{#2}%
  }%
  \def\magicnum@@add#1#2#3#4{%
    \def#1{#3}%
    \def#2{#4}%
    \edef#1{%
      \expandafter\strip@prefix\meaning#1%
    }%
    \edef#2{%
      \expandafter\strip@prefix\meaning#2%
    }%
  }%
  \expandafter\ifx\csname strip@prefix\endcsname\relax
    \def\strip@prefix#1->{}%
  \fi
\else
  \def\magicnum@add#1#2#3{%
    \expandafter\edef\csname MG@#1.#2\endcsname{%
      \detokenize{#3}%
    }%
    \expandafter\edef\csname MG@#1.#3\endcsname{%
      \detokenize{#2}%
    }%
  }%
\fi
%    \end{macrocode}
%    \end{macro}
%    \begin{macrocode}
\magicnum@add{tex.catcode}{escape}{0}
\magicnum@add{tex.catcode}{begingroup}{1}
\magicnum@add{tex.catcode}{endgroup}{2}
\magicnum@add{tex.catcode}{math}{3}
\magicnum@add{tex.catcode}{align}{4}
\magicnum@add{tex.catcode}{eol}{5}
\magicnum@add{tex.catcode}{parameter}{6}
\magicnum@add{tex.catcode}{superscript}{7}
\magicnum@add{tex.catcode}{subscript}{8}
\magicnum@add{tex.catcode}{ignore}{9}
\magicnum@add{tex.catcode}{space}{10}
\magicnum@add{tex.catcode}{letter}{11}
\magicnum@add{tex.catcode}{other}{12}
\magicnum@add{tex.catcode}{active}{13}
\magicnum@add{tex.catcode}{comment}{14}
\magicnum@add{tex.catcode}{invalid}{15}
\magicnum@add{etex.grouptype}{bottomlevel}{0}
\magicnum@add{etex.grouptype}{simple}{1}
\magicnum@add{etex.grouptype}{hbox}{2}
\magicnum@add{etex.grouptype}{adjustedhbox}{3}
\magicnum@add{etex.grouptype}{vbox}{4}
\magicnum@add{etex.grouptype}{align}{5}
\magicnum@add{etex.grouptype}{noalign}{6}
\magicnum@add{etex.grouptype}{output}{8}
\magicnum@add{etex.grouptype}{math}{9}
\magicnum@add{etex.grouptype}{disc}{10}
\magicnum@add{etex.grouptype}{insert}{11}
\magicnum@add{etex.grouptype}{vcenter}{12}
\magicnum@add{etex.grouptype}{mathchoice}{13}
\magicnum@add{etex.grouptype}{semisimple}{14}
\magicnum@add{etex.grouptype}{mathshift}{15}
\magicnum@add{etex.grouptype}{mathleft}{16}
\magicnum@add{etex.iftype}{none}{0}
\magicnum@add{etex.iftype}{char}{1}
\magicnum@add{etex.iftype}{cat}{2}
\magicnum@add{etex.iftype}{num}{3}
\magicnum@add{etex.iftype}{dim}{4}
\magicnum@add{etex.iftype}{odd}{5}
\magicnum@add{etex.iftype}{vmode}{6}
\magicnum@add{etex.iftype}{hmode}{7}
\magicnum@add{etex.iftype}{mmode}{8}
\magicnum@add{etex.iftype}{inner}{9}
\magicnum@add{etex.iftype}{void}{10}
\magicnum@add{etex.iftype}{hbox}{11}
\magicnum@add{etex.iftype}{vbox}{12}
\magicnum@add{etex.iftype}{x}{13}
\magicnum@add{etex.iftype}{eof}{14}
\magicnum@add{etex.iftype}{true}{15}
\magicnum@add{etex.iftype}{false}{16}
\magicnum@add{etex.iftype}{case}{17}
\magicnum@add{etex.iftype}{defined}{18}
\magicnum@add{etex.iftype}{csname}{19}
\magicnum@add{etex.iftype}{fontchar}{20}
\magicnum@add{etex.nodetype}{none}{-1}
\magicnum@add{etex.nodetype}{char}{0}
\magicnum@add{etex.nodetype}{hlist}{1}
\magicnum@add{etex.nodetype}{vlist}{2}
\magicnum@add{etex.nodetype}{rule}{3}
\magicnum@add{etex.nodetype}{ins}{4}
\magicnum@add{etex.nodetype}{mark}{5}
\magicnum@add{etex.nodetype}{adjust}{6}
\magicnum@add{etex.nodetype}{ligature}{7}
\magicnum@add{etex.nodetype}{disc}{8}
\magicnum@add{etex.nodetype}{whatsit}{9}
\magicnum@add{etex.nodetype}{math}{10}
\magicnum@add{etex.nodetype}{glue}{11}
\magicnum@add{etex.nodetype}{kern}{12}
\magicnum@add{etex.nodetype}{penalty}{13}
\magicnum@add{etex.nodetype}{unset}{14}
\magicnum@add{etex.nodetype}{maths}{15}
\magicnum@add{etex.interactionmode}{batch}{0}
\magicnum@add{etex.interactionmode}{nonstop}{1}
\magicnum@add{etex.interactionmode}{scroll}{2}
\magicnum@add{etex.interactionmode}{errorstop}{3}
\magicnum@add{luatex.pdfliteral.mode}{setorigin}{0}
\magicnum@add{luatex.pdfliteral.mode}{page}{1}
\magicnum@add{luatex.pdfliteral.mode}{direct}{2}
%    \end{macrocode}
%    \begin{macrocode}
\magicnum@AtEnd%
%</package>
%    \end{macrocode}
%
% \subsubsection{Lua module}
%
%    \begin{macrocode}
%<*lua>
%    \end{macrocode}
%    \begin{macrocode}
module("oberdiek.magicnum", package.seeall)
%    \end{macrocode}
%    \begin{macrocode}
function getversion()
  tex.write("2011/04/10 v1.4")
end
%    \end{macrocode}
%    \begin{macrocode}
local data = {
  ["tex.catcode"] = {
    [0] = "escape",
    [1] = "begingroup",
    [2] = "endgroup",
    [3] = "math",
    [4] = "align",
    [5] = "eol",
    [6] = "parameter",
    [7] = "superscript",
    [8] = "subscript",
    [9] = "ignore",
    [10] = "space",
    [11] = "letter",
    [12] = "other",
    [13] = "active",
    [14] = "comment",
    [15] = "invalid",
    ["active"] = 13,
    ["align"] = 4,
    ["begingroup"] = 1,
    ["comment"] = 14,
    ["endgroup"] = 2,
    ["eol"] = 5,
    ["escape"] = 0,
    ["ignore"] = 9,
    ["invalid"] = 15,
    ["letter"] = 11,
    ["math"] = 3,
    ["other"] = 12,
    ["parameter"] = 6,
    ["space"] = 10,
    ["subscript"] = 8,
    ["superscript"] = 7
  },
  ["etex.grouptype"] = {
    [0] = "bottomlevel",
    [1] = "simple",
    [2] = "hbox",
    [3] = "adjustedhbox",
    [4] = "vbox",
    [5] = "align",
    [6] = "noalign",
    [8] = "output",
    [9] = "math",
    [10] = "disc",
    [11] = "insert",
    [12] = "vcenter",
    [13] = "mathchoice",
    [14] = "semisimple",
    [15] = "mathshift",
    [16] = "mathleft",
    ["adjustedhbox"] = 3,
    ["align"] = 5,
    ["bottomlevel"] = 0,
    ["disc"] = 10,
    ["hbox"] = 2,
    ["insert"] = 11,
    ["math"] = 9,
    ["mathchoice"] = 13,
    ["mathleft"] = 16,
    ["mathshift"] = 15,
    ["noalign"] = 6,
    ["output"] = 8,
    ["semisimple"] = 14,
    ["simple"] = 1,
    ["vbox"] = 4,
    ["vcenter"] = 12
  },
  ["etex.iftype"] = {
    [0] = "none",
    [1] = "char",
    [2] = "cat",
    [3] = "num",
    [4] = "dim",
    [5] = "odd",
    [6] = "vmode",
    [7] = "hmode",
    [8] = "mmode",
    [9] = "inner",
    [10] = "void",
    [11] = "hbox",
    [12] = "vbox",
    [13] = "x",
    [14] = "eof",
    [15] = "true",
    [16] = "false",
    [17] = "case",
    [18] = "defined",
    [19] = "csname",
    [20] = "fontchar",
    ["case"] = 17,
    ["cat"] = 2,
    ["char"] = 1,
    ["csname"] = 19,
    ["defined"] = 18,
    ["dim"] = 4,
    ["eof"] = 14,
    ["false"] = 16,
    ["fontchar"] = 20,
    ["hbox"] = 11,
    ["hmode"] = 7,
    ["inner"] = 9,
    ["mmode"] = 8,
    ["none"] = 0,
    ["num"] = 3,
    ["odd"] = 5,
    ["true"] = 15,
    ["vbox"] = 12,
    ["vmode"] = 6,
    ["void"] = 10,
    ["x"] = 13
  },
  ["etex.nodetype"] = {
    [-1] = "none",
    [0] = "char",
    [1] = "hlist",
    [2] = "vlist",
    [3] = "rule",
    [4] = "ins",
    [5] = "mark",
    [6] = "adjust",
    [7] = "ligature",
    [8] = "disc",
    [9] = "whatsit",
    [10] = "math",
    [11] = "glue",
    [12] = "kern",
    [13] = "penalty",
    [14] = "unset",
    [15] = "maths",
    ["adjust"] = 6,
    ["char"] = 0,
    ["disc"] = 8,
    ["glue"] = 11,
    ["hlist"] = 1,
    ["ins"] = 4,
    ["kern"] = 12,
    ["ligature"] = 7,
    ["mark"] = 5,
    ["math"] = 10,
    ["maths"] = 15,
    ["none"] = -1,
    ["penalty"] = 13,
    ["rule"] = 3,
    ["unset"] = 14,
    ["vlist"] = 2,
    ["whatsit"] = 9
  },
  ["etex.interactionmode"] = {
    [0] = "batch",
    [1] = "nonstop",
    [2] = "scroll",
    [3] = "errorstop",
    ["batch"] = 0,
    ["errorstop"] = 3,
    ["nonstop"] = 1,
    ["scroll"] = 2
  },
  ["luatex.pdfliteral.mode"] = {
    [0] = "setorigin",
    [1] = "page",
    [2] = "direct",
    ["direct"] = 2,
    ["page"] = 1,
    ["setorigin"] = 0
  }
}
%    \end{macrocode}
%    \begin{macrocode}
function get(name)
  local startpos, endpos, category, entry =
      string.find(name, "^(%a[%a%d%.]*)%.(-?[%a%d]+)$")
  if not entry then
    return
  end
  local node = data[category]
  if not node then
    return
  end
  local num = tonumber(entry)
  local value
  if num then
    value = node[num]
    if not value then
      return
    end
  else
    value = node[entry]
    if not value then
      return
    end
    value = "" .. value
  end
  tex.write(value)
end
%    \end{macrocode}
%
%    \begin{macrocode}
%</lua>
%    \end{macrocode}
%
% \section{Test}
%
% \subsection{Catcode checks for loading}
%
%    \begin{macrocode}
%<*test1>
%    \end{macrocode}
%    \begin{macrocode}
\catcode`\{=1 %
\catcode`\}=2 %
\catcode`\#=6 %
\catcode`\@=11 %
\expandafter\ifx\csname count@\endcsname\relax
  \countdef\count@=255 %
\fi
\expandafter\ifx\csname @gobble\endcsname\relax
  \long\def\@gobble#1{}%
\fi
\expandafter\ifx\csname @firstofone\endcsname\relax
  \long\def\@firstofone#1{#1}%
\fi
\expandafter\ifx\csname loop\endcsname\relax
  \expandafter\@firstofone
\else
  \expandafter\@gobble
\fi
{%
  \def\loop#1\repeat{%
    \def\body{#1}%
    \iterate
  }%
  \def\iterate{%
    \body
      \let\next\iterate
    \else
      \let\next\relax
    \fi
    \next
  }%
  \let\repeat=\fi
}%
\def\RestoreCatcodes{}
\count@=0 %
\loop
  \edef\RestoreCatcodes{%
    \RestoreCatcodes
    \catcode\the\count@=\the\catcode\count@\relax
  }%
\ifnum\count@<255 %
  \advance\count@ 1 %
\repeat

\def\RangeCatcodeInvalid#1#2{%
  \count@=#1\relax
  \loop
    \catcode\count@=15 %
  \ifnum\count@<#2\relax
    \advance\count@ 1 %
  \repeat
}
\def\RangeCatcodeCheck#1#2#3{%
  \count@=#1\relax
  \loop
    \ifnum#3=\catcode\count@
    \else
      \errmessage{%
        Character \the\count@\space
        with wrong catcode \the\catcode\count@\space
        instead of \number#3%
      }%
    \fi
  \ifnum\count@<#2\relax
    \advance\count@ 1 %
  \repeat
}
\def\space{ }
\expandafter\ifx\csname LoadCommand\endcsname\relax
  \def\LoadCommand{\input magicnum.sty\relax}%
\fi
\def\Test{%
  \RangeCatcodeInvalid{0}{47}%
  \RangeCatcodeInvalid{58}{64}%
  \RangeCatcodeInvalid{91}{96}%
  \RangeCatcodeInvalid{123}{255}%
  \catcode`\@=12 %
  \catcode`\\=0 %
  \catcode`\%=14 %
  \LoadCommand
  \RangeCatcodeCheck{0}{36}{15}%
  \RangeCatcodeCheck{37}{37}{14}%
  \RangeCatcodeCheck{38}{47}{15}%
  \RangeCatcodeCheck{48}{57}{12}%
  \RangeCatcodeCheck{58}{63}{15}%
  \RangeCatcodeCheck{64}{64}{12}%
  \RangeCatcodeCheck{65}{90}{11}%
  \RangeCatcodeCheck{91}{91}{15}%
  \RangeCatcodeCheck{92}{92}{0}%
  \RangeCatcodeCheck{93}{96}{15}%
  \RangeCatcodeCheck{97}{122}{11}%
  \RangeCatcodeCheck{123}{255}{15}%
  \RestoreCatcodes
}
\Test
\csname @@end\endcsname
\end
%    \end{macrocode}
%    \begin{macrocode}
%</test1>
%    \end{macrocode}
%
% \subsection{Test data}
%
%    \begin{macrocode}
%<*testplain>
\input magicnum.sty\relax
\def\Test#1#2{%
  \edef\result{\magicnum{#1}}%
  \edef\expect{#2}%
  \edef\expect{\expandafter\stripprefix\meaning\expect}%
  \ifx\result\expect
  \else
    \errmessage{%
      Failed: [#1] % hash-ok
      returns [\result] instead of [\expect]%
    }%
  \fi
}
\def\stripprefix#1->{}
%</testplain>
%    \end{macrocode}
%    \begin{macrocode}
%<*testlatex>
\NeedsTeXFormat{LaTeX2e}
\documentclass{minimal}
\usepackage{magicnum}[2011/04/10]
\usepackage{qstest}
\IncludeTests{*}
\LogTests{log}{*}{*}
\newcommand*{\Test}[2]{%
  \Expect*{\magicnum{#1}}{#2}%
}
\begin{qstest}{magicnum}{magicnum}
%</testlatex>
%    \end{macrocode}
%    \begin{macrocode}
%<*testdata>
\Test{tex.catcode.escape}{0}
\Test{tex.catcode.invalid}{15}
\Test{tex.catcode.unknown}{}
\Test{tex.catcode.0}{escape}
\Test{tex.catcode.15}{invalid}
\Test{etex.iftype.true}{15}
\Test{etex.iftype.false}{16}
\Test{etex.iftype.15}{true}
\Test{etex.iftype.16}{false}
\Test{etex.nodetype.none}{-1}
\Test{etex.nodetype.-1}{none}
\Test{luatex.pdfliteral.mode.direct}{2}
\Test{luatex.pdfliteral.mode.1}{page}
\Test{}{}
\Test{unknown}{}
\Test{unknown.foo.bar}{}
\Test{unknown.foo.4}{}
%</testdata>
%    \end{macrocode}
%    \begin{macrocode}
%<*testplain>
\csname @@end\endcsname
\end
%</testplain>
%<*testlatex>
\end{qstest}
\csname @@end\endcsname
%</testlatex>
%    \end{macrocode}
%
% \subsection{Small test for \hologo{iniTeX}}
%
%    \begin{macrocode}
%<*test4>
\catcode`\{=1
\catcode`\}=2
\catcode`\#=6
\input magicnum.sty\relax
\edef\x{\magicnum{tex.catcode.15}}
\edef\y{invalid}
\def\Strip#1>{}
\edef\y{\expandafter\Strip\meaning\y}
\ifx\x\y
  \immediate\write16{Ok}%
\else
  \errmessage{\x<>\y}%
\fi
\csname @@end\endcsname\end
%</test4>
%    \end{macrocode}
%
% \section{Installation}
%
% \subsection{Download}
%
% \paragraph{Package.} This package is available on
% CTAN\footnote{\url{ftp://ftp.ctan.org/tex-archive/}}:
% \begin{description}
% \item[\CTAN{macros/latex/contrib/oberdiek/magicnum.dtx}] The source file.
% \item[\CTAN{macros/latex/contrib/oberdiek/magicnum.pdf}] Documentation.
% \end{description}
%
%
% \paragraph{Bundle.} All the packages of the bundle `oberdiek'
% are also available in a TDS compliant ZIP archive. There
% the packages are already unpacked and the documentation files
% are generated. The files and directories obey the TDS standard.
% \begin{description}
% \item[\CTAN{install/macros/latex/contrib/oberdiek.tds.zip}]
% \end{description}
% \emph{TDS} refers to the standard ``A Directory Structure
% for \TeX\ Files'' (\CTAN{tds/tds.pdf}). Directories
% with \xfile{texmf} in their name are usually organized this way.
%
% \subsection{Bundle installation}
%
% \paragraph{Unpacking.} Unpack the \xfile{oberdiek.tds.zip} in the
% TDS tree (also known as \xfile{texmf} tree) of your choice.
% Example (linux):
% \begin{quote}
%   |unzip oberdiek.tds.zip -d ~/texmf|
% \end{quote}
%
% \paragraph{Script installation.}
% Check the directory \xfile{TDS:scripts/oberdiek/} for
% scripts that need further installation steps.
% Package \xpackage{attachfile2} comes with the Perl script
% \xfile{pdfatfi.pl} that should be installed in such a way
% that it can be called as \texttt{pdfatfi}.
% Example (linux):
% \begin{quote}
%   |chmod +x scripts/oberdiek/pdfatfi.pl|\\
%   |cp scripts/oberdiek/pdfatfi.pl /usr/local/bin/|
% \end{quote}
%
% \subsection{Package installation}
%
% \paragraph{Unpacking.} The \xfile{.dtx} file is a self-extracting
% \docstrip\ archive. The files are extracted by running the
% \xfile{.dtx} through \plainTeX:
% \begin{quote}
%   \verb|tex magicnum.dtx|
% \end{quote}
%
% \paragraph{TDS.} Now the different files must be moved into
% the different directories in your installation TDS tree
% (also known as \xfile{texmf} tree):
% \begin{quote}
% \def\t{^^A
% \begin{tabular}{@{}>{\ttfamily}l@{ $\rightarrow$ }>{\ttfamily}l@{}}
%   magicnum.sty & tex/generic/oberdiek/magicnum.sty\\
%   magicnum.lua & scripts/oberdiek/magicnum.lua\\
%   oberdiek.magicnum.lua & scripts/oberdiek/oberdiek.magicnum.lua\\
%   magicnum.pdf & doc/latex/oberdiek/magicnum.pdf\\
%   magicnum.txt & doc/latex/oberdiek/magicnum.txt\\
%   test/magicnum-test1.tex & doc/latex/oberdiek/test/magicnum-test1.tex\\
%   test/magicnum-test2.tex & doc/latex/oberdiek/test/magicnum-test2.tex\\
%   test/magicnum-test3.tex & doc/latex/oberdiek/test/magicnum-test3.tex\\
%   test/magicnum-test4.tex & doc/latex/oberdiek/test/magicnum-test4.tex\\
%   magicnum.dtx & source/latex/oberdiek/magicnum.dtx\\
% \end{tabular}^^A
% }^^A
% \sbox0{\t}^^A
% \ifdim\wd0>\linewidth
%   \begingroup
%     \advance\linewidth by\leftmargin
%     \advance\linewidth by\rightmargin
%   \edef\x{\endgroup
%     \def\noexpand\lw{\the\linewidth}^^A
%   }\x
%   \def\lwbox{^^A
%     \leavevmode
%     \hbox to \linewidth{^^A
%       \kern-\leftmargin\relax
%       \hss
%       \usebox0
%       \hss
%       \kern-\rightmargin\relax
%     }^^A
%   }^^A
%   \ifdim\wd0>\lw
%     \sbox0{\small\t}^^A
%     \ifdim\wd0>\linewidth
%       \ifdim\wd0>\lw
%         \sbox0{\footnotesize\t}^^A
%         \ifdim\wd0>\linewidth
%           \ifdim\wd0>\lw
%             \sbox0{\scriptsize\t}^^A
%             \ifdim\wd0>\linewidth
%               \ifdim\wd0>\lw
%                 \sbox0{\tiny\t}^^A
%                 \ifdim\wd0>\linewidth
%                   \lwbox
%                 \else
%                   \usebox0
%                 \fi
%               \else
%                 \lwbox
%               \fi
%             \else
%               \usebox0
%             \fi
%           \else
%             \lwbox
%           \fi
%         \else
%           \usebox0
%         \fi
%       \else
%         \lwbox
%       \fi
%     \else
%       \usebox0
%     \fi
%   \else
%     \lwbox
%   \fi
% \else
%   \usebox0
% \fi
% \end{quote}
% If you have a \xfile{docstrip.cfg} that configures and enables \docstrip's
% TDS installing feature, then some files can already be in the right
% place, see the documentation of \docstrip.
%
% \subsection{Refresh file name databases}
%
% If your \TeX~distribution
% (\teTeX, \mikTeX, \dots) relies on file name databases, you must refresh
% these. For example, \teTeX\ users run \verb|texhash| or
% \verb|mktexlsr|.
%
% \subsection{Some details for the interested}
%
% \paragraph{Attached source.}
%
% The PDF documentation on CTAN also includes the
% \xfile{.dtx} source file. It can be extracted by
% AcrobatReader 6 or higher. Another option is \textsf{pdftk},
% e.g. unpack the file into the current directory:
% \begin{quote}
%   \verb|pdftk magicnum.pdf unpack_files output .|
% \end{quote}
%
% \paragraph{Unpacking with \LaTeX.}
% The \xfile{.dtx} chooses its action depending on the format:
% \begin{description}
% \item[\plainTeX:] Run \docstrip\ and extract the files.
% \item[\LaTeX:] Generate the documentation.
% \end{description}
% If you insist on using \LaTeX\ for \docstrip\ (really,
% \docstrip\ does not need \LaTeX), then inform the autodetect routine
% about your intention:
% \begin{quote}
%   \verb|latex \let\install=y% \iffalse meta-comment
%
% File: magicnum.dtx
% Version: 2011/04/10 v1.4
% Info: Magic numbers
%
% Copyright (C) 2007, 2009-2011 by
%    Heiko Oberdiek <heiko.oberdiek at googlemail.com>
%
% This work may be distributed and/or modified under the
% conditions of the LaTeX Project Public License, either
% version 1.3c of this license or (at your option) any later
% version. This version of this license is in
%    http://www.latex-project.org/lppl/lppl-1-3c.txt
% and the latest version of this license is in
%    http://www.latex-project.org/lppl.txt
% and version 1.3 or later is part of all distributions of
% LaTeX version 2005/12/01 or later.
%
% This work has the LPPL maintenance status "maintained".
%
% This Current Maintainer of this work is Heiko Oberdiek.
%
% The Base Interpreter refers to any `TeX-Format',
% because some files are installed in TDS:tex/generic//.
%
% This work consists of the main source file magicnum.dtx
% and the derived files
%    magicnum.sty, magicnum.pdf, magicnum.ins, magicnum.drv, magicnum.txt,
%    magicnum-test1.tex, magicnum-test2.tex, magicnum-test3.tex,
%    magicnum-test4.tex, magicnum.lua, oberdiek.magicnum.lua.
%
% Distribution:
%    CTAN:macros/latex/contrib/oberdiek/magicnum.dtx
%    CTAN:macros/latex/contrib/oberdiek/magicnum.pdf
%
% Unpacking:
%    (a) If magicnum.ins is present:
%           tex magicnum.ins
%    (b) Without magicnum.ins:
%           tex magicnum.dtx
%    (c) If you insist on using LaTeX
%           latex \let\install=y\input{magicnum.dtx}
%        (quote the arguments according to the demands of your shell)
%
% Documentation:
%    (a) If magicnum.drv is present:
%           latex magicnum.drv
%    (b) Without magicnum.drv:
%           latex magicnum.dtx; ...
%    The class ltxdoc loads the configuration file ltxdoc.cfg
%    if available. Here you can specify further options, e.g.
%    use A4 as paper format:
%       \PassOptionsToClass{a4paper}{article}
%
%    Programm calls to get the documentation (example):
%       pdflatex magicnum.dtx
%       makeindex -s gind.ist magicnum.idx
%       pdflatex magicnum.dtx
%       makeindex -s gind.ist magicnum.idx
%       pdflatex magicnum.dtx
%
% Installation:
%    TDS:tex/generic/oberdiek/magicnum.sty
%    TDS:scripts/oberdiek/magicnum.lua
%    TDS:scripts/oberdiek/oberdiek.magicnum.lua
%    TDS:doc/latex/oberdiek/magicnum.pdf
%    TDS:doc/latex/oberdiek/magicnum.txt
%    TDS:doc/latex/oberdiek/test/magicnum-test1.tex
%    TDS:doc/latex/oberdiek/test/magicnum-test2.tex
%    TDS:doc/latex/oberdiek/test/magicnum-test3.tex
%    TDS:doc/latex/oberdiek/test/magicnum-test4.tex
%    TDS:source/latex/oberdiek/magicnum.dtx
%
%<*ignore>
\begingroup
  \catcode123=1 %
  \catcode125=2 %
  \def\x{LaTeX2e}%
\expandafter\endgroup
\ifcase 0\ifx\install y1\fi\expandafter
         \ifx\csname processbatchFile\endcsname\relax\else1\fi
         \ifx\fmtname\x\else 1\fi\relax
\else\csname fi\endcsname
%</ignore>
%<*install>
\input docstrip.tex
\Msg{************************************************************************}
\Msg{* Installation}
\Msg{* Package: magicnum 2011/04/10 v1.4 Magic numbers (HO)}
\Msg{************************************************************************}

\keepsilent
\askforoverwritefalse

\let\MetaPrefix\relax
\preamble

This is a generated file.

Project: magicnum
Version: 2011/04/10 v1.4

Copyright (C) 2007, 2009-2011 by
   Heiko Oberdiek <heiko.oberdiek at googlemail.com>

This work may be distributed and/or modified under the
conditions of the LaTeX Project Public License, either
version 1.3c of this license or (at your option) any later
version. This version of this license is in
   http://www.latex-project.org/lppl/lppl-1-3c.txt
and the latest version of this license is in
   http://www.latex-project.org/lppl.txt
and version 1.3 or later is part of all distributions of
LaTeX version 2005/12/01 or later.

This work has the LPPL maintenance status "maintained".

This Current Maintainer of this work is Heiko Oberdiek.

The Base Interpreter refers to any `TeX-Format',
because some files are installed in TDS:tex/generic//.

This work consists of the main source file magicnum.dtx
and the derived files
   magicnum.sty, magicnum.pdf, magicnum.ins, magicnum.drv, magicnum.txt,
   magicnum-test1.tex, magicnum-test2.tex, magicnum-test3.tex,
   magicnum-test4.tex, magicnum.lua, oberdiek.magicnum.lua.

\endpreamble
\let\MetaPrefix\DoubleperCent

\generate{%
  \file{magicnum.ins}{\from{magicnum.dtx}{install}}%
  \file{magicnum.drv}{\from{magicnum.dtx}{driver}}%
  \usedir{tex/generic/oberdiek}%
  \file{magicnum.sty}{\from{magicnum.dtx}{package}}%
  \usedir{doc/latex/oberdiek/test}%
  \file{magicnum-test1.tex}{\from{magicnum.dtx}{test1}}%
  \file{magicnum-test2.tex}{\from{magicnum.dtx}{testplain,testdata}}%
  \file{magicnum-test3.tex}{\from{magicnum.dtx}{testlatex,testdata}}%
  \file{magicnum-test4.tex}{\from{magicnum.dtx}{test4}}%
  \nopreamble
  \nopostamble
  \usedir{doc/latex/oberdiek}%
  \file{magicnum.txt}{\from{magicnum.dtx}{data}}%
  \usedir{source/latex/oberdiek/catalogue}%
  \file{magicnum.xml}{\from{magicnum.dtx}{catalogue}}%
}
\def\MetaPrefix{-- }
\def\defaultpostamble{%
  \MetaPrefix^^J%
  \MetaPrefix\space End of File `\outFileName'.%
}
\def\currentpostamble{\defaultpostamble}%
\generate{%
  \usedir{scripts/oberdiek}%
  \file{magicnum.lua}{\from{magicnum.dtx}{lua}}%
  \file{oberdiek.magicnum.lua}{\from{magicnum.dtx}{lua}}%
}

\catcode32=13\relax% active space
\let =\space%
\Msg{************************************************************************}
\Msg{*}
\Msg{* To finish the installation you have to move the following}
\Msg{* file into a directory searched by TeX:}
\Msg{*}
\Msg{*     magicnum.sty}
\Msg{*}
\Msg{* And install the following script files:}
\Msg{*}
\Msg{*     magicnum.lua, oberdiek.magicnum.lua}
\Msg{*}
\Msg{* To produce the documentation run the file `magicnum.drv'}
\Msg{* through LaTeX.}
\Msg{*}
\Msg{* Happy TeXing!}
\Msg{*}
\Msg{************************************************************************}

\endbatchfile
%</install>
%<*ignore>
\fi
%</ignore>
%<*driver>
\NeedsTeXFormat{LaTeX2e}
\ProvidesFile{magicnum.drv}%
  [2011/04/10 v1.4 Magic numbers (HO)]%
\documentclass{ltxdoc}
\usepackage{holtxdoc}[2011/11/22]
\usepackage{array}
\begin{document}
  \DocInput{magicnum.dtx}%
\end{document}
%</driver>
% \fi
%
% \CheckSum{755}
%
% \CharacterTable
%  {Upper-case    \A\B\C\D\E\F\G\H\I\J\K\L\M\N\O\P\Q\R\S\T\U\V\W\X\Y\Z
%   Lower-case    \a\b\c\d\e\f\g\h\i\j\k\l\m\n\o\p\q\r\s\t\u\v\w\x\y\z
%   Digits        \0\1\2\3\4\5\6\7\8\9
%   Exclamation   \!     Double quote  \"     Hash (number) \#
%   Dollar        \$     Percent       \%     Ampersand     \&
%   Acute accent  \'     Left paren    \(     Right paren   \)
%   Asterisk      \*     Plus          \+     Comma         \,
%   Minus         \-     Point         \.     Solidus       \/
%   Colon         \:     Semicolon     \;     Less than     \<
%   Equals        \=     Greater than  \>     Question mark \?
%   Commercial at \@     Left bracket  \[     Backslash     \\
%   Right bracket \]     Circumflex    \^     Underscore    \_
%   Grave accent  \`     Left brace    \{     Vertical bar  \|
%   Right brace   \}     Tilde         \~}
%
% \GetFileInfo{magicnum.drv}
%
% \title{The \xpackage{magicnum} package}
% \date{2011/04/10 v1.4}
% \author{Heiko Oberdiek\\\xemail{heiko.oberdiek at googlemail.com}}
%
% \maketitle
%
% \begin{abstract}
% This packages allows to access magic numbers by a hierarchical
% name system.
% \end{abstract}
%
% \tableofcontents
%
% \hypersetup{bookmarksopenlevel=2}
% \section{Documentation}
%
% \subsection{Introduction}
%
% Especially since \eTeX\ there are many integer values
% with special meanings, such as catcodes, group types, \dots
% Package \xpackage{etex}, enabled by options, defines
% macros in the user namespace for these values.
%
% This package goes another approach for storing the names and values.
% \begin{itemize}
% \item If \LuaTeX\ is available, they
% are stored in Lua tables.
% \item Without \LuaTeX\ they are remembered using internal
% macros.
% \end{itemize}
%
% \subsection{User interface}
%
% The integer values and names are organized in a hierarchical
% scheme of categories with the property names as leaves.
% Example: \eTeX's \cs{currentgrouplevel} reports |2| for a
% group caused by \cs{hbox}. This package has choosen to organize
% the group types in a main category |etex| and its subcategory
% |grouptype|:
% \begin{quote}
%   |etex.grouptype.hbox| = |2|
% \end{quote}
% The property name |hbox| in category |etex.grouptype| has value |2|.
% Dots are used to separate components.
%
% If you want to have the value, the access key is constructed by
% the category with all its components and the property name.
% For the opposite the value is used instead of the property name.
%
% Values are always integers (including negative numbers).
%
% \subsubsection{\cs{magicnum}}
%
% \begin{declcs}{magicnum} \M{access key}
% \end{declcs}
% Macro \cs{magicnum} expects an access key as argument and
% expands to the requested data. The macro is always expandable.
% In case of errors the expansion result is empty.
%
% The same macro is also used for getting a property name.
% In this case the property name part in the access key is
% replaced by the value.
%
% The catcodes
% of the resulting numbers and strings follow \TeX's tradition of
% \cs{string}, \cs{meaning}, \dots: The space has catcode 10
% (|tex.catcode.space|) and the other characters have catcode
% 12 (|tex.catcode.other|).
%
% Examples:
% \begin{quote}
%   |\magicnum{etex.grouptype.hbox}| $\Rightarrow$ |2|\\
%   |\magicnum{tex.catcode.14}| $\Rightarrow$ |comment|\\
%   |\magicnum{tex.catcode.undefined}| $\Rightarrow$ $\emptyset$
% \end{quote}
%
% \subsubsection{Properties}
%
% \begin{itemize}
% \item The components of a category are either subcategories or
%       key value pairs, but not both.
% \item The full specified property names are unique and thus
%       has one integer value exactly.
% \item Also the values inside a category are unique.
%       This condition is a prerequisite for the reverse mapping
%       of \cs{magicnum}.
% \item All names start with a letter. Only letters or digits
%       may follow.
% \end{itemize}
%
% \subsection{Data}
%
%  \subsubsection{\texorpdfstring{Category }{}\texttt{tex.catcode}}
%
% \begin{quote}
% \begin{tabular}{@{}>{\ttfamily}l>{\ttfamily}l@{}}
%    tex.catcode.escape & 0\\
%    tex.catcode.begingroup & 1\\
%    tex.catcode.endgroup & 2\\
%    tex.catcode.math & 3\\
%    tex.catcode.align & 4\\
%    tex.catcode.eol & 5\\
%    tex.catcode.parameter & 6\\
%    tex.catcode.superscript & 7\\
%    tex.catcode.subscript & 8\\
%    tex.catcode.ignore & 9\\
%    tex.catcode.space & 10\\
%    tex.catcode.letter & 11\\
%    tex.catcode.other & 12\\
%    tex.catcode.active & 13\\
%    tex.catcode.comment & 14\\
%    tex.catcode.invalid & 15\\
%  \end{tabular}
%  \end{quote}
%
%  \subsubsection{\texorpdfstring{Category }{}\texttt{etex.grouptype}}
%
% \begin{quote}
% \begin{tabular}{@{}>{\ttfamily}l>{\ttfamily}l@{}}
%    etex.grouptype.bottomlevel & 0\\
%    etex.grouptype.simple & 1\\
%    etex.grouptype.hbox & 2\\
%    etex.grouptype.adjustedhbox & 3\\
%    etex.grouptype.vbox & 4\\
%    etex.grouptype.align & 5\\
%    etex.grouptype.noalign & 6\\
%    etex.grouptype.output & 8\\
%    etex.grouptype.math & 9\\
%    etex.grouptype.disc & 10\\
%    etex.grouptype.insert & 11\\
%    etex.grouptype.vcenter & 12\\
%    etex.grouptype.mathchoice & 13\\
%    etex.grouptype.semisimple & 14\\
%    etex.grouptype.mathshift & 15\\
%    etex.grouptype.mathleft & 16\\
%  \end{tabular}
%  \end{quote}
%
%  \subsubsection{\texorpdfstring{Category }{}\texttt{etex.iftype}}
%
% \begin{quote}
% \begin{tabular}{@{}>{\ttfamily}l>{\ttfamily}l@{}}
%    etex.iftype.none & 0\\
%    etex.iftype.char & 1\\
%    etex.iftype.cat & 2\\
%    etex.iftype.num & 3\\
%    etex.iftype.dim & 4\\
%    etex.iftype.odd & 5\\
%    etex.iftype.vmode & 6\\
%    etex.iftype.hmode & 7\\
%    etex.iftype.mmode & 8\\
%    etex.iftype.inner & 9\\
%    etex.iftype.void & 10\\
%    etex.iftype.hbox & 11\\
%    etex.iftype.vbox & 12\\
%    etex.iftype.x & 13\\
%    etex.iftype.eof & 14\\
%    etex.iftype.true & 15\\
%    etex.iftype.false & 16\\
%    etex.iftype.case & 17\\
%    etex.iftype.defined & 18\\
%    etex.iftype.csname & 19\\
%    etex.iftype.fontchar & 20\\
%  \end{tabular}
%  \end{quote}
%
%  \subsubsection{\texorpdfstring{Category }{}\texttt{etex.nodetype}}
%
% \begin{quote}
% \begin{tabular}{@{}>{\ttfamily}l>{\ttfamily}l@{}}
%    etex.nodetype.none & -1\\
%    etex.nodetype.char & 0\\
%    etex.nodetype.hlist & 1\\
%    etex.nodetype.vlist & 2\\
%    etex.nodetype.rule & 3\\
%    etex.nodetype.ins & 4\\
%    etex.nodetype.mark & 5\\
%    etex.nodetype.adjust & 6\\
%    etex.nodetype.ligature & 7\\
%    etex.nodetype.disc & 8\\
%    etex.nodetype.whatsit & 9\\
%    etex.nodetype.math & 10\\
%    etex.nodetype.glue & 11\\
%    etex.nodetype.kern & 12\\
%    etex.nodetype.penalty & 13\\
%    etex.nodetype.unset & 14\\
%    etex.nodetype.maths & 15\\
%  \end{tabular}
%  \end{quote}
%
%  \subsubsection{\texorpdfstring{Category }{}\texttt{etex.interactionmode}}
%
% \begin{quote}
% \begin{tabular}{@{}>{\ttfamily}l>{\ttfamily}l@{}}
%    etex.interactionmode.batch & 0\\
%    etex.interactionmode.nonstop & 1\\
%    etex.interactionmode.scroll & 2\\
%    etex.interactionmode.errorstop & 3\\
%  \end{tabular}
%  \end{quote}
%
%  \subsubsection{\texorpdfstring{Category }{}\texttt{luatex.pdfliteral.mode}}
%
% \begin{quote}
% \begin{tabular}{@{}>{\ttfamily}l>{\ttfamily}l@{}}
%    luatex.pdfliteral.mode.setorigin & 0\\
%    luatex.pdfliteral.mode.page & 1\\
%    luatex.pdfliteral.mode.direct & 2\\
%  \end{tabular}
%  \end{quote}
%
%
% \hypersetup{bookmarksopenlevel=1}
%
% \StopEventually{
% }
%
% \section{Implementation}
%
%    \begin{macrocode}
%<*package>
%    \end{macrocode}
%
% \subsection{Reload check and package identification}
%    Reload check, especially if the package is not used with \LaTeX.
%    \begin{macrocode}
\begingroup\catcode61\catcode48\catcode32=10\relax%
  \catcode13=5 % ^^M
  \endlinechar=13 %
  \catcode35=6 % #
  \catcode39=12 % '
  \catcode44=12 % ,
  \catcode45=12 % -
  \catcode46=12 % .
  \catcode58=12 % :
  \catcode64=11 % @
  \catcode123=1 % {
  \catcode125=2 % }
  \expandafter\let\expandafter\x\csname ver@magicnum.sty\endcsname
  \ifx\x\relax % plain-TeX, first loading
  \else
    \def\empty{}%
    \ifx\x\empty % LaTeX, first loading,
      % variable is initialized, but \ProvidesPackage not yet seen
    \else
      \expandafter\ifx\csname PackageInfo\endcsname\relax
        \def\x#1#2{%
          \immediate\write-1{Package #1 Info: #2.}%
        }%
      \else
        \def\x#1#2{\PackageInfo{#1}{#2, stopped}}%
      \fi
      \x{magicnum}{The package is already loaded}%
      \aftergroup\endinput
    \fi
  \fi
\endgroup%
%    \end{macrocode}
%    Package identification:
%    \begin{macrocode}
\begingroup\catcode61\catcode48\catcode32=10\relax%
  \catcode13=5 % ^^M
  \endlinechar=13 %
  \catcode35=6 % #
  \catcode39=12 % '
  \catcode40=12 % (
  \catcode41=12 % )
  \catcode44=12 % ,
  \catcode45=12 % -
  \catcode46=12 % .
  \catcode47=12 % /
  \catcode58=12 % :
  \catcode64=11 % @
  \catcode91=12 % [
  \catcode93=12 % ]
  \catcode123=1 % {
  \catcode125=2 % }
  \expandafter\ifx\csname ProvidesPackage\endcsname\relax
    \def\x#1#2#3[#4]{\endgroup
      \immediate\write-1{Package: #3 #4}%
      \xdef#1{#4}%
    }%
  \else
    \def\x#1#2[#3]{\endgroup
      #2[{#3}]%
      \ifx#1\@undefined
        \xdef#1{#3}%
      \fi
      \ifx#1\relax
        \xdef#1{#3}%
      \fi
    }%
  \fi
\expandafter\x\csname ver@magicnum.sty\endcsname
\ProvidesPackage{magicnum}%
  [2011/04/10 v1.4 Magic numbers (HO)]%
%    \end{macrocode}
%
% \subsection{Catcodes}
%
%    \begin{macrocode}
\begingroup\catcode61\catcode48\catcode32=10\relax%
  \catcode13=5 % ^^M
  \endlinechar=13 %
  \catcode123=1 % {
  \catcode125=2 % }
  \catcode64=11 % @
  \def\x{\endgroup
    \expandafter\edef\csname magicnum@AtEnd\endcsname{%
      \endlinechar=\the\endlinechar\relax
      \catcode13=\the\catcode13\relax
      \catcode32=\the\catcode32\relax
      \catcode35=\the\catcode35\relax
      \catcode61=\the\catcode61\relax
      \catcode64=\the\catcode64\relax
      \catcode123=\the\catcode123\relax
      \catcode125=\the\catcode125\relax
    }%
  }%
\x\catcode61\catcode48\catcode32=10\relax%
\catcode13=5 % ^^M
\endlinechar=13 %
\catcode35=6 % #
\catcode64=11 % @
\catcode123=1 % {
\catcode125=2 % }
\def\TMP@EnsureCode#1#2{%
  \edef\magicnum@AtEnd{%
    \magicnum@AtEnd
    \catcode#1=\the\catcode#1\relax
  }%
  \catcode#1=#2\relax
}
\TMP@EnsureCode{34}{12}% "
\TMP@EnsureCode{39}{12}% '
\TMP@EnsureCode{40}{12}% (
\TMP@EnsureCode{41}{12}% )
\TMP@EnsureCode{42}{12}% *
\TMP@EnsureCode{44}{12}% ,
\TMP@EnsureCode{45}{12}% -
\TMP@EnsureCode{46}{12}% .
\TMP@EnsureCode{47}{12}% /
\TMP@EnsureCode{58}{12}% :
\TMP@EnsureCode{60}{12}% <
\TMP@EnsureCode{62}{12}% >
\TMP@EnsureCode{91}{12}% [
\TMP@EnsureCode{93}{12}% ]
\edef\magicnum@AtEnd{\magicnum@AtEnd\noexpand\endinput}
%    \end{macrocode}
%
% \subsection{Check for previous definition}
%
%    \begin{macrocode}
\begingroup\expandafter\expandafter\expandafter\endgroup
\expandafter\ifx\csname newcommand\endcsname\relax
  \expandafter\ifx\csname magicnum\endcsname\relax
  \else
    \input infwarerr.sty\relax
    \@PackageError{magicnum}{%
      \string\magicnum\space is already defined%
    }\@ehc
  \fi
\else
  \newcommand*{\magicnum}{}%
\fi
%    \end{macrocode}
%
% \subsection{Without \LuaTeX}
%
%    \begin{macrocode}
\begingroup\expandafter\expandafter\expandafter\endgroup
\expandafter\ifx\csname directlua\endcsname\relax
%    \end{macrocode}
%
%    \begin{macro}{\magicnum}
%    \begin{macrocode}
  \begingroup\expandafter\expandafter\expandafter\endgroup
  \expandafter\ifx\csname ifcsname\endcsname\relax
    \def\magicnum#1{%
      \expandafter\ifx\csname MG@#1\endcsname\relax
      \else
        \csname MG@#1\endcsname
      \fi
    }%
  \else
    \begingroup
      \edef\x{\endgroup
        \def\noexpand\magicnum##1{%
          \expandafter\noexpand\csname
          ifcsname\endcsname MG@##1\noexpand\endcsname
            \noexpand\csname MG@##1%
                 \noexpand\expandafter\noexpand\endcsname
          \expandafter\noexpand\csname fi\endcsname
        }%
      }%
    \x
  \fi
%    \end{macrocode}
%    \end{macro}
%
%    \begin{macrocode}
\else
%    \end{macrocode}
%
% \subsection{With \LuaTeX}
%
%    \begin{macrocode}
  \begingroup\expandafter\expandafter\expandafter\endgroup
  \expandafter\ifx\csname RequirePackage\endcsname\relax
    \input ifluatex.sty\relax
    \input infwarerr.sty\relax
  \else
    \RequirePackage{ifluatex}[2010/03/01]%
    \RequirePackage{infwarerr}[2010/04/08]%
  \fi
%    \end{macrocode}
%
%    \begin{macro}{\magicnum@directlua}
%    \begin{macrocode}
  \ifnum\luatexversion<36 %
    \def\magicnum@directlua{\directlua0 }%
  \else
    \let\magicnum@directlua\directlua
  \fi
%    \end{macrocode}
%    \end{macro}
%    \begin{macrocode}
  \magicnum@directlua{%
    require("oberdiek.magicnum")%
  }%
  \begingroup
    \def\x{2011/04/10 v1.4}%
    \def\StripPrefix#1>{}%
    \edef\x{\expandafter\StripPrefix\meaning\x}%
    \edef\y{%
      \magicnum@directlua{%
        if oberdiek.magicnum.getversion then %
          oberdiek.magicnum.getversion()%
        end%
      }%
    }%
    \ifx\x\y
    \else
      \@PackageError{magicnum}{%
        Wrong version of lua module.\MessageBreak
        Package version: \x\MessageBreak
        Lua module: \y
      }\@ehc
    \fi
  \endgroup
%    \end{macrocode}
%    \begin{macro}{\luaescapestring}
%    \begin{macrocode}
  \begingroup
    \expandafter\ifx\csname luaescapestring\endcsname\relax
      \directlua{%
        if tex.enableprimitives then %
          tex.enableprimitives('magicnum@', {'luaescapestring'})%
        end%
      }%
      \global\let\luaescapestring\magicnum@luaescapestring
    \fi
    \expandafter\ifx\csname luaescapestring\endcsname\relax
      \escapechar=92 %
      \@PackageError{magicnum}{%
        Missing \string\luaescapestring
      }\@ehc
    \fi
  \endgroup
%    \end{macrocode}
%    \end{macro}
%    \begin{macro}{\magicnum}
%    \begin{macrocode}
  \def\magicnum#1{%
    \magicnum@directlua{%
      oberdiek.magicnum.get("\luaescapestring{#1}")%
    }%
  }%
%    \end{macrocode}
%    \end{macro}
%
%    \begin{macrocode}
  \expandafter\magicnum@AtEnd
\fi%
%</package>
%    \end{macrocode}
%
% \subsection{Data}
%
% \subsubsection{Plain data}
%
%    \begin{macrocode}
%<*data>
tex.catcode
  escape = 0
  begingroup = 1
  endgroup = 2
  math = 3
  align = 4
  eol = 5
  parameter = 6
  superscript = 7
  subscript = 8
  ignore = 9
  space = 10
  letter = 11
  other = 12
  active = 13
  comment = 14
  invalid = 15
etex.grouptype
  bottomlevel = 0
  simple = 1
  hbox = 2
  adjustedhbox = 3
  vbox = 4
  align = 5
  noalign = 6
  output = 8
  math = 9
  disc = 10
  insert = 11
  vcenter = 12
  mathchoice = 13
  semisimple = 14
  mathshift = 15
  mathleft = 16
etex.iftype
  none = 0
  char = 1
  cat = 2
  num = 3
  dim = 4
  odd = 5
  vmode = 6
  hmode = 7
  mmode = 8
  inner = 9
  void = 10
  hbox = 11
  vbox = 12
  x = 13
  eof = 14
  true = 15
  false = 16
  case = 17
  defined = 18
  csname = 19
  fontchar = 20
etex.nodetype
  none = -1
  char = 0
  hlist = 1
  vlist = 2
  rule = 3
  ins = 4
  mark = 5
  adjust = 6
  ligature = 7
  disc = 8
  whatsit = 9
  math = 10
  glue = 11
  kern = 12
  penalty = 13
  unset = 14
  maths = 15
etex.interactionmode
  batch = 0
  nonstop = 1
  scroll = 2
  errorstop = 3
luatex.pdfliteral.mode
  setorigin = 0
  page = 1
  direct = 2
%</data>
%    \end{macrocode}
%
% \subsubsection{Data for \TeX}
%
%    \begin{macrocode}
%<*package>
%    \end{macrocode}
%    \begin{macro}{\magicnum@add}
%    \begin{macrocode}
\begingroup\expandafter\expandafter\expandafter\endgroup
\expandafter\ifx\csname detokenize\endcsname\relax
  \def\magicnum@add#1#2#3{%
    \expandafter\magicnum@@add
        \csname MG@#1.#2\expandafter\endcsname
        \csname MG@#1.#3\endcsname
       {#3}{#2}%
  }%
  \def\magicnum@@add#1#2#3#4{%
    \def#1{#3}%
    \def#2{#4}%
    \edef#1{%
      \expandafter\strip@prefix\meaning#1%
    }%
    \edef#2{%
      \expandafter\strip@prefix\meaning#2%
    }%
  }%
  \expandafter\ifx\csname strip@prefix\endcsname\relax
    \def\strip@prefix#1->{}%
  \fi
\else
  \def\magicnum@add#1#2#3{%
    \expandafter\edef\csname MG@#1.#2\endcsname{%
      \detokenize{#3}%
    }%
    \expandafter\edef\csname MG@#1.#3\endcsname{%
      \detokenize{#2}%
    }%
  }%
\fi
%    \end{macrocode}
%    \end{macro}
%    \begin{macrocode}
\magicnum@add{tex.catcode}{escape}{0}
\magicnum@add{tex.catcode}{begingroup}{1}
\magicnum@add{tex.catcode}{endgroup}{2}
\magicnum@add{tex.catcode}{math}{3}
\magicnum@add{tex.catcode}{align}{4}
\magicnum@add{tex.catcode}{eol}{5}
\magicnum@add{tex.catcode}{parameter}{6}
\magicnum@add{tex.catcode}{superscript}{7}
\magicnum@add{tex.catcode}{subscript}{8}
\magicnum@add{tex.catcode}{ignore}{9}
\magicnum@add{tex.catcode}{space}{10}
\magicnum@add{tex.catcode}{letter}{11}
\magicnum@add{tex.catcode}{other}{12}
\magicnum@add{tex.catcode}{active}{13}
\magicnum@add{tex.catcode}{comment}{14}
\magicnum@add{tex.catcode}{invalid}{15}
\magicnum@add{etex.grouptype}{bottomlevel}{0}
\magicnum@add{etex.grouptype}{simple}{1}
\magicnum@add{etex.grouptype}{hbox}{2}
\magicnum@add{etex.grouptype}{adjustedhbox}{3}
\magicnum@add{etex.grouptype}{vbox}{4}
\magicnum@add{etex.grouptype}{align}{5}
\magicnum@add{etex.grouptype}{noalign}{6}
\magicnum@add{etex.grouptype}{output}{8}
\magicnum@add{etex.grouptype}{math}{9}
\magicnum@add{etex.grouptype}{disc}{10}
\magicnum@add{etex.grouptype}{insert}{11}
\magicnum@add{etex.grouptype}{vcenter}{12}
\magicnum@add{etex.grouptype}{mathchoice}{13}
\magicnum@add{etex.grouptype}{semisimple}{14}
\magicnum@add{etex.grouptype}{mathshift}{15}
\magicnum@add{etex.grouptype}{mathleft}{16}
\magicnum@add{etex.iftype}{none}{0}
\magicnum@add{etex.iftype}{char}{1}
\magicnum@add{etex.iftype}{cat}{2}
\magicnum@add{etex.iftype}{num}{3}
\magicnum@add{etex.iftype}{dim}{4}
\magicnum@add{etex.iftype}{odd}{5}
\magicnum@add{etex.iftype}{vmode}{6}
\magicnum@add{etex.iftype}{hmode}{7}
\magicnum@add{etex.iftype}{mmode}{8}
\magicnum@add{etex.iftype}{inner}{9}
\magicnum@add{etex.iftype}{void}{10}
\magicnum@add{etex.iftype}{hbox}{11}
\magicnum@add{etex.iftype}{vbox}{12}
\magicnum@add{etex.iftype}{x}{13}
\magicnum@add{etex.iftype}{eof}{14}
\magicnum@add{etex.iftype}{true}{15}
\magicnum@add{etex.iftype}{false}{16}
\magicnum@add{etex.iftype}{case}{17}
\magicnum@add{etex.iftype}{defined}{18}
\magicnum@add{etex.iftype}{csname}{19}
\magicnum@add{etex.iftype}{fontchar}{20}
\magicnum@add{etex.nodetype}{none}{-1}
\magicnum@add{etex.nodetype}{char}{0}
\magicnum@add{etex.nodetype}{hlist}{1}
\magicnum@add{etex.nodetype}{vlist}{2}
\magicnum@add{etex.nodetype}{rule}{3}
\magicnum@add{etex.nodetype}{ins}{4}
\magicnum@add{etex.nodetype}{mark}{5}
\magicnum@add{etex.nodetype}{adjust}{6}
\magicnum@add{etex.nodetype}{ligature}{7}
\magicnum@add{etex.nodetype}{disc}{8}
\magicnum@add{etex.nodetype}{whatsit}{9}
\magicnum@add{etex.nodetype}{math}{10}
\magicnum@add{etex.nodetype}{glue}{11}
\magicnum@add{etex.nodetype}{kern}{12}
\magicnum@add{etex.nodetype}{penalty}{13}
\magicnum@add{etex.nodetype}{unset}{14}
\magicnum@add{etex.nodetype}{maths}{15}
\magicnum@add{etex.interactionmode}{batch}{0}
\magicnum@add{etex.interactionmode}{nonstop}{1}
\magicnum@add{etex.interactionmode}{scroll}{2}
\magicnum@add{etex.interactionmode}{errorstop}{3}
\magicnum@add{luatex.pdfliteral.mode}{setorigin}{0}
\magicnum@add{luatex.pdfliteral.mode}{page}{1}
\magicnum@add{luatex.pdfliteral.mode}{direct}{2}
%    \end{macrocode}
%    \begin{macrocode}
\magicnum@AtEnd%
%</package>
%    \end{macrocode}
%
% \subsubsection{Lua module}
%
%    \begin{macrocode}
%<*lua>
%    \end{macrocode}
%    \begin{macrocode}
module("oberdiek.magicnum", package.seeall)
%    \end{macrocode}
%    \begin{macrocode}
function getversion()
  tex.write("2011/04/10 v1.4")
end
%    \end{macrocode}
%    \begin{macrocode}
local data = {
  ["tex.catcode"] = {
    [0] = "escape",
    [1] = "begingroup",
    [2] = "endgroup",
    [3] = "math",
    [4] = "align",
    [5] = "eol",
    [6] = "parameter",
    [7] = "superscript",
    [8] = "subscript",
    [9] = "ignore",
    [10] = "space",
    [11] = "letter",
    [12] = "other",
    [13] = "active",
    [14] = "comment",
    [15] = "invalid",
    ["active"] = 13,
    ["align"] = 4,
    ["begingroup"] = 1,
    ["comment"] = 14,
    ["endgroup"] = 2,
    ["eol"] = 5,
    ["escape"] = 0,
    ["ignore"] = 9,
    ["invalid"] = 15,
    ["letter"] = 11,
    ["math"] = 3,
    ["other"] = 12,
    ["parameter"] = 6,
    ["space"] = 10,
    ["subscript"] = 8,
    ["superscript"] = 7
  },
  ["etex.grouptype"] = {
    [0] = "bottomlevel",
    [1] = "simple",
    [2] = "hbox",
    [3] = "adjustedhbox",
    [4] = "vbox",
    [5] = "align",
    [6] = "noalign",
    [8] = "output",
    [9] = "math",
    [10] = "disc",
    [11] = "insert",
    [12] = "vcenter",
    [13] = "mathchoice",
    [14] = "semisimple",
    [15] = "mathshift",
    [16] = "mathleft",
    ["adjustedhbox"] = 3,
    ["align"] = 5,
    ["bottomlevel"] = 0,
    ["disc"] = 10,
    ["hbox"] = 2,
    ["insert"] = 11,
    ["math"] = 9,
    ["mathchoice"] = 13,
    ["mathleft"] = 16,
    ["mathshift"] = 15,
    ["noalign"] = 6,
    ["output"] = 8,
    ["semisimple"] = 14,
    ["simple"] = 1,
    ["vbox"] = 4,
    ["vcenter"] = 12
  },
  ["etex.iftype"] = {
    [0] = "none",
    [1] = "char",
    [2] = "cat",
    [3] = "num",
    [4] = "dim",
    [5] = "odd",
    [6] = "vmode",
    [7] = "hmode",
    [8] = "mmode",
    [9] = "inner",
    [10] = "void",
    [11] = "hbox",
    [12] = "vbox",
    [13] = "x",
    [14] = "eof",
    [15] = "true",
    [16] = "false",
    [17] = "case",
    [18] = "defined",
    [19] = "csname",
    [20] = "fontchar",
    ["case"] = 17,
    ["cat"] = 2,
    ["char"] = 1,
    ["csname"] = 19,
    ["defined"] = 18,
    ["dim"] = 4,
    ["eof"] = 14,
    ["false"] = 16,
    ["fontchar"] = 20,
    ["hbox"] = 11,
    ["hmode"] = 7,
    ["inner"] = 9,
    ["mmode"] = 8,
    ["none"] = 0,
    ["num"] = 3,
    ["odd"] = 5,
    ["true"] = 15,
    ["vbox"] = 12,
    ["vmode"] = 6,
    ["void"] = 10,
    ["x"] = 13
  },
  ["etex.nodetype"] = {
    [-1] = "none",
    [0] = "char",
    [1] = "hlist",
    [2] = "vlist",
    [3] = "rule",
    [4] = "ins",
    [5] = "mark",
    [6] = "adjust",
    [7] = "ligature",
    [8] = "disc",
    [9] = "whatsit",
    [10] = "math",
    [11] = "glue",
    [12] = "kern",
    [13] = "penalty",
    [14] = "unset",
    [15] = "maths",
    ["adjust"] = 6,
    ["char"] = 0,
    ["disc"] = 8,
    ["glue"] = 11,
    ["hlist"] = 1,
    ["ins"] = 4,
    ["kern"] = 12,
    ["ligature"] = 7,
    ["mark"] = 5,
    ["math"] = 10,
    ["maths"] = 15,
    ["none"] = -1,
    ["penalty"] = 13,
    ["rule"] = 3,
    ["unset"] = 14,
    ["vlist"] = 2,
    ["whatsit"] = 9
  },
  ["etex.interactionmode"] = {
    [0] = "batch",
    [1] = "nonstop",
    [2] = "scroll",
    [3] = "errorstop",
    ["batch"] = 0,
    ["errorstop"] = 3,
    ["nonstop"] = 1,
    ["scroll"] = 2
  },
  ["luatex.pdfliteral.mode"] = {
    [0] = "setorigin",
    [1] = "page",
    [2] = "direct",
    ["direct"] = 2,
    ["page"] = 1,
    ["setorigin"] = 0
  }
}
%    \end{macrocode}
%    \begin{macrocode}
function get(name)
  local startpos, endpos, category, entry =
      string.find(name, "^(%a[%a%d%.]*)%.(-?[%a%d]+)$")
  if not entry then
    return
  end
  local node = data[category]
  if not node then
    return
  end
  local num = tonumber(entry)
  local value
  if num then
    value = node[num]
    if not value then
      return
    end
  else
    value = node[entry]
    if not value then
      return
    end
    value = "" .. value
  end
  tex.write(value)
end
%    \end{macrocode}
%
%    \begin{macrocode}
%</lua>
%    \end{macrocode}
%
% \section{Test}
%
% \subsection{Catcode checks for loading}
%
%    \begin{macrocode}
%<*test1>
%    \end{macrocode}
%    \begin{macrocode}
\catcode`\{=1 %
\catcode`\}=2 %
\catcode`\#=6 %
\catcode`\@=11 %
\expandafter\ifx\csname count@\endcsname\relax
  \countdef\count@=255 %
\fi
\expandafter\ifx\csname @gobble\endcsname\relax
  \long\def\@gobble#1{}%
\fi
\expandafter\ifx\csname @firstofone\endcsname\relax
  \long\def\@firstofone#1{#1}%
\fi
\expandafter\ifx\csname loop\endcsname\relax
  \expandafter\@firstofone
\else
  \expandafter\@gobble
\fi
{%
  \def\loop#1\repeat{%
    \def\body{#1}%
    \iterate
  }%
  \def\iterate{%
    \body
      \let\next\iterate
    \else
      \let\next\relax
    \fi
    \next
  }%
  \let\repeat=\fi
}%
\def\RestoreCatcodes{}
\count@=0 %
\loop
  \edef\RestoreCatcodes{%
    \RestoreCatcodes
    \catcode\the\count@=\the\catcode\count@\relax
  }%
\ifnum\count@<255 %
  \advance\count@ 1 %
\repeat

\def\RangeCatcodeInvalid#1#2{%
  \count@=#1\relax
  \loop
    \catcode\count@=15 %
  \ifnum\count@<#2\relax
    \advance\count@ 1 %
  \repeat
}
\def\RangeCatcodeCheck#1#2#3{%
  \count@=#1\relax
  \loop
    \ifnum#3=\catcode\count@
    \else
      \errmessage{%
        Character \the\count@\space
        with wrong catcode \the\catcode\count@\space
        instead of \number#3%
      }%
    \fi
  \ifnum\count@<#2\relax
    \advance\count@ 1 %
  \repeat
}
\def\space{ }
\expandafter\ifx\csname LoadCommand\endcsname\relax
  \def\LoadCommand{\input magicnum.sty\relax}%
\fi
\def\Test{%
  \RangeCatcodeInvalid{0}{47}%
  \RangeCatcodeInvalid{58}{64}%
  \RangeCatcodeInvalid{91}{96}%
  \RangeCatcodeInvalid{123}{255}%
  \catcode`\@=12 %
  \catcode`\\=0 %
  \catcode`\%=14 %
  \LoadCommand
  \RangeCatcodeCheck{0}{36}{15}%
  \RangeCatcodeCheck{37}{37}{14}%
  \RangeCatcodeCheck{38}{47}{15}%
  \RangeCatcodeCheck{48}{57}{12}%
  \RangeCatcodeCheck{58}{63}{15}%
  \RangeCatcodeCheck{64}{64}{12}%
  \RangeCatcodeCheck{65}{90}{11}%
  \RangeCatcodeCheck{91}{91}{15}%
  \RangeCatcodeCheck{92}{92}{0}%
  \RangeCatcodeCheck{93}{96}{15}%
  \RangeCatcodeCheck{97}{122}{11}%
  \RangeCatcodeCheck{123}{255}{15}%
  \RestoreCatcodes
}
\Test
\csname @@end\endcsname
\end
%    \end{macrocode}
%    \begin{macrocode}
%</test1>
%    \end{macrocode}
%
% \subsection{Test data}
%
%    \begin{macrocode}
%<*testplain>
\input magicnum.sty\relax
\def\Test#1#2{%
  \edef\result{\magicnum{#1}}%
  \edef\expect{#2}%
  \edef\expect{\expandafter\stripprefix\meaning\expect}%
  \ifx\result\expect
  \else
    \errmessage{%
      Failed: [#1] % hash-ok
      returns [\result] instead of [\expect]%
    }%
  \fi
}
\def\stripprefix#1->{}
%</testplain>
%    \end{macrocode}
%    \begin{macrocode}
%<*testlatex>
\NeedsTeXFormat{LaTeX2e}
\documentclass{minimal}
\usepackage{magicnum}[2011/04/10]
\usepackage{qstest}
\IncludeTests{*}
\LogTests{log}{*}{*}
\newcommand*{\Test}[2]{%
  \Expect*{\magicnum{#1}}{#2}%
}
\begin{qstest}{magicnum}{magicnum}
%</testlatex>
%    \end{macrocode}
%    \begin{macrocode}
%<*testdata>
\Test{tex.catcode.escape}{0}
\Test{tex.catcode.invalid}{15}
\Test{tex.catcode.unknown}{}
\Test{tex.catcode.0}{escape}
\Test{tex.catcode.15}{invalid}
\Test{etex.iftype.true}{15}
\Test{etex.iftype.false}{16}
\Test{etex.iftype.15}{true}
\Test{etex.iftype.16}{false}
\Test{etex.nodetype.none}{-1}
\Test{etex.nodetype.-1}{none}
\Test{luatex.pdfliteral.mode.direct}{2}
\Test{luatex.pdfliteral.mode.1}{page}
\Test{}{}
\Test{unknown}{}
\Test{unknown.foo.bar}{}
\Test{unknown.foo.4}{}
%</testdata>
%    \end{macrocode}
%    \begin{macrocode}
%<*testplain>
\csname @@end\endcsname
\end
%</testplain>
%<*testlatex>
\end{qstest}
\csname @@end\endcsname
%</testlatex>
%    \end{macrocode}
%
% \subsection{Small test for \hologo{iniTeX}}
%
%    \begin{macrocode}
%<*test4>
\catcode`\{=1
\catcode`\}=2
\catcode`\#=6
\input magicnum.sty\relax
\edef\x{\magicnum{tex.catcode.15}}
\edef\y{invalid}
\def\Strip#1>{}
\edef\y{\expandafter\Strip\meaning\y}
\ifx\x\y
  \immediate\write16{Ok}%
\else
  \errmessage{\x<>\y}%
\fi
\csname @@end\endcsname\end
%</test4>
%    \end{macrocode}
%
% \section{Installation}
%
% \subsection{Download}
%
% \paragraph{Package.} This package is available on
% CTAN\footnote{\url{ftp://ftp.ctan.org/tex-archive/}}:
% \begin{description}
% \item[\CTAN{macros/latex/contrib/oberdiek/magicnum.dtx}] The source file.
% \item[\CTAN{macros/latex/contrib/oberdiek/magicnum.pdf}] Documentation.
% \end{description}
%
%
% \paragraph{Bundle.} All the packages of the bundle `oberdiek'
% are also available in a TDS compliant ZIP archive. There
% the packages are already unpacked and the documentation files
% are generated. The files and directories obey the TDS standard.
% \begin{description}
% \item[\CTAN{install/macros/latex/contrib/oberdiek.tds.zip}]
% \end{description}
% \emph{TDS} refers to the standard ``A Directory Structure
% for \TeX\ Files'' (\CTAN{tds/tds.pdf}). Directories
% with \xfile{texmf} in their name are usually organized this way.
%
% \subsection{Bundle installation}
%
% \paragraph{Unpacking.} Unpack the \xfile{oberdiek.tds.zip} in the
% TDS tree (also known as \xfile{texmf} tree) of your choice.
% Example (linux):
% \begin{quote}
%   |unzip oberdiek.tds.zip -d ~/texmf|
% \end{quote}
%
% \paragraph{Script installation.}
% Check the directory \xfile{TDS:scripts/oberdiek/} for
% scripts that need further installation steps.
% Package \xpackage{attachfile2} comes with the Perl script
% \xfile{pdfatfi.pl} that should be installed in such a way
% that it can be called as \texttt{pdfatfi}.
% Example (linux):
% \begin{quote}
%   |chmod +x scripts/oberdiek/pdfatfi.pl|\\
%   |cp scripts/oberdiek/pdfatfi.pl /usr/local/bin/|
% \end{quote}
%
% \subsection{Package installation}
%
% \paragraph{Unpacking.} The \xfile{.dtx} file is a self-extracting
% \docstrip\ archive. The files are extracted by running the
% \xfile{.dtx} through \plainTeX:
% \begin{quote}
%   \verb|tex magicnum.dtx|
% \end{quote}
%
% \paragraph{TDS.} Now the different files must be moved into
% the different directories in your installation TDS tree
% (also known as \xfile{texmf} tree):
% \begin{quote}
% \def\t{^^A
% \begin{tabular}{@{}>{\ttfamily}l@{ $\rightarrow$ }>{\ttfamily}l@{}}
%   magicnum.sty & tex/generic/oberdiek/magicnum.sty\\
%   magicnum.lua & scripts/oberdiek/magicnum.lua\\
%   oberdiek.magicnum.lua & scripts/oberdiek/oberdiek.magicnum.lua\\
%   magicnum.pdf & doc/latex/oberdiek/magicnum.pdf\\
%   magicnum.txt & doc/latex/oberdiek/magicnum.txt\\
%   test/magicnum-test1.tex & doc/latex/oberdiek/test/magicnum-test1.tex\\
%   test/magicnum-test2.tex & doc/latex/oberdiek/test/magicnum-test2.tex\\
%   test/magicnum-test3.tex & doc/latex/oberdiek/test/magicnum-test3.tex\\
%   test/magicnum-test4.tex & doc/latex/oberdiek/test/magicnum-test4.tex\\
%   magicnum.dtx & source/latex/oberdiek/magicnum.dtx\\
% \end{tabular}^^A
% }^^A
% \sbox0{\t}^^A
% \ifdim\wd0>\linewidth
%   \begingroup
%     \advance\linewidth by\leftmargin
%     \advance\linewidth by\rightmargin
%   \edef\x{\endgroup
%     \def\noexpand\lw{\the\linewidth}^^A
%   }\x
%   \def\lwbox{^^A
%     \leavevmode
%     \hbox to \linewidth{^^A
%       \kern-\leftmargin\relax
%       \hss
%       \usebox0
%       \hss
%       \kern-\rightmargin\relax
%     }^^A
%   }^^A
%   \ifdim\wd0>\lw
%     \sbox0{\small\t}^^A
%     \ifdim\wd0>\linewidth
%       \ifdim\wd0>\lw
%         \sbox0{\footnotesize\t}^^A
%         \ifdim\wd0>\linewidth
%           \ifdim\wd0>\lw
%             \sbox0{\scriptsize\t}^^A
%             \ifdim\wd0>\linewidth
%               \ifdim\wd0>\lw
%                 \sbox0{\tiny\t}^^A
%                 \ifdim\wd0>\linewidth
%                   \lwbox
%                 \else
%                   \usebox0
%                 \fi
%               \else
%                 \lwbox
%               \fi
%             \else
%               \usebox0
%             \fi
%           \else
%             \lwbox
%           \fi
%         \else
%           \usebox0
%         \fi
%       \else
%         \lwbox
%       \fi
%     \else
%       \usebox0
%     \fi
%   \else
%     \lwbox
%   \fi
% \else
%   \usebox0
% \fi
% \end{quote}
% If you have a \xfile{docstrip.cfg} that configures and enables \docstrip's
% TDS installing feature, then some files can already be in the right
% place, see the documentation of \docstrip.
%
% \subsection{Refresh file name databases}
%
% If your \TeX~distribution
% (\teTeX, \mikTeX, \dots) relies on file name databases, you must refresh
% these. For example, \teTeX\ users run \verb|texhash| or
% \verb|mktexlsr|.
%
% \subsection{Some details for the interested}
%
% \paragraph{Attached source.}
%
% The PDF documentation on CTAN also includes the
% \xfile{.dtx} source file. It can be extracted by
% AcrobatReader 6 or higher. Another option is \textsf{pdftk},
% e.g. unpack the file into the current directory:
% \begin{quote}
%   \verb|pdftk magicnum.pdf unpack_files output .|
% \end{quote}
%
% \paragraph{Unpacking with \LaTeX.}
% The \xfile{.dtx} chooses its action depending on the format:
% \begin{description}
% \item[\plainTeX:] Run \docstrip\ and extract the files.
% \item[\LaTeX:] Generate the documentation.
% \end{description}
% If you insist on using \LaTeX\ for \docstrip\ (really,
% \docstrip\ does not need \LaTeX), then inform the autodetect routine
% about your intention:
% \begin{quote}
%   \verb|latex \let\install=y\input{magicnum.dtx}|
% \end{quote}
% Do not forget to quote the argument according to the demands
% of your shell.
%
% \paragraph{Generating the documentation.}
% You can use both the \xfile{.dtx} or the \xfile{.drv} to generate
% the documentation. The process can be configured by the
% configuration file \xfile{ltxdoc.cfg}. For instance, put this
% line into this file, if you want to have A4 as paper format:
% \begin{quote}
%   \verb|\PassOptionsToClass{a4paper}{article}|
% \end{quote}
% An example follows how to generate the
% documentation with pdf\LaTeX:
% \begin{quote}
%\begin{verbatim}
%pdflatex magicnum.dtx
%makeindex -s gind.ist magicnum.idx
%pdflatex magicnum.dtx
%makeindex -s gind.ist magicnum.idx
%pdflatex magicnum.dtx
%\end{verbatim}
% \end{quote}
%
% \section{Catalogue}
%
% The following XML file can be used as source for the
% \href{http://mirror.ctan.org/help/Catalogue/catalogue.html}{\TeX\ Catalogue}.
% The elements \texttt{caption} and \texttt{description} are imported
% from the original XML file from the Catalogue.
% The name of the XML file in the Catalogue is \xfile{magicnum.xml}.
%    \begin{macrocode}
%<*catalogue>
<?xml version='1.0' encoding='us-ascii'?>
<!DOCTYPE entry SYSTEM 'catalogue.dtd'>
<entry datestamp='$Date$' modifier='$Author$' id='magicnum'>
  <name>magicnum</name>
  <caption>Access TeX systems' "magic numbers".</caption>
  <authorref id='auth:oberdiek'/>
  <copyright owner='Heiko Oberdiek' year='2007,2009-2011'/>
  <license type='lppl1.3'/>
  <version number='1.4'/>
  <description>
    This package allows access to the various parameter values in
    TeX (catcode values), e-TeX (group, if and node types, and
    interaction mode), and LuaTeX (pdfliteral mode) by a hierarchical
    name system.
    <p/>
    The package is part of the <xref refid='oberdiek'>oberdiek</xref> bundle.
  </description>
  <documentation details='Package documentation'
      href='ctan:/macros/latex/contrib/oberdiek/magicnum.pdf'/>
  <ctan file='true' path='/macros/latex/contrib/oberdiek/magicnum.dtx'/>
  <miktex location='oberdiek'/>
  <texlive location='oberdiek'/>
  <install path='/macros/latex/contrib/oberdiek/oberdiek.tds.zip'/>
</entry>
%</catalogue>
%    \end{macrocode}
%
% \begin{History}
%   \begin{Version}{2007/12/12 v1.0}
%   \item
%     First public version.
%   \end{Version}
%   \begin{Version}{2009/04/10 v1.1}
%   \item
%     Adaptation to \LuaTeX\ 0.40.
%   \end{Version}
%   \begin{Version}{2010/03/09 v1.2}
%   \item
%     Adaptation to package \xpackage{luatex} 0.4.
%   \end{Version}
%   \begin{Version}{2011/03/24 v1.3}
%   \item
%     Catcode fixes.
%   \end{Version}
%   \begin{Version}{2011/04/10 v1.4}
%   \item
%     Compatibility for \hologo{iniTeX}.
%   \item
%     Dependency from package \xpackage{luatex} removed.
%   \item
%     Version check for lua module.
%   \end{Version}
% \end{History}
%
% \PrintIndex
%
% \Finale
\endinput
|
% \end{quote}
% Do not forget to quote the argument according to the demands
% of your shell.
%
% \paragraph{Generating the documentation.}
% You can use both the \xfile{.dtx} or the \xfile{.drv} to generate
% the documentation. The process can be configured by the
% configuration file \xfile{ltxdoc.cfg}. For instance, put this
% line into this file, if you want to have A4 as paper format:
% \begin{quote}
%   \verb|\PassOptionsToClass{a4paper}{article}|
% \end{quote}
% An example follows how to generate the
% documentation with pdf\LaTeX:
% \begin{quote}
%\begin{verbatim}
%pdflatex magicnum.dtx
%makeindex -s gind.ist magicnum.idx
%pdflatex magicnum.dtx
%makeindex -s gind.ist magicnum.idx
%pdflatex magicnum.dtx
%\end{verbatim}
% \end{quote}
%
% \section{Catalogue}
%
% The following XML file can be used as source for the
% \href{http://mirror.ctan.org/help/Catalogue/catalogue.html}{\TeX\ Catalogue}.
% The elements \texttt{caption} and \texttt{description} are imported
% from the original XML file from the Catalogue.
% The name of the XML file in the Catalogue is \xfile{magicnum.xml}.
%    \begin{macrocode}
%<*catalogue>
<?xml version='1.0' encoding='us-ascii'?>
<!DOCTYPE entry SYSTEM 'catalogue.dtd'>
<entry datestamp='$Date$' modifier='$Author$' id='magicnum'>
  <name>magicnum</name>
  <caption>Access TeX systems' "magic numbers".</caption>
  <authorref id='auth:oberdiek'/>
  <copyright owner='Heiko Oberdiek' year='2007,2009-2011'/>
  <license type='lppl1.3'/>
  <version number='1.4'/>
  <description>
    This package allows access to the various parameter values in
    TeX (catcode values), e-TeX (group, if and node types, and
    interaction mode), and LuaTeX (pdfliteral mode) by a hierarchical
    name system.
    <p/>
    The package is part of the <xref refid='oberdiek'>oberdiek</xref> bundle.
  </description>
  <documentation details='Package documentation'
      href='ctan:/macros/latex/contrib/oberdiek/magicnum.pdf'/>
  <ctan file='true' path='/macros/latex/contrib/oberdiek/magicnum.dtx'/>
  <miktex location='oberdiek'/>
  <texlive location='oberdiek'/>
  <install path='/macros/latex/contrib/oberdiek/oberdiek.tds.zip'/>
</entry>
%</catalogue>
%    \end{macrocode}
%
% \begin{History}
%   \begin{Version}{2007/12/12 v1.0}
%   \item
%     First public version.
%   \end{Version}
%   \begin{Version}{2009/04/10 v1.1}
%   \item
%     Adaptation to \LuaTeX\ 0.40.
%   \end{Version}
%   \begin{Version}{2010/03/09 v1.2}
%   \item
%     Adaptation to package \xpackage{luatex} 0.4.
%   \end{Version}
%   \begin{Version}{2011/03/24 v1.3}
%   \item
%     Catcode fixes.
%   \end{Version}
%   \begin{Version}{2011/04/10 v1.4}
%   \item
%     Compatibility for \hologo{iniTeX}.
%   \item
%     Dependency from package \xpackage{luatex} removed.
%   \item
%     Version check for lua module.
%   \end{Version}
% \end{History}
%
% \PrintIndex
%
% \Finale
\endinput

%        (quote the arguments according to the demands of your shell)
%
% Documentation:
%    (a) If magicnum.drv is present:
%           latex magicnum.drv
%    (b) Without magicnum.drv:
%           latex magicnum.dtx; ...
%    The class ltxdoc loads the configuration file ltxdoc.cfg
%    if available. Here you can specify further options, e.g.
%    use A4 as paper format:
%       \PassOptionsToClass{a4paper}{article}
%
%    Programm calls to get the documentation (example):
%       pdflatex magicnum.dtx
%       makeindex -s gind.ist magicnum.idx
%       pdflatex magicnum.dtx
%       makeindex -s gind.ist magicnum.idx
%       pdflatex magicnum.dtx
%
% Installation:
%    TDS:tex/generic/oberdiek/magicnum.sty
%    TDS:scripts/oberdiek/magicnum.lua
%    TDS:scripts/oberdiek/oberdiek.magicnum.lua
%    TDS:doc/latex/oberdiek/magicnum.pdf
%    TDS:doc/latex/oberdiek/magicnum.txt
%    TDS:doc/latex/oberdiek/test/magicnum-test1.tex
%    TDS:doc/latex/oberdiek/test/magicnum-test2.tex
%    TDS:doc/latex/oberdiek/test/magicnum-test3.tex
%    TDS:doc/latex/oberdiek/test/magicnum-test4.tex
%    TDS:source/latex/oberdiek/magicnum.dtx
%
%<*ignore>
\begingroup
  \catcode123=1 %
  \catcode125=2 %
  \def\x{LaTeX2e}%
\expandafter\endgroup
\ifcase 0\ifx\install y1\fi\expandafter
         \ifx\csname processbatchFile\endcsname\relax\else1\fi
         \ifx\fmtname\x\else 1\fi\relax
\else\csname fi\endcsname
%</ignore>
%<*install>
\input docstrip.tex
\Msg{************************************************************************}
\Msg{* Installation}
\Msg{* Package: magicnum 2011/04/10 v1.4 Magic numbers (HO)}
\Msg{************************************************************************}

\keepsilent
\askforoverwritefalse

\let\MetaPrefix\relax
\preamble

This is a generated file.

Project: magicnum
Version: 2011/04/10 v1.4

Copyright (C) 2007, 2009-2011 by
   Heiko Oberdiek <heiko.oberdiek at googlemail.com>

This work may be distributed and/or modified under the
conditions of the LaTeX Project Public License, either
version 1.3c of this license or (at your option) any later
version. This version of this license is in
   http://www.latex-project.org/lppl/lppl-1-3c.txt
and the latest version of this license is in
   http://www.latex-project.org/lppl.txt
and version 1.3 or later is part of all distributions of
LaTeX version 2005/12/01 or later.

This work has the LPPL maintenance status "maintained".

This Current Maintainer of this work is Heiko Oberdiek.

The Base Interpreter refers to any `TeX-Format',
because some files are installed in TDS:tex/generic//.

This work consists of the main source file magicnum.dtx
and the derived files
   magicnum.sty, magicnum.pdf, magicnum.ins, magicnum.drv, magicnum.txt,
   magicnum-test1.tex, magicnum-test2.tex, magicnum-test3.tex,
   magicnum-test4.tex, magicnum.lua, oberdiek.magicnum.lua.

\endpreamble
\let\MetaPrefix\DoubleperCent

\generate{%
  \file{magicnum.ins}{\from{magicnum.dtx}{install}}%
  \file{magicnum.drv}{\from{magicnum.dtx}{driver}}%
  \usedir{tex/generic/oberdiek}%
  \file{magicnum.sty}{\from{magicnum.dtx}{package}}%
  \usedir{doc/latex/oberdiek/test}%
  \file{magicnum-test1.tex}{\from{magicnum.dtx}{test1}}%
  \file{magicnum-test2.tex}{\from{magicnum.dtx}{testplain,testdata}}%
  \file{magicnum-test3.tex}{\from{magicnum.dtx}{testlatex,testdata}}%
  \file{magicnum-test4.tex}{\from{magicnum.dtx}{test4}}%
  \nopreamble
  \nopostamble
  \usedir{doc/latex/oberdiek}%
  \file{magicnum.txt}{\from{magicnum.dtx}{data}}%
  \usedir{source/latex/oberdiek/catalogue}%
  \file{magicnum.xml}{\from{magicnum.dtx}{catalogue}}%
}
\def\MetaPrefix{-- }
\def\defaultpostamble{%
  \MetaPrefix^^J%
  \MetaPrefix\space End of File `\outFileName'.%
}
\def\currentpostamble{\defaultpostamble}%
\generate{%
  \usedir{scripts/oberdiek}%
  \file{magicnum.lua}{\from{magicnum.dtx}{lua}}%
  \file{oberdiek.magicnum.lua}{\from{magicnum.dtx}{lua}}%
}

\catcode32=13\relax% active space
\let =\space%
\Msg{************************************************************************}
\Msg{*}
\Msg{* To finish the installation you have to move the following}
\Msg{* file into a directory searched by TeX:}
\Msg{*}
\Msg{*     magicnum.sty}
\Msg{*}
\Msg{* And install the following script files:}
\Msg{*}
\Msg{*     magicnum.lua, oberdiek.magicnum.lua}
\Msg{*}
\Msg{* To produce the documentation run the file `magicnum.drv'}
\Msg{* through LaTeX.}
\Msg{*}
\Msg{* Happy TeXing!}
\Msg{*}
\Msg{************************************************************************}

\endbatchfile
%</install>
%<*ignore>
\fi
%</ignore>
%<*driver>
\NeedsTeXFormat{LaTeX2e}
\ProvidesFile{magicnum.drv}%
  [2011/04/10 v1.4 Magic numbers (HO)]%
\documentclass{ltxdoc}
\usepackage{holtxdoc}[2011/11/22]
\usepackage{array}
\begin{document}
  \DocInput{magicnum.dtx}%
\end{document}
%</driver>
% \fi
%
% \CheckSum{755}
%
% \CharacterTable
%  {Upper-case    \A\B\C\D\E\F\G\H\I\J\K\L\M\N\O\P\Q\R\S\T\U\V\W\X\Y\Z
%   Lower-case    \a\b\c\d\e\f\g\h\i\j\k\l\m\n\o\p\q\r\s\t\u\v\w\x\y\z
%   Digits        \0\1\2\3\4\5\6\7\8\9
%   Exclamation   \!     Double quote  \"     Hash (number) \#
%   Dollar        \$     Percent       \%     Ampersand     \&
%   Acute accent  \'     Left paren    \(     Right paren   \)
%   Asterisk      \*     Plus          \+     Comma         \,
%   Minus         \-     Point         \.     Solidus       \/
%   Colon         \:     Semicolon     \;     Less than     \<
%   Equals        \=     Greater than  \>     Question mark \?
%   Commercial at \@     Left bracket  \[     Backslash     \\
%   Right bracket \]     Circumflex    \^     Underscore    \_
%   Grave accent  \`     Left brace    \{     Vertical bar  \|
%   Right brace   \}     Tilde         \~}
%
% \GetFileInfo{magicnum.drv}
%
% \title{The \xpackage{magicnum} package}
% \date{2011/04/10 v1.4}
% \author{Heiko Oberdiek\\\xemail{heiko.oberdiek at googlemail.com}}
%
% \maketitle
%
% \begin{abstract}
% This packages allows to access magic numbers by a hierarchical
% name system.
% \end{abstract}
%
% \tableofcontents
%
% \hypersetup{bookmarksopenlevel=2}
% \section{Documentation}
%
% \subsection{Introduction}
%
% Especially since \eTeX\ there are many integer values
% with special meanings, such as catcodes, group types, \dots
% Package \xpackage{etex}, enabled by options, defines
% macros in the user namespace for these values.
%
% This package goes another approach for storing the names and values.
% \begin{itemize}
% \item If \LuaTeX\ is available, they
% are stored in Lua tables.
% \item Without \LuaTeX\ they are remembered using internal
% macros.
% \end{itemize}
%
% \subsection{User interface}
%
% The integer values and names are organized in a hierarchical
% scheme of categories with the property names as leaves.
% Example: \eTeX's \cs{currentgrouplevel} reports |2| for a
% group caused by \cs{hbox}. This package has choosen to organize
% the group types in a main category |etex| and its subcategory
% |grouptype|:
% \begin{quote}
%   |etex.grouptype.hbox| = |2|
% \end{quote}
% The property name |hbox| in category |etex.grouptype| has value |2|.
% Dots are used to separate components.
%
% If you want to have the value, the access key is constructed by
% the category with all its components and the property name.
% For the opposite the value is used instead of the property name.
%
% Values are always integers (including negative numbers).
%
% \subsubsection{\cs{magicnum}}
%
% \begin{declcs}{magicnum} \M{access key}
% \end{declcs}
% Macro \cs{magicnum} expects an access key as argument and
% expands to the requested data. The macro is always expandable.
% In case of errors the expansion result is empty.
%
% The same macro is also used for getting a property name.
% In this case the property name part in the access key is
% replaced by the value.
%
% The catcodes
% of the resulting numbers and strings follow \TeX's tradition of
% \cs{string}, \cs{meaning}, \dots: The space has catcode 10
% (|tex.catcode.space|) and the other characters have catcode
% 12 (|tex.catcode.other|).
%
% Examples:
% \begin{quote}
%   |\magicnum{etex.grouptype.hbox}| $\Rightarrow$ |2|\\
%   |\magicnum{tex.catcode.14}| $\Rightarrow$ |comment|\\
%   |\magicnum{tex.catcode.undefined}| $\Rightarrow$ $\emptyset$
% \end{quote}
%
% \subsubsection{Properties}
%
% \begin{itemize}
% \item The components of a category are either subcategories or
%       key value pairs, but not both.
% \item The full specified property names are unique and thus
%       has one integer value exactly.
% \item Also the values inside a category are unique.
%       This condition is a prerequisite for the reverse mapping
%       of \cs{magicnum}.
% \item All names start with a letter. Only letters or digits
%       may follow.
% \end{itemize}
%
% \subsection{Data}
%
%  \subsubsection{\texorpdfstring{Category }{}\texttt{tex.catcode}}
%
% \begin{quote}
% \begin{tabular}{@{}>{\ttfamily}l>{\ttfamily}l@{}}
%    tex.catcode.escape & 0\\
%    tex.catcode.begingroup & 1\\
%    tex.catcode.endgroup & 2\\
%    tex.catcode.math & 3\\
%    tex.catcode.align & 4\\
%    tex.catcode.eol & 5\\
%    tex.catcode.parameter & 6\\
%    tex.catcode.superscript & 7\\
%    tex.catcode.subscript & 8\\
%    tex.catcode.ignore & 9\\
%    tex.catcode.space & 10\\
%    tex.catcode.letter & 11\\
%    tex.catcode.other & 12\\
%    tex.catcode.active & 13\\
%    tex.catcode.comment & 14\\
%    tex.catcode.invalid & 15\\
%  \end{tabular}
%  \end{quote}
%
%  \subsubsection{\texorpdfstring{Category }{}\texttt{etex.grouptype}}
%
% \begin{quote}
% \begin{tabular}{@{}>{\ttfamily}l>{\ttfamily}l@{}}
%    etex.grouptype.bottomlevel & 0\\
%    etex.grouptype.simple & 1\\
%    etex.grouptype.hbox & 2\\
%    etex.grouptype.adjustedhbox & 3\\
%    etex.grouptype.vbox & 4\\
%    etex.grouptype.align & 5\\
%    etex.grouptype.noalign & 6\\
%    etex.grouptype.output & 8\\
%    etex.grouptype.math & 9\\
%    etex.grouptype.disc & 10\\
%    etex.grouptype.insert & 11\\
%    etex.grouptype.vcenter & 12\\
%    etex.grouptype.mathchoice & 13\\
%    etex.grouptype.semisimple & 14\\
%    etex.grouptype.mathshift & 15\\
%    etex.grouptype.mathleft & 16\\
%  \end{tabular}
%  \end{quote}
%
%  \subsubsection{\texorpdfstring{Category }{}\texttt{etex.iftype}}
%
% \begin{quote}
% \begin{tabular}{@{}>{\ttfamily}l>{\ttfamily}l@{}}
%    etex.iftype.none & 0\\
%    etex.iftype.char & 1\\
%    etex.iftype.cat & 2\\
%    etex.iftype.num & 3\\
%    etex.iftype.dim & 4\\
%    etex.iftype.odd & 5\\
%    etex.iftype.vmode & 6\\
%    etex.iftype.hmode & 7\\
%    etex.iftype.mmode & 8\\
%    etex.iftype.inner & 9\\
%    etex.iftype.void & 10\\
%    etex.iftype.hbox & 11\\
%    etex.iftype.vbox & 12\\
%    etex.iftype.x & 13\\
%    etex.iftype.eof & 14\\
%    etex.iftype.true & 15\\
%    etex.iftype.false & 16\\
%    etex.iftype.case & 17\\
%    etex.iftype.defined & 18\\
%    etex.iftype.csname & 19\\
%    etex.iftype.fontchar & 20\\
%  \end{tabular}
%  \end{quote}
%
%  \subsubsection{\texorpdfstring{Category }{}\texttt{etex.nodetype}}
%
% \begin{quote}
% \begin{tabular}{@{}>{\ttfamily}l>{\ttfamily}l@{}}
%    etex.nodetype.none & -1\\
%    etex.nodetype.char & 0\\
%    etex.nodetype.hlist & 1\\
%    etex.nodetype.vlist & 2\\
%    etex.nodetype.rule & 3\\
%    etex.nodetype.ins & 4\\
%    etex.nodetype.mark & 5\\
%    etex.nodetype.adjust & 6\\
%    etex.nodetype.ligature & 7\\
%    etex.nodetype.disc & 8\\
%    etex.nodetype.whatsit & 9\\
%    etex.nodetype.math & 10\\
%    etex.nodetype.glue & 11\\
%    etex.nodetype.kern & 12\\
%    etex.nodetype.penalty & 13\\
%    etex.nodetype.unset & 14\\
%    etex.nodetype.maths & 15\\
%  \end{tabular}
%  \end{quote}
%
%  \subsubsection{\texorpdfstring{Category }{}\texttt{etex.interactionmode}}
%
% \begin{quote}
% \begin{tabular}{@{}>{\ttfamily}l>{\ttfamily}l@{}}
%    etex.interactionmode.batch & 0\\
%    etex.interactionmode.nonstop & 1\\
%    etex.interactionmode.scroll & 2\\
%    etex.interactionmode.errorstop & 3\\
%  \end{tabular}
%  \end{quote}
%
%  \subsubsection{\texorpdfstring{Category }{}\texttt{luatex.pdfliteral.mode}}
%
% \begin{quote}
% \begin{tabular}{@{}>{\ttfamily}l>{\ttfamily}l@{}}
%    luatex.pdfliteral.mode.setorigin & 0\\
%    luatex.pdfliteral.mode.page & 1\\
%    luatex.pdfliteral.mode.direct & 2\\
%  \end{tabular}
%  \end{quote}
%
%
% \hypersetup{bookmarksopenlevel=1}
%
% \StopEventually{
% }
%
% \section{Implementation}
%
%    \begin{macrocode}
%<*package>
%    \end{macrocode}
%
% \subsection{Reload check and package identification}
%    Reload check, especially if the package is not used with \LaTeX.
%    \begin{macrocode}
\begingroup\catcode61\catcode48\catcode32=10\relax%
  \catcode13=5 % ^^M
  \endlinechar=13 %
  \catcode35=6 % #
  \catcode39=12 % '
  \catcode44=12 % ,
  \catcode45=12 % -
  \catcode46=12 % .
  \catcode58=12 % :
  \catcode64=11 % @
  \catcode123=1 % {
  \catcode125=2 % }
  \expandafter\let\expandafter\x\csname ver@magicnum.sty\endcsname
  \ifx\x\relax % plain-TeX, first loading
  \else
    \def\empty{}%
    \ifx\x\empty % LaTeX, first loading,
      % variable is initialized, but \ProvidesPackage not yet seen
    \else
      \expandafter\ifx\csname PackageInfo\endcsname\relax
        \def\x#1#2{%
          \immediate\write-1{Package #1 Info: #2.}%
        }%
      \else
        \def\x#1#2{\PackageInfo{#1}{#2, stopped}}%
      \fi
      \x{magicnum}{The package is already loaded}%
      \aftergroup\endinput
    \fi
  \fi
\endgroup%
%    \end{macrocode}
%    Package identification:
%    \begin{macrocode}
\begingroup\catcode61\catcode48\catcode32=10\relax%
  \catcode13=5 % ^^M
  \endlinechar=13 %
  \catcode35=6 % #
  \catcode39=12 % '
  \catcode40=12 % (
  \catcode41=12 % )
  \catcode44=12 % ,
  \catcode45=12 % -
  \catcode46=12 % .
  \catcode47=12 % /
  \catcode58=12 % :
  \catcode64=11 % @
  \catcode91=12 % [
  \catcode93=12 % ]
  \catcode123=1 % {
  \catcode125=2 % }
  \expandafter\ifx\csname ProvidesPackage\endcsname\relax
    \def\x#1#2#3[#4]{\endgroup
      \immediate\write-1{Package: #3 #4}%
      \xdef#1{#4}%
    }%
  \else
    \def\x#1#2[#3]{\endgroup
      #2[{#3}]%
      \ifx#1\@undefined
        \xdef#1{#3}%
      \fi
      \ifx#1\relax
        \xdef#1{#3}%
      \fi
    }%
  \fi
\expandafter\x\csname ver@magicnum.sty\endcsname
\ProvidesPackage{magicnum}%
  [2011/04/10 v1.4 Magic numbers (HO)]%
%    \end{macrocode}
%
% \subsection{Catcodes}
%
%    \begin{macrocode}
\begingroup\catcode61\catcode48\catcode32=10\relax%
  \catcode13=5 % ^^M
  \endlinechar=13 %
  \catcode123=1 % {
  \catcode125=2 % }
  \catcode64=11 % @
  \def\x{\endgroup
    \expandafter\edef\csname magicnum@AtEnd\endcsname{%
      \endlinechar=\the\endlinechar\relax
      \catcode13=\the\catcode13\relax
      \catcode32=\the\catcode32\relax
      \catcode35=\the\catcode35\relax
      \catcode61=\the\catcode61\relax
      \catcode64=\the\catcode64\relax
      \catcode123=\the\catcode123\relax
      \catcode125=\the\catcode125\relax
    }%
  }%
\x\catcode61\catcode48\catcode32=10\relax%
\catcode13=5 % ^^M
\endlinechar=13 %
\catcode35=6 % #
\catcode64=11 % @
\catcode123=1 % {
\catcode125=2 % }
\def\TMP@EnsureCode#1#2{%
  \edef\magicnum@AtEnd{%
    \magicnum@AtEnd
    \catcode#1=\the\catcode#1\relax
  }%
  \catcode#1=#2\relax
}
\TMP@EnsureCode{34}{12}% "
\TMP@EnsureCode{39}{12}% '
\TMP@EnsureCode{40}{12}% (
\TMP@EnsureCode{41}{12}% )
\TMP@EnsureCode{42}{12}% *
\TMP@EnsureCode{44}{12}% ,
\TMP@EnsureCode{45}{12}% -
\TMP@EnsureCode{46}{12}% .
\TMP@EnsureCode{47}{12}% /
\TMP@EnsureCode{58}{12}% :
\TMP@EnsureCode{60}{12}% <
\TMP@EnsureCode{62}{12}% >
\TMP@EnsureCode{91}{12}% [
\TMP@EnsureCode{93}{12}% ]
\edef\magicnum@AtEnd{\magicnum@AtEnd\noexpand\endinput}
%    \end{macrocode}
%
% \subsection{Check for previous definition}
%
%    \begin{macrocode}
\begingroup\expandafter\expandafter\expandafter\endgroup
\expandafter\ifx\csname newcommand\endcsname\relax
  \expandafter\ifx\csname magicnum\endcsname\relax
  \else
    \input infwarerr.sty\relax
    \@PackageError{magicnum}{%
      \string\magicnum\space is already defined%
    }\@ehc
  \fi
\else
  \newcommand*{\magicnum}{}%
\fi
%    \end{macrocode}
%
% \subsection{Without \LuaTeX}
%
%    \begin{macrocode}
\begingroup\expandafter\expandafter\expandafter\endgroup
\expandafter\ifx\csname directlua\endcsname\relax
%    \end{macrocode}
%
%    \begin{macro}{\magicnum}
%    \begin{macrocode}
  \begingroup\expandafter\expandafter\expandafter\endgroup
  \expandafter\ifx\csname ifcsname\endcsname\relax
    \def\magicnum#1{%
      \expandafter\ifx\csname MG@#1\endcsname\relax
      \else
        \csname MG@#1\endcsname
      \fi
    }%
  \else
    \begingroup
      \edef\x{\endgroup
        \def\noexpand\magicnum##1{%
          \expandafter\noexpand\csname
          ifcsname\endcsname MG@##1\noexpand\endcsname
            \noexpand\csname MG@##1%
                 \noexpand\expandafter\noexpand\endcsname
          \expandafter\noexpand\csname fi\endcsname
        }%
      }%
    \x
  \fi
%    \end{macrocode}
%    \end{macro}
%
%    \begin{macrocode}
\else
%    \end{macrocode}
%
% \subsection{With \LuaTeX}
%
%    \begin{macrocode}
  \begingroup\expandafter\expandafter\expandafter\endgroup
  \expandafter\ifx\csname RequirePackage\endcsname\relax
    \input ifluatex.sty\relax
    \input infwarerr.sty\relax
  \else
    \RequirePackage{ifluatex}[2010/03/01]%
    \RequirePackage{infwarerr}[2010/04/08]%
  \fi
%    \end{macrocode}
%
%    \begin{macro}{\magicnum@directlua}
%    \begin{macrocode}
  \ifnum\luatexversion<36 %
    \def\magicnum@directlua{\directlua0 }%
  \else
    \let\magicnum@directlua\directlua
  \fi
%    \end{macrocode}
%    \end{macro}
%    \begin{macrocode}
  \magicnum@directlua{%
    require("oberdiek.magicnum")%
  }%
  \begingroup
    \def\x{2011/04/10 v1.4}%
    \def\StripPrefix#1>{}%
    \edef\x{\expandafter\StripPrefix\meaning\x}%
    \edef\y{%
      \magicnum@directlua{%
        if oberdiek.magicnum.getversion then %
          oberdiek.magicnum.getversion()%
        end%
      }%
    }%
    \ifx\x\y
    \else
      \@PackageError{magicnum}{%
        Wrong version of lua module.\MessageBreak
        Package version: \x\MessageBreak
        Lua module: \y
      }\@ehc
    \fi
  \endgroup
%    \end{macrocode}
%    \begin{macro}{\luaescapestring}
%    \begin{macrocode}
  \begingroup
    \expandafter\ifx\csname luaescapestring\endcsname\relax
      \directlua{%
        if tex.enableprimitives then %
          tex.enableprimitives('magicnum@', {'luaescapestring'})%
        end%
      }%
      \global\let\luaescapestring\magicnum@luaescapestring
    \fi
    \expandafter\ifx\csname luaescapestring\endcsname\relax
      \escapechar=92 %
      \@PackageError{magicnum}{%
        Missing \string\luaescapestring
      }\@ehc
    \fi
  \endgroup
%    \end{macrocode}
%    \end{macro}
%    \begin{macro}{\magicnum}
%    \begin{macrocode}
  \def\magicnum#1{%
    \magicnum@directlua{%
      oberdiek.magicnum.get("\luaescapestring{#1}")%
    }%
  }%
%    \end{macrocode}
%    \end{macro}
%
%    \begin{macrocode}
  \expandafter\magicnum@AtEnd
\fi%
%</package>
%    \end{macrocode}
%
% \subsection{Data}
%
% \subsubsection{Plain data}
%
%    \begin{macrocode}
%<*data>
tex.catcode
  escape = 0
  begingroup = 1
  endgroup = 2
  math = 3
  align = 4
  eol = 5
  parameter = 6
  superscript = 7
  subscript = 8
  ignore = 9
  space = 10
  letter = 11
  other = 12
  active = 13
  comment = 14
  invalid = 15
etex.grouptype
  bottomlevel = 0
  simple = 1
  hbox = 2
  adjustedhbox = 3
  vbox = 4
  align = 5
  noalign = 6
  output = 8
  math = 9
  disc = 10
  insert = 11
  vcenter = 12
  mathchoice = 13
  semisimple = 14
  mathshift = 15
  mathleft = 16
etex.iftype
  none = 0
  char = 1
  cat = 2
  num = 3
  dim = 4
  odd = 5
  vmode = 6
  hmode = 7
  mmode = 8
  inner = 9
  void = 10
  hbox = 11
  vbox = 12
  x = 13
  eof = 14
  true = 15
  false = 16
  case = 17
  defined = 18
  csname = 19
  fontchar = 20
etex.nodetype
  none = -1
  char = 0
  hlist = 1
  vlist = 2
  rule = 3
  ins = 4
  mark = 5
  adjust = 6
  ligature = 7
  disc = 8
  whatsit = 9
  math = 10
  glue = 11
  kern = 12
  penalty = 13
  unset = 14
  maths = 15
etex.interactionmode
  batch = 0
  nonstop = 1
  scroll = 2
  errorstop = 3
luatex.pdfliteral.mode
  setorigin = 0
  page = 1
  direct = 2
%</data>
%    \end{macrocode}
%
% \subsubsection{Data for \TeX}
%
%    \begin{macrocode}
%<*package>
%    \end{macrocode}
%    \begin{macro}{\magicnum@add}
%    \begin{macrocode}
\begingroup\expandafter\expandafter\expandafter\endgroup
\expandafter\ifx\csname detokenize\endcsname\relax
  \def\magicnum@add#1#2#3{%
    \expandafter\magicnum@@add
        \csname MG@#1.#2\expandafter\endcsname
        \csname MG@#1.#3\endcsname
       {#3}{#2}%
  }%
  \def\magicnum@@add#1#2#3#4{%
    \def#1{#3}%
    \def#2{#4}%
    \edef#1{%
      \expandafter\strip@prefix\meaning#1%
    }%
    \edef#2{%
      \expandafter\strip@prefix\meaning#2%
    }%
  }%
  \expandafter\ifx\csname strip@prefix\endcsname\relax
    \def\strip@prefix#1->{}%
  \fi
\else
  \def\magicnum@add#1#2#3{%
    \expandafter\edef\csname MG@#1.#2\endcsname{%
      \detokenize{#3}%
    }%
    \expandafter\edef\csname MG@#1.#3\endcsname{%
      \detokenize{#2}%
    }%
  }%
\fi
%    \end{macrocode}
%    \end{macro}
%    \begin{macrocode}
\magicnum@add{tex.catcode}{escape}{0}
\magicnum@add{tex.catcode}{begingroup}{1}
\magicnum@add{tex.catcode}{endgroup}{2}
\magicnum@add{tex.catcode}{math}{3}
\magicnum@add{tex.catcode}{align}{4}
\magicnum@add{tex.catcode}{eol}{5}
\magicnum@add{tex.catcode}{parameter}{6}
\magicnum@add{tex.catcode}{superscript}{7}
\magicnum@add{tex.catcode}{subscript}{8}
\magicnum@add{tex.catcode}{ignore}{9}
\magicnum@add{tex.catcode}{space}{10}
\magicnum@add{tex.catcode}{letter}{11}
\magicnum@add{tex.catcode}{other}{12}
\magicnum@add{tex.catcode}{active}{13}
\magicnum@add{tex.catcode}{comment}{14}
\magicnum@add{tex.catcode}{invalid}{15}
\magicnum@add{etex.grouptype}{bottomlevel}{0}
\magicnum@add{etex.grouptype}{simple}{1}
\magicnum@add{etex.grouptype}{hbox}{2}
\magicnum@add{etex.grouptype}{adjustedhbox}{3}
\magicnum@add{etex.grouptype}{vbox}{4}
\magicnum@add{etex.grouptype}{align}{5}
\magicnum@add{etex.grouptype}{noalign}{6}
\magicnum@add{etex.grouptype}{output}{8}
\magicnum@add{etex.grouptype}{math}{9}
\magicnum@add{etex.grouptype}{disc}{10}
\magicnum@add{etex.grouptype}{insert}{11}
\magicnum@add{etex.grouptype}{vcenter}{12}
\magicnum@add{etex.grouptype}{mathchoice}{13}
\magicnum@add{etex.grouptype}{semisimple}{14}
\magicnum@add{etex.grouptype}{mathshift}{15}
\magicnum@add{etex.grouptype}{mathleft}{16}
\magicnum@add{etex.iftype}{none}{0}
\magicnum@add{etex.iftype}{char}{1}
\magicnum@add{etex.iftype}{cat}{2}
\magicnum@add{etex.iftype}{num}{3}
\magicnum@add{etex.iftype}{dim}{4}
\magicnum@add{etex.iftype}{odd}{5}
\magicnum@add{etex.iftype}{vmode}{6}
\magicnum@add{etex.iftype}{hmode}{7}
\magicnum@add{etex.iftype}{mmode}{8}
\magicnum@add{etex.iftype}{inner}{9}
\magicnum@add{etex.iftype}{void}{10}
\magicnum@add{etex.iftype}{hbox}{11}
\magicnum@add{etex.iftype}{vbox}{12}
\magicnum@add{etex.iftype}{x}{13}
\magicnum@add{etex.iftype}{eof}{14}
\magicnum@add{etex.iftype}{true}{15}
\magicnum@add{etex.iftype}{false}{16}
\magicnum@add{etex.iftype}{case}{17}
\magicnum@add{etex.iftype}{defined}{18}
\magicnum@add{etex.iftype}{csname}{19}
\magicnum@add{etex.iftype}{fontchar}{20}
\magicnum@add{etex.nodetype}{none}{-1}
\magicnum@add{etex.nodetype}{char}{0}
\magicnum@add{etex.nodetype}{hlist}{1}
\magicnum@add{etex.nodetype}{vlist}{2}
\magicnum@add{etex.nodetype}{rule}{3}
\magicnum@add{etex.nodetype}{ins}{4}
\magicnum@add{etex.nodetype}{mark}{5}
\magicnum@add{etex.nodetype}{adjust}{6}
\magicnum@add{etex.nodetype}{ligature}{7}
\magicnum@add{etex.nodetype}{disc}{8}
\magicnum@add{etex.nodetype}{whatsit}{9}
\magicnum@add{etex.nodetype}{math}{10}
\magicnum@add{etex.nodetype}{glue}{11}
\magicnum@add{etex.nodetype}{kern}{12}
\magicnum@add{etex.nodetype}{penalty}{13}
\magicnum@add{etex.nodetype}{unset}{14}
\magicnum@add{etex.nodetype}{maths}{15}
\magicnum@add{etex.interactionmode}{batch}{0}
\magicnum@add{etex.interactionmode}{nonstop}{1}
\magicnum@add{etex.interactionmode}{scroll}{2}
\magicnum@add{etex.interactionmode}{errorstop}{3}
\magicnum@add{luatex.pdfliteral.mode}{setorigin}{0}
\magicnum@add{luatex.pdfliteral.mode}{page}{1}
\magicnum@add{luatex.pdfliteral.mode}{direct}{2}
%    \end{macrocode}
%    \begin{macrocode}
\magicnum@AtEnd%
%</package>
%    \end{macrocode}
%
% \subsubsection{Lua module}
%
%    \begin{macrocode}
%<*lua>
%    \end{macrocode}
%    \begin{macrocode}
module("oberdiek.magicnum", package.seeall)
%    \end{macrocode}
%    \begin{macrocode}
function getversion()
  tex.write("2011/04/10 v1.4")
end
%    \end{macrocode}
%    \begin{macrocode}
local data = {
  ["tex.catcode"] = {
    [0] = "escape",
    [1] = "begingroup",
    [2] = "endgroup",
    [3] = "math",
    [4] = "align",
    [5] = "eol",
    [6] = "parameter",
    [7] = "superscript",
    [8] = "subscript",
    [9] = "ignore",
    [10] = "space",
    [11] = "letter",
    [12] = "other",
    [13] = "active",
    [14] = "comment",
    [15] = "invalid",
    ["active"] = 13,
    ["align"] = 4,
    ["begingroup"] = 1,
    ["comment"] = 14,
    ["endgroup"] = 2,
    ["eol"] = 5,
    ["escape"] = 0,
    ["ignore"] = 9,
    ["invalid"] = 15,
    ["letter"] = 11,
    ["math"] = 3,
    ["other"] = 12,
    ["parameter"] = 6,
    ["space"] = 10,
    ["subscript"] = 8,
    ["superscript"] = 7
  },
  ["etex.grouptype"] = {
    [0] = "bottomlevel",
    [1] = "simple",
    [2] = "hbox",
    [3] = "adjustedhbox",
    [4] = "vbox",
    [5] = "align",
    [6] = "noalign",
    [8] = "output",
    [9] = "math",
    [10] = "disc",
    [11] = "insert",
    [12] = "vcenter",
    [13] = "mathchoice",
    [14] = "semisimple",
    [15] = "mathshift",
    [16] = "mathleft",
    ["adjustedhbox"] = 3,
    ["align"] = 5,
    ["bottomlevel"] = 0,
    ["disc"] = 10,
    ["hbox"] = 2,
    ["insert"] = 11,
    ["math"] = 9,
    ["mathchoice"] = 13,
    ["mathleft"] = 16,
    ["mathshift"] = 15,
    ["noalign"] = 6,
    ["output"] = 8,
    ["semisimple"] = 14,
    ["simple"] = 1,
    ["vbox"] = 4,
    ["vcenter"] = 12
  },
  ["etex.iftype"] = {
    [0] = "none",
    [1] = "char",
    [2] = "cat",
    [3] = "num",
    [4] = "dim",
    [5] = "odd",
    [6] = "vmode",
    [7] = "hmode",
    [8] = "mmode",
    [9] = "inner",
    [10] = "void",
    [11] = "hbox",
    [12] = "vbox",
    [13] = "x",
    [14] = "eof",
    [15] = "true",
    [16] = "false",
    [17] = "case",
    [18] = "defined",
    [19] = "csname",
    [20] = "fontchar",
    ["case"] = 17,
    ["cat"] = 2,
    ["char"] = 1,
    ["csname"] = 19,
    ["defined"] = 18,
    ["dim"] = 4,
    ["eof"] = 14,
    ["false"] = 16,
    ["fontchar"] = 20,
    ["hbox"] = 11,
    ["hmode"] = 7,
    ["inner"] = 9,
    ["mmode"] = 8,
    ["none"] = 0,
    ["num"] = 3,
    ["odd"] = 5,
    ["true"] = 15,
    ["vbox"] = 12,
    ["vmode"] = 6,
    ["void"] = 10,
    ["x"] = 13
  },
  ["etex.nodetype"] = {
    [-1] = "none",
    [0] = "char",
    [1] = "hlist",
    [2] = "vlist",
    [3] = "rule",
    [4] = "ins",
    [5] = "mark",
    [6] = "adjust",
    [7] = "ligature",
    [8] = "disc",
    [9] = "whatsit",
    [10] = "math",
    [11] = "glue",
    [12] = "kern",
    [13] = "penalty",
    [14] = "unset",
    [15] = "maths",
    ["adjust"] = 6,
    ["char"] = 0,
    ["disc"] = 8,
    ["glue"] = 11,
    ["hlist"] = 1,
    ["ins"] = 4,
    ["kern"] = 12,
    ["ligature"] = 7,
    ["mark"] = 5,
    ["math"] = 10,
    ["maths"] = 15,
    ["none"] = -1,
    ["penalty"] = 13,
    ["rule"] = 3,
    ["unset"] = 14,
    ["vlist"] = 2,
    ["whatsit"] = 9
  },
  ["etex.interactionmode"] = {
    [0] = "batch",
    [1] = "nonstop",
    [2] = "scroll",
    [3] = "errorstop",
    ["batch"] = 0,
    ["errorstop"] = 3,
    ["nonstop"] = 1,
    ["scroll"] = 2
  },
  ["luatex.pdfliteral.mode"] = {
    [0] = "setorigin",
    [1] = "page",
    [2] = "direct",
    ["direct"] = 2,
    ["page"] = 1,
    ["setorigin"] = 0
  }
}
%    \end{macrocode}
%    \begin{macrocode}
function get(name)
  local startpos, endpos, category, entry =
      string.find(name, "^(%a[%a%d%.]*)%.(-?[%a%d]+)$")
  if not entry then
    return
  end
  local node = data[category]
  if not node then
    return
  end
  local num = tonumber(entry)
  local value
  if num then
    value = node[num]
    if not value then
      return
    end
  else
    value = node[entry]
    if not value then
      return
    end
    value = "" .. value
  end
  tex.write(value)
end
%    \end{macrocode}
%
%    \begin{macrocode}
%</lua>
%    \end{macrocode}
%
% \section{Test}
%
% \subsection{Catcode checks for loading}
%
%    \begin{macrocode}
%<*test1>
%    \end{macrocode}
%    \begin{macrocode}
\catcode`\{=1 %
\catcode`\}=2 %
\catcode`\#=6 %
\catcode`\@=11 %
\expandafter\ifx\csname count@\endcsname\relax
  \countdef\count@=255 %
\fi
\expandafter\ifx\csname @gobble\endcsname\relax
  \long\def\@gobble#1{}%
\fi
\expandafter\ifx\csname @firstofone\endcsname\relax
  \long\def\@firstofone#1{#1}%
\fi
\expandafter\ifx\csname loop\endcsname\relax
  \expandafter\@firstofone
\else
  \expandafter\@gobble
\fi
{%
  \def\loop#1\repeat{%
    \def\body{#1}%
    \iterate
  }%
  \def\iterate{%
    \body
      \let\next\iterate
    \else
      \let\next\relax
    \fi
    \next
  }%
  \let\repeat=\fi
}%
\def\RestoreCatcodes{}
\count@=0 %
\loop
  \edef\RestoreCatcodes{%
    \RestoreCatcodes
    \catcode\the\count@=\the\catcode\count@\relax
  }%
\ifnum\count@<255 %
  \advance\count@ 1 %
\repeat

\def\RangeCatcodeInvalid#1#2{%
  \count@=#1\relax
  \loop
    \catcode\count@=15 %
  \ifnum\count@<#2\relax
    \advance\count@ 1 %
  \repeat
}
\def\RangeCatcodeCheck#1#2#3{%
  \count@=#1\relax
  \loop
    \ifnum#3=\catcode\count@
    \else
      \errmessage{%
        Character \the\count@\space
        with wrong catcode \the\catcode\count@\space
        instead of \number#3%
      }%
    \fi
  \ifnum\count@<#2\relax
    \advance\count@ 1 %
  \repeat
}
\def\space{ }
\expandafter\ifx\csname LoadCommand\endcsname\relax
  \def\LoadCommand{\input magicnum.sty\relax}%
\fi
\def\Test{%
  \RangeCatcodeInvalid{0}{47}%
  \RangeCatcodeInvalid{58}{64}%
  \RangeCatcodeInvalid{91}{96}%
  \RangeCatcodeInvalid{123}{255}%
  \catcode`\@=12 %
  \catcode`\\=0 %
  \catcode`\%=14 %
  \LoadCommand
  \RangeCatcodeCheck{0}{36}{15}%
  \RangeCatcodeCheck{37}{37}{14}%
  \RangeCatcodeCheck{38}{47}{15}%
  \RangeCatcodeCheck{48}{57}{12}%
  \RangeCatcodeCheck{58}{63}{15}%
  \RangeCatcodeCheck{64}{64}{12}%
  \RangeCatcodeCheck{65}{90}{11}%
  \RangeCatcodeCheck{91}{91}{15}%
  \RangeCatcodeCheck{92}{92}{0}%
  \RangeCatcodeCheck{93}{96}{15}%
  \RangeCatcodeCheck{97}{122}{11}%
  \RangeCatcodeCheck{123}{255}{15}%
  \RestoreCatcodes
}
\Test
\csname @@end\endcsname
\end
%    \end{macrocode}
%    \begin{macrocode}
%</test1>
%    \end{macrocode}
%
% \subsection{Test data}
%
%    \begin{macrocode}
%<*testplain>
\input magicnum.sty\relax
\def\Test#1#2{%
  \edef\result{\magicnum{#1}}%
  \edef\expect{#2}%
  \edef\expect{\expandafter\stripprefix\meaning\expect}%
  \ifx\result\expect
  \else
    \errmessage{%
      Failed: [#1] % hash-ok
      returns [\result] instead of [\expect]%
    }%
  \fi
}
\def\stripprefix#1->{}
%</testplain>
%    \end{macrocode}
%    \begin{macrocode}
%<*testlatex>
\NeedsTeXFormat{LaTeX2e}
\documentclass{minimal}
\usepackage{magicnum}[2011/04/10]
\usepackage{qstest}
\IncludeTests{*}
\LogTests{log}{*}{*}
\newcommand*{\Test}[2]{%
  \Expect*{\magicnum{#1}}{#2}%
}
\begin{qstest}{magicnum}{magicnum}
%</testlatex>
%    \end{macrocode}
%    \begin{macrocode}
%<*testdata>
\Test{tex.catcode.escape}{0}
\Test{tex.catcode.invalid}{15}
\Test{tex.catcode.unknown}{}
\Test{tex.catcode.0}{escape}
\Test{tex.catcode.15}{invalid}
\Test{etex.iftype.true}{15}
\Test{etex.iftype.false}{16}
\Test{etex.iftype.15}{true}
\Test{etex.iftype.16}{false}
\Test{etex.nodetype.none}{-1}
\Test{etex.nodetype.-1}{none}
\Test{luatex.pdfliteral.mode.direct}{2}
\Test{luatex.pdfliteral.mode.1}{page}
\Test{}{}
\Test{unknown}{}
\Test{unknown.foo.bar}{}
\Test{unknown.foo.4}{}
%</testdata>
%    \end{macrocode}
%    \begin{macrocode}
%<*testplain>
\csname @@end\endcsname
\end
%</testplain>
%<*testlatex>
\end{qstest}
\csname @@end\endcsname
%</testlatex>
%    \end{macrocode}
%
% \subsection{Small test for \hologo{iniTeX}}
%
%    \begin{macrocode}
%<*test4>
\catcode`\{=1
\catcode`\}=2
\catcode`\#=6
\input magicnum.sty\relax
\edef\x{\magicnum{tex.catcode.15}}
\edef\y{invalid}
\def\Strip#1>{}
\edef\y{\expandafter\Strip\meaning\y}
\ifx\x\y
  \immediate\write16{Ok}%
\else
  \errmessage{\x<>\y}%
\fi
\csname @@end\endcsname\end
%</test4>
%    \end{macrocode}
%
% \section{Installation}
%
% \subsection{Download}
%
% \paragraph{Package.} This package is available on
% CTAN\footnote{\url{ftp://ftp.ctan.org/tex-archive/}}:
% \begin{description}
% \item[\CTAN{macros/latex/contrib/oberdiek/magicnum.dtx}] The source file.
% \item[\CTAN{macros/latex/contrib/oberdiek/magicnum.pdf}] Documentation.
% \end{description}
%
%
% \paragraph{Bundle.} All the packages of the bundle `oberdiek'
% are also available in a TDS compliant ZIP archive. There
% the packages are already unpacked and the documentation files
% are generated. The files and directories obey the TDS standard.
% \begin{description}
% \item[\CTAN{install/macros/latex/contrib/oberdiek.tds.zip}]
% \end{description}
% \emph{TDS} refers to the standard ``A Directory Structure
% for \TeX\ Files'' (\CTAN{tds/tds.pdf}). Directories
% with \xfile{texmf} in their name are usually organized this way.
%
% \subsection{Bundle installation}
%
% \paragraph{Unpacking.} Unpack the \xfile{oberdiek.tds.zip} in the
% TDS tree (also known as \xfile{texmf} tree) of your choice.
% Example (linux):
% \begin{quote}
%   |unzip oberdiek.tds.zip -d ~/texmf|
% \end{quote}
%
% \paragraph{Script installation.}
% Check the directory \xfile{TDS:scripts/oberdiek/} for
% scripts that need further installation steps.
% Package \xpackage{attachfile2} comes with the Perl script
% \xfile{pdfatfi.pl} that should be installed in such a way
% that it can be called as \texttt{pdfatfi}.
% Example (linux):
% \begin{quote}
%   |chmod +x scripts/oberdiek/pdfatfi.pl|\\
%   |cp scripts/oberdiek/pdfatfi.pl /usr/local/bin/|
% \end{quote}
%
% \subsection{Package installation}
%
% \paragraph{Unpacking.} The \xfile{.dtx} file is a self-extracting
% \docstrip\ archive. The files are extracted by running the
% \xfile{.dtx} through \plainTeX:
% \begin{quote}
%   \verb|tex magicnum.dtx|
% \end{quote}
%
% \paragraph{TDS.} Now the different files must be moved into
% the different directories in your installation TDS tree
% (also known as \xfile{texmf} tree):
% \begin{quote}
% \def\t{^^A
% \begin{tabular}{@{}>{\ttfamily}l@{ $\rightarrow$ }>{\ttfamily}l@{}}
%   magicnum.sty & tex/generic/oberdiek/magicnum.sty\\
%   magicnum.lua & scripts/oberdiek/magicnum.lua\\
%   oberdiek.magicnum.lua & scripts/oberdiek/oberdiek.magicnum.lua\\
%   magicnum.pdf & doc/latex/oberdiek/magicnum.pdf\\
%   magicnum.txt & doc/latex/oberdiek/magicnum.txt\\
%   test/magicnum-test1.tex & doc/latex/oberdiek/test/magicnum-test1.tex\\
%   test/magicnum-test2.tex & doc/latex/oberdiek/test/magicnum-test2.tex\\
%   test/magicnum-test3.tex & doc/latex/oberdiek/test/magicnum-test3.tex\\
%   test/magicnum-test4.tex & doc/latex/oberdiek/test/magicnum-test4.tex\\
%   magicnum.dtx & source/latex/oberdiek/magicnum.dtx\\
% \end{tabular}^^A
% }^^A
% \sbox0{\t}^^A
% \ifdim\wd0>\linewidth
%   \begingroup
%     \advance\linewidth by\leftmargin
%     \advance\linewidth by\rightmargin
%   \edef\x{\endgroup
%     \def\noexpand\lw{\the\linewidth}^^A
%   }\x
%   \def\lwbox{^^A
%     \leavevmode
%     \hbox to \linewidth{^^A
%       \kern-\leftmargin\relax
%       \hss
%       \usebox0
%       \hss
%       \kern-\rightmargin\relax
%     }^^A
%   }^^A
%   \ifdim\wd0>\lw
%     \sbox0{\small\t}^^A
%     \ifdim\wd0>\linewidth
%       \ifdim\wd0>\lw
%         \sbox0{\footnotesize\t}^^A
%         \ifdim\wd0>\linewidth
%           \ifdim\wd0>\lw
%             \sbox0{\scriptsize\t}^^A
%             \ifdim\wd0>\linewidth
%               \ifdim\wd0>\lw
%                 \sbox0{\tiny\t}^^A
%                 \ifdim\wd0>\linewidth
%                   \lwbox
%                 \else
%                   \usebox0
%                 \fi
%               \else
%                 \lwbox
%               \fi
%             \else
%               \usebox0
%             \fi
%           \else
%             \lwbox
%           \fi
%         \else
%           \usebox0
%         \fi
%       \else
%         \lwbox
%       \fi
%     \else
%       \usebox0
%     \fi
%   \else
%     \lwbox
%   \fi
% \else
%   \usebox0
% \fi
% \end{quote}
% If you have a \xfile{docstrip.cfg} that configures and enables \docstrip's
% TDS installing feature, then some files can already be in the right
% place, see the documentation of \docstrip.
%
% \subsection{Refresh file name databases}
%
% If your \TeX~distribution
% (\teTeX, \mikTeX, \dots) relies on file name databases, you must refresh
% these. For example, \teTeX\ users run \verb|texhash| or
% \verb|mktexlsr|.
%
% \subsection{Some details for the interested}
%
% \paragraph{Attached source.}
%
% The PDF documentation on CTAN also includes the
% \xfile{.dtx} source file. It can be extracted by
% AcrobatReader 6 or higher. Another option is \textsf{pdftk},
% e.g. unpack the file into the current directory:
% \begin{quote}
%   \verb|pdftk magicnum.pdf unpack_files output .|
% \end{quote}
%
% \paragraph{Unpacking with \LaTeX.}
% The \xfile{.dtx} chooses its action depending on the format:
% \begin{description}
% \item[\plainTeX:] Run \docstrip\ and extract the files.
% \item[\LaTeX:] Generate the documentation.
% \end{description}
% If you insist on using \LaTeX\ for \docstrip\ (really,
% \docstrip\ does not need \LaTeX), then inform the autodetect routine
% about your intention:
% \begin{quote}
%   \verb|latex \let\install=y% \iffalse meta-comment
%
% File: magicnum.dtx
% Version: 2011/04/10 v1.4
% Info: Magic numbers
%
% Copyright (C) 2007, 2009-2011 by
%    Heiko Oberdiek <heiko.oberdiek at googlemail.com>
%
% This work may be distributed and/or modified under the
% conditions of the LaTeX Project Public License, either
% version 1.3c of this license or (at your option) any later
% version. This version of this license is in
%    http://www.latex-project.org/lppl/lppl-1-3c.txt
% and the latest version of this license is in
%    http://www.latex-project.org/lppl.txt
% and version 1.3 or later is part of all distributions of
% LaTeX version 2005/12/01 or later.
%
% This work has the LPPL maintenance status "maintained".
%
% This Current Maintainer of this work is Heiko Oberdiek.
%
% The Base Interpreter refers to any `TeX-Format',
% because some files are installed in TDS:tex/generic//.
%
% This work consists of the main source file magicnum.dtx
% and the derived files
%    magicnum.sty, magicnum.pdf, magicnum.ins, magicnum.drv, magicnum.txt,
%    magicnum-test1.tex, magicnum-test2.tex, magicnum-test3.tex,
%    magicnum-test4.tex, magicnum.lua, oberdiek.magicnum.lua.
%
% Distribution:
%    CTAN:macros/latex/contrib/oberdiek/magicnum.dtx
%    CTAN:macros/latex/contrib/oberdiek/magicnum.pdf
%
% Unpacking:
%    (a) If magicnum.ins is present:
%           tex magicnum.ins
%    (b) Without magicnum.ins:
%           tex magicnum.dtx
%    (c) If you insist on using LaTeX
%           latex \let\install=y% \iffalse meta-comment
%
% File: magicnum.dtx
% Version: 2011/04/10 v1.4
% Info: Magic numbers
%
% Copyright (C) 2007, 2009-2011 by
%    Heiko Oberdiek <heiko.oberdiek at googlemail.com>
%
% This work may be distributed and/or modified under the
% conditions of the LaTeX Project Public License, either
% version 1.3c of this license or (at your option) any later
% version. This version of this license is in
%    http://www.latex-project.org/lppl/lppl-1-3c.txt
% and the latest version of this license is in
%    http://www.latex-project.org/lppl.txt
% and version 1.3 or later is part of all distributions of
% LaTeX version 2005/12/01 or later.
%
% This work has the LPPL maintenance status "maintained".
%
% This Current Maintainer of this work is Heiko Oberdiek.
%
% The Base Interpreter refers to any `TeX-Format',
% because some files are installed in TDS:tex/generic//.
%
% This work consists of the main source file magicnum.dtx
% and the derived files
%    magicnum.sty, magicnum.pdf, magicnum.ins, magicnum.drv, magicnum.txt,
%    magicnum-test1.tex, magicnum-test2.tex, magicnum-test3.tex,
%    magicnum-test4.tex, magicnum.lua, oberdiek.magicnum.lua.
%
% Distribution:
%    CTAN:macros/latex/contrib/oberdiek/magicnum.dtx
%    CTAN:macros/latex/contrib/oberdiek/magicnum.pdf
%
% Unpacking:
%    (a) If magicnum.ins is present:
%           tex magicnum.ins
%    (b) Without magicnum.ins:
%           tex magicnum.dtx
%    (c) If you insist on using LaTeX
%           latex \let\install=y\input{magicnum.dtx}
%        (quote the arguments according to the demands of your shell)
%
% Documentation:
%    (a) If magicnum.drv is present:
%           latex magicnum.drv
%    (b) Without magicnum.drv:
%           latex magicnum.dtx; ...
%    The class ltxdoc loads the configuration file ltxdoc.cfg
%    if available. Here you can specify further options, e.g.
%    use A4 as paper format:
%       \PassOptionsToClass{a4paper}{article}
%
%    Programm calls to get the documentation (example):
%       pdflatex magicnum.dtx
%       makeindex -s gind.ist magicnum.idx
%       pdflatex magicnum.dtx
%       makeindex -s gind.ist magicnum.idx
%       pdflatex magicnum.dtx
%
% Installation:
%    TDS:tex/generic/oberdiek/magicnum.sty
%    TDS:scripts/oberdiek/magicnum.lua
%    TDS:scripts/oberdiek/oberdiek.magicnum.lua
%    TDS:doc/latex/oberdiek/magicnum.pdf
%    TDS:doc/latex/oberdiek/magicnum.txt
%    TDS:doc/latex/oberdiek/test/magicnum-test1.tex
%    TDS:doc/latex/oberdiek/test/magicnum-test2.tex
%    TDS:doc/latex/oberdiek/test/magicnum-test3.tex
%    TDS:doc/latex/oberdiek/test/magicnum-test4.tex
%    TDS:source/latex/oberdiek/magicnum.dtx
%
%<*ignore>
\begingroup
  \catcode123=1 %
  \catcode125=2 %
  \def\x{LaTeX2e}%
\expandafter\endgroup
\ifcase 0\ifx\install y1\fi\expandafter
         \ifx\csname processbatchFile\endcsname\relax\else1\fi
         \ifx\fmtname\x\else 1\fi\relax
\else\csname fi\endcsname
%</ignore>
%<*install>
\input docstrip.tex
\Msg{************************************************************************}
\Msg{* Installation}
\Msg{* Package: magicnum 2011/04/10 v1.4 Magic numbers (HO)}
\Msg{************************************************************************}

\keepsilent
\askforoverwritefalse

\let\MetaPrefix\relax
\preamble

This is a generated file.

Project: magicnum
Version: 2011/04/10 v1.4

Copyright (C) 2007, 2009-2011 by
   Heiko Oberdiek <heiko.oberdiek at googlemail.com>

This work may be distributed and/or modified under the
conditions of the LaTeX Project Public License, either
version 1.3c of this license or (at your option) any later
version. This version of this license is in
   http://www.latex-project.org/lppl/lppl-1-3c.txt
and the latest version of this license is in
   http://www.latex-project.org/lppl.txt
and version 1.3 or later is part of all distributions of
LaTeX version 2005/12/01 or later.

This work has the LPPL maintenance status "maintained".

This Current Maintainer of this work is Heiko Oberdiek.

The Base Interpreter refers to any `TeX-Format',
because some files are installed in TDS:tex/generic//.

This work consists of the main source file magicnum.dtx
and the derived files
   magicnum.sty, magicnum.pdf, magicnum.ins, magicnum.drv, magicnum.txt,
   magicnum-test1.tex, magicnum-test2.tex, magicnum-test3.tex,
   magicnum-test4.tex, magicnum.lua, oberdiek.magicnum.lua.

\endpreamble
\let\MetaPrefix\DoubleperCent

\generate{%
  \file{magicnum.ins}{\from{magicnum.dtx}{install}}%
  \file{magicnum.drv}{\from{magicnum.dtx}{driver}}%
  \usedir{tex/generic/oberdiek}%
  \file{magicnum.sty}{\from{magicnum.dtx}{package}}%
  \usedir{doc/latex/oberdiek/test}%
  \file{magicnum-test1.tex}{\from{magicnum.dtx}{test1}}%
  \file{magicnum-test2.tex}{\from{magicnum.dtx}{testplain,testdata}}%
  \file{magicnum-test3.tex}{\from{magicnum.dtx}{testlatex,testdata}}%
  \file{magicnum-test4.tex}{\from{magicnum.dtx}{test4}}%
  \nopreamble
  \nopostamble
  \usedir{doc/latex/oberdiek}%
  \file{magicnum.txt}{\from{magicnum.dtx}{data}}%
  \usedir{source/latex/oberdiek/catalogue}%
  \file{magicnum.xml}{\from{magicnum.dtx}{catalogue}}%
}
\def\MetaPrefix{-- }
\def\defaultpostamble{%
  \MetaPrefix^^J%
  \MetaPrefix\space End of File `\outFileName'.%
}
\def\currentpostamble{\defaultpostamble}%
\generate{%
  \usedir{scripts/oberdiek}%
  \file{magicnum.lua}{\from{magicnum.dtx}{lua}}%
  \file{oberdiek.magicnum.lua}{\from{magicnum.dtx}{lua}}%
}

\catcode32=13\relax% active space
\let =\space%
\Msg{************************************************************************}
\Msg{*}
\Msg{* To finish the installation you have to move the following}
\Msg{* file into a directory searched by TeX:}
\Msg{*}
\Msg{*     magicnum.sty}
\Msg{*}
\Msg{* And install the following script files:}
\Msg{*}
\Msg{*     magicnum.lua, oberdiek.magicnum.lua}
\Msg{*}
\Msg{* To produce the documentation run the file `magicnum.drv'}
\Msg{* through LaTeX.}
\Msg{*}
\Msg{* Happy TeXing!}
\Msg{*}
\Msg{************************************************************************}

\endbatchfile
%</install>
%<*ignore>
\fi
%</ignore>
%<*driver>
\NeedsTeXFormat{LaTeX2e}
\ProvidesFile{magicnum.drv}%
  [2011/04/10 v1.4 Magic numbers (HO)]%
\documentclass{ltxdoc}
\usepackage{holtxdoc}[2011/11/22]
\usepackage{array}
\begin{document}
  \DocInput{magicnum.dtx}%
\end{document}
%</driver>
% \fi
%
% \CheckSum{755}
%
% \CharacterTable
%  {Upper-case    \A\B\C\D\E\F\G\H\I\J\K\L\M\N\O\P\Q\R\S\T\U\V\W\X\Y\Z
%   Lower-case    \a\b\c\d\e\f\g\h\i\j\k\l\m\n\o\p\q\r\s\t\u\v\w\x\y\z
%   Digits        \0\1\2\3\4\5\6\7\8\9
%   Exclamation   \!     Double quote  \"     Hash (number) \#
%   Dollar        \$     Percent       \%     Ampersand     \&
%   Acute accent  \'     Left paren    \(     Right paren   \)
%   Asterisk      \*     Plus          \+     Comma         \,
%   Minus         \-     Point         \.     Solidus       \/
%   Colon         \:     Semicolon     \;     Less than     \<
%   Equals        \=     Greater than  \>     Question mark \?
%   Commercial at \@     Left bracket  \[     Backslash     \\
%   Right bracket \]     Circumflex    \^     Underscore    \_
%   Grave accent  \`     Left brace    \{     Vertical bar  \|
%   Right brace   \}     Tilde         \~}
%
% \GetFileInfo{magicnum.drv}
%
% \title{The \xpackage{magicnum} package}
% \date{2011/04/10 v1.4}
% \author{Heiko Oberdiek\\\xemail{heiko.oberdiek at googlemail.com}}
%
% \maketitle
%
% \begin{abstract}
% This packages allows to access magic numbers by a hierarchical
% name system.
% \end{abstract}
%
% \tableofcontents
%
% \hypersetup{bookmarksopenlevel=2}
% \section{Documentation}
%
% \subsection{Introduction}
%
% Especially since \eTeX\ there are many integer values
% with special meanings, such as catcodes, group types, \dots
% Package \xpackage{etex}, enabled by options, defines
% macros in the user namespace for these values.
%
% This package goes another approach for storing the names and values.
% \begin{itemize}
% \item If \LuaTeX\ is available, they
% are stored in Lua tables.
% \item Without \LuaTeX\ they are remembered using internal
% macros.
% \end{itemize}
%
% \subsection{User interface}
%
% The integer values and names are organized in a hierarchical
% scheme of categories with the property names as leaves.
% Example: \eTeX's \cs{currentgrouplevel} reports |2| for a
% group caused by \cs{hbox}. This package has choosen to organize
% the group types in a main category |etex| and its subcategory
% |grouptype|:
% \begin{quote}
%   |etex.grouptype.hbox| = |2|
% \end{quote}
% The property name |hbox| in category |etex.grouptype| has value |2|.
% Dots are used to separate components.
%
% If you want to have the value, the access key is constructed by
% the category with all its components and the property name.
% For the opposite the value is used instead of the property name.
%
% Values are always integers (including negative numbers).
%
% \subsubsection{\cs{magicnum}}
%
% \begin{declcs}{magicnum} \M{access key}
% \end{declcs}
% Macro \cs{magicnum} expects an access key as argument and
% expands to the requested data. The macro is always expandable.
% In case of errors the expansion result is empty.
%
% The same macro is also used for getting a property name.
% In this case the property name part in the access key is
% replaced by the value.
%
% The catcodes
% of the resulting numbers and strings follow \TeX's tradition of
% \cs{string}, \cs{meaning}, \dots: The space has catcode 10
% (|tex.catcode.space|) and the other characters have catcode
% 12 (|tex.catcode.other|).
%
% Examples:
% \begin{quote}
%   |\magicnum{etex.grouptype.hbox}| $\Rightarrow$ |2|\\
%   |\magicnum{tex.catcode.14}| $\Rightarrow$ |comment|\\
%   |\magicnum{tex.catcode.undefined}| $\Rightarrow$ $\emptyset$
% \end{quote}
%
% \subsubsection{Properties}
%
% \begin{itemize}
% \item The components of a category are either subcategories or
%       key value pairs, but not both.
% \item The full specified property names are unique and thus
%       has one integer value exactly.
% \item Also the values inside a category are unique.
%       This condition is a prerequisite for the reverse mapping
%       of \cs{magicnum}.
% \item All names start with a letter. Only letters or digits
%       may follow.
% \end{itemize}
%
% \subsection{Data}
%
%  \subsubsection{\texorpdfstring{Category }{}\texttt{tex.catcode}}
%
% \begin{quote}
% \begin{tabular}{@{}>{\ttfamily}l>{\ttfamily}l@{}}
%    tex.catcode.escape & 0\\
%    tex.catcode.begingroup & 1\\
%    tex.catcode.endgroup & 2\\
%    tex.catcode.math & 3\\
%    tex.catcode.align & 4\\
%    tex.catcode.eol & 5\\
%    tex.catcode.parameter & 6\\
%    tex.catcode.superscript & 7\\
%    tex.catcode.subscript & 8\\
%    tex.catcode.ignore & 9\\
%    tex.catcode.space & 10\\
%    tex.catcode.letter & 11\\
%    tex.catcode.other & 12\\
%    tex.catcode.active & 13\\
%    tex.catcode.comment & 14\\
%    tex.catcode.invalid & 15\\
%  \end{tabular}
%  \end{quote}
%
%  \subsubsection{\texorpdfstring{Category }{}\texttt{etex.grouptype}}
%
% \begin{quote}
% \begin{tabular}{@{}>{\ttfamily}l>{\ttfamily}l@{}}
%    etex.grouptype.bottomlevel & 0\\
%    etex.grouptype.simple & 1\\
%    etex.grouptype.hbox & 2\\
%    etex.grouptype.adjustedhbox & 3\\
%    etex.grouptype.vbox & 4\\
%    etex.grouptype.align & 5\\
%    etex.grouptype.noalign & 6\\
%    etex.grouptype.output & 8\\
%    etex.grouptype.math & 9\\
%    etex.grouptype.disc & 10\\
%    etex.grouptype.insert & 11\\
%    etex.grouptype.vcenter & 12\\
%    etex.grouptype.mathchoice & 13\\
%    etex.grouptype.semisimple & 14\\
%    etex.grouptype.mathshift & 15\\
%    etex.grouptype.mathleft & 16\\
%  \end{tabular}
%  \end{quote}
%
%  \subsubsection{\texorpdfstring{Category }{}\texttt{etex.iftype}}
%
% \begin{quote}
% \begin{tabular}{@{}>{\ttfamily}l>{\ttfamily}l@{}}
%    etex.iftype.none & 0\\
%    etex.iftype.char & 1\\
%    etex.iftype.cat & 2\\
%    etex.iftype.num & 3\\
%    etex.iftype.dim & 4\\
%    etex.iftype.odd & 5\\
%    etex.iftype.vmode & 6\\
%    etex.iftype.hmode & 7\\
%    etex.iftype.mmode & 8\\
%    etex.iftype.inner & 9\\
%    etex.iftype.void & 10\\
%    etex.iftype.hbox & 11\\
%    etex.iftype.vbox & 12\\
%    etex.iftype.x & 13\\
%    etex.iftype.eof & 14\\
%    etex.iftype.true & 15\\
%    etex.iftype.false & 16\\
%    etex.iftype.case & 17\\
%    etex.iftype.defined & 18\\
%    etex.iftype.csname & 19\\
%    etex.iftype.fontchar & 20\\
%  \end{tabular}
%  \end{quote}
%
%  \subsubsection{\texorpdfstring{Category }{}\texttt{etex.nodetype}}
%
% \begin{quote}
% \begin{tabular}{@{}>{\ttfamily}l>{\ttfamily}l@{}}
%    etex.nodetype.none & -1\\
%    etex.nodetype.char & 0\\
%    etex.nodetype.hlist & 1\\
%    etex.nodetype.vlist & 2\\
%    etex.nodetype.rule & 3\\
%    etex.nodetype.ins & 4\\
%    etex.nodetype.mark & 5\\
%    etex.nodetype.adjust & 6\\
%    etex.nodetype.ligature & 7\\
%    etex.nodetype.disc & 8\\
%    etex.nodetype.whatsit & 9\\
%    etex.nodetype.math & 10\\
%    etex.nodetype.glue & 11\\
%    etex.nodetype.kern & 12\\
%    etex.nodetype.penalty & 13\\
%    etex.nodetype.unset & 14\\
%    etex.nodetype.maths & 15\\
%  \end{tabular}
%  \end{quote}
%
%  \subsubsection{\texorpdfstring{Category }{}\texttt{etex.interactionmode}}
%
% \begin{quote}
% \begin{tabular}{@{}>{\ttfamily}l>{\ttfamily}l@{}}
%    etex.interactionmode.batch & 0\\
%    etex.interactionmode.nonstop & 1\\
%    etex.interactionmode.scroll & 2\\
%    etex.interactionmode.errorstop & 3\\
%  \end{tabular}
%  \end{quote}
%
%  \subsubsection{\texorpdfstring{Category }{}\texttt{luatex.pdfliteral.mode}}
%
% \begin{quote}
% \begin{tabular}{@{}>{\ttfamily}l>{\ttfamily}l@{}}
%    luatex.pdfliteral.mode.setorigin & 0\\
%    luatex.pdfliteral.mode.page & 1\\
%    luatex.pdfliteral.mode.direct & 2\\
%  \end{tabular}
%  \end{quote}
%
%
% \hypersetup{bookmarksopenlevel=1}
%
% \StopEventually{
% }
%
% \section{Implementation}
%
%    \begin{macrocode}
%<*package>
%    \end{macrocode}
%
% \subsection{Reload check and package identification}
%    Reload check, especially if the package is not used with \LaTeX.
%    \begin{macrocode}
\begingroup\catcode61\catcode48\catcode32=10\relax%
  \catcode13=5 % ^^M
  \endlinechar=13 %
  \catcode35=6 % #
  \catcode39=12 % '
  \catcode44=12 % ,
  \catcode45=12 % -
  \catcode46=12 % .
  \catcode58=12 % :
  \catcode64=11 % @
  \catcode123=1 % {
  \catcode125=2 % }
  \expandafter\let\expandafter\x\csname ver@magicnum.sty\endcsname
  \ifx\x\relax % plain-TeX, first loading
  \else
    \def\empty{}%
    \ifx\x\empty % LaTeX, first loading,
      % variable is initialized, but \ProvidesPackage not yet seen
    \else
      \expandafter\ifx\csname PackageInfo\endcsname\relax
        \def\x#1#2{%
          \immediate\write-1{Package #1 Info: #2.}%
        }%
      \else
        \def\x#1#2{\PackageInfo{#1}{#2, stopped}}%
      \fi
      \x{magicnum}{The package is already loaded}%
      \aftergroup\endinput
    \fi
  \fi
\endgroup%
%    \end{macrocode}
%    Package identification:
%    \begin{macrocode}
\begingroup\catcode61\catcode48\catcode32=10\relax%
  \catcode13=5 % ^^M
  \endlinechar=13 %
  \catcode35=6 % #
  \catcode39=12 % '
  \catcode40=12 % (
  \catcode41=12 % )
  \catcode44=12 % ,
  \catcode45=12 % -
  \catcode46=12 % .
  \catcode47=12 % /
  \catcode58=12 % :
  \catcode64=11 % @
  \catcode91=12 % [
  \catcode93=12 % ]
  \catcode123=1 % {
  \catcode125=2 % }
  \expandafter\ifx\csname ProvidesPackage\endcsname\relax
    \def\x#1#2#3[#4]{\endgroup
      \immediate\write-1{Package: #3 #4}%
      \xdef#1{#4}%
    }%
  \else
    \def\x#1#2[#3]{\endgroup
      #2[{#3}]%
      \ifx#1\@undefined
        \xdef#1{#3}%
      \fi
      \ifx#1\relax
        \xdef#1{#3}%
      \fi
    }%
  \fi
\expandafter\x\csname ver@magicnum.sty\endcsname
\ProvidesPackage{magicnum}%
  [2011/04/10 v1.4 Magic numbers (HO)]%
%    \end{macrocode}
%
% \subsection{Catcodes}
%
%    \begin{macrocode}
\begingroup\catcode61\catcode48\catcode32=10\relax%
  \catcode13=5 % ^^M
  \endlinechar=13 %
  \catcode123=1 % {
  \catcode125=2 % }
  \catcode64=11 % @
  \def\x{\endgroup
    \expandafter\edef\csname magicnum@AtEnd\endcsname{%
      \endlinechar=\the\endlinechar\relax
      \catcode13=\the\catcode13\relax
      \catcode32=\the\catcode32\relax
      \catcode35=\the\catcode35\relax
      \catcode61=\the\catcode61\relax
      \catcode64=\the\catcode64\relax
      \catcode123=\the\catcode123\relax
      \catcode125=\the\catcode125\relax
    }%
  }%
\x\catcode61\catcode48\catcode32=10\relax%
\catcode13=5 % ^^M
\endlinechar=13 %
\catcode35=6 % #
\catcode64=11 % @
\catcode123=1 % {
\catcode125=2 % }
\def\TMP@EnsureCode#1#2{%
  \edef\magicnum@AtEnd{%
    \magicnum@AtEnd
    \catcode#1=\the\catcode#1\relax
  }%
  \catcode#1=#2\relax
}
\TMP@EnsureCode{34}{12}% "
\TMP@EnsureCode{39}{12}% '
\TMP@EnsureCode{40}{12}% (
\TMP@EnsureCode{41}{12}% )
\TMP@EnsureCode{42}{12}% *
\TMP@EnsureCode{44}{12}% ,
\TMP@EnsureCode{45}{12}% -
\TMP@EnsureCode{46}{12}% .
\TMP@EnsureCode{47}{12}% /
\TMP@EnsureCode{58}{12}% :
\TMP@EnsureCode{60}{12}% <
\TMP@EnsureCode{62}{12}% >
\TMP@EnsureCode{91}{12}% [
\TMP@EnsureCode{93}{12}% ]
\edef\magicnum@AtEnd{\magicnum@AtEnd\noexpand\endinput}
%    \end{macrocode}
%
% \subsection{Check for previous definition}
%
%    \begin{macrocode}
\begingroup\expandafter\expandafter\expandafter\endgroup
\expandafter\ifx\csname newcommand\endcsname\relax
  \expandafter\ifx\csname magicnum\endcsname\relax
  \else
    \input infwarerr.sty\relax
    \@PackageError{magicnum}{%
      \string\magicnum\space is already defined%
    }\@ehc
  \fi
\else
  \newcommand*{\magicnum}{}%
\fi
%    \end{macrocode}
%
% \subsection{Without \LuaTeX}
%
%    \begin{macrocode}
\begingroup\expandafter\expandafter\expandafter\endgroup
\expandafter\ifx\csname directlua\endcsname\relax
%    \end{macrocode}
%
%    \begin{macro}{\magicnum}
%    \begin{macrocode}
  \begingroup\expandafter\expandafter\expandafter\endgroup
  \expandafter\ifx\csname ifcsname\endcsname\relax
    \def\magicnum#1{%
      \expandafter\ifx\csname MG@#1\endcsname\relax
      \else
        \csname MG@#1\endcsname
      \fi
    }%
  \else
    \begingroup
      \edef\x{\endgroup
        \def\noexpand\magicnum##1{%
          \expandafter\noexpand\csname
          ifcsname\endcsname MG@##1\noexpand\endcsname
            \noexpand\csname MG@##1%
                 \noexpand\expandafter\noexpand\endcsname
          \expandafter\noexpand\csname fi\endcsname
        }%
      }%
    \x
  \fi
%    \end{macrocode}
%    \end{macro}
%
%    \begin{macrocode}
\else
%    \end{macrocode}
%
% \subsection{With \LuaTeX}
%
%    \begin{macrocode}
  \begingroup\expandafter\expandafter\expandafter\endgroup
  \expandafter\ifx\csname RequirePackage\endcsname\relax
    \input ifluatex.sty\relax
    \input infwarerr.sty\relax
  \else
    \RequirePackage{ifluatex}[2010/03/01]%
    \RequirePackage{infwarerr}[2010/04/08]%
  \fi
%    \end{macrocode}
%
%    \begin{macro}{\magicnum@directlua}
%    \begin{macrocode}
  \ifnum\luatexversion<36 %
    \def\magicnum@directlua{\directlua0 }%
  \else
    \let\magicnum@directlua\directlua
  \fi
%    \end{macrocode}
%    \end{macro}
%    \begin{macrocode}
  \magicnum@directlua{%
    require("oberdiek.magicnum")%
  }%
  \begingroup
    \def\x{2011/04/10 v1.4}%
    \def\StripPrefix#1>{}%
    \edef\x{\expandafter\StripPrefix\meaning\x}%
    \edef\y{%
      \magicnum@directlua{%
        if oberdiek.magicnum.getversion then %
          oberdiek.magicnum.getversion()%
        end%
      }%
    }%
    \ifx\x\y
    \else
      \@PackageError{magicnum}{%
        Wrong version of lua module.\MessageBreak
        Package version: \x\MessageBreak
        Lua module: \y
      }\@ehc
    \fi
  \endgroup
%    \end{macrocode}
%    \begin{macro}{\luaescapestring}
%    \begin{macrocode}
  \begingroup
    \expandafter\ifx\csname luaescapestring\endcsname\relax
      \directlua{%
        if tex.enableprimitives then %
          tex.enableprimitives('magicnum@', {'luaescapestring'})%
        end%
      }%
      \global\let\luaescapestring\magicnum@luaescapestring
    \fi
    \expandafter\ifx\csname luaescapestring\endcsname\relax
      \escapechar=92 %
      \@PackageError{magicnum}{%
        Missing \string\luaescapestring
      }\@ehc
    \fi
  \endgroup
%    \end{macrocode}
%    \end{macro}
%    \begin{macro}{\magicnum}
%    \begin{macrocode}
  \def\magicnum#1{%
    \magicnum@directlua{%
      oberdiek.magicnum.get("\luaescapestring{#1}")%
    }%
  }%
%    \end{macrocode}
%    \end{macro}
%
%    \begin{macrocode}
  \expandafter\magicnum@AtEnd
\fi%
%</package>
%    \end{macrocode}
%
% \subsection{Data}
%
% \subsubsection{Plain data}
%
%    \begin{macrocode}
%<*data>
tex.catcode
  escape = 0
  begingroup = 1
  endgroup = 2
  math = 3
  align = 4
  eol = 5
  parameter = 6
  superscript = 7
  subscript = 8
  ignore = 9
  space = 10
  letter = 11
  other = 12
  active = 13
  comment = 14
  invalid = 15
etex.grouptype
  bottomlevel = 0
  simple = 1
  hbox = 2
  adjustedhbox = 3
  vbox = 4
  align = 5
  noalign = 6
  output = 8
  math = 9
  disc = 10
  insert = 11
  vcenter = 12
  mathchoice = 13
  semisimple = 14
  mathshift = 15
  mathleft = 16
etex.iftype
  none = 0
  char = 1
  cat = 2
  num = 3
  dim = 4
  odd = 5
  vmode = 6
  hmode = 7
  mmode = 8
  inner = 9
  void = 10
  hbox = 11
  vbox = 12
  x = 13
  eof = 14
  true = 15
  false = 16
  case = 17
  defined = 18
  csname = 19
  fontchar = 20
etex.nodetype
  none = -1
  char = 0
  hlist = 1
  vlist = 2
  rule = 3
  ins = 4
  mark = 5
  adjust = 6
  ligature = 7
  disc = 8
  whatsit = 9
  math = 10
  glue = 11
  kern = 12
  penalty = 13
  unset = 14
  maths = 15
etex.interactionmode
  batch = 0
  nonstop = 1
  scroll = 2
  errorstop = 3
luatex.pdfliteral.mode
  setorigin = 0
  page = 1
  direct = 2
%</data>
%    \end{macrocode}
%
% \subsubsection{Data for \TeX}
%
%    \begin{macrocode}
%<*package>
%    \end{macrocode}
%    \begin{macro}{\magicnum@add}
%    \begin{macrocode}
\begingroup\expandafter\expandafter\expandafter\endgroup
\expandafter\ifx\csname detokenize\endcsname\relax
  \def\magicnum@add#1#2#3{%
    \expandafter\magicnum@@add
        \csname MG@#1.#2\expandafter\endcsname
        \csname MG@#1.#3\endcsname
       {#3}{#2}%
  }%
  \def\magicnum@@add#1#2#3#4{%
    \def#1{#3}%
    \def#2{#4}%
    \edef#1{%
      \expandafter\strip@prefix\meaning#1%
    }%
    \edef#2{%
      \expandafter\strip@prefix\meaning#2%
    }%
  }%
  \expandafter\ifx\csname strip@prefix\endcsname\relax
    \def\strip@prefix#1->{}%
  \fi
\else
  \def\magicnum@add#1#2#3{%
    \expandafter\edef\csname MG@#1.#2\endcsname{%
      \detokenize{#3}%
    }%
    \expandafter\edef\csname MG@#1.#3\endcsname{%
      \detokenize{#2}%
    }%
  }%
\fi
%    \end{macrocode}
%    \end{macro}
%    \begin{macrocode}
\magicnum@add{tex.catcode}{escape}{0}
\magicnum@add{tex.catcode}{begingroup}{1}
\magicnum@add{tex.catcode}{endgroup}{2}
\magicnum@add{tex.catcode}{math}{3}
\magicnum@add{tex.catcode}{align}{4}
\magicnum@add{tex.catcode}{eol}{5}
\magicnum@add{tex.catcode}{parameter}{6}
\magicnum@add{tex.catcode}{superscript}{7}
\magicnum@add{tex.catcode}{subscript}{8}
\magicnum@add{tex.catcode}{ignore}{9}
\magicnum@add{tex.catcode}{space}{10}
\magicnum@add{tex.catcode}{letter}{11}
\magicnum@add{tex.catcode}{other}{12}
\magicnum@add{tex.catcode}{active}{13}
\magicnum@add{tex.catcode}{comment}{14}
\magicnum@add{tex.catcode}{invalid}{15}
\magicnum@add{etex.grouptype}{bottomlevel}{0}
\magicnum@add{etex.grouptype}{simple}{1}
\magicnum@add{etex.grouptype}{hbox}{2}
\magicnum@add{etex.grouptype}{adjustedhbox}{3}
\magicnum@add{etex.grouptype}{vbox}{4}
\magicnum@add{etex.grouptype}{align}{5}
\magicnum@add{etex.grouptype}{noalign}{6}
\magicnum@add{etex.grouptype}{output}{8}
\magicnum@add{etex.grouptype}{math}{9}
\magicnum@add{etex.grouptype}{disc}{10}
\magicnum@add{etex.grouptype}{insert}{11}
\magicnum@add{etex.grouptype}{vcenter}{12}
\magicnum@add{etex.grouptype}{mathchoice}{13}
\magicnum@add{etex.grouptype}{semisimple}{14}
\magicnum@add{etex.grouptype}{mathshift}{15}
\magicnum@add{etex.grouptype}{mathleft}{16}
\magicnum@add{etex.iftype}{none}{0}
\magicnum@add{etex.iftype}{char}{1}
\magicnum@add{etex.iftype}{cat}{2}
\magicnum@add{etex.iftype}{num}{3}
\magicnum@add{etex.iftype}{dim}{4}
\magicnum@add{etex.iftype}{odd}{5}
\magicnum@add{etex.iftype}{vmode}{6}
\magicnum@add{etex.iftype}{hmode}{7}
\magicnum@add{etex.iftype}{mmode}{8}
\magicnum@add{etex.iftype}{inner}{9}
\magicnum@add{etex.iftype}{void}{10}
\magicnum@add{etex.iftype}{hbox}{11}
\magicnum@add{etex.iftype}{vbox}{12}
\magicnum@add{etex.iftype}{x}{13}
\magicnum@add{etex.iftype}{eof}{14}
\magicnum@add{etex.iftype}{true}{15}
\magicnum@add{etex.iftype}{false}{16}
\magicnum@add{etex.iftype}{case}{17}
\magicnum@add{etex.iftype}{defined}{18}
\magicnum@add{etex.iftype}{csname}{19}
\magicnum@add{etex.iftype}{fontchar}{20}
\magicnum@add{etex.nodetype}{none}{-1}
\magicnum@add{etex.nodetype}{char}{0}
\magicnum@add{etex.nodetype}{hlist}{1}
\magicnum@add{etex.nodetype}{vlist}{2}
\magicnum@add{etex.nodetype}{rule}{3}
\magicnum@add{etex.nodetype}{ins}{4}
\magicnum@add{etex.nodetype}{mark}{5}
\magicnum@add{etex.nodetype}{adjust}{6}
\magicnum@add{etex.nodetype}{ligature}{7}
\magicnum@add{etex.nodetype}{disc}{8}
\magicnum@add{etex.nodetype}{whatsit}{9}
\magicnum@add{etex.nodetype}{math}{10}
\magicnum@add{etex.nodetype}{glue}{11}
\magicnum@add{etex.nodetype}{kern}{12}
\magicnum@add{etex.nodetype}{penalty}{13}
\magicnum@add{etex.nodetype}{unset}{14}
\magicnum@add{etex.nodetype}{maths}{15}
\magicnum@add{etex.interactionmode}{batch}{0}
\magicnum@add{etex.interactionmode}{nonstop}{1}
\magicnum@add{etex.interactionmode}{scroll}{2}
\magicnum@add{etex.interactionmode}{errorstop}{3}
\magicnum@add{luatex.pdfliteral.mode}{setorigin}{0}
\magicnum@add{luatex.pdfliteral.mode}{page}{1}
\magicnum@add{luatex.pdfliteral.mode}{direct}{2}
%    \end{macrocode}
%    \begin{macrocode}
\magicnum@AtEnd%
%</package>
%    \end{macrocode}
%
% \subsubsection{Lua module}
%
%    \begin{macrocode}
%<*lua>
%    \end{macrocode}
%    \begin{macrocode}
module("oberdiek.magicnum", package.seeall)
%    \end{macrocode}
%    \begin{macrocode}
function getversion()
  tex.write("2011/04/10 v1.4")
end
%    \end{macrocode}
%    \begin{macrocode}
local data = {
  ["tex.catcode"] = {
    [0] = "escape",
    [1] = "begingroup",
    [2] = "endgroup",
    [3] = "math",
    [4] = "align",
    [5] = "eol",
    [6] = "parameter",
    [7] = "superscript",
    [8] = "subscript",
    [9] = "ignore",
    [10] = "space",
    [11] = "letter",
    [12] = "other",
    [13] = "active",
    [14] = "comment",
    [15] = "invalid",
    ["active"] = 13,
    ["align"] = 4,
    ["begingroup"] = 1,
    ["comment"] = 14,
    ["endgroup"] = 2,
    ["eol"] = 5,
    ["escape"] = 0,
    ["ignore"] = 9,
    ["invalid"] = 15,
    ["letter"] = 11,
    ["math"] = 3,
    ["other"] = 12,
    ["parameter"] = 6,
    ["space"] = 10,
    ["subscript"] = 8,
    ["superscript"] = 7
  },
  ["etex.grouptype"] = {
    [0] = "bottomlevel",
    [1] = "simple",
    [2] = "hbox",
    [3] = "adjustedhbox",
    [4] = "vbox",
    [5] = "align",
    [6] = "noalign",
    [8] = "output",
    [9] = "math",
    [10] = "disc",
    [11] = "insert",
    [12] = "vcenter",
    [13] = "mathchoice",
    [14] = "semisimple",
    [15] = "mathshift",
    [16] = "mathleft",
    ["adjustedhbox"] = 3,
    ["align"] = 5,
    ["bottomlevel"] = 0,
    ["disc"] = 10,
    ["hbox"] = 2,
    ["insert"] = 11,
    ["math"] = 9,
    ["mathchoice"] = 13,
    ["mathleft"] = 16,
    ["mathshift"] = 15,
    ["noalign"] = 6,
    ["output"] = 8,
    ["semisimple"] = 14,
    ["simple"] = 1,
    ["vbox"] = 4,
    ["vcenter"] = 12
  },
  ["etex.iftype"] = {
    [0] = "none",
    [1] = "char",
    [2] = "cat",
    [3] = "num",
    [4] = "dim",
    [5] = "odd",
    [6] = "vmode",
    [7] = "hmode",
    [8] = "mmode",
    [9] = "inner",
    [10] = "void",
    [11] = "hbox",
    [12] = "vbox",
    [13] = "x",
    [14] = "eof",
    [15] = "true",
    [16] = "false",
    [17] = "case",
    [18] = "defined",
    [19] = "csname",
    [20] = "fontchar",
    ["case"] = 17,
    ["cat"] = 2,
    ["char"] = 1,
    ["csname"] = 19,
    ["defined"] = 18,
    ["dim"] = 4,
    ["eof"] = 14,
    ["false"] = 16,
    ["fontchar"] = 20,
    ["hbox"] = 11,
    ["hmode"] = 7,
    ["inner"] = 9,
    ["mmode"] = 8,
    ["none"] = 0,
    ["num"] = 3,
    ["odd"] = 5,
    ["true"] = 15,
    ["vbox"] = 12,
    ["vmode"] = 6,
    ["void"] = 10,
    ["x"] = 13
  },
  ["etex.nodetype"] = {
    [-1] = "none",
    [0] = "char",
    [1] = "hlist",
    [2] = "vlist",
    [3] = "rule",
    [4] = "ins",
    [5] = "mark",
    [6] = "adjust",
    [7] = "ligature",
    [8] = "disc",
    [9] = "whatsit",
    [10] = "math",
    [11] = "glue",
    [12] = "kern",
    [13] = "penalty",
    [14] = "unset",
    [15] = "maths",
    ["adjust"] = 6,
    ["char"] = 0,
    ["disc"] = 8,
    ["glue"] = 11,
    ["hlist"] = 1,
    ["ins"] = 4,
    ["kern"] = 12,
    ["ligature"] = 7,
    ["mark"] = 5,
    ["math"] = 10,
    ["maths"] = 15,
    ["none"] = -1,
    ["penalty"] = 13,
    ["rule"] = 3,
    ["unset"] = 14,
    ["vlist"] = 2,
    ["whatsit"] = 9
  },
  ["etex.interactionmode"] = {
    [0] = "batch",
    [1] = "nonstop",
    [2] = "scroll",
    [3] = "errorstop",
    ["batch"] = 0,
    ["errorstop"] = 3,
    ["nonstop"] = 1,
    ["scroll"] = 2
  },
  ["luatex.pdfliteral.mode"] = {
    [0] = "setorigin",
    [1] = "page",
    [2] = "direct",
    ["direct"] = 2,
    ["page"] = 1,
    ["setorigin"] = 0
  }
}
%    \end{macrocode}
%    \begin{macrocode}
function get(name)
  local startpos, endpos, category, entry =
      string.find(name, "^(%a[%a%d%.]*)%.(-?[%a%d]+)$")
  if not entry then
    return
  end
  local node = data[category]
  if not node then
    return
  end
  local num = tonumber(entry)
  local value
  if num then
    value = node[num]
    if not value then
      return
    end
  else
    value = node[entry]
    if not value then
      return
    end
    value = "" .. value
  end
  tex.write(value)
end
%    \end{macrocode}
%
%    \begin{macrocode}
%</lua>
%    \end{macrocode}
%
% \section{Test}
%
% \subsection{Catcode checks for loading}
%
%    \begin{macrocode}
%<*test1>
%    \end{macrocode}
%    \begin{macrocode}
\catcode`\{=1 %
\catcode`\}=2 %
\catcode`\#=6 %
\catcode`\@=11 %
\expandafter\ifx\csname count@\endcsname\relax
  \countdef\count@=255 %
\fi
\expandafter\ifx\csname @gobble\endcsname\relax
  \long\def\@gobble#1{}%
\fi
\expandafter\ifx\csname @firstofone\endcsname\relax
  \long\def\@firstofone#1{#1}%
\fi
\expandafter\ifx\csname loop\endcsname\relax
  \expandafter\@firstofone
\else
  \expandafter\@gobble
\fi
{%
  \def\loop#1\repeat{%
    \def\body{#1}%
    \iterate
  }%
  \def\iterate{%
    \body
      \let\next\iterate
    \else
      \let\next\relax
    \fi
    \next
  }%
  \let\repeat=\fi
}%
\def\RestoreCatcodes{}
\count@=0 %
\loop
  \edef\RestoreCatcodes{%
    \RestoreCatcodes
    \catcode\the\count@=\the\catcode\count@\relax
  }%
\ifnum\count@<255 %
  \advance\count@ 1 %
\repeat

\def\RangeCatcodeInvalid#1#2{%
  \count@=#1\relax
  \loop
    \catcode\count@=15 %
  \ifnum\count@<#2\relax
    \advance\count@ 1 %
  \repeat
}
\def\RangeCatcodeCheck#1#2#3{%
  \count@=#1\relax
  \loop
    \ifnum#3=\catcode\count@
    \else
      \errmessage{%
        Character \the\count@\space
        with wrong catcode \the\catcode\count@\space
        instead of \number#3%
      }%
    \fi
  \ifnum\count@<#2\relax
    \advance\count@ 1 %
  \repeat
}
\def\space{ }
\expandafter\ifx\csname LoadCommand\endcsname\relax
  \def\LoadCommand{\input magicnum.sty\relax}%
\fi
\def\Test{%
  \RangeCatcodeInvalid{0}{47}%
  \RangeCatcodeInvalid{58}{64}%
  \RangeCatcodeInvalid{91}{96}%
  \RangeCatcodeInvalid{123}{255}%
  \catcode`\@=12 %
  \catcode`\\=0 %
  \catcode`\%=14 %
  \LoadCommand
  \RangeCatcodeCheck{0}{36}{15}%
  \RangeCatcodeCheck{37}{37}{14}%
  \RangeCatcodeCheck{38}{47}{15}%
  \RangeCatcodeCheck{48}{57}{12}%
  \RangeCatcodeCheck{58}{63}{15}%
  \RangeCatcodeCheck{64}{64}{12}%
  \RangeCatcodeCheck{65}{90}{11}%
  \RangeCatcodeCheck{91}{91}{15}%
  \RangeCatcodeCheck{92}{92}{0}%
  \RangeCatcodeCheck{93}{96}{15}%
  \RangeCatcodeCheck{97}{122}{11}%
  \RangeCatcodeCheck{123}{255}{15}%
  \RestoreCatcodes
}
\Test
\csname @@end\endcsname
\end
%    \end{macrocode}
%    \begin{macrocode}
%</test1>
%    \end{macrocode}
%
% \subsection{Test data}
%
%    \begin{macrocode}
%<*testplain>
\input magicnum.sty\relax
\def\Test#1#2{%
  \edef\result{\magicnum{#1}}%
  \edef\expect{#2}%
  \edef\expect{\expandafter\stripprefix\meaning\expect}%
  \ifx\result\expect
  \else
    \errmessage{%
      Failed: [#1] % hash-ok
      returns [\result] instead of [\expect]%
    }%
  \fi
}
\def\stripprefix#1->{}
%</testplain>
%    \end{macrocode}
%    \begin{macrocode}
%<*testlatex>
\NeedsTeXFormat{LaTeX2e}
\documentclass{minimal}
\usepackage{magicnum}[2011/04/10]
\usepackage{qstest}
\IncludeTests{*}
\LogTests{log}{*}{*}
\newcommand*{\Test}[2]{%
  \Expect*{\magicnum{#1}}{#2}%
}
\begin{qstest}{magicnum}{magicnum}
%</testlatex>
%    \end{macrocode}
%    \begin{macrocode}
%<*testdata>
\Test{tex.catcode.escape}{0}
\Test{tex.catcode.invalid}{15}
\Test{tex.catcode.unknown}{}
\Test{tex.catcode.0}{escape}
\Test{tex.catcode.15}{invalid}
\Test{etex.iftype.true}{15}
\Test{etex.iftype.false}{16}
\Test{etex.iftype.15}{true}
\Test{etex.iftype.16}{false}
\Test{etex.nodetype.none}{-1}
\Test{etex.nodetype.-1}{none}
\Test{luatex.pdfliteral.mode.direct}{2}
\Test{luatex.pdfliteral.mode.1}{page}
\Test{}{}
\Test{unknown}{}
\Test{unknown.foo.bar}{}
\Test{unknown.foo.4}{}
%</testdata>
%    \end{macrocode}
%    \begin{macrocode}
%<*testplain>
\csname @@end\endcsname
\end
%</testplain>
%<*testlatex>
\end{qstest}
\csname @@end\endcsname
%</testlatex>
%    \end{macrocode}
%
% \subsection{Small test for \hologo{iniTeX}}
%
%    \begin{macrocode}
%<*test4>
\catcode`\{=1
\catcode`\}=2
\catcode`\#=6
\input magicnum.sty\relax
\edef\x{\magicnum{tex.catcode.15}}
\edef\y{invalid}
\def\Strip#1>{}
\edef\y{\expandafter\Strip\meaning\y}
\ifx\x\y
  \immediate\write16{Ok}%
\else
  \errmessage{\x<>\y}%
\fi
\csname @@end\endcsname\end
%</test4>
%    \end{macrocode}
%
% \section{Installation}
%
% \subsection{Download}
%
% \paragraph{Package.} This package is available on
% CTAN\footnote{\url{ftp://ftp.ctan.org/tex-archive/}}:
% \begin{description}
% \item[\CTAN{macros/latex/contrib/oberdiek/magicnum.dtx}] The source file.
% \item[\CTAN{macros/latex/contrib/oberdiek/magicnum.pdf}] Documentation.
% \end{description}
%
%
% \paragraph{Bundle.} All the packages of the bundle `oberdiek'
% are also available in a TDS compliant ZIP archive. There
% the packages are already unpacked and the documentation files
% are generated. The files and directories obey the TDS standard.
% \begin{description}
% \item[\CTAN{install/macros/latex/contrib/oberdiek.tds.zip}]
% \end{description}
% \emph{TDS} refers to the standard ``A Directory Structure
% for \TeX\ Files'' (\CTAN{tds/tds.pdf}). Directories
% with \xfile{texmf} in their name are usually organized this way.
%
% \subsection{Bundle installation}
%
% \paragraph{Unpacking.} Unpack the \xfile{oberdiek.tds.zip} in the
% TDS tree (also known as \xfile{texmf} tree) of your choice.
% Example (linux):
% \begin{quote}
%   |unzip oberdiek.tds.zip -d ~/texmf|
% \end{quote}
%
% \paragraph{Script installation.}
% Check the directory \xfile{TDS:scripts/oberdiek/} for
% scripts that need further installation steps.
% Package \xpackage{attachfile2} comes with the Perl script
% \xfile{pdfatfi.pl} that should be installed in such a way
% that it can be called as \texttt{pdfatfi}.
% Example (linux):
% \begin{quote}
%   |chmod +x scripts/oberdiek/pdfatfi.pl|\\
%   |cp scripts/oberdiek/pdfatfi.pl /usr/local/bin/|
% \end{quote}
%
% \subsection{Package installation}
%
% \paragraph{Unpacking.} The \xfile{.dtx} file is a self-extracting
% \docstrip\ archive. The files are extracted by running the
% \xfile{.dtx} through \plainTeX:
% \begin{quote}
%   \verb|tex magicnum.dtx|
% \end{quote}
%
% \paragraph{TDS.} Now the different files must be moved into
% the different directories in your installation TDS tree
% (also known as \xfile{texmf} tree):
% \begin{quote}
% \def\t{^^A
% \begin{tabular}{@{}>{\ttfamily}l@{ $\rightarrow$ }>{\ttfamily}l@{}}
%   magicnum.sty & tex/generic/oberdiek/magicnum.sty\\
%   magicnum.lua & scripts/oberdiek/magicnum.lua\\
%   oberdiek.magicnum.lua & scripts/oberdiek/oberdiek.magicnum.lua\\
%   magicnum.pdf & doc/latex/oberdiek/magicnum.pdf\\
%   magicnum.txt & doc/latex/oberdiek/magicnum.txt\\
%   test/magicnum-test1.tex & doc/latex/oberdiek/test/magicnum-test1.tex\\
%   test/magicnum-test2.tex & doc/latex/oberdiek/test/magicnum-test2.tex\\
%   test/magicnum-test3.tex & doc/latex/oberdiek/test/magicnum-test3.tex\\
%   test/magicnum-test4.tex & doc/latex/oberdiek/test/magicnum-test4.tex\\
%   magicnum.dtx & source/latex/oberdiek/magicnum.dtx\\
% \end{tabular}^^A
% }^^A
% \sbox0{\t}^^A
% \ifdim\wd0>\linewidth
%   \begingroup
%     \advance\linewidth by\leftmargin
%     \advance\linewidth by\rightmargin
%   \edef\x{\endgroup
%     \def\noexpand\lw{\the\linewidth}^^A
%   }\x
%   \def\lwbox{^^A
%     \leavevmode
%     \hbox to \linewidth{^^A
%       \kern-\leftmargin\relax
%       \hss
%       \usebox0
%       \hss
%       \kern-\rightmargin\relax
%     }^^A
%   }^^A
%   \ifdim\wd0>\lw
%     \sbox0{\small\t}^^A
%     \ifdim\wd0>\linewidth
%       \ifdim\wd0>\lw
%         \sbox0{\footnotesize\t}^^A
%         \ifdim\wd0>\linewidth
%           \ifdim\wd0>\lw
%             \sbox0{\scriptsize\t}^^A
%             \ifdim\wd0>\linewidth
%               \ifdim\wd0>\lw
%                 \sbox0{\tiny\t}^^A
%                 \ifdim\wd0>\linewidth
%                   \lwbox
%                 \else
%                   \usebox0
%                 \fi
%               \else
%                 \lwbox
%               \fi
%             \else
%               \usebox0
%             \fi
%           \else
%             \lwbox
%           \fi
%         \else
%           \usebox0
%         \fi
%       \else
%         \lwbox
%       \fi
%     \else
%       \usebox0
%     \fi
%   \else
%     \lwbox
%   \fi
% \else
%   \usebox0
% \fi
% \end{quote}
% If you have a \xfile{docstrip.cfg} that configures and enables \docstrip's
% TDS installing feature, then some files can already be in the right
% place, see the documentation of \docstrip.
%
% \subsection{Refresh file name databases}
%
% If your \TeX~distribution
% (\teTeX, \mikTeX, \dots) relies on file name databases, you must refresh
% these. For example, \teTeX\ users run \verb|texhash| or
% \verb|mktexlsr|.
%
% \subsection{Some details for the interested}
%
% \paragraph{Attached source.}
%
% The PDF documentation on CTAN also includes the
% \xfile{.dtx} source file. It can be extracted by
% AcrobatReader 6 or higher. Another option is \textsf{pdftk},
% e.g. unpack the file into the current directory:
% \begin{quote}
%   \verb|pdftk magicnum.pdf unpack_files output .|
% \end{quote}
%
% \paragraph{Unpacking with \LaTeX.}
% The \xfile{.dtx} chooses its action depending on the format:
% \begin{description}
% \item[\plainTeX:] Run \docstrip\ and extract the files.
% \item[\LaTeX:] Generate the documentation.
% \end{description}
% If you insist on using \LaTeX\ for \docstrip\ (really,
% \docstrip\ does not need \LaTeX), then inform the autodetect routine
% about your intention:
% \begin{quote}
%   \verb|latex \let\install=y\input{magicnum.dtx}|
% \end{quote}
% Do not forget to quote the argument according to the demands
% of your shell.
%
% \paragraph{Generating the documentation.}
% You can use both the \xfile{.dtx} or the \xfile{.drv} to generate
% the documentation. The process can be configured by the
% configuration file \xfile{ltxdoc.cfg}. For instance, put this
% line into this file, if you want to have A4 as paper format:
% \begin{quote}
%   \verb|\PassOptionsToClass{a4paper}{article}|
% \end{quote}
% An example follows how to generate the
% documentation with pdf\LaTeX:
% \begin{quote}
%\begin{verbatim}
%pdflatex magicnum.dtx
%makeindex -s gind.ist magicnum.idx
%pdflatex magicnum.dtx
%makeindex -s gind.ist magicnum.idx
%pdflatex magicnum.dtx
%\end{verbatim}
% \end{quote}
%
% \section{Catalogue}
%
% The following XML file can be used as source for the
% \href{http://mirror.ctan.org/help/Catalogue/catalogue.html}{\TeX\ Catalogue}.
% The elements \texttt{caption} and \texttt{description} are imported
% from the original XML file from the Catalogue.
% The name of the XML file in the Catalogue is \xfile{magicnum.xml}.
%    \begin{macrocode}
%<*catalogue>
<?xml version='1.0' encoding='us-ascii'?>
<!DOCTYPE entry SYSTEM 'catalogue.dtd'>
<entry datestamp='$Date$' modifier='$Author$' id='magicnum'>
  <name>magicnum</name>
  <caption>Access TeX systems' "magic numbers".</caption>
  <authorref id='auth:oberdiek'/>
  <copyright owner='Heiko Oberdiek' year='2007,2009-2011'/>
  <license type='lppl1.3'/>
  <version number='1.4'/>
  <description>
    This package allows access to the various parameter values in
    TeX (catcode values), e-TeX (group, if and node types, and
    interaction mode), and LuaTeX (pdfliteral mode) by a hierarchical
    name system.
    <p/>
    The package is part of the <xref refid='oberdiek'>oberdiek</xref> bundle.
  </description>
  <documentation details='Package documentation'
      href='ctan:/macros/latex/contrib/oberdiek/magicnum.pdf'/>
  <ctan file='true' path='/macros/latex/contrib/oberdiek/magicnum.dtx'/>
  <miktex location='oberdiek'/>
  <texlive location='oberdiek'/>
  <install path='/macros/latex/contrib/oberdiek/oberdiek.tds.zip'/>
</entry>
%</catalogue>
%    \end{macrocode}
%
% \begin{History}
%   \begin{Version}{2007/12/12 v1.0}
%   \item
%     First public version.
%   \end{Version}
%   \begin{Version}{2009/04/10 v1.1}
%   \item
%     Adaptation to \LuaTeX\ 0.40.
%   \end{Version}
%   \begin{Version}{2010/03/09 v1.2}
%   \item
%     Adaptation to package \xpackage{luatex} 0.4.
%   \end{Version}
%   \begin{Version}{2011/03/24 v1.3}
%   \item
%     Catcode fixes.
%   \end{Version}
%   \begin{Version}{2011/04/10 v1.4}
%   \item
%     Compatibility for \hologo{iniTeX}.
%   \item
%     Dependency from package \xpackage{luatex} removed.
%   \item
%     Version check for lua module.
%   \end{Version}
% \end{History}
%
% \PrintIndex
%
% \Finale
\endinput

%        (quote the arguments according to the demands of your shell)
%
% Documentation:
%    (a) If magicnum.drv is present:
%           latex magicnum.drv
%    (b) Without magicnum.drv:
%           latex magicnum.dtx; ...
%    The class ltxdoc loads the configuration file ltxdoc.cfg
%    if available. Here you can specify further options, e.g.
%    use A4 as paper format:
%       \PassOptionsToClass{a4paper}{article}
%
%    Programm calls to get the documentation (example):
%       pdflatex magicnum.dtx
%       makeindex -s gind.ist magicnum.idx
%       pdflatex magicnum.dtx
%       makeindex -s gind.ist magicnum.idx
%       pdflatex magicnum.dtx
%
% Installation:
%    TDS:tex/generic/oberdiek/magicnum.sty
%    TDS:scripts/oberdiek/magicnum.lua
%    TDS:scripts/oberdiek/oberdiek.magicnum.lua
%    TDS:doc/latex/oberdiek/magicnum.pdf
%    TDS:doc/latex/oberdiek/magicnum.txt
%    TDS:doc/latex/oberdiek/test/magicnum-test1.tex
%    TDS:doc/latex/oberdiek/test/magicnum-test2.tex
%    TDS:doc/latex/oberdiek/test/magicnum-test3.tex
%    TDS:doc/latex/oberdiek/test/magicnum-test4.tex
%    TDS:source/latex/oberdiek/magicnum.dtx
%
%<*ignore>
\begingroup
  \catcode123=1 %
  \catcode125=2 %
  \def\x{LaTeX2e}%
\expandafter\endgroup
\ifcase 0\ifx\install y1\fi\expandafter
         \ifx\csname processbatchFile\endcsname\relax\else1\fi
         \ifx\fmtname\x\else 1\fi\relax
\else\csname fi\endcsname
%</ignore>
%<*install>
\input docstrip.tex
\Msg{************************************************************************}
\Msg{* Installation}
\Msg{* Package: magicnum 2011/04/10 v1.4 Magic numbers (HO)}
\Msg{************************************************************************}

\keepsilent
\askforoverwritefalse

\let\MetaPrefix\relax
\preamble

This is a generated file.

Project: magicnum
Version: 2011/04/10 v1.4

Copyright (C) 2007, 2009-2011 by
   Heiko Oberdiek <heiko.oberdiek at googlemail.com>

This work may be distributed and/or modified under the
conditions of the LaTeX Project Public License, either
version 1.3c of this license or (at your option) any later
version. This version of this license is in
   http://www.latex-project.org/lppl/lppl-1-3c.txt
and the latest version of this license is in
   http://www.latex-project.org/lppl.txt
and version 1.3 or later is part of all distributions of
LaTeX version 2005/12/01 or later.

This work has the LPPL maintenance status "maintained".

This Current Maintainer of this work is Heiko Oberdiek.

The Base Interpreter refers to any `TeX-Format',
because some files are installed in TDS:tex/generic//.

This work consists of the main source file magicnum.dtx
and the derived files
   magicnum.sty, magicnum.pdf, magicnum.ins, magicnum.drv, magicnum.txt,
   magicnum-test1.tex, magicnum-test2.tex, magicnum-test3.tex,
   magicnum-test4.tex, magicnum.lua, oberdiek.magicnum.lua.

\endpreamble
\let\MetaPrefix\DoubleperCent

\generate{%
  \file{magicnum.ins}{\from{magicnum.dtx}{install}}%
  \file{magicnum.drv}{\from{magicnum.dtx}{driver}}%
  \usedir{tex/generic/oberdiek}%
  \file{magicnum.sty}{\from{magicnum.dtx}{package}}%
  \usedir{doc/latex/oberdiek/test}%
  \file{magicnum-test1.tex}{\from{magicnum.dtx}{test1}}%
  \file{magicnum-test2.tex}{\from{magicnum.dtx}{testplain,testdata}}%
  \file{magicnum-test3.tex}{\from{magicnum.dtx}{testlatex,testdata}}%
  \file{magicnum-test4.tex}{\from{magicnum.dtx}{test4}}%
  \nopreamble
  \nopostamble
  \usedir{doc/latex/oberdiek}%
  \file{magicnum.txt}{\from{magicnum.dtx}{data}}%
  \usedir{source/latex/oberdiek/catalogue}%
  \file{magicnum.xml}{\from{magicnum.dtx}{catalogue}}%
}
\def\MetaPrefix{-- }
\def\defaultpostamble{%
  \MetaPrefix^^J%
  \MetaPrefix\space End of File `\outFileName'.%
}
\def\currentpostamble{\defaultpostamble}%
\generate{%
  \usedir{scripts/oberdiek}%
  \file{magicnum.lua}{\from{magicnum.dtx}{lua}}%
  \file{oberdiek.magicnum.lua}{\from{magicnum.dtx}{lua}}%
}

\catcode32=13\relax% active space
\let =\space%
\Msg{************************************************************************}
\Msg{*}
\Msg{* To finish the installation you have to move the following}
\Msg{* file into a directory searched by TeX:}
\Msg{*}
\Msg{*     magicnum.sty}
\Msg{*}
\Msg{* And install the following script files:}
\Msg{*}
\Msg{*     magicnum.lua, oberdiek.magicnum.lua}
\Msg{*}
\Msg{* To produce the documentation run the file `magicnum.drv'}
\Msg{* through LaTeX.}
\Msg{*}
\Msg{* Happy TeXing!}
\Msg{*}
\Msg{************************************************************************}

\endbatchfile
%</install>
%<*ignore>
\fi
%</ignore>
%<*driver>
\NeedsTeXFormat{LaTeX2e}
\ProvidesFile{magicnum.drv}%
  [2011/04/10 v1.4 Magic numbers (HO)]%
\documentclass{ltxdoc}
\usepackage{holtxdoc}[2011/11/22]
\usepackage{array}
\begin{document}
  \DocInput{magicnum.dtx}%
\end{document}
%</driver>
% \fi
%
% \CheckSum{755}
%
% \CharacterTable
%  {Upper-case    \A\B\C\D\E\F\G\H\I\J\K\L\M\N\O\P\Q\R\S\T\U\V\W\X\Y\Z
%   Lower-case    \a\b\c\d\e\f\g\h\i\j\k\l\m\n\o\p\q\r\s\t\u\v\w\x\y\z
%   Digits        \0\1\2\3\4\5\6\7\8\9
%   Exclamation   \!     Double quote  \"     Hash (number) \#
%   Dollar        \$     Percent       \%     Ampersand     \&
%   Acute accent  \'     Left paren    \(     Right paren   \)
%   Asterisk      \*     Plus          \+     Comma         \,
%   Minus         \-     Point         \.     Solidus       \/
%   Colon         \:     Semicolon     \;     Less than     \<
%   Equals        \=     Greater than  \>     Question mark \?
%   Commercial at \@     Left bracket  \[     Backslash     \\
%   Right bracket \]     Circumflex    \^     Underscore    \_
%   Grave accent  \`     Left brace    \{     Vertical bar  \|
%   Right brace   \}     Tilde         \~}
%
% \GetFileInfo{magicnum.drv}
%
% \title{The \xpackage{magicnum} package}
% \date{2011/04/10 v1.4}
% \author{Heiko Oberdiek\\\xemail{heiko.oberdiek at googlemail.com}}
%
% \maketitle
%
% \begin{abstract}
% This packages allows to access magic numbers by a hierarchical
% name system.
% \end{abstract}
%
% \tableofcontents
%
% \hypersetup{bookmarksopenlevel=2}
% \section{Documentation}
%
% \subsection{Introduction}
%
% Especially since \eTeX\ there are many integer values
% with special meanings, such as catcodes, group types, \dots
% Package \xpackage{etex}, enabled by options, defines
% macros in the user namespace for these values.
%
% This package goes another approach for storing the names and values.
% \begin{itemize}
% \item If \LuaTeX\ is available, they
% are stored in Lua tables.
% \item Without \LuaTeX\ they are remembered using internal
% macros.
% \end{itemize}
%
% \subsection{User interface}
%
% The integer values and names are organized in a hierarchical
% scheme of categories with the property names as leaves.
% Example: \eTeX's \cs{currentgrouplevel} reports |2| for a
% group caused by \cs{hbox}. This package has choosen to organize
% the group types in a main category |etex| and its subcategory
% |grouptype|:
% \begin{quote}
%   |etex.grouptype.hbox| = |2|
% \end{quote}
% The property name |hbox| in category |etex.grouptype| has value |2|.
% Dots are used to separate components.
%
% If you want to have the value, the access key is constructed by
% the category with all its components and the property name.
% For the opposite the value is used instead of the property name.
%
% Values are always integers (including negative numbers).
%
% \subsubsection{\cs{magicnum}}
%
% \begin{declcs}{magicnum} \M{access key}
% \end{declcs}
% Macro \cs{magicnum} expects an access key as argument and
% expands to the requested data. The macro is always expandable.
% In case of errors the expansion result is empty.
%
% The same macro is also used for getting a property name.
% In this case the property name part in the access key is
% replaced by the value.
%
% The catcodes
% of the resulting numbers and strings follow \TeX's tradition of
% \cs{string}, \cs{meaning}, \dots: The space has catcode 10
% (|tex.catcode.space|) and the other characters have catcode
% 12 (|tex.catcode.other|).
%
% Examples:
% \begin{quote}
%   |\magicnum{etex.grouptype.hbox}| $\Rightarrow$ |2|\\
%   |\magicnum{tex.catcode.14}| $\Rightarrow$ |comment|\\
%   |\magicnum{tex.catcode.undefined}| $\Rightarrow$ $\emptyset$
% \end{quote}
%
% \subsubsection{Properties}
%
% \begin{itemize}
% \item The components of a category are either subcategories or
%       key value pairs, but not both.
% \item The full specified property names are unique and thus
%       has one integer value exactly.
% \item Also the values inside a category are unique.
%       This condition is a prerequisite for the reverse mapping
%       of \cs{magicnum}.
% \item All names start with a letter. Only letters or digits
%       may follow.
% \end{itemize}
%
% \subsection{Data}
%
%  \subsubsection{\texorpdfstring{Category }{}\texttt{tex.catcode}}
%
% \begin{quote}
% \begin{tabular}{@{}>{\ttfamily}l>{\ttfamily}l@{}}
%    tex.catcode.escape & 0\\
%    tex.catcode.begingroup & 1\\
%    tex.catcode.endgroup & 2\\
%    tex.catcode.math & 3\\
%    tex.catcode.align & 4\\
%    tex.catcode.eol & 5\\
%    tex.catcode.parameter & 6\\
%    tex.catcode.superscript & 7\\
%    tex.catcode.subscript & 8\\
%    tex.catcode.ignore & 9\\
%    tex.catcode.space & 10\\
%    tex.catcode.letter & 11\\
%    tex.catcode.other & 12\\
%    tex.catcode.active & 13\\
%    tex.catcode.comment & 14\\
%    tex.catcode.invalid & 15\\
%  \end{tabular}
%  \end{quote}
%
%  \subsubsection{\texorpdfstring{Category }{}\texttt{etex.grouptype}}
%
% \begin{quote}
% \begin{tabular}{@{}>{\ttfamily}l>{\ttfamily}l@{}}
%    etex.grouptype.bottomlevel & 0\\
%    etex.grouptype.simple & 1\\
%    etex.grouptype.hbox & 2\\
%    etex.grouptype.adjustedhbox & 3\\
%    etex.grouptype.vbox & 4\\
%    etex.grouptype.align & 5\\
%    etex.grouptype.noalign & 6\\
%    etex.grouptype.output & 8\\
%    etex.grouptype.math & 9\\
%    etex.grouptype.disc & 10\\
%    etex.grouptype.insert & 11\\
%    etex.grouptype.vcenter & 12\\
%    etex.grouptype.mathchoice & 13\\
%    etex.grouptype.semisimple & 14\\
%    etex.grouptype.mathshift & 15\\
%    etex.grouptype.mathleft & 16\\
%  \end{tabular}
%  \end{quote}
%
%  \subsubsection{\texorpdfstring{Category }{}\texttt{etex.iftype}}
%
% \begin{quote}
% \begin{tabular}{@{}>{\ttfamily}l>{\ttfamily}l@{}}
%    etex.iftype.none & 0\\
%    etex.iftype.char & 1\\
%    etex.iftype.cat & 2\\
%    etex.iftype.num & 3\\
%    etex.iftype.dim & 4\\
%    etex.iftype.odd & 5\\
%    etex.iftype.vmode & 6\\
%    etex.iftype.hmode & 7\\
%    etex.iftype.mmode & 8\\
%    etex.iftype.inner & 9\\
%    etex.iftype.void & 10\\
%    etex.iftype.hbox & 11\\
%    etex.iftype.vbox & 12\\
%    etex.iftype.x & 13\\
%    etex.iftype.eof & 14\\
%    etex.iftype.true & 15\\
%    etex.iftype.false & 16\\
%    etex.iftype.case & 17\\
%    etex.iftype.defined & 18\\
%    etex.iftype.csname & 19\\
%    etex.iftype.fontchar & 20\\
%  \end{tabular}
%  \end{quote}
%
%  \subsubsection{\texorpdfstring{Category }{}\texttt{etex.nodetype}}
%
% \begin{quote}
% \begin{tabular}{@{}>{\ttfamily}l>{\ttfamily}l@{}}
%    etex.nodetype.none & -1\\
%    etex.nodetype.char & 0\\
%    etex.nodetype.hlist & 1\\
%    etex.nodetype.vlist & 2\\
%    etex.nodetype.rule & 3\\
%    etex.nodetype.ins & 4\\
%    etex.nodetype.mark & 5\\
%    etex.nodetype.adjust & 6\\
%    etex.nodetype.ligature & 7\\
%    etex.nodetype.disc & 8\\
%    etex.nodetype.whatsit & 9\\
%    etex.nodetype.math & 10\\
%    etex.nodetype.glue & 11\\
%    etex.nodetype.kern & 12\\
%    etex.nodetype.penalty & 13\\
%    etex.nodetype.unset & 14\\
%    etex.nodetype.maths & 15\\
%  \end{tabular}
%  \end{quote}
%
%  \subsubsection{\texorpdfstring{Category }{}\texttt{etex.interactionmode}}
%
% \begin{quote}
% \begin{tabular}{@{}>{\ttfamily}l>{\ttfamily}l@{}}
%    etex.interactionmode.batch & 0\\
%    etex.interactionmode.nonstop & 1\\
%    etex.interactionmode.scroll & 2\\
%    etex.interactionmode.errorstop & 3\\
%  \end{tabular}
%  \end{quote}
%
%  \subsubsection{\texorpdfstring{Category }{}\texttt{luatex.pdfliteral.mode}}
%
% \begin{quote}
% \begin{tabular}{@{}>{\ttfamily}l>{\ttfamily}l@{}}
%    luatex.pdfliteral.mode.setorigin & 0\\
%    luatex.pdfliteral.mode.page & 1\\
%    luatex.pdfliteral.mode.direct & 2\\
%  \end{tabular}
%  \end{quote}
%
%
% \hypersetup{bookmarksopenlevel=1}
%
% \StopEventually{
% }
%
% \section{Implementation}
%
%    \begin{macrocode}
%<*package>
%    \end{macrocode}
%
% \subsection{Reload check and package identification}
%    Reload check, especially if the package is not used with \LaTeX.
%    \begin{macrocode}
\begingroup\catcode61\catcode48\catcode32=10\relax%
  \catcode13=5 % ^^M
  \endlinechar=13 %
  \catcode35=6 % #
  \catcode39=12 % '
  \catcode44=12 % ,
  \catcode45=12 % -
  \catcode46=12 % .
  \catcode58=12 % :
  \catcode64=11 % @
  \catcode123=1 % {
  \catcode125=2 % }
  \expandafter\let\expandafter\x\csname ver@magicnum.sty\endcsname
  \ifx\x\relax % plain-TeX, first loading
  \else
    \def\empty{}%
    \ifx\x\empty % LaTeX, first loading,
      % variable is initialized, but \ProvidesPackage not yet seen
    \else
      \expandafter\ifx\csname PackageInfo\endcsname\relax
        \def\x#1#2{%
          \immediate\write-1{Package #1 Info: #2.}%
        }%
      \else
        \def\x#1#2{\PackageInfo{#1}{#2, stopped}}%
      \fi
      \x{magicnum}{The package is already loaded}%
      \aftergroup\endinput
    \fi
  \fi
\endgroup%
%    \end{macrocode}
%    Package identification:
%    \begin{macrocode}
\begingroup\catcode61\catcode48\catcode32=10\relax%
  \catcode13=5 % ^^M
  \endlinechar=13 %
  \catcode35=6 % #
  \catcode39=12 % '
  \catcode40=12 % (
  \catcode41=12 % )
  \catcode44=12 % ,
  \catcode45=12 % -
  \catcode46=12 % .
  \catcode47=12 % /
  \catcode58=12 % :
  \catcode64=11 % @
  \catcode91=12 % [
  \catcode93=12 % ]
  \catcode123=1 % {
  \catcode125=2 % }
  \expandafter\ifx\csname ProvidesPackage\endcsname\relax
    \def\x#1#2#3[#4]{\endgroup
      \immediate\write-1{Package: #3 #4}%
      \xdef#1{#4}%
    }%
  \else
    \def\x#1#2[#3]{\endgroup
      #2[{#3}]%
      \ifx#1\@undefined
        \xdef#1{#3}%
      \fi
      \ifx#1\relax
        \xdef#1{#3}%
      \fi
    }%
  \fi
\expandafter\x\csname ver@magicnum.sty\endcsname
\ProvidesPackage{magicnum}%
  [2011/04/10 v1.4 Magic numbers (HO)]%
%    \end{macrocode}
%
% \subsection{Catcodes}
%
%    \begin{macrocode}
\begingroup\catcode61\catcode48\catcode32=10\relax%
  \catcode13=5 % ^^M
  \endlinechar=13 %
  \catcode123=1 % {
  \catcode125=2 % }
  \catcode64=11 % @
  \def\x{\endgroup
    \expandafter\edef\csname magicnum@AtEnd\endcsname{%
      \endlinechar=\the\endlinechar\relax
      \catcode13=\the\catcode13\relax
      \catcode32=\the\catcode32\relax
      \catcode35=\the\catcode35\relax
      \catcode61=\the\catcode61\relax
      \catcode64=\the\catcode64\relax
      \catcode123=\the\catcode123\relax
      \catcode125=\the\catcode125\relax
    }%
  }%
\x\catcode61\catcode48\catcode32=10\relax%
\catcode13=5 % ^^M
\endlinechar=13 %
\catcode35=6 % #
\catcode64=11 % @
\catcode123=1 % {
\catcode125=2 % }
\def\TMP@EnsureCode#1#2{%
  \edef\magicnum@AtEnd{%
    \magicnum@AtEnd
    \catcode#1=\the\catcode#1\relax
  }%
  \catcode#1=#2\relax
}
\TMP@EnsureCode{34}{12}% "
\TMP@EnsureCode{39}{12}% '
\TMP@EnsureCode{40}{12}% (
\TMP@EnsureCode{41}{12}% )
\TMP@EnsureCode{42}{12}% *
\TMP@EnsureCode{44}{12}% ,
\TMP@EnsureCode{45}{12}% -
\TMP@EnsureCode{46}{12}% .
\TMP@EnsureCode{47}{12}% /
\TMP@EnsureCode{58}{12}% :
\TMP@EnsureCode{60}{12}% <
\TMP@EnsureCode{62}{12}% >
\TMP@EnsureCode{91}{12}% [
\TMP@EnsureCode{93}{12}% ]
\edef\magicnum@AtEnd{\magicnum@AtEnd\noexpand\endinput}
%    \end{macrocode}
%
% \subsection{Check for previous definition}
%
%    \begin{macrocode}
\begingroup\expandafter\expandafter\expandafter\endgroup
\expandafter\ifx\csname newcommand\endcsname\relax
  \expandafter\ifx\csname magicnum\endcsname\relax
  \else
    \input infwarerr.sty\relax
    \@PackageError{magicnum}{%
      \string\magicnum\space is already defined%
    }\@ehc
  \fi
\else
  \newcommand*{\magicnum}{}%
\fi
%    \end{macrocode}
%
% \subsection{Without \LuaTeX}
%
%    \begin{macrocode}
\begingroup\expandafter\expandafter\expandafter\endgroup
\expandafter\ifx\csname directlua\endcsname\relax
%    \end{macrocode}
%
%    \begin{macro}{\magicnum}
%    \begin{macrocode}
  \begingroup\expandafter\expandafter\expandafter\endgroup
  \expandafter\ifx\csname ifcsname\endcsname\relax
    \def\magicnum#1{%
      \expandafter\ifx\csname MG@#1\endcsname\relax
      \else
        \csname MG@#1\endcsname
      \fi
    }%
  \else
    \begingroup
      \edef\x{\endgroup
        \def\noexpand\magicnum##1{%
          \expandafter\noexpand\csname
          ifcsname\endcsname MG@##1\noexpand\endcsname
            \noexpand\csname MG@##1%
                 \noexpand\expandafter\noexpand\endcsname
          \expandafter\noexpand\csname fi\endcsname
        }%
      }%
    \x
  \fi
%    \end{macrocode}
%    \end{macro}
%
%    \begin{macrocode}
\else
%    \end{macrocode}
%
% \subsection{With \LuaTeX}
%
%    \begin{macrocode}
  \begingroup\expandafter\expandafter\expandafter\endgroup
  \expandafter\ifx\csname RequirePackage\endcsname\relax
    \input ifluatex.sty\relax
    \input infwarerr.sty\relax
  \else
    \RequirePackage{ifluatex}[2010/03/01]%
    \RequirePackage{infwarerr}[2010/04/08]%
  \fi
%    \end{macrocode}
%
%    \begin{macro}{\magicnum@directlua}
%    \begin{macrocode}
  \ifnum\luatexversion<36 %
    \def\magicnum@directlua{\directlua0 }%
  \else
    \let\magicnum@directlua\directlua
  \fi
%    \end{macrocode}
%    \end{macro}
%    \begin{macrocode}
  \magicnum@directlua{%
    require("oberdiek.magicnum")%
  }%
  \begingroup
    \def\x{2011/04/10 v1.4}%
    \def\StripPrefix#1>{}%
    \edef\x{\expandafter\StripPrefix\meaning\x}%
    \edef\y{%
      \magicnum@directlua{%
        if oberdiek.magicnum.getversion then %
          oberdiek.magicnum.getversion()%
        end%
      }%
    }%
    \ifx\x\y
    \else
      \@PackageError{magicnum}{%
        Wrong version of lua module.\MessageBreak
        Package version: \x\MessageBreak
        Lua module: \y
      }\@ehc
    \fi
  \endgroup
%    \end{macrocode}
%    \begin{macro}{\luaescapestring}
%    \begin{macrocode}
  \begingroup
    \expandafter\ifx\csname luaescapestring\endcsname\relax
      \directlua{%
        if tex.enableprimitives then %
          tex.enableprimitives('magicnum@', {'luaescapestring'})%
        end%
      }%
      \global\let\luaescapestring\magicnum@luaescapestring
    \fi
    \expandafter\ifx\csname luaescapestring\endcsname\relax
      \escapechar=92 %
      \@PackageError{magicnum}{%
        Missing \string\luaescapestring
      }\@ehc
    \fi
  \endgroup
%    \end{macrocode}
%    \end{macro}
%    \begin{macro}{\magicnum}
%    \begin{macrocode}
  \def\magicnum#1{%
    \magicnum@directlua{%
      oberdiek.magicnum.get("\luaescapestring{#1}")%
    }%
  }%
%    \end{macrocode}
%    \end{macro}
%
%    \begin{macrocode}
  \expandafter\magicnum@AtEnd
\fi%
%</package>
%    \end{macrocode}
%
% \subsection{Data}
%
% \subsubsection{Plain data}
%
%    \begin{macrocode}
%<*data>
tex.catcode
  escape = 0
  begingroup = 1
  endgroup = 2
  math = 3
  align = 4
  eol = 5
  parameter = 6
  superscript = 7
  subscript = 8
  ignore = 9
  space = 10
  letter = 11
  other = 12
  active = 13
  comment = 14
  invalid = 15
etex.grouptype
  bottomlevel = 0
  simple = 1
  hbox = 2
  adjustedhbox = 3
  vbox = 4
  align = 5
  noalign = 6
  output = 8
  math = 9
  disc = 10
  insert = 11
  vcenter = 12
  mathchoice = 13
  semisimple = 14
  mathshift = 15
  mathleft = 16
etex.iftype
  none = 0
  char = 1
  cat = 2
  num = 3
  dim = 4
  odd = 5
  vmode = 6
  hmode = 7
  mmode = 8
  inner = 9
  void = 10
  hbox = 11
  vbox = 12
  x = 13
  eof = 14
  true = 15
  false = 16
  case = 17
  defined = 18
  csname = 19
  fontchar = 20
etex.nodetype
  none = -1
  char = 0
  hlist = 1
  vlist = 2
  rule = 3
  ins = 4
  mark = 5
  adjust = 6
  ligature = 7
  disc = 8
  whatsit = 9
  math = 10
  glue = 11
  kern = 12
  penalty = 13
  unset = 14
  maths = 15
etex.interactionmode
  batch = 0
  nonstop = 1
  scroll = 2
  errorstop = 3
luatex.pdfliteral.mode
  setorigin = 0
  page = 1
  direct = 2
%</data>
%    \end{macrocode}
%
% \subsubsection{Data for \TeX}
%
%    \begin{macrocode}
%<*package>
%    \end{macrocode}
%    \begin{macro}{\magicnum@add}
%    \begin{macrocode}
\begingroup\expandafter\expandafter\expandafter\endgroup
\expandafter\ifx\csname detokenize\endcsname\relax
  \def\magicnum@add#1#2#3{%
    \expandafter\magicnum@@add
        \csname MG@#1.#2\expandafter\endcsname
        \csname MG@#1.#3\endcsname
       {#3}{#2}%
  }%
  \def\magicnum@@add#1#2#3#4{%
    \def#1{#3}%
    \def#2{#4}%
    \edef#1{%
      \expandafter\strip@prefix\meaning#1%
    }%
    \edef#2{%
      \expandafter\strip@prefix\meaning#2%
    }%
  }%
  \expandafter\ifx\csname strip@prefix\endcsname\relax
    \def\strip@prefix#1->{}%
  \fi
\else
  \def\magicnum@add#1#2#3{%
    \expandafter\edef\csname MG@#1.#2\endcsname{%
      \detokenize{#3}%
    }%
    \expandafter\edef\csname MG@#1.#3\endcsname{%
      \detokenize{#2}%
    }%
  }%
\fi
%    \end{macrocode}
%    \end{macro}
%    \begin{macrocode}
\magicnum@add{tex.catcode}{escape}{0}
\magicnum@add{tex.catcode}{begingroup}{1}
\magicnum@add{tex.catcode}{endgroup}{2}
\magicnum@add{tex.catcode}{math}{3}
\magicnum@add{tex.catcode}{align}{4}
\magicnum@add{tex.catcode}{eol}{5}
\magicnum@add{tex.catcode}{parameter}{6}
\magicnum@add{tex.catcode}{superscript}{7}
\magicnum@add{tex.catcode}{subscript}{8}
\magicnum@add{tex.catcode}{ignore}{9}
\magicnum@add{tex.catcode}{space}{10}
\magicnum@add{tex.catcode}{letter}{11}
\magicnum@add{tex.catcode}{other}{12}
\magicnum@add{tex.catcode}{active}{13}
\magicnum@add{tex.catcode}{comment}{14}
\magicnum@add{tex.catcode}{invalid}{15}
\magicnum@add{etex.grouptype}{bottomlevel}{0}
\magicnum@add{etex.grouptype}{simple}{1}
\magicnum@add{etex.grouptype}{hbox}{2}
\magicnum@add{etex.grouptype}{adjustedhbox}{3}
\magicnum@add{etex.grouptype}{vbox}{4}
\magicnum@add{etex.grouptype}{align}{5}
\magicnum@add{etex.grouptype}{noalign}{6}
\magicnum@add{etex.grouptype}{output}{8}
\magicnum@add{etex.grouptype}{math}{9}
\magicnum@add{etex.grouptype}{disc}{10}
\magicnum@add{etex.grouptype}{insert}{11}
\magicnum@add{etex.grouptype}{vcenter}{12}
\magicnum@add{etex.grouptype}{mathchoice}{13}
\magicnum@add{etex.grouptype}{semisimple}{14}
\magicnum@add{etex.grouptype}{mathshift}{15}
\magicnum@add{etex.grouptype}{mathleft}{16}
\magicnum@add{etex.iftype}{none}{0}
\magicnum@add{etex.iftype}{char}{1}
\magicnum@add{etex.iftype}{cat}{2}
\magicnum@add{etex.iftype}{num}{3}
\magicnum@add{etex.iftype}{dim}{4}
\magicnum@add{etex.iftype}{odd}{5}
\magicnum@add{etex.iftype}{vmode}{6}
\magicnum@add{etex.iftype}{hmode}{7}
\magicnum@add{etex.iftype}{mmode}{8}
\magicnum@add{etex.iftype}{inner}{9}
\magicnum@add{etex.iftype}{void}{10}
\magicnum@add{etex.iftype}{hbox}{11}
\magicnum@add{etex.iftype}{vbox}{12}
\magicnum@add{etex.iftype}{x}{13}
\magicnum@add{etex.iftype}{eof}{14}
\magicnum@add{etex.iftype}{true}{15}
\magicnum@add{etex.iftype}{false}{16}
\magicnum@add{etex.iftype}{case}{17}
\magicnum@add{etex.iftype}{defined}{18}
\magicnum@add{etex.iftype}{csname}{19}
\magicnum@add{etex.iftype}{fontchar}{20}
\magicnum@add{etex.nodetype}{none}{-1}
\magicnum@add{etex.nodetype}{char}{0}
\magicnum@add{etex.nodetype}{hlist}{1}
\magicnum@add{etex.nodetype}{vlist}{2}
\magicnum@add{etex.nodetype}{rule}{3}
\magicnum@add{etex.nodetype}{ins}{4}
\magicnum@add{etex.nodetype}{mark}{5}
\magicnum@add{etex.nodetype}{adjust}{6}
\magicnum@add{etex.nodetype}{ligature}{7}
\magicnum@add{etex.nodetype}{disc}{8}
\magicnum@add{etex.nodetype}{whatsit}{9}
\magicnum@add{etex.nodetype}{math}{10}
\magicnum@add{etex.nodetype}{glue}{11}
\magicnum@add{etex.nodetype}{kern}{12}
\magicnum@add{etex.nodetype}{penalty}{13}
\magicnum@add{etex.nodetype}{unset}{14}
\magicnum@add{etex.nodetype}{maths}{15}
\magicnum@add{etex.interactionmode}{batch}{0}
\magicnum@add{etex.interactionmode}{nonstop}{1}
\magicnum@add{etex.interactionmode}{scroll}{2}
\magicnum@add{etex.interactionmode}{errorstop}{3}
\magicnum@add{luatex.pdfliteral.mode}{setorigin}{0}
\magicnum@add{luatex.pdfliteral.mode}{page}{1}
\magicnum@add{luatex.pdfliteral.mode}{direct}{2}
%    \end{macrocode}
%    \begin{macrocode}
\magicnum@AtEnd%
%</package>
%    \end{macrocode}
%
% \subsubsection{Lua module}
%
%    \begin{macrocode}
%<*lua>
%    \end{macrocode}
%    \begin{macrocode}
module("oberdiek.magicnum", package.seeall)
%    \end{macrocode}
%    \begin{macrocode}
function getversion()
  tex.write("2011/04/10 v1.4")
end
%    \end{macrocode}
%    \begin{macrocode}
local data = {
  ["tex.catcode"] = {
    [0] = "escape",
    [1] = "begingroup",
    [2] = "endgroup",
    [3] = "math",
    [4] = "align",
    [5] = "eol",
    [6] = "parameter",
    [7] = "superscript",
    [8] = "subscript",
    [9] = "ignore",
    [10] = "space",
    [11] = "letter",
    [12] = "other",
    [13] = "active",
    [14] = "comment",
    [15] = "invalid",
    ["active"] = 13,
    ["align"] = 4,
    ["begingroup"] = 1,
    ["comment"] = 14,
    ["endgroup"] = 2,
    ["eol"] = 5,
    ["escape"] = 0,
    ["ignore"] = 9,
    ["invalid"] = 15,
    ["letter"] = 11,
    ["math"] = 3,
    ["other"] = 12,
    ["parameter"] = 6,
    ["space"] = 10,
    ["subscript"] = 8,
    ["superscript"] = 7
  },
  ["etex.grouptype"] = {
    [0] = "bottomlevel",
    [1] = "simple",
    [2] = "hbox",
    [3] = "adjustedhbox",
    [4] = "vbox",
    [5] = "align",
    [6] = "noalign",
    [8] = "output",
    [9] = "math",
    [10] = "disc",
    [11] = "insert",
    [12] = "vcenter",
    [13] = "mathchoice",
    [14] = "semisimple",
    [15] = "mathshift",
    [16] = "mathleft",
    ["adjustedhbox"] = 3,
    ["align"] = 5,
    ["bottomlevel"] = 0,
    ["disc"] = 10,
    ["hbox"] = 2,
    ["insert"] = 11,
    ["math"] = 9,
    ["mathchoice"] = 13,
    ["mathleft"] = 16,
    ["mathshift"] = 15,
    ["noalign"] = 6,
    ["output"] = 8,
    ["semisimple"] = 14,
    ["simple"] = 1,
    ["vbox"] = 4,
    ["vcenter"] = 12
  },
  ["etex.iftype"] = {
    [0] = "none",
    [1] = "char",
    [2] = "cat",
    [3] = "num",
    [4] = "dim",
    [5] = "odd",
    [6] = "vmode",
    [7] = "hmode",
    [8] = "mmode",
    [9] = "inner",
    [10] = "void",
    [11] = "hbox",
    [12] = "vbox",
    [13] = "x",
    [14] = "eof",
    [15] = "true",
    [16] = "false",
    [17] = "case",
    [18] = "defined",
    [19] = "csname",
    [20] = "fontchar",
    ["case"] = 17,
    ["cat"] = 2,
    ["char"] = 1,
    ["csname"] = 19,
    ["defined"] = 18,
    ["dim"] = 4,
    ["eof"] = 14,
    ["false"] = 16,
    ["fontchar"] = 20,
    ["hbox"] = 11,
    ["hmode"] = 7,
    ["inner"] = 9,
    ["mmode"] = 8,
    ["none"] = 0,
    ["num"] = 3,
    ["odd"] = 5,
    ["true"] = 15,
    ["vbox"] = 12,
    ["vmode"] = 6,
    ["void"] = 10,
    ["x"] = 13
  },
  ["etex.nodetype"] = {
    [-1] = "none",
    [0] = "char",
    [1] = "hlist",
    [2] = "vlist",
    [3] = "rule",
    [4] = "ins",
    [5] = "mark",
    [6] = "adjust",
    [7] = "ligature",
    [8] = "disc",
    [9] = "whatsit",
    [10] = "math",
    [11] = "glue",
    [12] = "kern",
    [13] = "penalty",
    [14] = "unset",
    [15] = "maths",
    ["adjust"] = 6,
    ["char"] = 0,
    ["disc"] = 8,
    ["glue"] = 11,
    ["hlist"] = 1,
    ["ins"] = 4,
    ["kern"] = 12,
    ["ligature"] = 7,
    ["mark"] = 5,
    ["math"] = 10,
    ["maths"] = 15,
    ["none"] = -1,
    ["penalty"] = 13,
    ["rule"] = 3,
    ["unset"] = 14,
    ["vlist"] = 2,
    ["whatsit"] = 9
  },
  ["etex.interactionmode"] = {
    [0] = "batch",
    [1] = "nonstop",
    [2] = "scroll",
    [3] = "errorstop",
    ["batch"] = 0,
    ["errorstop"] = 3,
    ["nonstop"] = 1,
    ["scroll"] = 2
  },
  ["luatex.pdfliteral.mode"] = {
    [0] = "setorigin",
    [1] = "page",
    [2] = "direct",
    ["direct"] = 2,
    ["page"] = 1,
    ["setorigin"] = 0
  }
}
%    \end{macrocode}
%    \begin{macrocode}
function get(name)
  local startpos, endpos, category, entry =
      string.find(name, "^(%a[%a%d%.]*)%.(-?[%a%d]+)$")
  if not entry then
    return
  end
  local node = data[category]
  if not node then
    return
  end
  local num = tonumber(entry)
  local value
  if num then
    value = node[num]
    if not value then
      return
    end
  else
    value = node[entry]
    if not value then
      return
    end
    value = "" .. value
  end
  tex.write(value)
end
%    \end{macrocode}
%
%    \begin{macrocode}
%</lua>
%    \end{macrocode}
%
% \section{Test}
%
% \subsection{Catcode checks for loading}
%
%    \begin{macrocode}
%<*test1>
%    \end{macrocode}
%    \begin{macrocode}
\catcode`\{=1 %
\catcode`\}=2 %
\catcode`\#=6 %
\catcode`\@=11 %
\expandafter\ifx\csname count@\endcsname\relax
  \countdef\count@=255 %
\fi
\expandafter\ifx\csname @gobble\endcsname\relax
  \long\def\@gobble#1{}%
\fi
\expandafter\ifx\csname @firstofone\endcsname\relax
  \long\def\@firstofone#1{#1}%
\fi
\expandafter\ifx\csname loop\endcsname\relax
  \expandafter\@firstofone
\else
  \expandafter\@gobble
\fi
{%
  \def\loop#1\repeat{%
    \def\body{#1}%
    \iterate
  }%
  \def\iterate{%
    \body
      \let\next\iterate
    \else
      \let\next\relax
    \fi
    \next
  }%
  \let\repeat=\fi
}%
\def\RestoreCatcodes{}
\count@=0 %
\loop
  \edef\RestoreCatcodes{%
    \RestoreCatcodes
    \catcode\the\count@=\the\catcode\count@\relax
  }%
\ifnum\count@<255 %
  \advance\count@ 1 %
\repeat

\def\RangeCatcodeInvalid#1#2{%
  \count@=#1\relax
  \loop
    \catcode\count@=15 %
  \ifnum\count@<#2\relax
    \advance\count@ 1 %
  \repeat
}
\def\RangeCatcodeCheck#1#2#3{%
  \count@=#1\relax
  \loop
    \ifnum#3=\catcode\count@
    \else
      \errmessage{%
        Character \the\count@\space
        with wrong catcode \the\catcode\count@\space
        instead of \number#3%
      }%
    \fi
  \ifnum\count@<#2\relax
    \advance\count@ 1 %
  \repeat
}
\def\space{ }
\expandafter\ifx\csname LoadCommand\endcsname\relax
  \def\LoadCommand{\input magicnum.sty\relax}%
\fi
\def\Test{%
  \RangeCatcodeInvalid{0}{47}%
  \RangeCatcodeInvalid{58}{64}%
  \RangeCatcodeInvalid{91}{96}%
  \RangeCatcodeInvalid{123}{255}%
  \catcode`\@=12 %
  \catcode`\\=0 %
  \catcode`\%=14 %
  \LoadCommand
  \RangeCatcodeCheck{0}{36}{15}%
  \RangeCatcodeCheck{37}{37}{14}%
  \RangeCatcodeCheck{38}{47}{15}%
  \RangeCatcodeCheck{48}{57}{12}%
  \RangeCatcodeCheck{58}{63}{15}%
  \RangeCatcodeCheck{64}{64}{12}%
  \RangeCatcodeCheck{65}{90}{11}%
  \RangeCatcodeCheck{91}{91}{15}%
  \RangeCatcodeCheck{92}{92}{0}%
  \RangeCatcodeCheck{93}{96}{15}%
  \RangeCatcodeCheck{97}{122}{11}%
  \RangeCatcodeCheck{123}{255}{15}%
  \RestoreCatcodes
}
\Test
\csname @@end\endcsname
\end
%    \end{macrocode}
%    \begin{macrocode}
%</test1>
%    \end{macrocode}
%
% \subsection{Test data}
%
%    \begin{macrocode}
%<*testplain>
\input magicnum.sty\relax
\def\Test#1#2{%
  \edef\result{\magicnum{#1}}%
  \edef\expect{#2}%
  \edef\expect{\expandafter\stripprefix\meaning\expect}%
  \ifx\result\expect
  \else
    \errmessage{%
      Failed: [#1] % hash-ok
      returns [\result] instead of [\expect]%
    }%
  \fi
}
\def\stripprefix#1->{}
%</testplain>
%    \end{macrocode}
%    \begin{macrocode}
%<*testlatex>
\NeedsTeXFormat{LaTeX2e}
\documentclass{minimal}
\usepackage{magicnum}[2011/04/10]
\usepackage{qstest}
\IncludeTests{*}
\LogTests{log}{*}{*}
\newcommand*{\Test}[2]{%
  \Expect*{\magicnum{#1}}{#2}%
}
\begin{qstest}{magicnum}{magicnum}
%</testlatex>
%    \end{macrocode}
%    \begin{macrocode}
%<*testdata>
\Test{tex.catcode.escape}{0}
\Test{tex.catcode.invalid}{15}
\Test{tex.catcode.unknown}{}
\Test{tex.catcode.0}{escape}
\Test{tex.catcode.15}{invalid}
\Test{etex.iftype.true}{15}
\Test{etex.iftype.false}{16}
\Test{etex.iftype.15}{true}
\Test{etex.iftype.16}{false}
\Test{etex.nodetype.none}{-1}
\Test{etex.nodetype.-1}{none}
\Test{luatex.pdfliteral.mode.direct}{2}
\Test{luatex.pdfliteral.mode.1}{page}
\Test{}{}
\Test{unknown}{}
\Test{unknown.foo.bar}{}
\Test{unknown.foo.4}{}
%</testdata>
%    \end{macrocode}
%    \begin{macrocode}
%<*testplain>
\csname @@end\endcsname
\end
%</testplain>
%<*testlatex>
\end{qstest}
\csname @@end\endcsname
%</testlatex>
%    \end{macrocode}
%
% \subsection{Small test for \hologo{iniTeX}}
%
%    \begin{macrocode}
%<*test4>
\catcode`\{=1
\catcode`\}=2
\catcode`\#=6
\input magicnum.sty\relax
\edef\x{\magicnum{tex.catcode.15}}
\edef\y{invalid}
\def\Strip#1>{}
\edef\y{\expandafter\Strip\meaning\y}
\ifx\x\y
  \immediate\write16{Ok}%
\else
  \errmessage{\x<>\y}%
\fi
\csname @@end\endcsname\end
%</test4>
%    \end{macrocode}
%
% \section{Installation}
%
% \subsection{Download}
%
% \paragraph{Package.} This package is available on
% CTAN\footnote{\url{ftp://ftp.ctan.org/tex-archive/}}:
% \begin{description}
% \item[\CTAN{macros/latex/contrib/oberdiek/magicnum.dtx}] The source file.
% \item[\CTAN{macros/latex/contrib/oberdiek/magicnum.pdf}] Documentation.
% \end{description}
%
%
% \paragraph{Bundle.} All the packages of the bundle `oberdiek'
% are also available in a TDS compliant ZIP archive. There
% the packages are already unpacked and the documentation files
% are generated. The files and directories obey the TDS standard.
% \begin{description}
% \item[\CTAN{install/macros/latex/contrib/oberdiek.tds.zip}]
% \end{description}
% \emph{TDS} refers to the standard ``A Directory Structure
% for \TeX\ Files'' (\CTAN{tds/tds.pdf}). Directories
% with \xfile{texmf} in their name are usually organized this way.
%
% \subsection{Bundle installation}
%
% \paragraph{Unpacking.} Unpack the \xfile{oberdiek.tds.zip} in the
% TDS tree (also known as \xfile{texmf} tree) of your choice.
% Example (linux):
% \begin{quote}
%   |unzip oberdiek.tds.zip -d ~/texmf|
% \end{quote}
%
% \paragraph{Script installation.}
% Check the directory \xfile{TDS:scripts/oberdiek/} for
% scripts that need further installation steps.
% Package \xpackage{attachfile2} comes with the Perl script
% \xfile{pdfatfi.pl} that should be installed in such a way
% that it can be called as \texttt{pdfatfi}.
% Example (linux):
% \begin{quote}
%   |chmod +x scripts/oberdiek/pdfatfi.pl|\\
%   |cp scripts/oberdiek/pdfatfi.pl /usr/local/bin/|
% \end{quote}
%
% \subsection{Package installation}
%
% \paragraph{Unpacking.} The \xfile{.dtx} file is a self-extracting
% \docstrip\ archive. The files are extracted by running the
% \xfile{.dtx} through \plainTeX:
% \begin{quote}
%   \verb|tex magicnum.dtx|
% \end{quote}
%
% \paragraph{TDS.} Now the different files must be moved into
% the different directories in your installation TDS tree
% (also known as \xfile{texmf} tree):
% \begin{quote}
% \def\t{^^A
% \begin{tabular}{@{}>{\ttfamily}l@{ $\rightarrow$ }>{\ttfamily}l@{}}
%   magicnum.sty & tex/generic/oberdiek/magicnum.sty\\
%   magicnum.lua & scripts/oberdiek/magicnum.lua\\
%   oberdiek.magicnum.lua & scripts/oberdiek/oberdiek.magicnum.lua\\
%   magicnum.pdf & doc/latex/oberdiek/magicnum.pdf\\
%   magicnum.txt & doc/latex/oberdiek/magicnum.txt\\
%   test/magicnum-test1.tex & doc/latex/oberdiek/test/magicnum-test1.tex\\
%   test/magicnum-test2.tex & doc/latex/oberdiek/test/magicnum-test2.tex\\
%   test/magicnum-test3.tex & doc/latex/oberdiek/test/magicnum-test3.tex\\
%   test/magicnum-test4.tex & doc/latex/oberdiek/test/magicnum-test4.tex\\
%   magicnum.dtx & source/latex/oberdiek/magicnum.dtx\\
% \end{tabular}^^A
% }^^A
% \sbox0{\t}^^A
% \ifdim\wd0>\linewidth
%   \begingroup
%     \advance\linewidth by\leftmargin
%     \advance\linewidth by\rightmargin
%   \edef\x{\endgroup
%     \def\noexpand\lw{\the\linewidth}^^A
%   }\x
%   \def\lwbox{^^A
%     \leavevmode
%     \hbox to \linewidth{^^A
%       \kern-\leftmargin\relax
%       \hss
%       \usebox0
%       \hss
%       \kern-\rightmargin\relax
%     }^^A
%   }^^A
%   \ifdim\wd0>\lw
%     \sbox0{\small\t}^^A
%     \ifdim\wd0>\linewidth
%       \ifdim\wd0>\lw
%         \sbox0{\footnotesize\t}^^A
%         \ifdim\wd0>\linewidth
%           \ifdim\wd0>\lw
%             \sbox0{\scriptsize\t}^^A
%             \ifdim\wd0>\linewidth
%               \ifdim\wd0>\lw
%                 \sbox0{\tiny\t}^^A
%                 \ifdim\wd0>\linewidth
%                   \lwbox
%                 \else
%                   \usebox0
%                 \fi
%               \else
%                 \lwbox
%               \fi
%             \else
%               \usebox0
%             \fi
%           \else
%             \lwbox
%           \fi
%         \else
%           \usebox0
%         \fi
%       \else
%         \lwbox
%       \fi
%     \else
%       \usebox0
%     \fi
%   \else
%     \lwbox
%   \fi
% \else
%   \usebox0
% \fi
% \end{quote}
% If you have a \xfile{docstrip.cfg} that configures and enables \docstrip's
% TDS installing feature, then some files can already be in the right
% place, see the documentation of \docstrip.
%
% \subsection{Refresh file name databases}
%
% If your \TeX~distribution
% (\teTeX, \mikTeX, \dots) relies on file name databases, you must refresh
% these. For example, \teTeX\ users run \verb|texhash| or
% \verb|mktexlsr|.
%
% \subsection{Some details for the interested}
%
% \paragraph{Attached source.}
%
% The PDF documentation on CTAN also includes the
% \xfile{.dtx} source file. It can be extracted by
% AcrobatReader 6 or higher. Another option is \textsf{pdftk},
% e.g. unpack the file into the current directory:
% \begin{quote}
%   \verb|pdftk magicnum.pdf unpack_files output .|
% \end{quote}
%
% \paragraph{Unpacking with \LaTeX.}
% The \xfile{.dtx} chooses its action depending on the format:
% \begin{description}
% \item[\plainTeX:] Run \docstrip\ and extract the files.
% \item[\LaTeX:] Generate the documentation.
% \end{description}
% If you insist on using \LaTeX\ for \docstrip\ (really,
% \docstrip\ does not need \LaTeX), then inform the autodetect routine
% about your intention:
% \begin{quote}
%   \verb|latex \let\install=y% \iffalse meta-comment
%
% File: magicnum.dtx
% Version: 2011/04/10 v1.4
% Info: Magic numbers
%
% Copyright (C) 2007, 2009-2011 by
%    Heiko Oberdiek <heiko.oberdiek at googlemail.com>
%
% This work may be distributed and/or modified under the
% conditions of the LaTeX Project Public License, either
% version 1.3c of this license or (at your option) any later
% version. This version of this license is in
%    http://www.latex-project.org/lppl/lppl-1-3c.txt
% and the latest version of this license is in
%    http://www.latex-project.org/lppl.txt
% and version 1.3 or later is part of all distributions of
% LaTeX version 2005/12/01 or later.
%
% This work has the LPPL maintenance status "maintained".
%
% This Current Maintainer of this work is Heiko Oberdiek.
%
% The Base Interpreter refers to any `TeX-Format',
% because some files are installed in TDS:tex/generic//.
%
% This work consists of the main source file magicnum.dtx
% and the derived files
%    magicnum.sty, magicnum.pdf, magicnum.ins, magicnum.drv, magicnum.txt,
%    magicnum-test1.tex, magicnum-test2.tex, magicnum-test3.tex,
%    magicnum-test4.tex, magicnum.lua, oberdiek.magicnum.lua.
%
% Distribution:
%    CTAN:macros/latex/contrib/oberdiek/magicnum.dtx
%    CTAN:macros/latex/contrib/oberdiek/magicnum.pdf
%
% Unpacking:
%    (a) If magicnum.ins is present:
%           tex magicnum.ins
%    (b) Without magicnum.ins:
%           tex magicnum.dtx
%    (c) If you insist on using LaTeX
%           latex \let\install=y\input{magicnum.dtx}
%        (quote the arguments according to the demands of your shell)
%
% Documentation:
%    (a) If magicnum.drv is present:
%           latex magicnum.drv
%    (b) Without magicnum.drv:
%           latex magicnum.dtx; ...
%    The class ltxdoc loads the configuration file ltxdoc.cfg
%    if available. Here you can specify further options, e.g.
%    use A4 as paper format:
%       \PassOptionsToClass{a4paper}{article}
%
%    Programm calls to get the documentation (example):
%       pdflatex magicnum.dtx
%       makeindex -s gind.ist magicnum.idx
%       pdflatex magicnum.dtx
%       makeindex -s gind.ist magicnum.idx
%       pdflatex magicnum.dtx
%
% Installation:
%    TDS:tex/generic/oberdiek/magicnum.sty
%    TDS:scripts/oberdiek/magicnum.lua
%    TDS:scripts/oberdiek/oberdiek.magicnum.lua
%    TDS:doc/latex/oberdiek/magicnum.pdf
%    TDS:doc/latex/oberdiek/magicnum.txt
%    TDS:doc/latex/oberdiek/test/magicnum-test1.tex
%    TDS:doc/latex/oberdiek/test/magicnum-test2.tex
%    TDS:doc/latex/oberdiek/test/magicnum-test3.tex
%    TDS:doc/latex/oberdiek/test/magicnum-test4.tex
%    TDS:source/latex/oberdiek/magicnum.dtx
%
%<*ignore>
\begingroup
  \catcode123=1 %
  \catcode125=2 %
  \def\x{LaTeX2e}%
\expandafter\endgroup
\ifcase 0\ifx\install y1\fi\expandafter
         \ifx\csname processbatchFile\endcsname\relax\else1\fi
         \ifx\fmtname\x\else 1\fi\relax
\else\csname fi\endcsname
%</ignore>
%<*install>
\input docstrip.tex
\Msg{************************************************************************}
\Msg{* Installation}
\Msg{* Package: magicnum 2011/04/10 v1.4 Magic numbers (HO)}
\Msg{************************************************************************}

\keepsilent
\askforoverwritefalse

\let\MetaPrefix\relax
\preamble

This is a generated file.

Project: magicnum
Version: 2011/04/10 v1.4

Copyright (C) 2007, 2009-2011 by
   Heiko Oberdiek <heiko.oberdiek at googlemail.com>

This work may be distributed and/or modified under the
conditions of the LaTeX Project Public License, either
version 1.3c of this license or (at your option) any later
version. This version of this license is in
   http://www.latex-project.org/lppl/lppl-1-3c.txt
and the latest version of this license is in
   http://www.latex-project.org/lppl.txt
and version 1.3 or later is part of all distributions of
LaTeX version 2005/12/01 or later.

This work has the LPPL maintenance status "maintained".

This Current Maintainer of this work is Heiko Oberdiek.

The Base Interpreter refers to any `TeX-Format',
because some files are installed in TDS:tex/generic//.

This work consists of the main source file magicnum.dtx
and the derived files
   magicnum.sty, magicnum.pdf, magicnum.ins, magicnum.drv, magicnum.txt,
   magicnum-test1.tex, magicnum-test2.tex, magicnum-test3.tex,
   magicnum-test4.tex, magicnum.lua, oberdiek.magicnum.lua.

\endpreamble
\let\MetaPrefix\DoubleperCent

\generate{%
  \file{magicnum.ins}{\from{magicnum.dtx}{install}}%
  \file{magicnum.drv}{\from{magicnum.dtx}{driver}}%
  \usedir{tex/generic/oberdiek}%
  \file{magicnum.sty}{\from{magicnum.dtx}{package}}%
  \usedir{doc/latex/oberdiek/test}%
  \file{magicnum-test1.tex}{\from{magicnum.dtx}{test1}}%
  \file{magicnum-test2.tex}{\from{magicnum.dtx}{testplain,testdata}}%
  \file{magicnum-test3.tex}{\from{magicnum.dtx}{testlatex,testdata}}%
  \file{magicnum-test4.tex}{\from{magicnum.dtx}{test4}}%
  \nopreamble
  \nopostamble
  \usedir{doc/latex/oberdiek}%
  \file{magicnum.txt}{\from{magicnum.dtx}{data}}%
  \usedir{source/latex/oberdiek/catalogue}%
  \file{magicnum.xml}{\from{magicnum.dtx}{catalogue}}%
}
\def\MetaPrefix{-- }
\def\defaultpostamble{%
  \MetaPrefix^^J%
  \MetaPrefix\space End of File `\outFileName'.%
}
\def\currentpostamble{\defaultpostamble}%
\generate{%
  \usedir{scripts/oberdiek}%
  \file{magicnum.lua}{\from{magicnum.dtx}{lua}}%
  \file{oberdiek.magicnum.lua}{\from{magicnum.dtx}{lua}}%
}

\catcode32=13\relax% active space
\let =\space%
\Msg{************************************************************************}
\Msg{*}
\Msg{* To finish the installation you have to move the following}
\Msg{* file into a directory searched by TeX:}
\Msg{*}
\Msg{*     magicnum.sty}
\Msg{*}
\Msg{* And install the following script files:}
\Msg{*}
\Msg{*     magicnum.lua, oberdiek.magicnum.lua}
\Msg{*}
\Msg{* To produce the documentation run the file `magicnum.drv'}
\Msg{* through LaTeX.}
\Msg{*}
\Msg{* Happy TeXing!}
\Msg{*}
\Msg{************************************************************************}

\endbatchfile
%</install>
%<*ignore>
\fi
%</ignore>
%<*driver>
\NeedsTeXFormat{LaTeX2e}
\ProvidesFile{magicnum.drv}%
  [2011/04/10 v1.4 Magic numbers (HO)]%
\documentclass{ltxdoc}
\usepackage{holtxdoc}[2011/11/22]
\usepackage{array}
\begin{document}
  \DocInput{magicnum.dtx}%
\end{document}
%</driver>
% \fi
%
% \CheckSum{755}
%
% \CharacterTable
%  {Upper-case    \A\B\C\D\E\F\G\H\I\J\K\L\M\N\O\P\Q\R\S\T\U\V\W\X\Y\Z
%   Lower-case    \a\b\c\d\e\f\g\h\i\j\k\l\m\n\o\p\q\r\s\t\u\v\w\x\y\z
%   Digits        \0\1\2\3\4\5\6\7\8\9
%   Exclamation   \!     Double quote  \"     Hash (number) \#
%   Dollar        \$     Percent       \%     Ampersand     \&
%   Acute accent  \'     Left paren    \(     Right paren   \)
%   Asterisk      \*     Plus          \+     Comma         \,
%   Minus         \-     Point         \.     Solidus       \/
%   Colon         \:     Semicolon     \;     Less than     \<
%   Equals        \=     Greater than  \>     Question mark \?
%   Commercial at \@     Left bracket  \[     Backslash     \\
%   Right bracket \]     Circumflex    \^     Underscore    \_
%   Grave accent  \`     Left brace    \{     Vertical bar  \|
%   Right brace   \}     Tilde         \~}
%
% \GetFileInfo{magicnum.drv}
%
% \title{The \xpackage{magicnum} package}
% \date{2011/04/10 v1.4}
% \author{Heiko Oberdiek\\\xemail{heiko.oberdiek at googlemail.com}}
%
% \maketitle
%
% \begin{abstract}
% This packages allows to access magic numbers by a hierarchical
% name system.
% \end{abstract}
%
% \tableofcontents
%
% \hypersetup{bookmarksopenlevel=2}
% \section{Documentation}
%
% \subsection{Introduction}
%
% Especially since \eTeX\ there are many integer values
% with special meanings, such as catcodes, group types, \dots
% Package \xpackage{etex}, enabled by options, defines
% macros in the user namespace for these values.
%
% This package goes another approach for storing the names and values.
% \begin{itemize}
% \item If \LuaTeX\ is available, they
% are stored in Lua tables.
% \item Without \LuaTeX\ they are remembered using internal
% macros.
% \end{itemize}
%
% \subsection{User interface}
%
% The integer values and names are organized in a hierarchical
% scheme of categories with the property names as leaves.
% Example: \eTeX's \cs{currentgrouplevel} reports |2| for a
% group caused by \cs{hbox}. This package has choosen to organize
% the group types in a main category |etex| and its subcategory
% |grouptype|:
% \begin{quote}
%   |etex.grouptype.hbox| = |2|
% \end{quote}
% The property name |hbox| in category |etex.grouptype| has value |2|.
% Dots are used to separate components.
%
% If you want to have the value, the access key is constructed by
% the category with all its components and the property name.
% For the opposite the value is used instead of the property name.
%
% Values are always integers (including negative numbers).
%
% \subsubsection{\cs{magicnum}}
%
% \begin{declcs}{magicnum} \M{access key}
% \end{declcs}
% Macro \cs{magicnum} expects an access key as argument and
% expands to the requested data. The macro is always expandable.
% In case of errors the expansion result is empty.
%
% The same macro is also used for getting a property name.
% In this case the property name part in the access key is
% replaced by the value.
%
% The catcodes
% of the resulting numbers and strings follow \TeX's tradition of
% \cs{string}, \cs{meaning}, \dots: The space has catcode 10
% (|tex.catcode.space|) and the other characters have catcode
% 12 (|tex.catcode.other|).
%
% Examples:
% \begin{quote}
%   |\magicnum{etex.grouptype.hbox}| $\Rightarrow$ |2|\\
%   |\magicnum{tex.catcode.14}| $\Rightarrow$ |comment|\\
%   |\magicnum{tex.catcode.undefined}| $\Rightarrow$ $\emptyset$
% \end{quote}
%
% \subsubsection{Properties}
%
% \begin{itemize}
% \item The components of a category are either subcategories or
%       key value pairs, but not both.
% \item The full specified property names are unique and thus
%       has one integer value exactly.
% \item Also the values inside a category are unique.
%       This condition is a prerequisite for the reverse mapping
%       of \cs{magicnum}.
% \item All names start with a letter. Only letters or digits
%       may follow.
% \end{itemize}
%
% \subsection{Data}
%
%  \subsubsection{\texorpdfstring{Category }{}\texttt{tex.catcode}}
%
% \begin{quote}
% \begin{tabular}{@{}>{\ttfamily}l>{\ttfamily}l@{}}
%    tex.catcode.escape & 0\\
%    tex.catcode.begingroup & 1\\
%    tex.catcode.endgroup & 2\\
%    tex.catcode.math & 3\\
%    tex.catcode.align & 4\\
%    tex.catcode.eol & 5\\
%    tex.catcode.parameter & 6\\
%    tex.catcode.superscript & 7\\
%    tex.catcode.subscript & 8\\
%    tex.catcode.ignore & 9\\
%    tex.catcode.space & 10\\
%    tex.catcode.letter & 11\\
%    tex.catcode.other & 12\\
%    tex.catcode.active & 13\\
%    tex.catcode.comment & 14\\
%    tex.catcode.invalid & 15\\
%  \end{tabular}
%  \end{quote}
%
%  \subsubsection{\texorpdfstring{Category }{}\texttt{etex.grouptype}}
%
% \begin{quote}
% \begin{tabular}{@{}>{\ttfamily}l>{\ttfamily}l@{}}
%    etex.grouptype.bottomlevel & 0\\
%    etex.grouptype.simple & 1\\
%    etex.grouptype.hbox & 2\\
%    etex.grouptype.adjustedhbox & 3\\
%    etex.grouptype.vbox & 4\\
%    etex.grouptype.align & 5\\
%    etex.grouptype.noalign & 6\\
%    etex.grouptype.output & 8\\
%    etex.grouptype.math & 9\\
%    etex.grouptype.disc & 10\\
%    etex.grouptype.insert & 11\\
%    etex.grouptype.vcenter & 12\\
%    etex.grouptype.mathchoice & 13\\
%    etex.grouptype.semisimple & 14\\
%    etex.grouptype.mathshift & 15\\
%    etex.grouptype.mathleft & 16\\
%  \end{tabular}
%  \end{quote}
%
%  \subsubsection{\texorpdfstring{Category }{}\texttt{etex.iftype}}
%
% \begin{quote}
% \begin{tabular}{@{}>{\ttfamily}l>{\ttfamily}l@{}}
%    etex.iftype.none & 0\\
%    etex.iftype.char & 1\\
%    etex.iftype.cat & 2\\
%    etex.iftype.num & 3\\
%    etex.iftype.dim & 4\\
%    etex.iftype.odd & 5\\
%    etex.iftype.vmode & 6\\
%    etex.iftype.hmode & 7\\
%    etex.iftype.mmode & 8\\
%    etex.iftype.inner & 9\\
%    etex.iftype.void & 10\\
%    etex.iftype.hbox & 11\\
%    etex.iftype.vbox & 12\\
%    etex.iftype.x & 13\\
%    etex.iftype.eof & 14\\
%    etex.iftype.true & 15\\
%    etex.iftype.false & 16\\
%    etex.iftype.case & 17\\
%    etex.iftype.defined & 18\\
%    etex.iftype.csname & 19\\
%    etex.iftype.fontchar & 20\\
%  \end{tabular}
%  \end{quote}
%
%  \subsubsection{\texorpdfstring{Category }{}\texttt{etex.nodetype}}
%
% \begin{quote}
% \begin{tabular}{@{}>{\ttfamily}l>{\ttfamily}l@{}}
%    etex.nodetype.none & -1\\
%    etex.nodetype.char & 0\\
%    etex.nodetype.hlist & 1\\
%    etex.nodetype.vlist & 2\\
%    etex.nodetype.rule & 3\\
%    etex.nodetype.ins & 4\\
%    etex.nodetype.mark & 5\\
%    etex.nodetype.adjust & 6\\
%    etex.nodetype.ligature & 7\\
%    etex.nodetype.disc & 8\\
%    etex.nodetype.whatsit & 9\\
%    etex.nodetype.math & 10\\
%    etex.nodetype.glue & 11\\
%    etex.nodetype.kern & 12\\
%    etex.nodetype.penalty & 13\\
%    etex.nodetype.unset & 14\\
%    etex.nodetype.maths & 15\\
%  \end{tabular}
%  \end{quote}
%
%  \subsubsection{\texorpdfstring{Category }{}\texttt{etex.interactionmode}}
%
% \begin{quote}
% \begin{tabular}{@{}>{\ttfamily}l>{\ttfamily}l@{}}
%    etex.interactionmode.batch & 0\\
%    etex.interactionmode.nonstop & 1\\
%    etex.interactionmode.scroll & 2\\
%    etex.interactionmode.errorstop & 3\\
%  \end{tabular}
%  \end{quote}
%
%  \subsubsection{\texorpdfstring{Category }{}\texttt{luatex.pdfliteral.mode}}
%
% \begin{quote}
% \begin{tabular}{@{}>{\ttfamily}l>{\ttfamily}l@{}}
%    luatex.pdfliteral.mode.setorigin & 0\\
%    luatex.pdfliteral.mode.page & 1\\
%    luatex.pdfliteral.mode.direct & 2\\
%  \end{tabular}
%  \end{quote}
%
%
% \hypersetup{bookmarksopenlevel=1}
%
% \StopEventually{
% }
%
% \section{Implementation}
%
%    \begin{macrocode}
%<*package>
%    \end{macrocode}
%
% \subsection{Reload check and package identification}
%    Reload check, especially if the package is not used with \LaTeX.
%    \begin{macrocode}
\begingroup\catcode61\catcode48\catcode32=10\relax%
  \catcode13=5 % ^^M
  \endlinechar=13 %
  \catcode35=6 % #
  \catcode39=12 % '
  \catcode44=12 % ,
  \catcode45=12 % -
  \catcode46=12 % .
  \catcode58=12 % :
  \catcode64=11 % @
  \catcode123=1 % {
  \catcode125=2 % }
  \expandafter\let\expandafter\x\csname ver@magicnum.sty\endcsname
  \ifx\x\relax % plain-TeX, first loading
  \else
    \def\empty{}%
    \ifx\x\empty % LaTeX, first loading,
      % variable is initialized, but \ProvidesPackage not yet seen
    \else
      \expandafter\ifx\csname PackageInfo\endcsname\relax
        \def\x#1#2{%
          \immediate\write-1{Package #1 Info: #2.}%
        }%
      \else
        \def\x#1#2{\PackageInfo{#1}{#2, stopped}}%
      \fi
      \x{magicnum}{The package is already loaded}%
      \aftergroup\endinput
    \fi
  \fi
\endgroup%
%    \end{macrocode}
%    Package identification:
%    \begin{macrocode}
\begingroup\catcode61\catcode48\catcode32=10\relax%
  \catcode13=5 % ^^M
  \endlinechar=13 %
  \catcode35=6 % #
  \catcode39=12 % '
  \catcode40=12 % (
  \catcode41=12 % )
  \catcode44=12 % ,
  \catcode45=12 % -
  \catcode46=12 % .
  \catcode47=12 % /
  \catcode58=12 % :
  \catcode64=11 % @
  \catcode91=12 % [
  \catcode93=12 % ]
  \catcode123=1 % {
  \catcode125=2 % }
  \expandafter\ifx\csname ProvidesPackage\endcsname\relax
    \def\x#1#2#3[#4]{\endgroup
      \immediate\write-1{Package: #3 #4}%
      \xdef#1{#4}%
    }%
  \else
    \def\x#1#2[#3]{\endgroup
      #2[{#3}]%
      \ifx#1\@undefined
        \xdef#1{#3}%
      \fi
      \ifx#1\relax
        \xdef#1{#3}%
      \fi
    }%
  \fi
\expandafter\x\csname ver@magicnum.sty\endcsname
\ProvidesPackage{magicnum}%
  [2011/04/10 v1.4 Magic numbers (HO)]%
%    \end{macrocode}
%
% \subsection{Catcodes}
%
%    \begin{macrocode}
\begingroup\catcode61\catcode48\catcode32=10\relax%
  \catcode13=5 % ^^M
  \endlinechar=13 %
  \catcode123=1 % {
  \catcode125=2 % }
  \catcode64=11 % @
  \def\x{\endgroup
    \expandafter\edef\csname magicnum@AtEnd\endcsname{%
      \endlinechar=\the\endlinechar\relax
      \catcode13=\the\catcode13\relax
      \catcode32=\the\catcode32\relax
      \catcode35=\the\catcode35\relax
      \catcode61=\the\catcode61\relax
      \catcode64=\the\catcode64\relax
      \catcode123=\the\catcode123\relax
      \catcode125=\the\catcode125\relax
    }%
  }%
\x\catcode61\catcode48\catcode32=10\relax%
\catcode13=5 % ^^M
\endlinechar=13 %
\catcode35=6 % #
\catcode64=11 % @
\catcode123=1 % {
\catcode125=2 % }
\def\TMP@EnsureCode#1#2{%
  \edef\magicnum@AtEnd{%
    \magicnum@AtEnd
    \catcode#1=\the\catcode#1\relax
  }%
  \catcode#1=#2\relax
}
\TMP@EnsureCode{34}{12}% "
\TMP@EnsureCode{39}{12}% '
\TMP@EnsureCode{40}{12}% (
\TMP@EnsureCode{41}{12}% )
\TMP@EnsureCode{42}{12}% *
\TMP@EnsureCode{44}{12}% ,
\TMP@EnsureCode{45}{12}% -
\TMP@EnsureCode{46}{12}% .
\TMP@EnsureCode{47}{12}% /
\TMP@EnsureCode{58}{12}% :
\TMP@EnsureCode{60}{12}% <
\TMP@EnsureCode{62}{12}% >
\TMP@EnsureCode{91}{12}% [
\TMP@EnsureCode{93}{12}% ]
\edef\magicnum@AtEnd{\magicnum@AtEnd\noexpand\endinput}
%    \end{macrocode}
%
% \subsection{Check for previous definition}
%
%    \begin{macrocode}
\begingroup\expandafter\expandafter\expandafter\endgroup
\expandafter\ifx\csname newcommand\endcsname\relax
  \expandafter\ifx\csname magicnum\endcsname\relax
  \else
    \input infwarerr.sty\relax
    \@PackageError{magicnum}{%
      \string\magicnum\space is already defined%
    }\@ehc
  \fi
\else
  \newcommand*{\magicnum}{}%
\fi
%    \end{macrocode}
%
% \subsection{Without \LuaTeX}
%
%    \begin{macrocode}
\begingroup\expandafter\expandafter\expandafter\endgroup
\expandafter\ifx\csname directlua\endcsname\relax
%    \end{macrocode}
%
%    \begin{macro}{\magicnum}
%    \begin{macrocode}
  \begingroup\expandafter\expandafter\expandafter\endgroup
  \expandafter\ifx\csname ifcsname\endcsname\relax
    \def\magicnum#1{%
      \expandafter\ifx\csname MG@#1\endcsname\relax
      \else
        \csname MG@#1\endcsname
      \fi
    }%
  \else
    \begingroup
      \edef\x{\endgroup
        \def\noexpand\magicnum##1{%
          \expandafter\noexpand\csname
          ifcsname\endcsname MG@##1\noexpand\endcsname
            \noexpand\csname MG@##1%
                 \noexpand\expandafter\noexpand\endcsname
          \expandafter\noexpand\csname fi\endcsname
        }%
      }%
    \x
  \fi
%    \end{macrocode}
%    \end{macro}
%
%    \begin{macrocode}
\else
%    \end{macrocode}
%
% \subsection{With \LuaTeX}
%
%    \begin{macrocode}
  \begingroup\expandafter\expandafter\expandafter\endgroup
  \expandafter\ifx\csname RequirePackage\endcsname\relax
    \input ifluatex.sty\relax
    \input infwarerr.sty\relax
  \else
    \RequirePackage{ifluatex}[2010/03/01]%
    \RequirePackage{infwarerr}[2010/04/08]%
  \fi
%    \end{macrocode}
%
%    \begin{macro}{\magicnum@directlua}
%    \begin{macrocode}
  \ifnum\luatexversion<36 %
    \def\magicnum@directlua{\directlua0 }%
  \else
    \let\magicnum@directlua\directlua
  \fi
%    \end{macrocode}
%    \end{macro}
%    \begin{macrocode}
  \magicnum@directlua{%
    require("oberdiek.magicnum")%
  }%
  \begingroup
    \def\x{2011/04/10 v1.4}%
    \def\StripPrefix#1>{}%
    \edef\x{\expandafter\StripPrefix\meaning\x}%
    \edef\y{%
      \magicnum@directlua{%
        if oberdiek.magicnum.getversion then %
          oberdiek.magicnum.getversion()%
        end%
      }%
    }%
    \ifx\x\y
    \else
      \@PackageError{magicnum}{%
        Wrong version of lua module.\MessageBreak
        Package version: \x\MessageBreak
        Lua module: \y
      }\@ehc
    \fi
  \endgroup
%    \end{macrocode}
%    \begin{macro}{\luaescapestring}
%    \begin{macrocode}
  \begingroup
    \expandafter\ifx\csname luaescapestring\endcsname\relax
      \directlua{%
        if tex.enableprimitives then %
          tex.enableprimitives('magicnum@', {'luaescapestring'})%
        end%
      }%
      \global\let\luaescapestring\magicnum@luaescapestring
    \fi
    \expandafter\ifx\csname luaescapestring\endcsname\relax
      \escapechar=92 %
      \@PackageError{magicnum}{%
        Missing \string\luaescapestring
      }\@ehc
    \fi
  \endgroup
%    \end{macrocode}
%    \end{macro}
%    \begin{macro}{\magicnum}
%    \begin{macrocode}
  \def\magicnum#1{%
    \magicnum@directlua{%
      oberdiek.magicnum.get("\luaescapestring{#1}")%
    }%
  }%
%    \end{macrocode}
%    \end{macro}
%
%    \begin{macrocode}
  \expandafter\magicnum@AtEnd
\fi%
%</package>
%    \end{macrocode}
%
% \subsection{Data}
%
% \subsubsection{Plain data}
%
%    \begin{macrocode}
%<*data>
tex.catcode
  escape = 0
  begingroup = 1
  endgroup = 2
  math = 3
  align = 4
  eol = 5
  parameter = 6
  superscript = 7
  subscript = 8
  ignore = 9
  space = 10
  letter = 11
  other = 12
  active = 13
  comment = 14
  invalid = 15
etex.grouptype
  bottomlevel = 0
  simple = 1
  hbox = 2
  adjustedhbox = 3
  vbox = 4
  align = 5
  noalign = 6
  output = 8
  math = 9
  disc = 10
  insert = 11
  vcenter = 12
  mathchoice = 13
  semisimple = 14
  mathshift = 15
  mathleft = 16
etex.iftype
  none = 0
  char = 1
  cat = 2
  num = 3
  dim = 4
  odd = 5
  vmode = 6
  hmode = 7
  mmode = 8
  inner = 9
  void = 10
  hbox = 11
  vbox = 12
  x = 13
  eof = 14
  true = 15
  false = 16
  case = 17
  defined = 18
  csname = 19
  fontchar = 20
etex.nodetype
  none = -1
  char = 0
  hlist = 1
  vlist = 2
  rule = 3
  ins = 4
  mark = 5
  adjust = 6
  ligature = 7
  disc = 8
  whatsit = 9
  math = 10
  glue = 11
  kern = 12
  penalty = 13
  unset = 14
  maths = 15
etex.interactionmode
  batch = 0
  nonstop = 1
  scroll = 2
  errorstop = 3
luatex.pdfliteral.mode
  setorigin = 0
  page = 1
  direct = 2
%</data>
%    \end{macrocode}
%
% \subsubsection{Data for \TeX}
%
%    \begin{macrocode}
%<*package>
%    \end{macrocode}
%    \begin{macro}{\magicnum@add}
%    \begin{macrocode}
\begingroup\expandafter\expandafter\expandafter\endgroup
\expandafter\ifx\csname detokenize\endcsname\relax
  \def\magicnum@add#1#2#3{%
    \expandafter\magicnum@@add
        \csname MG@#1.#2\expandafter\endcsname
        \csname MG@#1.#3\endcsname
       {#3}{#2}%
  }%
  \def\magicnum@@add#1#2#3#4{%
    \def#1{#3}%
    \def#2{#4}%
    \edef#1{%
      \expandafter\strip@prefix\meaning#1%
    }%
    \edef#2{%
      \expandafter\strip@prefix\meaning#2%
    }%
  }%
  \expandafter\ifx\csname strip@prefix\endcsname\relax
    \def\strip@prefix#1->{}%
  \fi
\else
  \def\magicnum@add#1#2#3{%
    \expandafter\edef\csname MG@#1.#2\endcsname{%
      \detokenize{#3}%
    }%
    \expandafter\edef\csname MG@#1.#3\endcsname{%
      \detokenize{#2}%
    }%
  }%
\fi
%    \end{macrocode}
%    \end{macro}
%    \begin{macrocode}
\magicnum@add{tex.catcode}{escape}{0}
\magicnum@add{tex.catcode}{begingroup}{1}
\magicnum@add{tex.catcode}{endgroup}{2}
\magicnum@add{tex.catcode}{math}{3}
\magicnum@add{tex.catcode}{align}{4}
\magicnum@add{tex.catcode}{eol}{5}
\magicnum@add{tex.catcode}{parameter}{6}
\magicnum@add{tex.catcode}{superscript}{7}
\magicnum@add{tex.catcode}{subscript}{8}
\magicnum@add{tex.catcode}{ignore}{9}
\magicnum@add{tex.catcode}{space}{10}
\magicnum@add{tex.catcode}{letter}{11}
\magicnum@add{tex.catcode}{other}{12}
\magicnum@add{tex.catcode}{active}{13}
\magicnum@add{tex.catcode}{comment}{14}
\magicnum@add{tex.catcode}{invalid}{15}
\magicnum@add{etex.grouptype}{bottomlevel}{0}
\magicnum@add{etex.grouptype}{simple}{1}
\magicnum@add{etex.grouptype}{hbox}{2}
\magicnum@add{etex.grouptype}{adjustedhbox}{3}
\magicnum@add{etex.grouptype}{vbox}{4}
\magicnum@add{etex.grouptype}{align}{5}
\magicnum@add{etex.grouptype}{noalign}{6}
\magicnum@add{etex.grouptype}{output}{8}
\magicnum@add{etex.grouptype}{math}{9}
\magicnum@add{etex.grouptype}{disc}{10}
\magicnum@add{etex.grouptype}{insert}{11}
\magicnum@add{etex.grouptype}{vcenter}{12}
\magicnum@add{etex.grouptype}{mathchoice}{13}
\magicnum@add{etex.grouptype}{semisimple}{14}
\magicnum@add{etex.grouptype}{mathshift}{15}
\magicnum@add{etex.grouptype}{mathleft}{16}
\magicnum@add{etex.iftype}{none}{0}
\magicnum@add{etex.iftype}{char}{1}
\magicnum@add{etex.iftype}{cat}{2}
\magicnum@add{etex.iftype}{num}{3}
\magicnum@add{etex.iftype}{dim}{4}
\magicnum@add{etex.iftype}{odd}{5}
\magicnum@add{etex.iftype}{vmode}{6}
\magicnum@add{etex.iftype}{hmode}{7}
\magicnum@add{etex.iftype}{mmode}{8}
\magicnum@add{etex.iftype}{inner}{9}
\magicnum@add{etex.iftype}{void}{10}
\magicnum@add{etex.iftype}{hbox}{11}
\magicnum@add{etex.iftype}{vbox}{12}
\magicnum@add{etex.iftype}{x}{13}
\magicnum@add{etex.iftype}{eof}{14}
\magicnum@add{etex.iftype}{true}{15}
\magicnum@add{etex.iftype}{false}{16}
\magicnum@add{etex.iftype}{case}{17}
\magicnum@add{etex.iftype}{defined}{18}
\magicnum@add{etex.iftype}{csname}{19}
\magicnum@add{etex.iftype}{fontchar}{20}
\magicnum@add{etex.nodetype}{none}{-1}
\magicnum@add{etex.nodetype}{char}{0}
\magicnum@add{etex.nodetype}{hlist}{1}
\magicnum@add{etex.nodetype}{vlist}{2}
\magicnum@add{etex.nodetype}{rule}{3}
\magicnum@add{etex.nodetype}{ins}{4}
\magicnum@add{etex.nodetype}{mark}{5}
\magicnum@add{etex.nodetype}{adjust}{6}
\magicnum@add{etex.nodetype}{ligature}{7}
\magicnum@add{etex.nodetype}{disc}{8}
\magicnum@add{etex.nodetype}{whatsit}{9}
\magicnum@add{etex.nodetype}{math}{10}
\magicnum@add{etex.nodetype}{glue}{11}
\magicnum@add{etex.nodetype}{kern}{12}
\magicnum@add{etex.nodetype}{penalty}{13}
\magicnum@add{etex.nodetype}{unset}{14}
\magicnum@add{etex.nodetype}{maths}{15}
\magicnum@add{etex.interactionmode}{batch}{0}
\magicnum@add{etex.interactionmode}{nonstop}{1}
\magicnum@add{etex.interactionmode}{scroll}{2}
\magicnum@add{etex.interactionmode}{errorstop}{3}
\magicnum@add{luatex.pdfliteral.mode}{setorigin}{0}
\magicnum@add{luatex.pdfliteral.mode}{page}{1}
\magicnum@add{luatex.pdfliteral.mode}{direct}{2}
%    \end{macrocode}
%    \begin{macrocode}
\magicnum@AtEnd%
%</package>
%    \end{macrocode}
%
% \subsubsection{Lua module}
%
%    \begin{macrocode}
%<*lua>
%    \end{macrocode}
%    \begin{macrocode}
module("oberdiek.magicnum", package.seeall)
%    \end{macrocode}
%    \begin{macrocode}
function getversion()
  tex.write("2011/04/10 v1.4")
end
%    \end{macrocode}
%    \begin{macrocode}
local data = {
  ["tex.catcode"] = {
    [0] = "escape",
    [1] = "begingroup",
    [2] = "endgroup",
    [3] = "math",
    [4] = "align",
    [5] = "eol",
    [6] = "parameter",
    [7] = "superscript",
    [8] = "subscript",
    [9] = "ignore",
    [10] = "space",
    [11] = "letter",
    [12] = "other",
    [13] = "active",
    [14] = "comment",
    [15] = "invalid",
    ["active"] = 13,
    ["align"] = 4,
    ["begingroup"] = 1,
    ["comment"] = 14,
    ["endgroup"] = 2,
    ["eol"] = 5,
    ["escape"] = 0,
    ["ignore"] = 9,
    ["invalid"] = 15,
    ["letter"] = 11,
    ["math"] = 3,
    ["other"] = 12,
    ["parameter"] = 6,
    ["space"] = 10,
    ["subscript"] = 8,
    ["superscript"] = 7
  },
  ["etex.grouptype"] = {
    [0] = "bottomlevel",
    [1] = "simple",
    [2] = "hbox",
    [3] = "adjustedhbox",
    [4] = "vbox",
    [5] = "align",
    [6] = "noalign",
    [8] = "output",
    [9] = "math",
    [10] = "disc",
    [11] = "insert",
    [12] = "vcenter",
    [13] = "mathchoice",
    [14] = "semisimple",
    [15] = "mathshift",
    [16] = "mathleft",
    ["adjustedhbox"] = 3,
    ["align"] = 5,
    ["bottomlevel"] = 0,
    ["disc"] = 10,
    ["hbox"] = 2,
    ["insert"] = 11,
    ["math"] = 9,
    ["mathchoice"] = 13,
    ["mathleft"] = 16,
    ["mathshift"] = 15,
    ["noalign"] = 6,
    ["output"] = 8,
    ["semisimple"] = 14,
    ["simple"] = 1,
    ["vbox"] = 4,
    ["vcenter"] = 12
  },
  ["etex.iftype"] = {
    [0] = "none",
    [1] = "char",
    [2] = "cat",
    [3] = "num",
    [4] = "dim",
    [5] = "odd",
    [6] = "vmode",
    [7] = "hmode",
    [8] = "mmode",
    [9] = "inner",
    [10] = "void",
    [11] = "hbox",
    [12] = "vbox",
    [13] = "x",
    [14] = "eof",
    [15] = "true",
    [16] = "false",
    [17] = "case",
    [18] = "defined",
    [19] = "csname",
    [20] = "fontchar",
    ["case"] = 17,
    ["cat"] = 2,
    ["char"] = 1,
    ["csname"] = 19,
    ["defined"] = 18,
    ["dim"] = 4,
    ["eof"] = 14,
    ["false"] = 16,
    ["fontchar"] = 20,
    ["hbox"] = 11,
    ["hmode"] = 7,
    ["inner"] = 9,
    ["mmode"] = 8,
    ["none"] = 0,
    ["num"] = 3,
    ["odd"] = 5,
    ["true"] = 15,
    ["vbox"] = 12,
    ["vmode"] = 6,
    ["void"] = 10,
    ["x"] = 13
  },
  ["etex.nodetype"] = {
    [-1] = "none",
    [0] = "char",
    [1] = "hlist",
    [2] = "vlist",
    [3] = "rule",
    [4] = "ins",
    [5] = "mark",
    [6] = "adjust",
    [7] = "ligature",
    [8] = "disc",
    [9] = "whatsit",
    [10] = "math",
    [11] = "glue",
    [12] = "kern",
    [13] = "penalty",
    [14] = "unset",
    [15] = "maths",
    ["adjust"] = 6,
    ["char"] = 0,
    ["disc"] = 8,
    ["glue"] = 11,
    ["hlist"] = 1,
    ["ins"] = 4,
    ["kern"] = 12,
    ["ligature"] = 7,
    ["mark"] = 5,
    ["math"] = 10,
    ["maths"] = 15,
    ["none"] = -1,
    ["penalty"] = 13,
    ["rule"] = 3,
    ["unset"] = 14,
    ["vlist"] = 2,
    ["whatsit"] = 9
  },
  ["etex.interactionmode"] = {
    [0] = "batch",
    [1] = "nonstop",
    [2] = "scroll",
    [3] = "errorstop",
    ["batch"] = 0,
    ["errorstop"] = 3,
    ["nonstop"] = 1,
    ["scroll"] = 2
  },
  ["luatex.pdfliteral.mode"] = {
    [0] = "setorigin",
    [1] = "page",
    [2] = "direct",
    ["direct"] = 2,
    ["page"] = 1,
    ["setorigin"] = 0
  }
}
%    \end{macrocode}
%    \begin{macrocode}
function get(name)
  local startpos, endpos, category, entry =
      string.find(name, "^(%a[%a%d%.]*)%.(-?[%a%d]+)$")
  if not entry then
    return
  end
  local node = data[category]
  if not node then
    return
  end
  local num = tonumber(entry)
  local value
  if num then
    value = node[num]
    if not value then
      return
    end
  else
    value = node[entry]
    if not value then
      return
    end
    value = "" .. value
  end
  tex.write(value)
end
%    \end{macrocode}
%
%    \begin{macrocode}
%</lua>
%    \end{macrocode}
%
% \section{Test}
%
% \subsection{Catcode checks for loading}
%
%    \begin{macrocode}
%<*test1>
%    \end{macrocode}
%    \begin{macrocode}
\catcode`\{=1 %
\catcode`\}=2 %
\catcode`\#=6 %
\catcode`\@=11 %
\expandafter\ifx\csname count@\endcsname\relax
  \countdef\count@=255 %
\fi
\expandafter\ifx\csname @gobble\endcsname\relax
  \long\def\@gobble#1{}%
\fi
\expandafter\ifx\csname @firstofone\endcsname\relax
  \long\def\@firstofone#1{#1}%
\fi
\expandafter\ifx\csname loop\endcsname\relax
  \expandafter\@firstofone
\else
  \expandafter\@gobble
\fi
{%
  \def\loop#1\repeat{%
    \def\body{#1}%
    \iterate
  }%
  \def\iterate{%
    \body
      \let\next\iterate
    \else
      \let\next\relax
    \fi
    \next
  }%
  \let\repeat=\fi
}%
\def\RestoreCatcodes{}
\count@=0 %
\loop
  \edef\RestoreCatcodes{%
    \RestoreCatcodes
    \catcode\the\count@=\the\catcode\count@\relax
  }%
\ifnum\count@<255 %
  \advance\count@ 1 %
\repeat

\def\RangeCatcodeInvalid#1#2{%
  \count@=#1\relax
  \loop
    \catcode\count@=15 %
  \ifnum\count@<#2\relax
    \advance\count@ 1 %
  \repeat
}
\def\RangeCatcodeCheck#1#2#3{%
  \count@=#1\relax
  \loop
    \ifnum#3=\catcode\count@
    \else
      \errmessage{%
        Character \the\count@\space
        with wrong catcode \the\catcode\count@\space
        instead of \number#3%
      }%
    \fi
  \ifnum\count@<#2\relax
    \advance\count@ 1 %
  \repeat
}
\def\space{ }
\expandafter\ifx\csname LoadCommand\endcsname\relax
  \def\LoadCommand{\input magicnum.sty\relax}%
\fi
\def\Test{%
  \RangeCatcodeInvalid{0}{47}%
  \RangeCatcodeInvalid{58}{64}%
  \RangeCatcodeInvalid{91}{96}%
  \RangeCatcodeInvalid{123}{255}%
  \catcode`\@=12 %
  \catcode`\\=0 %
  \catcode`\%=14 %
  \LoadCommand
  \RangeCatcodeCheck{0}{36}{15}%
  \RangeCatcodeCheck{37}{37}{14}%
  \RangeCatcodeCheck{38}{47}{15}%
  \RangeCatcodeCheck{48}{57}{12}%
  \RangeCatcodeCheck{58}{63}{15}%
  \RangeCatcodeCheck{64}{64}{12}%
  \RangeCatcodeCheck{65}{90}{11}%
  \RangeCatcodeCheck{91}{91}{15}%
  \RangeCatcodeCheck{92}{92}{0}%
  \RangeCatcodeCheck{93}{96}{15}%
  \RangeCatcodeCheck{97}{122}{11}%
  \RangeCatcodeCheck{123}{255}{15}%
  \RestoreCatcodes
}
\Test
\csname @@end\endcsname
\end
%    \end{macrocode}
%    \begin{macrocode}
%</test1>
%    \end{macrocode}
%
% \subsection{Test data}
%
%    \begin{macrocode}
%<*testplain>
\input magicnum.sty\relax
\def\Test#1#2{%
  \edef\result{\magicnum{#1}}%
  \edef\expect{#2}%
  \edef\expect{\expandafter\stripprefix\meaning\expect}%
  \ifx\result\expect
  \else
    \errmessage{%
      Failed: [#1] % hash-ok
      returns [\result] instead of [\expect]%
    }%
  \fi
}
\def\stripprefix#1->{}
%</testplain>
%    \end{macrocode}
%    \begin{macrocode}
%<*testlatex>
\NeedsTeXFormat{LaTeX2e}
\documentclass{minimal}
\usepackage{magicnum}[2011/04/10]
\usepackage{qstest}
\IncludeTests{*}
\LogTests{log}{*}{*}
\newcommand*{\Test}[2]{%
  \Expect*{\magicnum{#1}}{#2}%
}
\begin{qstest}{magicnum}{magicnum}
%</testlatex>
%    \end{macrocode}
%    \begin{macrocode}
%<*testdata>
\Test{tex.catcode.escape}{0}
\Test{tex.catcode.invalid}{15}
\Test{tex.catcode.unknown}{}
\Test{tex.catcode.0}{escape}
\Test{tex.catcode.15}{invalid}
\Test{etex.iftype.true}{15}
\Test{etex.iftype.false}{16}
\Test{etex.iftype.15}{true}
\Test{etex.iftype.16}{false}
\Test{etex.nodetype.none}{-1}
\Test{etex.nodetype.-1}{none}
\Test{luatex.pdfliteral.mode.direct}{2}
\Test{luatex.pdfliteral.mode.1}{page}
\Test{}{}
\Test{unknown}{}
\Test{unknown.foo.bar}{}
\Test{unknown.foo.4}{}
%</testdata>
%    \end{macrocode}
%    \begin{macrocode}
%<*testplain>
\csname @@end\endcsname
\end
%</testplain>
%<*testlatex>
\end{qstest}
\csname @@end\endcsname
%</testlatex>
%    \end{macrocode}
%
% \subsection{Small test for \hologo{iniTeX}}
%
%    \begin{macrocode}
%<*test4>
\catcode`\{=1
\catcode`\}=2
\catcode`\#=6
\input magicnum.sty\relax
\edef\x{\magicnum{tex.catcode.15}}
\edef\y{invalid}
\def\Strip#1>{}
\edef\y{\expandafter\Strip\meaning\y}
\ifx\x\y
  \immediate\write16{Ok}%
\else
  \errmessage{\x<>\y}%
\fi
\csname @@end\endcsname\end
%</test4>
%    \end{macrocode}
%
% \section{Installation}
%
% \subsection{Download}
%
% \paragraph{Package.} This package is available on
% CTAN\footnote{\url{ftp://ftp.ctan.org/tex-archive/}}:
% \begin{description}
% \item[\CTAN{macros/latex/contrib/oberdiek/magicnum.dtx}] The source file.
% \item[\CTAN{macros/latex/contrib/oberdiek/magicnum.pdf}] Documentation.
% \end{description}
%
%
% \paragraph{Bundle.} All the packages of the bundle `oberdiek'
% are also available in a TDS compliant ZIP archive. There
% the packages are already unpacked and the documentation files
% are generated. The files and directories obey the TDS standard.
% \begin{description}
% \item[\CTAN{install/macros/latex/contrib/oberdiek.tds.zip}]
% \end{description}
% \emph{TDS} refers to the standard ``A Directory Structure
% for \TeX\ Files'' (\CTAN{tds/tds.pdf}). Directories
% with \xfile{texmf} in their name are usually organized this way.
%
% \subsection{Bundle installation}
%
% \paragraph{Unpacking.} Unpack the \xfile{oberdiek.tds.zip} in the
% TDS tree (also known as \xfile{texmf} tree) of your choice.
% Example (linux):
% \begin{quote}
%   |unzip oberdiek.tds.zip -d ~/texmf|
% \end{quote}
%
% \paragraph{Script installation.}
% Check the directory \xfile{TDS:scripts/oberdiek/} for
% scripts that need further installation steps.
% Package \xpackage{attachfile2} comes with the Perl script
% \xfile{pdfatfi.pl} that should be installed in such a way
% that it can be called as \texttt{pdfatfi}.
% Example (linux):
% \begin{quote}
%   |chmod +x scripts/oberdiek/pdfatfi.pl|\\
%   |cp scripts/oberdiek/pdfatfi.pl /usr/local/bin/|
% \end{quote}
%
% \subsection{Package installation}
%
% \paragraph{Unpacking.} The \xfile{.dtx} file is a self-extracting
% \docstrip\ archive. The files are extracted by running the
% \xfile{.dtx} through \plainTeX:
% \begin{quote}
%   \verb|tex magicnum.dtx|
% \end{quote}
%
% \paragraph{TDS.} Now the different files must be moved into
% the different directories in your installation TDS tree
% (also known as \xfile{texmf} tree):
% \begin{quote}
% \def\t{^^A
% \begin{tabular}{@{}>{\ttfamily}l@{ $\rightarrow$ }>{\ttfamily}l@{}}
%   magicnum.sty & tex/generic/oberdiek/magicnum.sty\\
%   magicnum.lua & scripts/oberdiek/magicnum.lua\\
%   oberdiek.magicnum.lua & scripts/oberdiek/oberdiek.magicnum.lua\\
%   magicnum.pdf & doc/latex/oberdiek/magicnum.pdf\\
%   magicnum.txt & doc/latex/oberdiek/magicnum.txt\\
%   test/magicnum-test1.tex & doc/latex/oberdiek/test/magicnum-test1.tex\\
%   test/magicnum-test2.tex & doc/latex/oberdiek/test/magicnum-test2.tex\\
%   test/magicnum-test3.tex & doc/latex/oberdiek/test/magicnum-test3.tex\\
%   test/magicnum-test4.tex & doc/latex/oberdiek/test/magicnum-test4.tex\\
%   magicnum.dtx & source/latex/oberdiek/magicnum.dtx\\
% \end{tabular}^^A
% }^^A
% \sbox0{\t}^^A
% \ifdim\wd0>\linewidth
%   \begingroup
%     \advance\linewidth by\leftmargin
%     \advance\linewidth by\rightmargin
%   \edef\x{\endgroup
%     \def\noexpand\lw{\the\linewidth}^^A
%   }\x
%   \def\lwbox{^^A
%     \leavevmode
%     \hbox to \linewidth{^^A
%       \kern-\leftmargin\relax
%       \hss
%       \usebox0
%       \hss
%       \kern-\rightmargin\relax
%     }^^A
%   }^^A
%   \ifdim\wd0>\lw
%     \sbox0{\small\t}^^A
%     \ifdim\wd0>\linewidth
%       \ifdim\wd0>\lw
%         \sbox0{\footnotesize\t}^^A
%         \ifdim\wd0>\linewidth
%           \ifdim\wd0>\lw
%             \sbox0{\scriptsize\t}^^A
%             \ifdim\wd0>\linewidth
%               \ifdim\wd0>\lw
%                 \sbox0{\tiny\t}^^A
%                 \ifdim\wd0>\linewidth
%                   \lwbox
%                 \else
%                   \usebox0
%                 \fi
%               \else
%                 \lwbox
%               \fi
%             \else
%               \usebox0
%             \fi
%           \else
%             \lwbox
%           \fi
%         \else
%           \usebox0
%         \fi
%       \else
%         \lwbox
%       \fi
%     \else
%       \usebox0
%     \fi
%   \else
%     \lwbox
%   \fi
% \else
%   \usebox0
% \fi
% \end{quote}
% If you have a \xfile{docstrip.cfg} that configures and enables \docstrip's
% TDS installing feature, then some files can already be in the right
% place, see the documentation of \docstrip.
%
% \subsection{Refresh file name databases}
%
% If your \TeX~distribution
% (\teTeX, \mikTeX, \dots) relies on file name databases, you must refresh
% these. For example, \teTeX\ users run \verb|texhash| or
% \verb|mktexlsr|.
%
% \subsection{Some details for the interested}
%
% \paragraph{Attached source.}
%
% The PDF documentation on CTAN also includes the
% \xfile{.dtx} source file. It can be extracted by
% AcrobatReader 6 or higher. Another option is \textsf{pdftk},
% e.g. unpack the file into the current directory:
% \begin{quote}
%   \verb|pdftk magicnum.pdf unpack_files output .|
% \end{quote}
%
% \paragraph{Unpacking with \LaTeX.}
% The \xfile{.dtx} chooses its action depending on the format:
% \begin{description}
% \item[\plainTeX:] Run \docstrip\ and extract the files.
% \item[\LaTeX:] Generate the documentation.
% \end{description}
% If you insist on using \LaTeX\ for \docstrip\ (really,
% \docstrip\ does not need \LaTeX), then inform the autodetect routine
% about your intention:
% \begin{quote}
%   \verb|latex \let\install=y\input{magicnum.dtx}|
% \end{quote}
% Do not forget to quote the argument according to the demands
% of your shell.
%
% \paragraph{Generating the documentation.}
% You can use both the \xfile{.dtx} or the \xfile{.drv} to generate
% the documentation. The process can be configured by the
% configuration file \xfile{ltxdoc.cfg}. For instance, put this
% line into this file, if you want to have A4 as paper format:
% \begin{quote}
%   \verb|\PassOptionsToClass{a4paper}{article}|
% \end{quote}
% An example follows how to generate the
% documentation with pdf\LaTeX:
% \begin{quote}
%\begin{verbatim}
%pdflatex magicnum.dtx
%makeindex -s gind.ist magicnum.idx
%pdflatex magicnum.dtx
%makeindex -s gind.ist magicnum.idx
%pdflatex magicnum.dtx
%\end{verbatim}
% \end{quote}
%
% \section{Catalogue}
%
% The following XML file can be used as source for the
% \href{http://mirror.ctan.org/help/Catalogue/catalogue.html}{\TeX\ Catalogue}.
% The elements \texttt{caption} and \texttt{description} are imported
% from the original XML file from the Catalogue.
% The name of the XML file in the Catalogue is \xfile{magicnum.xml}.
%    \begin{macrocode}
%<*catalogue>
<?xml version='1.0' encoding='us-ascii'?>
<!DOCTYPE entry SYSTEM 'catalogue.dtd'>
<entry datestamp='$Date$' modifier='$Author$' id='magicnum'>
  <name>magicnum</name>
  <caption>Access TeX systems' "magic numbers".</caption>
  <authorref id='auth:oberdiek'/>
  <copyright owner='Heiko Oberdiek' year='2007,2009-2011'/>
  <license type='lppl1.3'/>
  <version number='1.4'/>
  <description>
    This package allows access to the various parameter values in
    TeX (catcode values), e-TeX (group, if and node types, and
    interaction mode), and LuaTeX (pdfliteral mode) by a hierarchical
    name system.
    <p/>
    The package is part of the <xref refid='oberdiek'>oberdiek</xref> bundle.
  </description>
  <documentation details='Package documentation'
      href='ctan:/macros/latex/contrib/oberdiek/magicnum.pdf'/>
  <ctan file='true' path='/macros/latex/contrib/oberdiek/magicnum.dtx'/>
  <miktex location='oberdiek'/>
  <texlive location='oberdiek'/>
  <install path='/macros/latex/contrib/oberdiek/oberdiek.tds.zip'/>
</entry>
%</catalogue>
%    \end{macrocode}
%
% \begin{History}
%   \begin{Version}{2007/12/12 v1.0}
%   \item
%     First public version.
%   \end{Version}
%   \begin{Version}{2009/04/10 v1.1}
%   \item
%     Adaptation to \LuaTeX\ 0.40.
%   \end{Version}
%   \begin{Version}{2010/03/09 v1.2}
%   \item
%     Adaptation to package \xpackage{luatex} 0.4.
%   \end{Version}
%   \begin{Version}{2011/03/24 v1.3}
%   \item
%     Catcode fixes.
%   \end{Version}
%   \begin{Version}{2011/04/10 v1.4}
%   \item
%     Compatibility for \hologo{iniTeX}.
%   \item
%     Dependency from package \xpackage{luatex} removed.
%   \item
%     Version check for lua module.
%   \end{Version}
% \end{History}
%
% \PrintIndex
%
% \Finale
\endinput
|
% \end{quote}
% Do not forget to quote the argument according to the demands
% of your shell.
%
% \paragraph{Generating the documentation.}
% You can use both the \xfile{.dtx} or the \xfile{.drv} to generate
% the documentation. The process can be configured by the
% configuration file \xfile{ltxdoc.cfg}. For instance, put this
% line into this file, if you want to have A4 as paper format:
% \begin{quote}
%   \verb|\PassOptionsToClass{a4paper}{article}|
% \end{quote}
% An example follows how to generate the
% documentation with pdf\LaTeX:
% \begin{quote}
%\begin{verbatim}
%pdflatex magicnum.dtx
%makeindex -s gind.ist magicnum.idx
%pdflatex magicnum.dtx
%makeindex -s gind.ist magicnum.idx
%pdflatex magicnum.dtx
%\end{verbatim}
% \end{quote}
%
% \section{Catalogue}
%
% The following XML file can be used as source for the
% \href{http://mirror.ctan.org/help/Catalogue/catalogue.html}{\TeX\ Catalogue}.
% The elements \texttt{caption} and \texttt{description} are imported
% from the original XML file from the Catalogue.
% The name of the XML file in the Catalogue is \xfile{magicnum.xml}.
%    \begin{macrocode}
%<*catalogue>
<?xml version='1.0' encoding='us-ascii'?>
<!DOCTYPE entry SYSTEM 'catalogue.dtd'>
<entry datestamp='$Date$' modifier='$Author$' id='magicnum'>
  <name>magicnum</name>
  <caption>Access TeX systems' "magic numbers".</caption>
  <authorref id='auth:oberdiek'/>
  <copyright owner='Heiko Oberdiek' year='2007,2009-2011'/>
  <license type='lppl1.3'/>
  <version number='1.4'/>
  <description>
    This package allows access to the various parameter values in
    TeX (catcode values), e-TeX (group, if and node types, and
    interaction mode), and LuaTeX (pdfliteral mode) by a hierarchical
    name system.
    <p/>
    The package is part of the <xref refid='oberdiek'>oberdiek</xref> bundle.
  </description>
  <documentation details='Package documentation'
      href='ctan:/macros/latex/contrib/oberdiek/magicnum.pdf'/>
  <ctan file='true' path='/macros/latex/contrib/oberdiek/magicnum.dtx'/>
  <miktex location='oberdiek'/>
  <texlive location='oberdiek'/>
  <install path='/macros/latex/contrib/oberdiek/oberdiek.tds.zip'/>
</entry>
%</catalogue>
%    \end{macrocode}
%
% \begin{History}
%   \begin{Version}{2007/12/12 v1.0}
%   \item
%     First public version.
%   \end{Version}
%   \begin{Version}{2009/04/10 v1.1}
%   \item
%     Adaptation to \LuaTeX\ 0.40.
%   \end{Version}
%   \begin{Version}{2010/03/09 v1.2}
%   \item
%     Adaptation to package \xpackage{luatex} 0.4.
%   \end{Version}
%   \begin{Version}{2011/03/24 v1.3}
%   \item
%     Catcode fixes.
%   \end{Version}
%   \begin{Version}{2011/04/10 v1.4}
%   \item
%     Compatibility for \hologo{iniTeX}.
%   \item
%     Dependency from package \xpackage{luatex} removed.
%   \item
%     Version check for lua module.
%   \end{Version}
% \end{History}
%
% \PrintIndex
%
% \Finale
\endinput
|
% \end{quote}
% Do not forget to quote the argument according to the demands
% of your shell.
%
% \paragraph{Generating the documentation.}
% You can use both the \xfile{.dtx} or the \xfile{.drv} to generate
% the documentation. The process can be configured by the
% configuration file \xfile{ltxdoc.cfg}. For instance, put this
% line into this file, if you want to have A4 as paper format:
% \begin{quote}
%   \verb|\PassOptionsToClass{a4paper}{article}|
% \end{quote}
% An example follows how to generate the
% documentation with pdf\LaTeX:
% \begin{quote}
%\begin{verbatim}
%pdflatex magicnum.dtx
%makeindex -s gind.ist magicnum.idx
%pdflatex magicnum.dtx
%makeindex -s gind.ist magicnum.idx
%pdflatex magicnum.dtx
%\end{verbatim}
% \end{quote}
%
% \section{Catalogue}
%
% The following XML file can be used as source for the
% \href{http://mirror.ctan.org/help/Catalogue/catalogue.html}{\TeX\ Catalogue}.
% The elements \texttt{caption} and \texttt{description} are imported
% from the original XML file from the Catalogue.
% The name of the XML file in the Catalogue is \xfile{magicnum.xml}.
%    \begin{macrocode}
%<*catalogue>
<?xml version='1.0' encoding='us-ascii'?>
<!DOCTYPE entry SYSTEM 'catalogue.dtd'>
<entry datestamp='$Date$' modifier='$Author$' id='magicnum'>
  <name>magicnum</name>
  <caption>Access TeX systems' "magic numbers".</caption>
  <authorref id='auth:oberdiek'/>
  <copyright owner='Heiko Oberdiek' year='2007,2009-2011'/>
  <license type='lppl1.3'/>
  <version number='1.4'/>
  <description>
    This package allows access to the various parameter values in
    TeX (catcode values), e-TeX (group, if and node types, and
    interaction mode), and LuaTeX (pdfliteral mode) by a hierarchical
    name system.
    <p/>
    The package is part of the <xref refid='oberdiek'>oberdiek</xref> bundle.
  </description>
  <documentation details='Package documentation'
      href='ctan:/macros/latex/contrib/oberdiek/magicnum.pdf'/>
  <ctan file='true' path='/macros/latex/contrib/oberdiek/magicnum.dtx'/>
  <miktex location='oberdiek'/>
  <texlive location='oberdiek'/>
  <install path='/macros/latex/contrib/oberdiek/oberdiek.tds.zip'/>
</entry>
%</catalogue>
%    \end{macrocode}
%
% \begin{History}
%   \begin{Version}{2007/12/12 v1.0}
%   \item
%     First public version.
%   \end{Version}
%   \begin{Version}{2009/04/10 v1.1}
%   \item
%     Adaptation to \LuaTeX\ 0.40.
%   \end{Version}
%   \begin{Version}{2010/03/09 v1.2}
%   \item
%     Adaptation to package \xpackage{luatex} 0.4.
%   \end{Version}
%   \begin{Version}{2011/03/24 v1.3}
%   \item
%     Catcode fixes.
%   \end{Version}
%   \begin{Version}{2011/04/10 v1.4}
%   \item
%     Compatibility for \hologo{iniTeX}.
%   \item
%     Dependency from package \xpackage{luatex} removed.
%   \item
%     Version check for lua module.
%   \end{Version}
% \end{History}
%
% \PrintIndex
%
% \Finale
\endinput
|
% \end{quote}
% Do not forget to quote the argument according to the demands
% of your shell.
%
% \paragraph{Generating the documentation.}
% You can use both the \xfile{.dtx} or the \xfile{.drv} to generate
% the documentation. The process can be configured by the
% configuration file \xfile{ltxdoc.cfg}. For instance, put this
% line into this file, if you want to have A4 as paper format:
% \begin{quote}
%   \verb|\PassOptionsToClass{a4paper}{article}|
% \end{quote}
% An example follows how to generate the
% documentation with pdf\LaTeX:
% \begin{quote}
%\begin{verbatim}
%pdflatex magicnum.dtx
%makeindex -s gind.ist magicnum.idx
%pdflatex magicnum.dtx
%makeindex -s gind.ist magicnum.idx
%pdflatex magicnum.dtx
%\end{verbatim}
% \end{quote}
%
% \section{Catalogue}
%
% The following XML file can be used as source for the
% \href{http://mirror.ctan.org/help/Catalogue/catalogue.html}{\TeX\ Catalogue}.
% The elements \texttt{caption} and \texttt{description} are imported
% from the original XML file from the Catalogue.
% The name of the XML file in the Catalogue is \xfile{magicnum.xml}.
%    \begin{macrocode}
%<*catalogue>
<?xml version='1.0' encoding='us-ascii'?>
<!DOCTYPE entry SYSTEM 'catalogue.dtd'>
<entry datestamp='$Date$' modifier='$Author$' id='magicnum'>
  <name>magicnum</name>
  <caption>Access TeX systems' "magic numbers".</caption>
  <authorref id='auth:oberdiek'/>
  <copyright owner='Heiko Oberdiek' year='2007,2009-2011'/>
  <license type='lppl1.3'/>
  <version number='1.4'/>
  <description>
    This package allows access to the various parameter values in
    TeX (catcode values), e-TeX (group, if and node types, and
    interaction mode), and LuaTeX (pdfliteral mode) by a hierarchical
    name system.
    <p/>
    The package is part of the <xref refid='oberdiek'>oberdiek</xref> bundle.
  </description>
  <documentation details='Package documentation'
      href='ctan:/macros/latex/contrib/oberdiek/magicnum.pdf'/>
  <ctan file='true' path='/macros/latex/contrib/oberdiek/magicnum.dtx'/>
  <miktex location='oberdiek'/>
  <texlive location='oberdiek'/>
  <install path='/macros/latex/contrib/oberdiek/oberdiek.tds.zip'/>
</entry>
%</catalogue>
%    \end{macrocode}
%
% \begin{History}
%   \begin{Version}{2007/12/12 v1.0}
%   \item
%     First public version.
%   \end{Version}
%   \begin{Version}{2009/04/10 v1.1}
%   \item
%     Adaptation to \LuaTeX\ 0.40.
%   \end{Version}
%   \begin{Version}{2010/03/09 v1.2}
%   \item
%     Adaptation to package \xpackage{luatex} 0.4.
%   \end{Version}
%   \begin{Version}{2011/03/24 v1.3}
%   \item
%     Catcode fixes.
%   \end{Version}
%   \begin{Version}{2011/04/10 v1.4}
%   \item
%     Compatibility for \hologo{iniTeX}.
%   \item
%     Dependency from package \xpackage{luatex} removed.
%   \item
%     Version check for lua module.
%   \end{Version}
% \end{History}
%
% \PrintIndex
%
% \Finale
\endinput
